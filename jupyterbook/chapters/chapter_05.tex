
\chapter{الحساب الشرعي}

\section{مقدمة}
هذه هي المقدمة للفصل الأول.




\section{الحساب داخل في المعاملات}

ومن رحمة الله بنا أنه أباح لنا التعامل بالمكيال حجما، وبالميزان وزنا، وبالحساب عدا لكل ما هو قابل للقياس كالحجم والوزن وغير ذلك. ومن حكمة الله وعدله أنه كلفنا ووصانا بالوفاء والقسط في ذلك كله كما جاء في العديد من الآيات منها قوله تعالى: \quranayah*[17][35]{ (\footnotesize \surahname*[17])}. وكل هذا حتى يتدرج الناس في إقامة العدل بسحب ما يستطيعون وبحسب ما علموا كما في قوله تعالى: \quranayah*[6][152][12-20]{ (\footnotesize \surahname*[6]: 152)}.

ولقد عرف الناس قديما الكيل لسهولته فيقاس بالصاع أي الحجم الثابت دون الحاجة إلى الميزان كما كان شائعا في عهد يوسف عليه السلام حيث قال لأخوته: \quranayah*[12][59][10]{ (\footnotesize \surahname*[12])}. ولم يأتي ذكر الميزان في قصة يوسف عليه السلام فدل ذلك على أن الميزان لم يكن شائعا في ذلك الزمان ولكن يوسف عليه السلام كان عالما بالحساب أيضا كما سيأتي بيانه فأوفى الكيل بالعد والحساب وهذا من تمام حفظه وعلمه عليه السلام. ومما يدل على أن يوسف عليه السلام أقام الكيل بالحجم لا بالوزن هو إستخدامه عليه السلام للسقاية وهي صواع الملك. وجاء في تفسير بن كثير رحمه الله تعالى أن السقاية هي إناء من فضة أو من ذهب وهي نفسها صواع الملك وهو صاعه الذي يكيل به \href{https://shamela.ws/book/8473/2199#p7}{\faExternalLink} \href{https://shamela.ws/book/8473/2200#p1}{\faExternalLink} \cite{tafsir_ibnKathir}.

ويكثر التعامل بالمكيال عند المزارعين لسهولته بينما يكثر التعامل بالميزان عند التجار لدقته. وقد أدرك ذلك النبي ﷺ وأقره في قوله: المِكْيالُ مِكْيالُ أَهْلِ المدينةِ، والميزانُ ميزانُ أَهْلِ مَكَّةَ.\footnote{صححة الألباني في هداية الرواة.} وهذا لأن أهل مكة عرفوا بالتجارة وأن أهل المدينة عرفوا بالزراعة. وأما في زماننا هذا فقد أصبح التعامل بالميزان والحساب أكثر من التعامل بالمكيال بسبب تطور العلوم والتكنولوجيا والتجارة. ولكن يبقى التعامل بالمكيال والميزان والحساب موجودا ومطلوبا في كل المعاملات. 

ومن المعلوم بالقياس أن الوزن يثبت والحجم يتغير وهذا لأن الأشياء في أصلها تثبت بالوزن ولكن قد تتغير في الحجم بحسب ما تتعرض له من حرارة أو ضغط أو غير ذلك من الظروف الأخرى. ولهذا العديد من قوانين الفيزياء في أصلها تبنى على ثبات الوزن كما هو معروف لأهل هذا التخصص. وهذا فيه فضل الميزان بالوزن على المكيال بالحجم فيكون الميزان أفضل وأكمل من المكيال. ومن كمال عدل الله أنه يحاسب المكلفين يوم القيامة بالميزان لا بالكيل كما في قوله تعالى: \quranayah*[21][47]{ (\footnotesize \surahname*[21])}. فجعل سبحانه العبرة في يوم الحساب بالوزن الذي يقدر بالميزان وليس بالحجم الذي يقدر بالمكيال. ولهذا فقد قال النبي ﷺ: إنَّه لَيَأْتي الرَّجُلُ العَظِيمُ السَّمِينُ يَومَ القِيامَةِ، لا يَزِنُ عِنْدَ اللَّهِ جَناحَ بَعُوضَةٍ، وقالَ اقْرَؤُوا: \quranayah*[18][105][9]{ (\footnotesize \surahname*[18]: 105)}  \href{https://shamela.ws/book/1284/3008#p3}{\faExternalLink} \href{https://shamela.ws/book/1727/6978#p3}{\faExternalLink} \cite{bukhari}.\footnote{صحيح البخاري: 4710، صحيح مسلم: 2785.} وهذا فيه أيضا أن موازين القسط ليوم القيامة ليس بالحجم والشكل وإنما بالوزن لما يحبه الله ويرضاه من الأعمال والتي توافق ما شرعه الله لعباده وهذا هو الميزان الشرعي كما سيأتي بيانه. 


\comment{

ومما يؤكد ما سبق أيضا قوله ﷺ:
الذهب بالذهب وزنًا بوزن، مثلًا بمثل، سواءً بسواء، يدًا بيد.
يقول الشيخ ابن باز رحمه الله تعالى في بيان معنى هذا الحديث: ولا فرق بين كونه جديدًا أو قديمًا، أو كون هذا أطيب وهذا أطيب، ما دام جنس الذهب لابد أن يكونا متساويين في الوزن يدًا بيد، يقبض في الحال [هـ]. وعن أبي سعيد الخدري رضي الله عنه أن رسول الله ﷺ قال: لا تبيعوا الذهب بالذهب إلا مثلا بمثل ولا تشفوا -أي تفاضلوا- بعضها على بعض، ولا تبيعوا منها غائبا بناجز
{\footnotesize (متفق عليه)}.

والحساب يبنى على التقدير العددي والتقدير الفكري. أما التقدير العددي فهو يضبط بالوزن والتقدير فكري يضبط بالحق. ولهذا يكون تقدير المخلوق محدود وناقص بما توفر لديه من علم وإدراك. وأما تقدير الخالق فهو تقدير كامل لا نقص فيه لأن الله هو العليم بكل شئ والقادر على كل شئ. ومن تقدير الله التقدير الكوني والتقدير الشرعي.ولما كان تقدير الله هو الحق وهو الميزان الذي لا نقص فيه, وافق تقديره الكوني سبحانه الميزان الكومي وتقديره الشرعي الميزان الشرعي.
}



\comment{
ويقول الشيخ الفوزان حفظه الله:
أما ما كان من علم الحساب الذي ينتفع به في معرفة المواقيت ومعرفة القبلة فهذا مباح وهو ما يسمى علم التسيير [.] وهو معرفة الحساب الذي به ينتفع الناس في مواقيت عباداتهم ومعاملاتهم ومواقيت زروعهم وغرس أشجارهم ويستدلون به على القبلة فهذا مباح وقد يجب تعلمه إذا كان يعين على أداء العبادات في مواقيتها [هـ].
}

\section{أمثلة حسابية من القرآن والسنة}
\subsection{مكوث أهل الكهف}
يقول جل جلاله عن مدة مكوث أهل الكهف في سورة الكهف:
\quranayah*[18][25]{\footnotesize \surahname*[18]}. فقد يسأل السائل لماذا جاء النص مع "وازدادوا تسعا" وهذا فيه الحكمة البالغة منه سبحانه. فمن المعلوم أن الرسول ﷺ كان مخاطبا لأهل الكتاب وأن أهل الكتاب يستعملون السنوات الشمسية وأما المسلمين فهم يستعملون السنوات القمرية. فالمطلوب هنا حساب المدة بعدد السنوات الشمسية ليوافق ذلك حساب أهل الكتاب وتحويل ذلك إلى عدد السنوات القمرية الذي يستعمله المسلمين في تاريخهم. وبالحساب الصحيح يتين الأتي:

عدد الأيام في السنة القمرية = 354 يوم

عدد الأيام في السنة الشمسية = 365 يوم

وبهذا يكون الفارق في عدد الأيام بينهما = 11 يوم

وعليه يكون في المئة سنة شمسية 36500 يوم وفي المئة سنة قمرية 35400 يوم.
الفارق هو  1100 يوم. بتقسيم هذا الفارق على 354 نجد أن الفارق هو 3 سنوات قمرية. وبهذا يعلم أن لكل 100 سنة شمسية توجد 103 سنة قمرية تقريبا. وعليه يكون في كل 300 سنة شمسية هناك 309 سنة قمرية. ويكون الفارق هو فقط تسعة سنوات ولهذا جاء لفظ "وازدادوا تسعا" للبيان فهي 300 سنة شمسية بالنسبة لأهل الكتاب وزيادة عليها 9 سنوات لتوافق بذلك 309 سنة قمرية بالنسبة للمسلمين. وهذا ما يعرف في علم الحساب بوحدة قياس الأعداد. فبالحساب يمكن تحويل وحدة قياس من نوع إلى أخر. وفي هذا المثال كانت وحدة القياس هي الزمن وبه علم التحويل من القياس الشمسي إلى القياس القمري والخلاف بينهما لا يعني التعارض بل لكل وحدة قياس حسابها الخاص. وهذا فيه بيان حكمة الله وعلمه سبحانه وأن كلامه هو الحق لهذا جاء بعد هذه الأية قوله تعالى:
\quranayah*[18][26]{\footnotesize \surahname*[18]}.
ويقول ابن كثير في تفسير الآية التي ذكر فيها عدد السنوات:
هذا خبر من الله تعالى لرسوله ﷺ بمقدار ما لبث أصحاب الكهف في كهفهم، منذ أرقدهم الله إلى أن بعثهم وأعثر عليهم أهل ذلك الزمان، وأنه كان مقداره ثلاثمائة سنة وتسع سنين بالهلالية، وهي ثلاثمائة سنة بالشمسية، فإن تفاوت ما بين كل مائة سنة بالقمرية إلى الشمسية ثلاث سنين؛ فلهذا قال بعد الثلاثمائة: (وازدادوا تسعا) [هـ].

\subsection{مكوث الوحي من عيسى عليه السلام إلى محمد ﷺ}
وفي مثال أخر يشه المثال السابق في تفسير قوله تعالى:
\quranayah*[5][19]{\footnotesize \surahname*[5]}. يقول ابن كثير في تفسيره عن هذا: (على فترة من الرسل) أي: بعد مدة متطاولة ما بين إرساله وعيسى ابن مريم.
وهو أنه ستمائة سنة. ومنهم من يقول: ستمائة وعشرون سنة. ولا منافاة بينهما، فإن القائل الأول أراد ستمائة سنة شمسية، والآخر أراد قمرية، وبين كل مائة سنة شمسية وبين القمرية نحو من ثلاث سنين; ولهذا قال تعالى في قصة أصحاب الكهف:
\quranayah*[18][25]{\footnotesize \surahname*[18]}. أي: قمرية ، لتكميل الثلاثمائة الشمسية التي كانت معلومة لأهل الكتاب
[هـ]. ولهذا فقد فهم المفسرين رحمهم الله بالقران الفرق بين عدد السنوات الشمسية والقمرية كما تقدم. وفيه أيضا عناية السلف رحمهم الله بالحساب وحساب الزمن وتحويله من وحدة قياس إلى أخرى.

\subsection{عدد ساعات اليوم والليلة}

عَنْ جَابِرِ بْنِ عَبْدِ اللَّهِ عَنْ رَسُولِ اللَّهِ ﷺ قَالَ: يَوْمُ الْجُمُعَةِ اثْنَتَا عَشْرَةَ سَاعَةً، لَا يُوجَدُ فِيهَا عَبْدٌ مُسْلِمٌ يَسْأَلُ اللَّهَ شَيْئًا إِلَّا آتَاهُ إِيَّاهُ فَالْتَمِسُوهَا آخِرَ سَاعَةٍ بَعْدَ الْعَصْرِ.{\footnotesize (أبو داود(1048), والنسائي(1389) وصححه الألباني)}. وفيه أن النبي ﷺ علم أن عدد ساعات اليوم بالمتوسط هو 12 ساعة. فإن كان لعدد ساعات الليلة مثل ذلك كانت عدد ساعات اليوم والليلة معا 24 ساعة. وهذا ما أعتاد عليه الناس في زماننا من حساب عدد ساعات اليوم واليلة. فإن ساعات اليوم والليلة تطول وتقصر خلال العام وتتغير بتغير المكان ولكن بالإجمال فهي 24 ساعة. وهذا فيه دليل نبوته ﷺ ففي زمانه لم يكن هناك الساعات الدقيقة التي نعرفها اليوم.

وعدد ساعات اليوم والليل تضبط بالتاريخ الشمسي كما موضخ في %\ref{fig:Hours}.


وكما موضح أن الأماكن القريبة من القطبين يغيب فيها النهار أو الليل خلال 24 ساعة وبهذا يمضي اليوم كاملا ويتعذر ضبط أوقات الصلاة بالطريقة المعتادة. وقد رخص أهل العلم على جواز ضبط وقت الصلاة فيها كما اعتاد الناس في سائر أوقات السنة أو قياسا على غيرها من الأماكن. فعَنِ النَّوَّاسِ بْنِ سَمْعَانَ الْكِلاَبِيِّ، قَالَ ذَكَرَ رَسُولُ اللَّهِ ﷺ الدَّجَّالَ فَقَالَ "إِنْ يَخْرُجْ وَأَنَا فِيكُمْ فَأَنَا حَجِيجُهُ دُونَكُمْ وَإِنْ يَخْرُجْ وَلَسْتُ فِيكُمْ فَامْرُؤٌ حَجِيجُ نَفْسِهِ وَاللَّهُ خَلِيفَتِي عَلَى كُلِّ مُسْلِمٍ فَمَنْ أَدْرَكَهُ مِنْكُمْ فَلْيَقْرَأْ عَلَيْهِ فَوَاتِحَ سُورَةِ الْكَهْفِ فَإِنَّهَا جِوَارُكُمْ مِنْ فِتْنَتِهِ". قُلْنَا وَمَا لُبْثُهُ فِي الأَرْضِ قَالَ " أَرْبَعُونَ يَوْمًا يَوْمٌ كَسَنَةٍ وَيَوْمٌ كَشَهْرٍ وَيَوْمٌ كَجُمُعَةٍ وَسَائِرُ أَيَّامِهِ كَأَيَّامِكُمْ". فَقُلْنَا يَا رَسُولَ اللَّهِ هَذَا الْيَوْمُ الَّذِي كَسَنَةٍ أَتَكْفِينَا فِيهِ صَلاَةُ يَوْمٍ وَلَيْلَةٍ قَالَ "لاَ اقْدُرُوا لَهُ قَدْرَهُ ثُمَّ يَنْزِلُ عِيسَى ابْنُ مَرْيَمَ عِنْدَ الْمَنَارَةِ الْبَيْضَاءِ شَرْقِيَّ دِمَشْقَ فَيُدْرِكُهُ عِنْدَ بَابِ لُدٍّ فَيَقْتُلُهُ {\footnotesize (صححه الألباني)}.
وفي هذا يقول الشيخ ابن باز رحمه الله:
الواجب على سكان هذه المناطق التي يطول فيها النهار أو الليل أن يصلوا الصلوات الخمس بالتقدير إذا لم يكن لديهم زوال ولا غروب لمدة أربع وعشرين ساعة، كما صح ذلك عن النبي ﷺ في حديث النواس بن سمعان، المخرج في صحيح مسلم في يوم الدجال الذي كسنة، سأل الصحابة رسول الله ﷺ عن ذلك، فقال: اقدروا له قدره. وهكذا حكم اليوم الثاني من أيام الدجال، وهو اليوم الذي كشهر، وهكذا اليوم الذي كأسبوع. أما المكان الذي يقصر فيه الليل ويطول فيه النهار أو العكس في أربع وعشرين ساعة فحكمه واضح: يصلون فيه كسائر الأيام، ولو قصر الليل جدا أو النهار؛ لعموم الأدلة، والله ولي التوفيق [هـ]. وهذا فيه أهمية الحساب وتسجيل بيانات أوقات الصلاة حتى يمكن تقديرها تقديرا صحيحا بقدر المستطاع إن تعذر معرفة ذلك من بغياب الليل أو النهار خلال 24 ساعة كما في فتنة المسيح الدجال.

\begin{figure}
    \centering
    \includegraphics[width=\textwidth] %,natwidth=512,natheight=288
    {Hours_of_daylight_vs_latitude_vs_day_of_year_with_tropical_and_polar_circles.png}
    \caption{عدد ساعات النهار بحسب خطوط العرض}
    \label{fig:Hours}
\end{figure}

\subsection{نسبية الوقت في القرآن}

قال تعالى في كتابه:
\quranayah*[22][47]{\footnotesize \surahname*[22]}.
ويقول ابن كثير في تفسيره: هو تعالى لا يعجل، فإن مقدار ألف سنة عند خلقه كيوم واحد عنده بالنسبة إلى حكمه. وهذا فيه الدليل على أن حساب الوقت نسبي ويختلف كما أخبر الله سبحانه وتعالى في أكثر من موضع. ومن ذلك قوله تعالى:
\quranayah*[32][5]{\footnotesize \surahname*[32]}.
وأيضا قوله تعالى:
\quranayah*[70][4]{\footnotesize \surahname*[70]}. ففي هذه الأيات الدليل الواضح على أن الوقت لا يجري بنفس السرعة فبين الله تعالى ذلك بالنسبة لوقتنا في قوله "مما تعدون".

ويمكن توضيح ذلك المعنى عن طريق حساب فارق الوقت بين الجنة والأرض. فإن حملنا معنى "عند ربك" أي في الجنة كما في قوله:
\quranayah*[3][169]{\footnotesize \surahname*[3]}, فبهذا يكون لكل يوم واحد في الجنة ألف سنة مما نعد في الأرض كما يلي:

\begin{center}
    1 يوم في الجنة = 1000 سنة في الأرض
\end{center}

نبدأ أولا بتحويل اليوم الواحد إلى ثواني. فإن علمنا أن اليوم به 12 ساعة بالمتوسط (اليوم والليلة معا 24 ساعة) كما في المثال السابق, وأن الساعة بها 60 دقيقة وأن الدقيقة بها 60 ثانية, عليه يكون اليوم الواحد به 43200 ثانية. ثانيا نحسب عدد الأيام في كل 1000 سنة. فإن إستخدمنا عدد الأيام في السنة القمرية يكون عدد الأيام الكلي في كل 1000 سنة قمرية 354000 يوم. وإبقاء الميزان وإستبدال وحدات القياس يكون الناتج:

\begin{center}
    43200 ثانية في الجنة = 354000 يوما في الأرض
\end{center}

فإن قسمنا عدد الثواني في الجنة على عدد الأيام يكون:

\begin{center}
    1 ثانية في الجنة = 1944.8 يوما في الأرض
\end{center}

وعليه بمرور ثانية واحدة في الجنة تمر علينا 8 أيام و4 ساعات و39 دقيقة و22 ثانية في الأرض مما نعد ونحسب. وهذا لا يجب أن يفهم أن الوقت في الجنة بطئ جدا بالنسبة لمن هو في الجنة ولكن فقط بالنسبة للأرض. فالوقت هو نفسه كوحدة قياس ولكن قياس نفس القيمة في نطاق مختلف لا يلزم التساوي وأنما كل قياس يرجع لنطاقه الذي قيس فيه.

وهذه الظاهرة تم إكتشافها حديثا في بداية القرن العشرين على يدي العالم الفيزيائي ألبيرت آنشتاين وتعرف بظاهرة التمدد الزمني وهي جزء من النظرية النسبية. وهي ظاهرة مثبتة ويمكن حسابها بدقة بإستخدام مفاهيم معروفة وأهمها ثبات سرعة الضوء في الفراغ. وهي ظاهرة مهمة جدا وتستخدم لموائمة أنظمة الإتصال وأنظمة تحديد الموقع مع الأقمار الصناعية. بدون الأخذ بعين الإعتبار ظاهرة التمدد الزمني كل هذه الأنظمة تتعطل.

ومن المثبت أيضا في النظرية النسبية أن مرور الوقت نسبي وهو يبطأ مع زيادة السرعة أو الجاذبية. ومن المعلوم أن الجاذبية تزداد مع زيادة كتلة الكواكب. وبهذا يعلم أن الوقت على القمر يمر أسرع بقليل بالنسبة للأرض بينما الوقت على الأرض يمر أسرع بالنسبة للشمس. وهذا الفارق يزيد من زيادة الفرق في الحجم. وعليه فإن مرور الوقت في الجنة أبطأ بكثير من الأرض دل على عظم الجنة بالنسبة للأرض. وهذا يتوافق مع قوله تعالى:
\quranayah*[3][133]{\footnotesize \surahname*[3]}.




\subsection{ظاهرة الغلاف الجوي}

\quranayah*[36][36-40]{\footnotesize \surahname*[36]}.


\subsection{ظاهرة تعاقب الليل والنهار}

\quranayah*[3][27]{\footnotesize \surahname*[3]}.

\quranayah*[24][44]{\footnotesize \surahname*[24]}.


\subsection{ظاهرة توسع الكون}

من أعظم بلايا هذا الزمان هو تصوير ووضع العلم في إيطار منفصل عن الايمان بل والاسوأ في ايطار انكار وجود الخالق وهذا والله من اعظم الضلال والجهل.

بينما في الحقيقة العلم هو الطريق للتأمل في آيات الله الكونية والتي جميعها تنادي بوجود الخالق وقدرته وعظمته. فكل ما نراه من تناسق في هذا الكون من ليل ونهار وشمس وقمر ومطر وشجر وحجر ودواب كلها من آيات الله الكونية.

ومن اعظم آيات الله الكونية أن الله عز وجل لم يجعل السماء ثابتة بل جعلها تتوسع فلو كانت ثابتة لقال الكثير ان هذا الكون ليس له بداية وهي ثابتة ازلا وبهذا ينكرون وجود الخالق. ولكن الله جل جلاله جعل السماء تتمدد ليكون هذا التمدد دليلا على ان اطراف السماء كلها جاءت من نقطة واحدة وهذا هو الدليل القاطع على بداية الكون.

الإقرار بأن هذا الكون له بداية كما تشير كل الأدلة والمفاهيم التي توصل لها البشر في القرن العشرين ومنها ما جاء في نظرية الإنفجار العظيم يبطل كل ما تم طرحه من أصحاب النظريات الإلحادية إذا يتعذر على شي له بداية أن يبدأ من لا شئ.

قال تعالى:
أَوَلَمْ يَرَ الَّذِينَ كَفَرُوا أَنَّ السَّمَاوَاتِ وَالْأَرْضَ كَانَتَا رَتْقًا فَفَتَقْنَاهُمَا ۖ وَجَعَلْنَا مِنَ الْمَاءِ كُلَّ شَيْءٍ حَيٍّ ۖ أَفَلَا يُؤْمِنُونَ [الأنبياء: 30]

فقد اتفق المفسرون على ان معنى هذه الآية ان السماوات والأرض كانتا ملتصقتين وهذا ما يتوافق مع فهمنا اليوم بناء على المشاهدة أن الكون بدأ من نقطة واحدة.

فقد جاء في تفسير القرطبي ان ابن عباس والحسن وعطاء والضحاك وقتادة قالوا في تفسير هذه الآية: يعني أنها كانت شيئا واحدا ملتزقتين ففصل الله بينهما بالهواء. وكذلك قال كعب: خلق الله السماوات والأرض بعضها على بعض ثم خلق ريحا بوسطها ففتحها بها ، وجعل السماوات سبعا والأرضين سبعا. وقول ثان قاله مجاهد والسدي وأبو صالح: كانت السماوات مؤتلفة طبقة واحدة ففتقها فجعلها سبع سماوات، وكذلك الأرضين كانت مرتتقة طبقة واحدة ففتقها فجعلها سبعا

وقال الطبري في تفسير هذه الآيات: أو لم ينظر هؤلاء الذي كفروا بالله بأبصار قلوبهم، فيروا بها، ويعلموا أن السماوات والأرض كانتا رَتْقا: يقول: ليس فيهما ثقب، بل كانتا ملتصقتين،

وقد جاء في تفسير ابن كثير:

ألم يروا ( أن السماوات والأرض كانتا رتقا ) أي: كان الجميع متصلا بعضه ببعض متلاصق متراكم، بعضه فوق بعض في ابتداء الأمر ، ففتق هذه من هذه. فجعل السماوات سبعا، والأرض سبعا، وفصل بين سماء الدنيا والأرض بالهواء، فأمطرت السماء وأنبتت الأرض; ولهذا قال: ( وجعلنا من الماء كل شيء حي أفلا يؤمنون ) أي: وهم يشاهدون المخلوقات تحدث شيئا فشيئا عيانا ، وذلك دليل على وجود الصانع الفاعل المختار القادر على ما يشاء

ويقول جل جلاله:
وَالسَّمَاءَ بَنَيْنَاهَا بِأَيْدٍ وَإِنَّا لَمُوسِعُونَ [الذاريات: 47]

وقد جاء في تفسير السعدي رحمه الله:
{ بِأَيْدٍ } أي: بقوة وقدرة عظيمة { وَإِنَّا لَمُوسِعُونَ } لأرجائها وأنحائها،


\subsection{ظاهرة المجال المغنطيسي}

\quranayah*[21][32]{\footnotesize \surahname*[21]}.


