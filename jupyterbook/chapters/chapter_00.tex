\chapter*{المقدمة}
\addcontentsline{toc}{chapter}{\protect\numberline{}المقدمة}

\begin{center}
    بسم الله الرحمن الرحيم
\end{center}

الحمد الله العزيز العليم فالق الحب والنوى خلق كل شئ بقدر معلوم فقدره تقديرا. فالق الإصباح وجعل الليل سكنا والشمس والقمر حسبانا فقدر للقمر منازلا وجعل الشمس تجري لمستقر لها وكل في فلك يسبحون. لا إله إلا هو وحده لا شريك له  إيمانا بروبويته وتسليما وإقرارا بألوهيته وتصديقا بأسمائه وصفاته على الوجه الذي يحب ويرضى من غير تحريف ولا تمثيل ولا تكييف ولا تعطيل. بعث الرسل بالحق والميزان مبشرين ومنذرين وليبينوا للناس أمور دنيهم ودنياهم ومن أجلها إقامة التوحيد بإفراده سبحانه وحده بالعبادة بالطريقة التي ارتضاها وإقامة الميزان بالقسط والعدل بين الناس. ومن رحمته أنه أرسل محمدا ﷺ خاتما للنبيئين وأنزل عليه القرآن هدى وبشرى للمتقين، أما بعد:

هذا كتاب ألفته لشرح علم الحساب وأهميته في دين الإسلام وإقامة أركان الحكم الرشيد\comment{( الثلاثة: 1. إقامة الحق، 2. إقامة الميزان، 3. الأخذ بالأسباب)}. فالعلم بالحساب يدخل في الكثير من العلوم: منها العلوم الشرعية كالمواريث، والمواقيت، والمعاملات، ومنها العلوم الكونية السببية النافعة مثل الهندسة، والفلك، والطب، والفيزياء، الكيمياء، وعلوم الحاسب الآلي كعلوم البيانات وعلوم الآلة والذكاء الإصطناعي وغيرها من العلوم النافعة التي لا يمكن فهمها ولا إتقانها إلا بالحساب الصحيح. وسعيا إلى بيان الحق الموافق لأمر الله الكوني والشرعي معا، وجدت أن هذا البحث لا يكتمل إلا بالجمع بين العلم الكوني الحديث والمؤكد مع العلم الشرعي الصحيح والثابت. ورغم قلة العلم وضعف العمل ولا حول ولا قوة إلا بالله، لم يكن لي إلا أن أستعين بالله العزيز العليم وأتوكل عليه، راجيا توفيقه وطالبا لفضله ورحمته، مستفيدا مما حصلته من العلوم الكونية في مجال الحساب والهندسة والفيزياء وعلوم البيانات والذكاء الإصطناعي، ومجتهدا في الأمور الشرعية المهمة المتعلقة بهذا البحث في العقيدة والتفسيير لبيان أهمية الحساب في الإسلام وفي إقامة الحكم الرشيد متبعا في ذلك ما جاء في كتاب الله عز وجل وسنة نبه محمد ﷺ وسبيل المؤمنين من سلف هذه الأمة وعلماءها.

سميت هذا الكتاب: "الحساب الصحيح من بنيان الحكم الرشيد". فالحساب هو وسيلة يحتاجها الناس لغايات عديدة منها ما هو فرض، ومنها ما هو نافع، ومنها ما هو بخلاف ذلك. ولا يكاد يوجد علم إلا والحساب داخل فيه. ولهذا فقد حثنا الله جل جلاله على الحساب في أكثر من موضع في كتابه العظيم. فالحساب موافق للفطرة والعقل وهو رياضة العقل وهو مفتاح العلوم الكونية النافعة وهو السبيل لفهمها وبه تكشف العديد من حقائق وأسرار هذا الكون. وهو الرياضيات وهو الأساس الذي تبنى عليه الهندسة بكافة أنواعها وفروعها. وهو العلم الذي به تبنى قوانين الفيزياء والكيمياء وغير ذلك من العلوم السببية. وهو أساس علوم البيانات والآلة والذكاء الإصطناعي. وفيه قوه الإستدلال والحجة في رد شبه الإلحاد. وفيه معرفة الله عز وجل بأسمائه وصفاته وأفعاله. فإن كانت الغاية من الحساب هي منفعة الناس بالعموم في أمور دينهم ودنياهم وإقامة الميزان الشرعي الذي أمر الله به، كان هذا الحساب صحيحا، وكان من بنيان الحكم الرشيد، وكان سببا في الزيادة في الإيمان وإقامة العدل بين الناس بالقسط، وسبيلا للتطور العلمي والحضاري، وقوة في دحر الأعداء ونشر الحق ونصرته. 

اسئل الله العلي العظيم أن يجعل هذا العمل خالصا لوجهه الكريم وأن يبارك فيه ويكتب له القبول في الأرض ويجعله سببا لعودة أمة الإسلام إلى الطريق المستقيم، الطريق الذي فيه يقام الحق والميزان معا. وأن يجعل دعوتنا دعوة الراسخين في العلم كما في قوله تعالى: \quranayah*[3][7][32]\quranayah*[3][8-9]{\footnotesize \surahname*[3]}. اللهم اجعل دعائنا كدعاء نبينا ﷺ كما جاء عن عائشة أم المؤمنين أن النبي ﷺ كان إذَا قَامَ مِنَ اللَّيْلِ افْتَتَحَ صَلَاتَهُ: اللَّهُمَّ رَبَّ جِبْرَائِيلَ، وَمِيكَائِيلَ، وإسْرَافِيلَ، فَاطِرَ السَّمَوَاتِ وَالأرْضِ، عَالِمَ الغَيْبِ وَالشَّهَادَةِ، أَنْتَ تَحْكُمُ بيْنَ عِبَادِكَ فِيما كَانُوا فيه يَخْتَلِفُونَ، اهْدِنِي لِما اخْتُلِفَ فيه مِنَ الحَقِّ بإذْنِكَ؛ إنَّكَ تَهْدِي مَن تَشَاءُ إلى صِرَاطٍ مُسْتَقِيمٍ {\footnotesize (صحيح مسلم)}. وكما جاء عن عبدالله بن عمر أن نبينا ﷺ قلَّما يقوم من مجلس حتى يدعو بهؤلاء الدعوات لأصحابه: اللهمَّ اقسِمْ لنا مِنْ خشيَتِكَ ما تحولُ بِهِ بينَنَا وبينَ معاصيكَ، ومِنْ طاعَتِكَ ما تُبَلِّغُنَا بِهِ جنتَكَ، ومِنَ اليقينِ ما تُهَوِّنُ بِهِ علَيْنَا مصائِبَ الدُّنيا، اللهمَّ متِّعْنَا بأسماعِنا، وأبصارِنا، وقوَّتِنا ما أحْيَيْتَنا، واجعلْهُ الوارِثَ مِنَّا، واجعَلْ ثَأْرَنا عَلَى مَنْ ظلَمَنا، وانصرْنا عَلَى مَنْ عادَانا، ولا تَجْعَلِ مُصِيبَتَنا في دينِنِا، ولَا تَجْعَلْ الدنيا أكبرَ هَمِّنَا، ولَا مَبْلَغَ عِلْمِنا، ولَا تُسَلِّطْ عَلَيْنا مَنْ لَا يرْحَمُنا {\footnotesize (صحيح الترمذي)}. وكما جاء عن زيد بن أرقم رضي الله عنه أنه قال: لا أُعلِّمُكم إلا ما كان رسولُ اللهِ ﷺ يُعلِّمُنا: اللَّهمَّ إني أعوذُ بك من العجزِ والكسلِ، والبخلِ والجبنِ، والهَرَمِ وعذابِ القبرِ، اللَّهمَّ آتِ نفسي تقْوَاها وزكِّها أنت خيرُ من زكَّاها أنت ولِيُّها ومولاها، اللَّهمَّ إني أعوذُ بك من قلبٍ لا يخشعُ ومن نفسٍ لا تشبعُ وعلمٍ لا ينفعُ ودعوةٌ لا يُستجابُ لها {\footnotesize (صحيح النسائي)}.

\vskip 0.5in

\begin{center}
\begin{tabular}{cc}
    اللهم اعنا على إقامة الحق واجعلنا من المهتدين & 
    واعنا على إقامة الميزان واجعلنا من المقسطين  \\
    واهدنا إلى الرشاد واجعلنا من المصلحين & 
    وزدنا علما نافعا واجعلنا من الخاشعين \\
    اللهم اهدنا إلى الإسلام واجعلنا من الذاكرين & 
    واهدنا إلى الإيمان واجعلنا من المخلصين \\
    واهدنا إلى الإحسان واجعلنا من الموقنين & 
    وتوفنا وأنت راض عنا واجعلنا من المتقين \\
    \multicolumn{2}{c}{اللهم إنَّا نسألك الصلاح والصدق والصبر والهدى والسداد  واليقين والتقى والعفاف والغنى} \\
    \multicolumn{2}{c}{واجعلنا اللهم من الشاكرين والفائزين} \\
    \multicolumn{2}{c}{ربنا أوزعنا أن نشكر نعمتك التي أنعمت علينا وعلى والدينا وأن نعمل صالحا ترضاه} \\
    \multicolumn{2}{c}{وأصلح لنا في ذريتنا أنا تبنا إليك وإنا من المسلمين وأدخلنا برحمتك في عبادك الصالحين} \\
    \multicolumn{2}{c}{اللهم اغفر لنا ولوالدينا ولإخواننا الذين سبقونا بالإيمان} \\
    \multicolumn{2}{c}{ولجميع المسلمين والمسلمات والمؤمنين والمؤمنات الأحياء منهم والأموات} \\
    \multicolumn{2}{c}{وأدخلنا جميعا في رحمتك وأنت أرحم الرحمين} \\
    \multicolumn{2}{c}{وصلى الله على نبينا محمد وعلى آله وصحبه وسلم} \\
\end{tabular}
\end{center}

\newpage
\vspace*{\fill}
\textbf{ملاحظة:}

هذا البحث ما هو إلا إجتهاد شخصي وقد يحتوي على أخطاء ونقص، فما وافق الحق فمن الله جل جلاله وما خالفه فمن نفسي واستغفر الله وأتوب إليه. هذا البحث إنما هو لرفع الجهل عن نفسي ولولا أن الله فرض علينا نشر الحق وبيانه وأن النبي ﷺ أمرنا بأداء النصيحة لأَئِمَّةِ المُسْلِمِينَ وعامَّتِهِمْ، لما نشرته وخصوصا أن هذا البحث لم يراجعه أهل العلم كما أني لست من أهل العلم ولا أدعي ذلك وإنما أنا أجهل أكثر مما أعلم ولكني أبحث عن الحق حيث كان بالحجة والبرهان. يمكن للقارئ الكريم المساهمة بالنقض البنَّاء في تصحيح وتحسين هذا الكتاب بإرسال ملاحظاته ومقترحاته وتعليقاته على البريد الإلكتروني. هذا بحث حر لا يتقيد إلا بالحق وما قامت عليه الحجة بالجمع بين الأدلة الصحيحة والثابتة دون الأخذ ببعضها ورد بعضها الأخر ودون التعصب لأقوال العلماء أو أصحاب المذاهب مهما كانت منزلتهم في الإسلام وخصوصا متى كان ذلك مخالفا لقول الله جل جلاله أو لقول المعصوم النبي محمد بن عبد الله القرشي عليه أفضل الصلاة والتسليم. هذا البحث فيه الجمع بين الحجة العقلية والحجة الشرعية وبين العلم الكوني المؤكد والعلم الشرعي الثابت. المنهجية المطروحة في هذا البحث فيها يتوافق العقل مع النقل لا إفراط بجعل العقل مقدما على الشرع ومقررا له، ولا تفريط بجعل الشرع مستغنيا عن العقل ومضادا له. وإنما الوسط بين ذلك وهو أن العقل في الحقيقة موافق للشرع ومقرا ومسلما له ومصدقا به، وبه تقام الحجة ويزداد الإيمان، وهو من طرق الإستدلال التي أرشد الله إليها في كتابه الكريم مع السمع والبصر وهو وسيلة وليس غاية منَّ الله بها على عباده المكلفين لمعرفة الحق والهدى من مصادر الإستدلال الكونية والشرعية. بل إن الله تبارك وتعالى رفع مكانة أصحاب العقول الذكية ووصفهم بأولي الألباب الذين يستدلون بآيات الله الكونية والشرعية على الحق وأنكر سبحانه على كل من خالف ذلك كما سيأتي بيان ذلك في هذا الكتاب، والحمد الله رب العالمين.
