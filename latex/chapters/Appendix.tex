\chapter{الملحق}
% \addcontentsline{toc}{chapter}{\protect\numberline{}الملحق}
\label{sec:appendix}


\section{مسألة محاجة إبراهيم لقومه}
\label{sec:app_ibrahim_argumentation}

قال تعالى: \quranayah*[6][74-79]{\footnotesize \surahname*[6]}. قال ابن كثير رحمه الله: وقد اختلف المفسرون في هذا المقام، هل هو مقام نظر أو مناظرة؟ فروى ابن جرير من طريق علي بن أبي طلحة، عن ابن عباس ما يقتضي أنه مقام نظر، واختاره ابن جرير مستدلا بقوله: (لئن لم يهدني ربي لأكونن من القوم الضالين) [.] والحق أن إبراهيم عليه الصلاة والسلام، كان في هذا المقام مناظرا لقومه \cite{tafsir_ibnKathir}. وهذا ما رجحه السعدي في تفسيره \cite{tafsir_Saadi}. 

ولعل ما رحجه ابن جرير عن ابن عباس هو الأصح أي أنه مقام نظر فهذا إنما كان في بداية هداية الله له عند تفكره في خلق الله تبارك وتعالى كما دل السياق عليه لأن الله جل جلاله بدأ هذه الآيات بقوله تعالى:  \quranayah*[6][75]{\footnotesize \surahname*[6]}. ولا يوجد ما يمنع أن يكون ذلك مقام نظر ومناظرة في نفس الوقت، والله فوق كل ذي علم عليم. وأما القول بأن هذا مقام مناظرة فقط دون نظر فهو يخالف سياق الآيات. فالإستدلال بقوله تعالى:  \quranayah*[6][74]{\footnotesize \surahname*[6]} على أن إبراهيم قد هدي قبل هذا المقام لا يثبت. فالعديد من القصص التي قصها الله تبارك وتعالى في كتابه الكريم تبدأ بذكر القصة على وجه الإجمال أو التقديم ثم يأتي التفصيل في ذلك تبعا دون التقيد بالترتيب الزمني كما في قصة أصحاب الكهف وقصص موسى عليه السلام.

\section{مسألة كروية الأفلاك لشيخ الإسلام ابن تيمية}
\label{sec:app_round_planets_teimia}

يقول شيخ الإسلام: هَذَا وَقَدْ ثَبَتَ بِالْكِتَابِ وَالسُّنَّةِ وَإِجْمَاعِ عُلَمَاءِ الْأُمَّةِ أَنَّ الْأَفْلَاكَ مُسْتَدِيرَةٌ قَالَ اللَّهُ تَعَالَى: (وَمِنْ آيَاتِهِ اللَّيْلُ وَالنَّهَارُ وَالشَّمْسُ وَالْقَمَرُ) وَقَالَ: (وَهُوَ الَّذِي خَلَقَ اللَّيْلَ وَالنَّهَارَ وَالشَّمْسَ وَالْقَمَرَ كُلٌّ فِي فَلَكٍ يَسْبَحُونَ) وَقَالَ تَعَالَى: (لَا الشَّمْسُ يَنْبَغِي لَهَا أَنْ تُدْرِكَ الْقَمَرَ وَلَا اللَّيْلُ سَابِقُ النَّهَارِ وَكُلٌّ فِي فَلَكٍ يَسْبَحُونَ) قَالَ ابْنِ عَبَّاسٍ: فِي فَلْكَةٍ مِثْلِ فَلْكَةِ الْمِغْزَلِ وَهَكَذَا هُوَ فِي لِسَانِ الْعَرَبِ الْفَلَكُ الشَّيْءُ الْمُسْتَدِيرُ. وَمِنْهُ يُقَالُ: تَفَلَّكَ ثَدْيُ الْجَارِيَةِ إذَا اسْتَدَارَ. قَالَ تَعَالَى: (يُكَوِّرُ اللَّيْلَ عَلَى النَّهَارِ وَيُكَوِّرُ النَّهَارَ عَلَى اللَّيْلِ) وَالتَّكْوِيرُ هُوَ التَّدْوِيرُ. وَمِنْهُ قِيلَ: كَارَ الْعِمَامَةَ وَكَوَّرَهَا إذَا أَدَارَهَا. وَمِنْهُ قِيلَ: لِلْكُرَةِ كُرَةٌ وَهِيَ الْجِسْمُ الْمُسْتَدِيرُ وَلِهَذَا يُقَالُ: لِلْأَفْلَاكِ كُرَوِيَّةُ الشَّكْلِ؛ لِأَنَّ أَصْلَ الْكُرَةِ كورة تَحَرَّكَتْ الْوَاوُ وَانْفَتَحَ مَا قَبْلَهَا فَقُلِبَتْ أَلِفًا وَكَوَّرْت الْكَارَةَ إذَا دَوَّرْتهَا وَمِنْهُ الْحَدِيثُ: (إنَّ الشَّمْسَ وَالْقَمَرَ يُكَوَّرَانِ يَوْمَ الْقِيَامَةِ كَأَنَّهُمَا ثَوْرَانِ فِي نَارِ جَهَنَّمَ) وَقَالَ تَعَالَى: (الشَّمْسُ وَالْقَمَرُ بِحُسْبَانٍ) مِثْلِ حُسْبَانِ الرَّحَا وَقَالَ: (مَا تَرَى فِي خَلْقِ الرَّحْمَنِ مِنْ تَفَاوُتٍ) وَهَذَا إنَّمَا يَكُونُ فِيمَا يَسْتَدِيرُ مِنْ أَشْكَالِ الْأَجْسَامِ دُونَ الْمُضَلَّعَاتِ مِنْ الْمُثَلَّثِ أَوْ الْمُرَبَّعِ أَوْ غَيْرِهِمَا فَإِنَّهُ يَتَفَاوَتُ لِأَنَّ زَوَايَاهُ مُخَالِفَةٌ لِقَوَائِمِهِ وَالْجِسْمِ الْمُسْتَدِيرِ مُتَشَابِهُ الْجَوَانِبِ وَالنَّوَاحِي لَيْسَ بَعْضُهُ مُخَالِفًا لِبَعْضِ. (وقال النَّبِيُّ صَلَّى اللَّهُ عَلَيْهِ وَسَلَّمَ لِلْأَعْرَابِيِّ الَّذِي قَالَ: إنَّا نَسْتَشْفِعُ بِك عَلَى اللَّهِ وَنَسْتَشْفِعُ بِاَللَّهِ عَلَيْك. فَقَالَ: وَيْحَك إنَّ اللَّهَ لَا يُسْتَشْفَعُ بِهِ عَلَى أَحَدٍ مِنْ خَلْقِهِ. إنَّ شَأْنَهُ أَعْظَمُ مِنْ ذَلِكَ إنَّ عَرْشَهُ عَلَى سَمَوَاتِهِ هَكَذَا وَقَالَ بِيَدِهِ مِثْلَ الْقُبَّةِ: وَإِنَّهُ لَيَئِطُّ بِهِ أَطِيطَ الرَّحْلِ الْجَدِيدِ بِرَاكِبِهِ) رَوَاهُ. أَبُو دَاوُد وَغَيْرُهُ مِنْ حَدِيثِ جُبَيْرِ بْنِ مُطْعِمٍ عَنْ النَّبِيِّ صَلَّى اللَّهُ عَلَيْهِ وَسَلَّمَ. وَفِي الصَّحِيحَيْنِ عَنْ أَبِي هُرَيْرَةَ عَنْ النَّبِيِّ صَلَّى اللَّهُ عَلَيْهِ وَسَلَّمَ أَنَّهُ قَالَ: (إذَا سَأَلْتُمْ اللَّهَ الْجَنَّةَ فَاسْأَلُوهُ الْفِرْدَوْسَ فَإِنَّهَا أَعْلَى الْجَنَّةِ وَأَوْسَطُ الْجَنَّةِ وَسَقْفُهَا عَرْشُ الرَّحْمَنِ) فَقَدْ أَخْبَرَ أَنَّ الْفِرْدَوْسَ هِيَ الْأَعْلَى وَالْأَوْسَطُ وَهَذَا لَا يَكُونُ إلَّا فِي الصُّورَةِ الْمُسْتَدِيرَةِ فَأَمَّا الْمُرَبَّعُ وَنَحْوُهُ فَلَيْسَ أَوْسَطُهُ أَعْلَاهُ بَلْ هُوَ مُتَسَاوٍ. وَأَمَّا إجْمَاعُ الْعُلَمَاءِ: فَقَالَ إيَاسُ بْنُ مُعَاوِيَةَ - الْإِمَامُ الْمَشْهُورُ قَاضِي الْبَصْرَةِ مِنْ التَّابِعِينَ -: السَّمَاءُ عَلَى الْأَرْضِ مِثْلُ الْقُبَّةِ. وَقَالَ الْإِمَامُ أَبُو الْحُسَيْنِ أَحْمَد بْنُ جَعْفَرِ بْنِ الْمُنَادِي مِنْ أَعْيَانِ الْعُلَمَاءِ الْمَشْهُورِينَ بِمَعْرِفَةِ الْآثَارِ وَالتَّصَانِيفِ الْكِبَارِ فِي فُنُونِ الْعُلُومِ الدِّينِيَّةِ مِنْ الطَّبَقَةِ الثَّانِيَةِ مِنْ أَصْحَابِ أَحْمَد: لَا خِلَافَ بَيْنِ الْعُلَمَاءِ أَنَّ السَّمَاءَ عَلَى مِثَالِ الْكَرَّةِ وَأَنَّهَا تَدُورُ بِجَمِيعِ مَا فِيهَا مِنْ الْكَوَاكِبِ كَدَوْرَةِ الْكُرَةِ عَلَى قُطْبَيْنِ ثَابِتَيْنِ غَيْرِ مُتَحَرِّكَيْنِ: أَحَدُهُمَا فِي نَاحِيَةِ الشَّمَالِ وَالْآخَرُ فِي نَاحِيَةِ الْجَنُوبِ. قَالَ: وَيَدُلُّ عَلَى ذَلِكَ أَنَّ الْكَوَاكِبَ جَمِيعَهَا تَدُورُ مِنْ الْمَشْرِقِ تَقَعُ قَلِيلًا عَلَى تَرْتِيبٍ وَاحِدٍ فِي حَرَكَاتِهَا وَمَقَادِيرِ أَجْزَائِهَا إلَى أَنْ تَتَوَسَّطَ السَّمَاءَ ثُمَّ تَنْحَدِرُ عَلَى ذَلِكَ التَّرْتِيبِ. كَأَنَّهَا ثَابِتَةٌ فِي كُرَةٍ تُدِيرُهَا جَمِيعَهَا دَوْرًا وَاحِدًا. قَالَ: وَكَذَلِكَ أَجْمَعُوا عَلَى أَنَّ الْأَرْضَ بِجَمِيعِ حَرَكَاتِهَا مِنْ الْبَرِّ وَالْبَحْرِ مِثْلُ الْكُرَةِ. قَالَ: وَيَدُلُّ عَلَيْهِ أَنَّ الشَّمْسَ وَالْقَمَرَ وَالْكَوَاكِبَ لَا يُوجَدُ طُلُوعُهَا وَغُرُوبُهَا عَلَى جَمِيعِ مَنْ فِي نَوَاحِي الْأَرْضِ فِي وَقْتٍ وَاحِدٍ بَلْ عَلَى الْمَشْرِقِ قَبْلَ الْمَغْرِبِ. قَالَ: فَكُرَةُ الْأَرْضِ مُثَبَّتَةٌ فِي وَسَطِ كَرَّةِ السَّمَاءِ كَالنُّقْطَةِ فِي الدَّائِرَةِ. يَدُلُّ عَلَى ذَلِكَ أَنَّ جُرْمَ كُلِّ كَوْكَبٍ يُرَى فِي جَمِيعِ نَوَاحِي السَّمَاءَ عَلَى قَدْرٍ وَاحِدٍ فَيَدُلُّ ذَلِكَ عَلَى بُعْدِ مَا بَيْنَ السَّمَاءِ وَالْأَرْضِ مِنْ جَمِيعِ الْجِهَاتِ بِقَدْرِ وَاحِدٍ فَاضْطِرَارُ أَنْ تَكُونَ الْأَرْضَ وَسَطَ السَّمَاءِ. وَقَدْ يَظُنُّ بَعْضُ النَّاسِ أَنَّ مَا جَاءَتْ بِهِ الْآثَارُ النَّبَوِيَّةُ مِنْ أَنَّ الْعَرْشَ سَقْفُ الْجَنَّةِ وَأَنَّ اللَّهَ عَلَى عَرْشِهِ مَعَ مَا دَلَّتْ عَلَيْهِ مِنْ أَنَّ الْأَفْلَاكَ مُسْتَدِيرَةٌ مُتَنَاقِضٌ أَوْ مُقْتَضٍ أَنْ يَكُونَ اللَّهُ تَحْتَ بَعْضِ خَلْقِهِ " كَمَا احْتَجَّ بَعْضُ الْجَهْمِيَّة عَلَى إنْكَارِ أَنْ يَكُونَ اللَّهُ فَوْقَ الْعَرْشِ بِاسْتِدَارَةِ الْأَفْلَاكِ وَأَنَّ ذَلِكَ مُسْتَلْزَمٌ كَوْنَ الرَّبِّ أَسْفَلَ. وَهَذَا مِنْ غَلَطِهِمْ فِي تَصَوُّرِ الْأَمْرِ وَمَنْ عَلِمَ أَنَّ الْأَفْلَاكَ مُسْتَدِيرَةٌ وَأَنَّ الْمُحِيطَ الَّذِي هُوَ السَّقْفُ هُوَ أَعْلَى عِلِّيِّينَ وَأَنَّ الْمَرْكَزَ الَّذِي هُوَ بَاطِنُ ذَلِكَ وَجَوْفُهُ وَهُوَ قَعْرُ الْأَرْضِ هُوَ " سِجِّينٌ " " وَأَسْفَلُ سَافِلِينَ " عَلِمَ مِنْ مُقَابَلَةِ اللَّهِ بَيْنَ أَعْلَى عِلِّيِّينَ وَبَيْنَ سِجِّينٍ مَعَ أَنَّ الْمُقَابَلَةَ: إنَّمَا تَكُونُ فِي الظَّاهِرِ بَيْنَ الْعُلُوِّ وَالسُّفْلِ أَوْ بَيْنَ السِّعَةِ وَالضِّيقِ وَذَلِكَ لِأَنَّ الْعُلُوَّ مُسْتَلْزَمٌ لِلسِّعَةِ وَالضِّيقِ مُسْتَلْزِمٌ لِلسُّفُولِ وَعَلِمَ أَنَّ السَّمَاءَ فَوْقَ الْأَرْضِ مُطْلَقًا لَا يُتَصَوَّرُ أَنْ تَكُونَ تَحْتَهَا قَطُّ وَإِنْ كَانَتْ مُسْتَدِيرَةً مُحِيطَةً وَكَذَلِكَ كُلَّمَا عَلَا كَانَ أَرْفَعَ وَأَشْمَلَ. وَعَلِمَ أَنَّ الْجِهَةَ قِسْمَانِ: قِسْمٌ ذَاتِيٌّ. وَهُوَ الْعُلُوُّ وَالسُّفُولُ فَقَطْ. وَقِسْمٌ إضَافِيٌّ: وَهُوَ مَا يُنْسَبُ إلَى الْحَيَوَانِ بِحَسَبِ حَرَكَتِهِ: فَمَا أَمَامَهُ يُقَالُ لَهُ: أَمَامٌ وَمَا خَلْفَهُ يُقَالُ لَهُ خَلْفٌ وَمَا عَنْ يَمِينِهِ يُقَالُ لَهُ الْيَمِينُ وَمَا عَنْ يَسْرَتِهِ يُقَالُ لَهُ الْيَسَارُ وَمَا فَوْقَ رَأْسِهِ يُقَالُ لَهُ فَوْقٌ وَمَا تَحْتَ قَدَمَيْهِ يُقَالُ لَهُ تَحْتٌ وَذَلِكَ أَمْرٌ إضَافِيٌّ. أَرَأَيْت لَوْ أَنَّ رَجُلًا عَلَّقَ رِجْلَيْهِ إلَى السَّمَاءِ وَرَأْسُهُ إلَى الْأَرْضِ أَلَيْسَتْ السَّمَاءُ فَوْقَهُ وَإِنْ قَابَلَهَا بِرِجْلَيْهِ وَكَذَلِكَ النَّمْلَةُ أَوْ غَيْرُهَا لَوْ مَشَى تَحْتَ السَّقْفِ مُقَابِلًا لَهُ بِرِجْلَيْهِ وَظَهْرُهُ إلَى الْأَرْضِ لَكَانَ الْعُلُوُّ مُحَاذِيًا لَرِجْلَيْهِ وَإِنْ كَانَ فَوْقَهُ وَأَسْفَلُ سَافِلِينَ يَنْتَهِي إلَى جَوْفِ الْأَرْضِ. وَالْكَوَاكِبُ الَّتِي فِي السَّمَاءِ وَإِنْ كَانَ بَعْضُهَا مُحَاذِيًا لِرُءُوسِنَا وَبَعْضُهَا فِي النِّصْفِ الْآخَرِ مِنْ الْفَلَكِ فَلَيْسَ شَيْءٌ مِنْهَا تَحْتَ شَيْءٍ بَلْ كُلُّهَا فَوْقَنَا فِي السَّمَاءِ وَلَمَّا كَانَ الْإِنْسَانُ إذَا تَصَوَّرَ هَذَا يَسْبِقُ إلَى وَهْمِهِ السُّفْلُ الْإِضَافِيُّ كَمَا احْتَجَّ بِهِ الجهمي الَّذِي أَنْكَرَ عُلُوَّ اللَّهِ عَلَى عَرْشِهِ وَخَيَّلَ عَلَى مَنْ لَا يَدْرِي أَنَّ مَنْ قَالَ: إنَّ اللَّهَ فَوْقَ الْعَرْشِ فَقَدْ جَعَلَهُ تَحْتَ نِصْفِ الْمَخْلُوقَاتِ أَوْ جَعَلَهُ فَلَكًا آخَرَ تَعَالَى اللَّهُ عَمَّا يَقُولُ الْجَاهِلُ. فَمَنْ ظَنَّ أَنَّهُ لَازِمٌ لِأَهْلِ الْإِسْلَامِ مِنْ الْأُمُورِ الَّتِي لَا تَلِيقُ بِاَللَّهِ وَلَا هِيَ لَازِمَةٌ بَلْ هَذَا يُصَدِّقُهُ الْحَدِيثُ الَّذِي رَوَاهُ أَحْمَد فِي مُسْنَدِهِ مِنْ حَدِيثِ الْحَسَنِ عَنْ أَبِي هُرَيْرَةَ وَرَوَاهُ التِّرْمِذِيُّ فِي حَدِيثِ الْإِدْلَاءِ؛ فَإِنَّ الْحَدِيثَ يَدُلُّ عَلَى أَنَّ اللَّهَ فَوْقَ الْعَرْشِ وَيَدُلُّ عَلَى إحَاطَةِ الْعَرْشِ كَوْنُهُ سَقْفَ الْمَخْلُوقَاتِ. وَمَنْ تَأَوَّلَهُ عَلَى قَوْلِهِ هَبَطَ عَلَى عِلْمِ اللَّهِ كَمَا فَعَلَ التِّرْمِذِيُّ لَمْ يَدْرِ كَيْفَ الْأَمْرُ وَلَكِنْ لَمَّا كَانَ مِنْ أَهْلِ السُّنَّةِ وَعَلِمَ أَنَّ اللَّهَ فَوْقَ الْعَرْشِ وَلَمْ يَعْرِفْ صُورَةَ الْمَخْلُوقَاتِ وَخَشِيَ أَنْ يَتَأَوَّلهُ الجهمي أَنَّهُ مُخْتَلِطٌ بِالْخَلْقِ قَالَ: هَكَذَا وَإِلَّا فَقَوْلُ رَسُولِ اللَّهِ صَلَّى اللَّهُ عَلَيْهِ وَسَلَّمَ كُلُّهُ حَقٌّ يُصَدِّقُ بَعْضُهُ بَعْضًا. وَمَا عُلِمَ بِالْمَعْقُولِ مِنْ الْعُلُومِ الصَّحِيحَةِ يُصَدِّقُ مَا جَاءَ بِهِ الرَّسُولُ وَيَشْهَدُ لَهُ. فَنَقُولُ: إذَا تَبَيَّنَ أَنَّا نَعْرِفُ مَا قَدْ عُرِفَ مِنْ اسْتِدَارَةِ الْأَفْلَاكِ عُلِمَ أَنَّ الْمُنْكِرَ لَهُ مُخَالِفٌ لِجَمِيعِ الْأَدِلَّةِ لَكِنْ الْمُتَوَقِّفُ فِي ذَلِكَ قَبْلَ الْبَيَانِ فَعَلَ الْوَاجِبَ وَكَذَلِكَ مَنْ لَمْ يَزَلْ يَسْتَفِيدُ ذَلِكَ مِنْ جِهَةٍ لَا يَثِقُ بِهَا \href{https://shamela.ws/book/7289/12641#p1}{\faExternalLink} \cite{ibnTaimia_Majmoo}.\footnote{مجموع الفتاوى 25/193.}

\section{مسألة دوران الأرض حول نفسها وحول الشمس}
\label{sec:app_solar_system_albani}

يقول الشيخ الألباني رحمه الله: 

نحن الحقيقة لا نشك في أن قضية دوران الأرض حقيقة علمية لا تقبل جدلا، في الوقت الذي نعتقد أنه ليس من وظيفة الشرع عموما والقرآن خصوصا أن يتحدث عن علم الفلك ودقائق علم الفلك وإنما هذه تدخل في عموم قوله عليه الصلاة والسلام الذي أخرجه مسلم في صحيحه من حديث أنس بن مالك رضي الله عنه في قصة تأبير النخل حينما قال لهم (سإنما هو ظن ظننته فإذا أمرتكم بشيء من أمر دينكم فأتوا منه ما استطعتم وما أمرتكم بشيء من أمور دنياكم فأنتم أعلم بأمور دنياكم).

فهذه القضايا ليس من المفروض أن يتحدث عنها الرسول عليه السلام وإن تحدث هو في حديثه أو ربنا عز وجل في كتابه فإنما لبالغة أو لآية أو لمعجزة أو نحو ذلك، ولذلك فنستطيع أن نقول أنه لا يوجد في الكتاب ولا في السنة ما ينافي هذه الحقيقة العلمية المعروفة اليوم والتي تقول أن الأرض كروية وأنها تدور بقدرة الله عز وجل في هذا الفضاء الواسع، بل يمكن للمسلم أن يجد ما يشعر إن لم نقل ما ينص على أن الأرض كالشمس وكالقمر من حيث أنه كلها في هذا الفراغ كما قال عز وجل (وَهُوَ الَّذِي خَلَقَ اللَّيْلَ وَالنَّهَارَ وَالشَّمْسَ وَالقَمَرَ كُلٌّ فِي فَلَكٍ يَسْبَحُونَ) ، لا سيما إذا استحضرنا أن قبل هذا التعميم الإلهي بلفظة (وكُلٌّ)تعني الكواكب الثلاثة حيث ابتدأ بالأرض فقال (وَآَيَةٌ لَهُمُ الأَرْضُ المَيْتَةُ أَحْيَيْنَاهَا وَأَخْرَجْنَا مِنْهَا حَبًّا فَمِنْهُ يَأْكُلُونَ) ثم قال (وَالشَّمْسُ تَجْرِي لِمُسْتَقَرٍّ لَهَا ذَلِكَ تَقْدِيرُ العَزِيزِ العَلِيمِ وَالقَمَرَ قَدَّرْنَاهُ مَنَازِلَ حَتَّى عَادَ كَالعُرْجُونِ القَدِيم لَا الشَّمْسُ يَنْبَغِي لَهَا أَنْ تُدْرِكَ القَمَرَ وَلَا اللَّيْلُ سَابِقُ النَّهَارِ وَكُلٌّ فِي فَلَكٍ يَسْبَحُونَ)، فلفظة (كُلٌّ) تشمل الآية الأولى الأرض ثم الشمس ثم القمر ثم قال تعالى (وَكُلٌّ فِي فَلَكٍ يَسْبَحُونَ) هذا ظاهر من سياق الآيات وهي بلا شك آيات في ملك الله عز وجل باهرة، أقول هذا مع العلم بأن العلماء في التفسير أعادوا اسم كل إلى أقرب مذكور وهو الشمس والقمر لكن ليس هناك ما يمنع أبدا من أن نوسّع معنى الكل فيشمل الأرض التي ذكرت قبل الشمس وقبل القمر، هذا أقوله فإن صح فبها ونعمت وإن لم يصح فأقل ما يقال أنه لا يوجد في القرآن كما قلت آنفا ولا في السنة ما ينفي هذه الحقيقة العلمية، أما ما يقال أو ما يستدل به من الآيات كجعل الله عز وجل الجبال رواسي أَنْ تَمِيدَ بِهم وكـ (الأرض بعد ذلك دحاها)ونحو ذلك من الآيات فهي في الحقيقة لا تنهض لإبطال هذه الحقيقة العلمية الكونية من جهة بل لعل بعضها تكون حجة على المستدلين بها، فجعل الله عز وجل للجبال كالأوتاد تشبيها بالأوتاد فهذا نص صريح بأن ذلك لا يمنع تحركها مطلقا وإنما يمنع تحرك الأرض تحركا اضطرابيا بحيث لا يتمكن الساكنون عليها من التمتع بما فيها بل من الحياة عليها، ذلك لأننا نعلم أن الرواسي بالنسبة للسفن لا تمنع حركتها مطلقا لكنها تمنع أن تفلت هكذا في خضم البحر فتضربها الأمواج يمينا ويسارا ثم يكون مصيرها الغرق، كذلك الأوتاد التي تضرب للخيل ونحو ذلك من الدواب فهي لا تمنع أبدا أن تتحرك تحركا في مدى محدود أراد ذاك الحيوان الواتد إن صح التعبير وهو الذي ضرب الوتد بالحيوان.
ونحن نرى في سوريا في بعض البساتين التي تزرع فيها بعض الحشائش التي هي طعام للخيل وللبقر ونحو ذلك من الحيوانات يسمّى عندنا في بلاد الشام بالفصّة وربما يسمى عندكم بالبرسيم. فهذا يزرع فيأتي الفلاح حينما ينبت فيضرب وتدا لفرسه أو لبقرته فتجد هذه البقرة تأكل من هذا البرسيم المقدار الذي يريده صاحبها فهي تتحرك لكن لا تتحرك كيف تشاء كحركة الفوضى كما لو أطلق لها الزمام وإنما تتحرك حركة نظامية ولذلك تجد قد شكلت دائرة، الفصة أو البرسيم الذي أكلته فأصبحت الأرض جرداء من الخضار وما حولها الخضار به لا يزال قائما، فتشبيه رب العالمين تبارك وتعالى للجبال بالنسبة للأرض كالمراسي بالنسبة للسفينة والأوتاد بالنسبة للحيوانات هذه أيضا بالنسبة للأرض، كل ذلك لا ينفي عن الأرض الحركة المنظمة بقدرة الله تبارك وتعالى، 

لذلك قلت أن هذه الآيات أو بعضها على الأقل هي أقرب إلى الدلالة على أن الأرض تتحرك أقل ما يقال وأنها ليست ثابتة جامدة كما يتوهم كثير من الناس.
فخلاصة القول لا يوجد في الشرع أبدا ما ينفي كروية الأرض، ثم كروية الأرض أصبحت اليوم حقيقة علمية ملموسة لمس اليد، يعني يتهم الإنسان في عقله وعلى الأقل في علمه فيما إذا جحد هذه الحقيقة لأنك اليوم تستطيع أن ترفع السماعة وتتصل مع صديق لك صادق وتقول له ماذا عندكم اليوم نهار أم ليل سيقول لك عندنا ليل في الوقت الذي يؤذن عندنا مثلا لأذان المغرب يؤذن عندهم لصلاة الفجر أو يكون قد طلعت الشمس وهذا لا يمكن تصوره أبدا إلا كما يقول العلم بالتجربة أن هذا ينتج بسبب أن الأرض تدور حول الشمس دائرة كاملة ينتج من ورائها الليل والنهار، ثم أدق من ذلك حصول الفصول الأربعة بسبب ابتعاد الأرض عن الشمس واقترابها وهذا له تفصيله في علم الفلك وعلم الجغرافيا لسنا في صدده، لكن الشاهد أنه لا يمكن أن تحصل هذه الأمور الواضحة إلا والأرض أولا كروية وإذا سلبت كرويتها فلا يمكن أن يقال بأنها ثابتة لأن البشر يسكنون هذه الأرض في كل جوانبها كما يقال اليوم في القطب الشمالي في القطب الجنوبي فلو كانت هي كروية وثابتة كيف يثبت من كانوا في أسفل القطب الجنوبي مثلا بل ومن كان في طرفيها لكنها لما كانت تدور بقدرة الله العجيبة الدوران الذي لا يجعل حياة المستوطنين والساكنين عليها مضطربة فهذا أمر في غاية الإعجاز الدالة على عظمة وقدرة الله تبارك وتعالى.

وأنا أريد أن أذكّر بشيء يقرّب هذا الشيء البعيد الذي لا يدخل في أذهان بعض الناس، وأنا في ألبانيا كنت أجيرا في دكان خالي كان حلّاقا، فكان يأتيه زبون مثلا فيطلب له فنجان قهوة، يأتي أجير القهوجي وفي يده صحن صينية مثل هذه الصواني أكبر منها قليلا لكن هذه لها حاملة يعني يمكن أن نصورها هكذا هنا يضع إصبعه ويمشي أولا يلهوا ويتسلى ويعمل فيها هكذا والفنجان على الصحن الصغير كما هي العادة لا يتحرك من مكانه، هذا مصغّر جداً جدا ليفهم الإنسان كيف تدور الأرض ولا يضطرب البشر عليها والبشر بشر كما قال تعالى و (لَقَدْ خَلَقْنَا الإِنْسَانَ فِي أَحْسَنِ تَقْوِيمٍ) فإذا كان هذا الفنجان وهو موضوع في الصينية والذي يحركه هو إنسان جاهل غشيم قدرته ومداركه محدودة ومع ذلك ربنا عز وجل أعطاه شيء من العقل وشيء من القدرة بحيث أنه يدير هذه الصينية وعليها الفنجان وهو فوق صحن صغير فلا يقطر منه قطرة، هذا كنا نراه ونحن صغار، فالله عز وجل ماذا نقول "وليس يصح في الأذهان شيء فاحتاج إلى دليل" فالله عزوجل على كل شيء قدير .
إذن القضية لا تحتاج إلا إلى شيء من العلم والإدراك الصحيح مع وجود الإيمان الكامل طبعا بعظمة وقدرة الله التي لا يمكن أن يتصورها إنسان \href{https://alathar.net/home/esound/index.php?op=codevi&coid=155590}{\faExternalLink} \href{https://shamela.ws/book/36190/3337#p8}{\faExternalLink}.

يقول الشيخ ابن باز رحمه الله تعالى:

الله جل وعلا أخبرنا أنه جعل الأرض قرارا، وأرساها بالجبال وثبتها وجعلها قرارا لعباده. عليها يسيرون وعليها ينامون وفيها يحرثون ويغرسون الأشجار. وفي بحارها كذلك يعملون ما يعملون لطلب الرزق. فإذا زعم زاعم أو صور مصور أنها تسبح في الفضاء، لم يلزم بذلك أن يكون صادقا سواء كان شيوعيا أو نصرانيا أو يهوديا أو مسلما. كلام الله أصدق من الجميع. أما الإنسان قد يتصور الشيء أنه يدور أو يسبح بالحركة وليس الأمر كما قال. وإنما يكون في الجو وربما يكون في رأيه الظاهري وهو بعيد عنه لا يمسه ولا يتيقن مما يقوله هؤلاء وما يصوره هؤلاء. فما أخبر الله عنه أنه يتحرك هو كما أخبر عنه سبحانه وتعالى. وما شاهده الناس من سير الكواكب هو كما أخبر، كما يرى ويشاهد. وأما زعم الزاعمين بأن هذا يدل على أن الأرض تدور وأنها تسبح في الفضاء وأنها محتركة والله يقول جعلها لنا قرارا وقال: وألقى في الأرض رواسي أن تميد بكم. وبين سبحانه وتعالى أنه ثبتها بالجبال وأرساها وجعلها لها أوتادا، فالواجب التمسك بهذا والأخذ بهذا وأنها لا تميد ولا تضطرب ولا تدور، ولو دارت لأحسوا بها العباد من أجل الزلازل. ولو زلازل قليلة عرفها الناس. وربما هلك من حولها إذا عظمت الزلزلة وتهدمت البيوت وسقطت الأشجار وهلك الناس بأقل زلزلة. 

فهذه التي يحكيها الناس من هؤلاء الفضائيين أو غيرهم يزعمون أنها تدل على حركة الأرض ودورانها ليس لنا أن نسلم لهم ذلك ولا يمكن أن نسلم لهم ذلك إلا بدليل من كتاب الله وسنة رسوله عليه الصلاة والسلام أو شئ نلمسه بأيدينا ونراه بأبصارنا ونعقله لا شبهة فيه. فإذا وجد  ذلك أمكن تأويل أن تميد بالإضطراب الذي يضر الناس وأن الحركة التي لا تضر الناس من دوران وغيره لا تخالف الميد الذي ذكره الله. أما أن نفسر الميد بالإضطراب فقط وأن الأرض تدور وتتحرك ولكن ليس ميدا فهذا يحتاج إلى دليل. ومن قنع لذلك، من رأى وشاهد واعتقد لا يضره ذلك. ومن لم يعتقد ذلك ولم يظهر له ما يخالف ذلك لا يضره اعتقاده الذي يراه صحيحا، ويراه موافقا لكتاب الله وكل واحد له اعتقاده، فمن اعتقد ما ظهر له من كتاب الله فهو غير ملوم. ومن شاهد أشياء وتيقنها يقينا وأن هناك حركة لا تمنع وصف الأرض بأنها غير مائدة وأنها قرار وأنه دوران خاص لا ينافي كونها قرارا ولا ينافي كونها قد أرسيت بالجبال ولا ينافي كونها لا تميد، من تيقن هذا وعرفه بقلبه وصدقه بعينه فلا لوم عليه إذا اعتقد ذلك. وليس له أن يلوم الآخرين. وليس له أن يقدح في الآخرين لأنهم لم يعلموا ما علم، وكل له علمه. كما أن من علم أن الحكم الفلاني هو التحريم أو الوجوب والآخر أشكل عليه الأمر فليس له أن يلوم من علم. فالحجة حجة على من لم يعلم. ومن علم وحفظ حجة على من لم يحفظ ولم يعلم. وكل له حجته وكل له دليل. فأنا أعتقد، وقد كتبت هذا في كتابا من مدة سنوات، أعتقد أنها قارة كما قال الله وأنها لا تدور ولا تضطرب ولا تتحرك بل هي ثابتة. وقد ذكرت كلام أهل العلم في ذلك. ومن زعم خلاف ذلك فإن كان متيقنا فلا لوم عليه وله ما اعتقد ولا يلزمنا أن نوافقه ونقلده، ولا يلزمنه أن يقلدنا ومن قال بقولنا \href{https://www.youtube.com/watch?v=nbzh7p2ZlFQ}{\faExternalLink}. 

\section{مختصر رسالة الشيخ أحمد شاكر في الأخذ بالحساب}
\label{sec:app_shuhur_ahmidShakir}

هذا مختصر لما قرره الشيخ أحمد محمد شاكر رحمه الله في رسالته: أوائل الشهور العربية: هل يجوز شرعا إثباتها بالحساب الفلكي؟ \cite{shuhur_ahmidShakir}.  ولعلنا نضع أهم ما جاء فيها في عدة نقاط مع شيء يسير من التصرف.

- أتفقت - أو كادت - كتب العلماء والفقهاء على أن العبرة في ثبوت الشهر بالرؤية وحدها دون الإعتبار بالحساب. مع قلة أو ندرة وجود الأقوال في الأخذ بالحساب على الإطلاق وإنما فقط على وجه التقييد كما جاء في المذهب الشافعي: أنه يجوز للحاسب العمل بحسابه في نفسه مع جواز غيره تقليده. واختلف العلماء أيضا في الإعتبار بإختلاف المطالع. فأما الشافعية فقد ذهبوا إلى أن لكل بلد رؤيتهم مع الخلاف في حد ذلك. كما أن النووي نقل إتفاق الإمام مالك وأبا حنيفة: أنه يلزم غير أهل البلد رؤية بلد أخر.

- الإختلاف في المطالع مع إختلاف أقوال العلماء في ذلك ترتب عليه الخلاف بين المسلمين في المواقيت ولم يجدوا لتوحيد الكلمة فيها سبيلا. من ذلك الخلاف في يوم عرفة وهو يوم الحج الأكبر وهو أعظم المواسم الإسلامية وشهر ذي الحجة أخطر الشهور أثرا، إذ أن يوم عرفة، وهو التاسع منه هو ظرف محدود لأداء ركن الحج وهو الوقوف بعرفة وهو لا يدور إلا مرة واحدة في السنة. وأكثر الحجاج لا يحجون إلا مرة في العمر فلعلهم إن أخطأهم الوقوف يخشون أن لا يكونوا قد أدوا الفريضة على وجهها الصحيح.

- مما لا شك فيه أن العرب قبل الإسلام وفي صدر الإسلام لم لكونوا يعرفون العلوم الفلكية معرفة علمية جازمة. أي أنهم كانوا أميين لا يكتبون ولا يحسبون، ومن عرف منهم شيئا من ذلك فإنما يعرف مبادئ أو قشورا عرفها بالملاحظة والتتبع، أو بالسماع والخبر، لم تبن على قواعد رياضية، ولا على براهين قطعية ترجع إلى مقدمات أولية يقينية. ولذلك جعل الرسول ﷺ مرجع إثبات الشهر إلى الأمر القطعي المشاهد الذي يكون برؤية الهلال بالعين المجردة. فإن هذا أحكم وأضبط وهو الذي يصل إليه اليقين مما في استطاعتهم ولا يكلف الله نفسا إلا وسعها. ولم يكن مما يوافق حكمة الشارع أن يجعل مناط إثبات الأهلة بالحساب وهم لا يعرفون شيئا من ذلك. فلو جعله لهم بالحساب لأعنتهم.

- كان أكثر الفقهاء لا يعرفون علوم الفلك (والحساب) أو يعرفون فقط بعض مبادئها. وكان بعضهم أو أغلبهم لا يثق بمن يعرفها. بل كان بعضهم يرمي المشتغل بها بالزيغ والإبتداع ظنا منه أن هذه العلوم يتوسل بها أهلها إلى إدعاء العلم بالغيب (التنجيم). وكان بعضهم يدعى ذلك فعلا فأساء إلى نفسه وإلى علمه. والفقهاء معذورون في ذلك. ومن كان من الفقهاء والعلماء يعرف هذه العلوم لم يستطع أن يحدد موفقها الصحيح بالنسبة إلى الدين والفقه، بل كان يشير إليها على تخوف. ومن ذلك ما جاء عن الإمام الكبير تقي الدين بن دقيق العيد في وجوب الأخذ بالحساب عند الإغمام إن دل الحساب على أن الهلال قد طلع على نحو يرى لولا وجود المانع ولوجود السبب الشرعي الذي لا يلزم بالرؤية.\footnote{وهذا أكثر موافقة للسنة من القول بوجوب الأخذ بالرؤية على الإطلاق حتى مع وجود المانع لتعارض ذلك مع قوله ﷺ "فإن غم عليكم فأقدروا له".} ومن ذلك أيضا ما جاء عن تقي الدين السبكي في الوجوب الأخذ بالحساب الثابت والصحيح قطعا ولو جاءت شهادة الرؤية بخلافه لإستحالة ذلك حسا وعقلا وشرعا.\footnote{هذا فيه تفصيل ولا يفهم منه أن الحساب القطعي الثابت يكون مخالف للرؤية وأنما موافق لها عند الدرجات التي يظهر فيها الهلال. وإنما المقصود عدم الإعتبار بالشهادة نفسها لأنها في حقيقتها غير صحيحة لإنتفاء أسبابها كأن يكون القمر في الدرجات التي لا يمكن أن يظهر فيها. وإلا لو كانت الشهادة صحيحة فلا يجب أن تخالف الرؤية ولا الحساب القطعي وإلا فإما الشهادة غير صحيحة وإما الحساب غير صحيح. ولهذا ليس كل حساب قطعي وثابت كما أن ليس كل شهادة صحيحة وثابتة ولكن متى ثبت إحداهما وجب الأخذ به.}

\section{مسألة الأخذ بالحساب لشيخ الإسلام ابن تيمية}
\label{sec:app_shuhur_ibnTaimia}

وقد  قال شيخ الإسلام ابن تيمية رحمه الله: قد بينا أن شريعة الإسلام ومعرفتها ليست موقوفة على شيء يتعلم من غير المسلمين أصلا وإن كان طريقا صحيحا. بل طرق الجبر والمقابلة فيها تطويل. يغني الله عنه بغيره كما ذكرنا في المنطق. وهكذا كل ما بعث به النبي صلى الله عليه وسلم مثل العلم بجهة القبلة والعلم بمواقيت الصلاة والعلم بطلوع الفجر والعلم بالهلال؛ فكل هذا يمكن العلم به بالطرق التي كان الصحابة والتابعون لهم بإحسان يسلكونها ولا يحتاجون معها إلى شيء آخر. وإن كان كثير من الناس قد أحدثوا طرقا أخر؛ وكثير منهم يظن أنه لا يمكن معرفة الشريعة إلا بها. وهذا من جهلهم كما يظن طائفة من الناس أن العلم بالقبلة لا يمكن إلا بمعرفة أطوال البلاد وعروضها. وهو وإن كان علما صحيحا حسابيا يعرف بالعقل لكن معرفة المسلمين بقبلتهم ليست موقوفة على هذا [.] فلهذا كان قدماء علماء "الهيئة" كبطليموس صاحب المجسطي وغيره لم يتكلموا في ذلك بحرف وإنما تكلم فيه بعض المتأخرين مثل كوشيار الديلمي ونحوه لما رأوا الشريعة جاءت باعتبار الرؤية. فأحبوا أن يعرفوا ذلك بالحساب فضلوا وأضلوا. \href{https://shamela.ws/book/7289/4480#p1}{\faExternalLink} \cite{ibnTaimia_Majmoo}.\footnote{مجموع الفتاوى 9/215.} 


ولقد ناقش شيخ الإسلام ابن تيمية رحمه الله العديد من الأمور فيما يخص بالأخذ بالرؤية بدلا من الحساب. وقد رحج شيخ الإسلام عدم الجواز بالأخذ بالحساب على الإطلاق في إثبات الرؤية أو نفيها وقال: "الطريق إلى معرفة طلوع الهلال هو الرؤية؛ لا غيرها". ورجح ذلك لثبوت الأدلة وإجماع المسلمون المتقدمين عليه. ونقل أن بعض المتأخرين أجاز للحاسب أن يعمل بالحساب في حق نفسه إذا غم الهلال وقال عن هذا القول شاذ. كما أنه نقل أن تعليق عموم الحكم العام بالحساب لم يقله مسلم. والحقيقة أن شيخ الإسلام بخس في أهمية علم الحساب في تقدير الأهلة وإستنقصه فقال: "تضييع زمان كثير واشتغال عما يعني الناس وما لا بد له منه وربما وقع فيه الغلط والاختلاف". كما أنه حمل حديث الرسول صلى الله عليه وسلم على وجوب عدم الأخذ بالكتابة والحساب بل الحفظ فقط فقال: "فأمتنا ليست مثل أهل الكتاب الذين لا يحفظون كتبهم في قلوبهم بل لو عدمت المصاحف كلها كان القرآن محفوظا في قلوب الأمة وبهذا الاعتبار فالمسلمون أمة أمية بعد نزول القرآن وحفظه". وقال: "إنا أمة أمية لا نحسب ولا نكتب": فلم يقل إنا لا نقرأ كتابا ولا نحفظ بل قال: لا نكتب ولا نحسب فديننا لا يحتاج أن يكتب ويحسب".

كما أن شيخ الإسلام قال أيضا: "والمعتمد على الحساب في الهلال كما أنه ضال في الشريعة مبتدع في الدين فهو مخطئ في العقل وعلم الحساب. فإن العلماء. بالهيئة يعرفون أن الرؤية لا تنضبط بأمر حسابي". وهذا فيه أن علم الجبر والمقابلة في زمن شيخ الإسلام كان ضعيفا عند المسلمين ولكن في الأصل فإن المسلمين هم من وضع أساس هذا العلم العظيم بناءا على آيات الله الشرعية التي دلت على تعلم الحساب وإقامة الميزان. وقد تقدمت الأدلة في هذا البحث على أن شريعة الإسلام جاءت لرفع الأمية عن أمة الإسلام. 


حتى أنه غلظ القول في من يأخذ بالحساب من أهل البدع المارقين الخارجين عن شريعة الإسلام فقال: "الأخذ بالحساب أو الكتاب كالجداول وحساب التقويم والتعديل المأخوذ من سيرهما. وغير ذلك الذي صرح رسول الله صلى الله عليه وسلم بنفيه عن أمته والنهي عنه. ولهذا ما زال العلماء يعدون من خرج إلى ذلك قد أدخل في الإسلام ما ليس منه فيقابلون هذه الأقوال بالإنكار الذي يقابل به أهل البدع [.] فأما الذين يعتمدون على حساب الشهور وتعديلها فيعتبرونه برمضان الماضي. أو برجب أو يضعون جدولا يعتمدون عليه فهم مع مخالفتهم لقوله صلى الله عليه وسلم {لا نكتب ولا نحسب} [.] وهذا القدر موافق في أكثر الأوقات؛ لأن الغالب على الشهور هكذا ولكنه غير مطرد فقد يتوالى شهران وثلاثة وأكثر ثلاثين وقد يتوالى شهران وثلاثة وأكثر تسعة وعشرين فينتقض كتابهم وحسابهم ويفسد دينهم الذي ليس بقيم وهذا من الأسباب الموجبة لئلا يعمل بالكتاب والحساب في الأهلة. فهذه طريقة هؤلاء المبتدعة المارقين الخارجين عن شريعة الإسلام الذين يحسبون ذلك الشهر بما قبله من الشهور إما في جميع السنين أو بعضها ويكتبون ذلك [.] وأما الفريق الثاني: فقوم من فقهاء البصريين ذهبوا إلى أن قوله: {فاقدروا له} تقدير حساب بمنازل القمر وقد روي عن محمد بن سيرين قال: خرجت في اليوم الذي شك فيه فلم أدخل على أحد يؤخذ عنه العلم إلا وجدته يأكل إلا رجلا كان يحسب ويأخذ بالحساب ولو لم يعلمه كان خيرا له. وقد قيل: إن الرجل مطرف بن عبد الله بن الشخير وهو رجل جليل القدر إلا أن هذا إن صح عنه فهي من زلات العلماء. وقد حكي هذا القول عن أبي العباس بن سريج أيضا [.] واحتجاج هؤلاء بحديث ابن عمر في غاية الفساد مع أن ابن عمر هو الراوي عن النبي صلى الله عليه وسلم {إنا أمة أمية لا نكتب ولا نحسب} فكيف يكون موجب حديثه العمل بالحساب. وهؤلاء يحسبون مسيره في ذلك الشهر ولياليه. وليس لأحد منهم طريقة منضبطة أصلا بل أية طريقة سلكوها فإن الخطأ واقع فيها أيضا فإن الله سبحانه لم يجعل لمطلع الهلال حسابا مستقيما بل لا يمكن أن يكون إلى رؤيته طريق مطرد إلا الرؤية وقد سلكوا طرقا كما سلك الأولون منهم من لم يضبطوا سيره إلا بالتعديل الذي يتفق الحساب على أنه غير مطرد.
". 



 كما أنه علق على أحاديث الرسول فقال: "فهذه الأحاديث المستفيضة المتلقاة بالقبول دلت على أمور. أحدها أن قوله: (إنا أمة أمية لا نكتب ولا نحسب) هو خبر تضمن نهيا. فإنه أخبر أن الأمة التي اتبعته هي الأمة الوسط أمية لا تكتب ولا تحسب. فمن كتب أو حسب أو لم يكن من هذه الأمة في هذا الحكم. بل يكون قد اتبع غير سبيل المؤمنين الذين هم هذه الأمة فيكون قد فعل ما ليس من دينها والخروج عنها محرم منهي عنه فيكون الكتاب والحساب المذكوران محرمين منهيا عنهما. وهذا كقوله: (المسلم من سلم المسلمون من لسانه ويده) أي هذه صفة المسلم فمن خرج عنها خرج عن الإسلام ومن خرج عن بعضها خرج عن الإسلام في ذلك البعض وكذلك قوله: (المؤمن من أمنه الناس على دمائهم وأموالهم) [.] وأما الأمور المميزة التي هي وسائل وأسباب إلى الفضائل مع إمكان الاستغناء عنها بغيرها فهذه مثل الكتاب الذي هو الخط والحساب فهذا إذا فقدها مع أن فضيلته في نفسه لا تتم بدونها وفقدها نقص إذا حصلها واستعان بها على كماله وفضله كالذي يتعلم الخط فيقرأ به القرآن؛ وكتب العلم النافعة أو يكتب للناس ما ينتفعون به: كان هذا فضلا في حقه وكمالا. وإن استعان به على تحصيل ما يضره أو يضر الناس كالذي يقرأ بها كتب الضلالة ويكتب بها ما يضر الناس كالذي يزور خطوط الأمراء والقضاة والشهود: كان هذا ضررا في حقه وسيئة ومنقصة ولهذا نهى عمر أن تعلم النساء الخط [.] وإن أمكن أن يستغنى عنها بالكلية بحيث ينال كمال العلوم من غيرها. وينال كمال التعليم بدونها: كان هذا أفضل له وأكمل. وهذه حال نبينا صلى الله عليه وسلم الذي قال الله فيه: {الذين يتبعون الرسول النبي الأمي الذي يجدونه مكتوبا عندهم في التوراة والإنجيل} فإن أميته لم تكن من جهة فقد العلم والقراءة عن ظهر قلب فإنه إمام الأئمة في هذا. وإنما كان من جهة أنه لا يكتب ولا يقرأ مكتوبا. كما قال الله فيه: {وما كنت تتلو من قبله من كتاب ولا تخطه بيمينك} [.] وأما سائر أكابر الصحابة كالخلفاء الأربعة وغيرهم فالغالب على كبارهم الكتابة لاحتياجهم إليها إذ لم يؤت أحد منهم من الوحي ما أوتيه صارت أميته المختصة به كمالا في حقه من جهة الغنى بما هو أفضل منها وأكمل ونقصا في حق غيره من جهة فقده الفضائل التي لا تتم إلا بالكتابة.


فبين النبي صلى الله عليه وسلم أنا أيتها الأمة الأمية لا نكتب هذا الكتاب ولا نحسب هذا الحساب فعاد كلامه إلى نفي الحساب والكتاب فيما يتعلق بأيام الشهر الذي يستدل به على استسرار الهلال وطلوعه. وقد قدمنا فيما تقدم أن النفي وإن كان على إطلاقه يكون عاما فإذا كان في سياق الكلام ما يبين المقصود علم به المقصود أخاص هو أم عام؟ فلما قرن ذلك بقوله: {الشهر ثلاثون} و {الشهر تسعة وعشرون} بين أن المراد به أنا لا نحتاج في أمر الهلال إلى كتاب ولا حساب إذ هو تارة كذلك وتارة كذلك. والفارق بينهما هو الرؤية فقط ليس بينها فرق آخر من كتاب ولا حساب كما سنبينه. فإن أرباب الكتاب والحساب لا يقدرون على أن يضبطوا الرؤية بضبط مستمر وإنما يقربوا ذلك فيصيبون تارة ويخطئون أخرى.

كما أنه فسر الأمية بأنها صفة مدح: 
وظهر بذلك أن الأمية المذكورة هنا صفة مدح وكمال من وجوه: من جهة الاستغناء عن الكتاب والحساب بما هو أبين منه وأظهر وهو الهلال. ومن جهة أن الكتاب والحساب هنا يدخلهما غلط. ومن جهة أن فيهما تعبا كثيرا بلا فائدة فإن ذلك شغل عن المصالح إذ هذا مقصود لغيره لا لنفسه وإذا كان نفي الكتاب والحساب عنهم للاستغناء عنه بخير منه وللمفسدة التي فيه كان الكتاب والحساب في ذلك نقصا وعيبا بل سيئة وذنبا فمن دخل فيه فقد خرج عن الأمة الأمية فيما هو من الكمال والفضل السالم عن المفسدة ودخل في أمر ناقص يؤديه إلى الفساد والاضطراب. وأيضا فإنه جعل هذا وصفا للأمة. كما جعلها وسطا في قوله تعالى {جعلناكم أمة وسطا} فالخروج عن ذلك اتباع غير سبيل المؤمنين.

التبرير: 

وأيضا فالشيء إذا كان صفة للأمة لأنه أصلح من غيره؛ ولأن غيره فيه مفسدة: كان ذلك مما يجب مراعاته ولا يجوز العدول عنه إلى غيره لوجهين: لما فيه من المفسدة ولأن صفة الكمال التي للأمة يجب حفظها عليها. فإن كان الواحد لا يجب عليه في نفسه تحصيل المستحبات فإن كل ما شرع للأمة جميعا صار من دينها وحفظ مجموع الدين واجب على الأمة فرض عين أو فرض كفاية. ولهذا وجب على مجموع الأمة حفظ جميع الكتاب وجميع السنن المتعلقة بالمستحبات والرغائب وإن لم يجب ذلك على آحادها. ولهذا أوجب على الأمة من تحصيل المستحبات العامة ما لا يجب على الأفراد [.] ونظائره كثيرة مما يوجب أن يحفظ للأمة - في أمرها العام في الأزمنة والأمكنة والأعمال - كمال دينها الذي قال الله فيه: {اليوم أكملت لكم دينكم وأتممت عليكم نعمتي ورضيت لكم الإسلام دينا} فما أفضى إلى نقص كمال دينها ولو بترك مستحب يفضي إلى تركه مطلقا كان تحصيله واجبا على الكفاية إما على الأئمة وإما على غيرهم. فالكمال والفضل الذي يحصل برؤية الهلال دون الحساب يزول بمراعاة الحساب لو لم يكن فيه مفسدة.

الوجه الثاني: ما دلت عليه الأحاديث ما في قوله صلى الله عليه وسلم {لا تصوموا حتى تروه ولا تفطروا حتى تروه} كما ثبت ذلك عنه من حديث ابن عمر فنهى عن الصوم قبل رؤيته وعن الفطر قبل رؤيته

وأما العقل: فاعلم أن المحققين من أهل الحساب كلهم متفقون على أنه لا يمكن ضبط الرؤية بحساب بحيث يحكم بأنه يرى لا محالة أو لا يرى ألبتة على وجه مطرد وإنما قد يتفق ذلك أو لا 

ولعل من دخل في ذلك منهم كان مرموقا بنفاق، فما النفاق من هؤلاء ببعيد أو يتقرب به إلى بعض الملوك الجهال ممن يحسن ظنه بالحساب مع انتسابه إلى الإسلام.


ومن معرفة الحساب الاستسرار والإبدار الذي هو الاجتماع والاستقبال فالناس يعبرون عن ذلك بالأمر الظاهر من الاستسرار الهلالي في آخر الشهر وظهوره في أوله وكمال نوره في وسطه والحساب يعبرون بالأمر الخفي من اجتماع القرصين الذي هو وقت الاستسرار ومن استقبال الشمس والقمر الذي هو وقت الإبدار فإن هذا يضبط بالحساب. وأما الإهلال فلا له عندهم من جهة الحساب ضبط؛ لأنه لا يضبط بحساب يعرف كما يعرف وقت الكسوف والخسوف


فنقول الحاسب غاية ما يمكنه إذا صح حسابه أن يعرف مثلا أن القرصين اجتمعا في الساعة الفلانية وأنه عند غروب الشمس يكون قد فارقها القمر إما بعشر درجات مثلا أو أقل أو أكثر. والدرجة هي جزء من ثلاثمائة وستين جزءا من الفلك.

\href{https://shamela.ws/book/7289/12579#p2}{\faExternalLink} \cite{ibnTaimia_Majmoo}.\footnote{مجموع الفتاوى 25/126.} 



\section{مختصر سيرة  مطرف بن عبد الله بن الشخير}
\label{sec:app_mutrif}


وكان يقول : عقول الناس على قدر زمانهم .

قال قتادة : قال مطرف : لأن أعافى فأشكر أحب إلى من أن أبتلى فأصبر .

وعن محمد بن واسع قال : كان مطرف يقول : اللهم ارض عنا ; فإن لم ترض عنا فاعف عنا ; فإن المولى قد يعفو عن عبده وهو عنه غير راض .

وعن مطرف أنه قال لبعض إخوانه : يا أبا فلان إذا كانت لك حاجة فلا تكلمني واكتبها في رقعة ; فإني أكره أن أرى في وجهك ذل السؤال .

وفي " الحلية " روى أبو الأشهب ، عن رجل ، قال مطرف بن عبد الله : لأن أبيت نائما وأصبح نادما أحب إلي من أبيت أن قائما وأصبح معجبا .

وقال مهدي بن ميمون : قال مطرف : لقد كاد خوف النار يحول بيني وبين أن أسأل الله الجنة .

كان ثقة لم ينج بالبصرة من فتنة ابن الأشعث إلا هو وابن سيرين

أبو عقيل بشير بن عقبة قال : قلت ليزيد بن الشخير : ما كان مطرف يصنع إذا هاج الناس؟ قال : يلزم قعر بيته ، ولا يقرب لهم جمعة ولا جماعة حتى تنجلي .

وقال أيوب : قال مطرف : لأن آخذ بالثقة في القعود أحب إلي من أن ألتمس فضل الجهاد بالتغرير .

قال قتادة : فكان مطرف إذا كانت الفتنة نهى عنها وهرب ، وكان الحسن ينهى عنها ولا يبرح .

وقال حميد بن هلال : أتت الحرورية مطرف بن عبد الله يدعونه إلى رأيهم ، فقال : يا هؤلاء ، لو كان لي نفسان بايعتكم بإحداهما وأمسكت الأخرى ; فإن كان الذي تقولون هدى أتبعتها الأخرى ، وإن كان ضلالة ، هلكت نفس وبقيت لي نفس ; ولكن هي نفس واحدة لا أغرر بها .

ابن أبي عروبة : عن قتادة ، عن مطرف قال : لقيت عليا -رضي الله عنه- فقال لي : يا أبا عبد الله ، ما بطأ بك؟ أحب عثمان ؟ ثم قال : لئن قلت ذاك ، لقد كان أوصلنا للرحم ، وأتقانا للرب .


قال سليمان بن المغيرة : كان مطرف إذا دخل بيته ، سبحت معه آنية بيته .

قال أبو نعيم حدثنا سليمان بن أحمد ، حدثنا إسحاق ، أنبأنا عبد الرزاق ، حدثنا معمر ، عن قتادة قال : كان مطرف بن عبد الله وصاحب له سريا في ليلة مظلمة ، فإذا طرف سوط أحدهما عنده ضوء ، فقال : أما إنه لو حدثنا الناس بهذا ، كذبونا . فقال مطرف : المكذب أكذب - يقول : المكذب بنعمة الله أكذب .

حدثنا أبو حامد بن جبلة : حدثنا محمد بن إسحاق ، حدثنا الحسين بن منصور ، حدثنا حجاج ، عن مهدي بن ميمون ، عن غيلان بن جرير ، قال : أقبل مطرف مع ابن أخ له من البادية -وكان يبدو- فبينا هو يسير سمع في طرف سوطه كالتسبيح فقال له ابن أخيه : لو حدثنا الناس بهذا ، كذبونا . فقال : المكذب أكذب الناس .

حدثنا أبو بكر بن مالك ، حدثنا عبد الله بن أحمد ، حدثنا محمد بن عبيد بن حساب ، حدثنا جعفر بن سليمان ، حدثنا أبو التياح قال : كان مطرف بن عبد الله يبدو ; فإذا كان ليلة الجمعة ، أدلج على فرسه ، فربما نور له سوطه ، فأدلج ليلة حتى إذا كان عند القبور ، هوم على فرسه ، قال : فرأيت أهل القبور ، صاحب كل قبر جالسا على قبره ، فلما رأوني ، قالوا : هذا مطرف يأتي الجمعة قلت : أتعلمون عندكم يوم الجمعة؟ قالوا : نعم ، نعلم ما تقول الطير فيه . قلت : وما تقول الطير؟ قالوا : تقول : سلام سلام من يوم صالح . إسنادها صحيح .


عبد الله بن جعفر الرقي ، حدثنا الحسن بن عمرو الفزاري ، عن ثابت [ ص: 194 ] البناني ورجل آخر ، أنهما دخلا على مطرف وهو مغمى عليه ، قال : فسطعت معه ثلاثة أنوار : نور من رأسه ، ونور من وسطه ، ونور من رجليه ، فهالنا ذلك ، فأفاق فقلنا : كيف أنت يا أبا عبد الله ؟ قال : صالح . فقيل : لقد رأينا شيئا هالنا . قال : وما هو؟ قلنا : أنوار سطعت منك . قال : وقد رأيتم ذلك؟ قالوا : نعم . قال : تلك تنزيل السجدة ، وهي تسع وعشرون آية ، سطع أولها من رأسي ووسطها من وسطي وآخرها من قدمي ، وقد صورت تشفع لي ، فهذه ثوابية تحرسني

وقال سليمان بن حرب : كان مطرف مجاب الدعوة ، قال لرجل : إن كنت كذبت فأرنا به . فمات مكانه .

قال مهدي بن ميمون : حدثنا غيلان بن جرير ، أنه كان بينه وبين رجل كلام ، فكذب عليه فقال : اللهم إن كان كاذبا فأمته . فخر ميتا مكانه ، قال : فرفع ذلك إلى زياد فقال : قتلت الرجل؟ قال : لا ; ولكنها دعوة وافقت أجلا .



\section{مسألة أول ما خلق الله}
\label{sec:app_first_creation}


لقد تبث عن النبي ﷺ أن الله عز وجل لم بخلق السموات والأرض إلا بعد كتابة المقادير وكان عرشه على الماء سبحانه. ولكن اختلف أهل العلم في أول ما خلق الله قبل أن يخلق السموات والأرض. فمنهم من قدم العرش والماء على القلم واللوح المحفوظ، ومنهم من قدم القلم واللوح المحفوظ على العرش والماء. ولكن أهل العلم الذين بحثوا في هذه المسألة وجمعوا أدلتها وإجماع السلف وجهور أهل العلم فيها، قدموا الماء والعرش على القلم واللوح المحفوظ وهذا هو الحق كما قرر ذلك شيخ الإسلام ابن تيمية وكذلك ابن القيم، وابن كثير، والهمداني، والعسقلاني والعديد من علماء الإسلام، والله أعلى وأعلم. وفيما يلي بيان أدلة وأقوال أهل العلم في هذه المسألة.

\subsection{الأحاديث الخاصة بالمسألة}

\subsubsection{الحديث الأول}
\label{sec:app_first_creation_hadith_1}

حَدَّثَنِي أَبُو الطَّاهِرِ، أَحْمَدُ بْنُ عَمْرِو بْنِ عبد الله بْنُ عَمْرِو بْنِ سَرْحٍ. حَدَّثَنَا ابْنُ وَهْبٍ. أخبرني أَبُو هَانِئٍ الْخَوْلَانِيُّ عَنْ أَبِي عَبْدِ الرَّحْمَنِ الْحُبُلِيِّ، عَنْ عَبْدِ اللَّهِ بْنِ عَمْرِو بْنِ الْعَاصِ، قَالَ: النبي ﷺ أنه قال:  كَتَبَ اللَّهُ مَقَادِيرَ الخَلَائِقِ قَبْلَ أَنْ يَخْلُقَ السَّمَوَاتِ وَالأرْضَ بِخَمْسِينَ أَلْفَ سَنَةٍ، وَعَرْشُهُ علَى المَاءِ \href{https://shamela.ws/book/1727/6683#p2}{\faExternalLink} \cite{muslim}.\footnote{صحيح مسلم: 2653، وصححه الألباني في شرح الطحاوية.} وهذا الحديث فيه تقديم العرش والماء على كتابة المقادير أي القلم واللوح المحفوظ، والكتابة كانت قبل خلق السموات والأرض بخمسين ألف سنة.

\subsubsection{الحديث الثاني}
\label{sec:app_first_creation_hadith_2}

حدثنا عُمَرُ بْنُ حَفْصِ بْنِ غِيَاثٍ، حَدَّثَنَا أَبِي، حَدَّثَنَا الْأَعْمَشُ، حَدَّثَنَا جَامِعُ بْنُ شَدَّادٍ، عَنْ صَفْوَانَ بْنِ مُحْرِزٍ، أَنَّهُ حَدَّثَهُ عَنْ عِمْرَانَ بْنِ حُصَيْنٍ رضي الله عنه قَالَ: دَخَلْتُ عَلَى النَّبِيِّ ﷺ وَعَقَلْتُ نَاقَتِي بِالْبَابِ، فَأَتَاهُ نَاسٌ مِنْ بَنِي تَمِيمٍ فَقَالَ: "اقْبَلُوا البُشْرَى يَا بَنِي تَمِيمٍ"، قَالُوا: قَدْ بَشَّرْتَنَا فَأَعْطِنَا - مَرَّتَيْنِ - ثُمَّ دَخَلَ عَلَيْهِ نَاسٌ مِنْ أَهْلِ اليَمَنِ فَقَالَ: "اقْبَلُوا البُشْرَى يَا أَهْلَ اليَمَنِ إِذْ لَمْ يَقْبَلْهَا بَنُو تَمِيمٍ"، قَالُوا: قَدْ قَبِلْنَا يَا رَسُولَ اللَّهِ. قَالُوا: جِئْنَاكَ نَسْأَلُكَ عَنْ هَذَا الأَمْرِ؟ قَالَ: "كَانَ اللَّهُ وَلَمْ يَكُنْ شَيْءٌ غَيْرُهُ، وَكَانَ عَرْشُهُ عَلَى المَاءِ، وَكَتَبَ فِي الذِّكْرِ كُلَّ شَيْءٍ، وَخَلَقَ السَّمَاوَاتِ وَالأَرْضَ". فَنَادَى مُنَادٍ: ذَهَبَتْ نَاقَتُكَ يَا ابْنَ الحُصَيْنِ، فَانْطَلَقْتُ فَإِذَا هِيَ يَقْطَعُ دُونَهَا السَّرَابُ، فَوَاللَّهِ لَوَدِدْتُ أَنِّي كُنْتُ تَرَكْتُهَا \href{https://shamela.ws/book/1284/2020#p2}{\faExternalLink} \cite{bukhari}.\footnote{صحيح البخاري: 3199} وهذا إن حمل على وجه الترتيب فيه تقديم العرش والماء على كتابة المقادير، ومن ثم خلق السموات والأرض.

وجاء أيضا في صحيح البخاري نفس الحديث بسند آخر:
حدثنا عَبْدَانُ، عَنْ أَبِي حَمْزَةَ، عَنِ الْأَعْمَشِ، عَنْ جَامِعِ بْنِ شَدَّادٍ، عَنْ صَفْوَانَ بْنِ مُحْرِزٍ، عَنْ عِمْرَانَ بْنِ حُصَيْنٍ قَالَ: إِنِّي عِنْدَ النَّبِيِّ ﷺ إِذْ جَاءَهُ قَوْمٌ مِنْ بَنِي تَمِيمٍ فَقَالَ: "اقْبَلُوا الْبُشْرَى يَا بَنِي تَمِيمٍ"، قَالُوا: بَشَّرْتَنَا فَأَعْطِنَا، فَدَخَلَ نَاسٌ مِنْ أَهْلِ الْيَمَنِ فَقَالَ: "اقْبَلُوا الْبُشْرَى يَا أَهْلَ الْيَمَنِ إِذْ لَمْ يَقْبَلْهَا بَنُو تَمِيمٍ"، قَالُوا: قَبِلْنَا، جِئْنَاكَ لِنَتَفَقَّهَ فِي الدِّينِ، وَلِنَسْأَلَكَ عَنْ أَوَّلِ هَذَا الْأَمْرِ مَا كَانَ؟ قَالَ: "كَانَ اللَّهُ وَلَمْ يَكُنْ شَيْءٌ قَبْلَهُ، وَكَانَ عَرْشُهُ عَلَى الْمَاءِ، ثُمَّ خَلَقَ السَّمَاوَاتِ وَالْأَرْضَ، وَكَتَبَ فِي الذِّكْرِ كُلَّ شَيْءٍ"، ثُمَّ أَتَانِي رَجُلٌ فَقَالَ: يَا عِمْرَانُ، أَدْرِكْ نَاقَتَكَ فَقَدْ ذَهَبَتْ، فَانْطَلَقْتُ أَطْلُبُهَا فَإِذَا السَّرَابُ يَنْقَطِعُ دُونَهَا، وَايْمُ اللَّهِ، لَوَدِدْتُ أَنَّهَا قَدْ ذَهَبَتْ وَلَمْ أَقُمْ \href{https://shamela.ws/book/1284/4624#p6}{\faExternalLink} \cite{bukhari}.\footnote{صحيح البخاري: 7413} وهذا فيه تقديم السموات والأرض على كتابة المقادير، وهذا إن حمل على وجه الترتيب فهو يتعارض مع أغلب الأحاديث الأخرى. ولهذا أغلب أهل العلم أخذ بالحديث السابق. 

\subsubsection{الحديث الثالث}
\label{sec:app_first_creation_hadith_3}

حدَّثنا جعفرُ بنُ مسافر الهُذلىُّ، حدَّثنا يحيى بن حسَّان، حدَّثنا الوليدُ بن رباحٍ، عن إبراهيمَ بن أبي عَبْلَة، عن أبي حفصةَ، قال: قال عبادةُ بن الصَّامت لابنه: يا بُنيَّ إنَّك لن تَجِدَ طعمَ حقيقةِ الإيمان حتّى تَعلمَ أن ما أصابَكَ لم يكُنْ ليُخْطئَكَ، وما أخطَاكَ لم يكُنْ ليُصيبَك، سمعتُ رسول الله ﷺ يقول: إنَّ أولَ ما خلق اللهُ القلمُ، فقال لهُ: اكتبْ، قال: ربِّ وماذا أكتبُ؟ قال: اكتُبْ مقاديرَ كلِّ شيءٍ حتى تقومَ الساعةُ. وفي رواية: اكتُبْ القدَرَ، ما كان و ما هو كائِنٌ إلى الأبَدِ. ومن مات على غيرِ هذا فليسَ مِني \href{https://shamela.ws/book/117359/3975#p2}{\faExternalLink} \cite{SunanAbiDawood}.\footnote{أبي داود: 4700 واللفظ له، أحمد: 22705، الترمذي: 3319، وصححه الألباني في صحيح الجامع.} وفي الأخذ بظاهر هذا الحديث دون الجمع مع الأحاديث السابقة تقديم القلم واللوح المحفوظ على العرش والماء، أي أن الله خلق القلم أولا ثم خلق اللوح المحفوظ وأمر القلم بكتابة المقادير على اللوح المحفوظ قبل العرش والماء وخلق السموات والأرض.

\subsubsection{الحديث الرابع}
\label{sec:app_first_creation_hadith_4}

حَدَّثَنَا يَزِيدُ بْنُ هَارُونَ، أَخْبَرَنَا حَمَّادُ بْنُ سَلَمَةَ، عَنْ يَعْلَى بْنِ عَطَاءٍ، عَنْ وَكِيعِ بْنِ عُدُسٍ، عَنْ عَمِّهِ أَبِي رَزِينٍ، قَالَ: قُلْتُ: يَا رَسُولَ اللهِ، أَيْنَ كَانَ رَبُّنَا عَزَّ وَجَلَّ قَبْلَ أَنْ يَخْلُقَ خَلْقَهُ؟ قَالَ: " كَانَ فِي عَمَاءٍ مَا تَحْتَهُ هَوَاءٌ، وَمَا فَوْقَهُ هَوَاءٌ، ثُمَّ خَلَقَ عَرْشَهُ عَلَى الْمَاءِ" \href{https://shamela.ws/book/25794/12650#p1}{\faExternalLink} \cite{ahmid}.\footnote{أحمد: 16188 واللفظ له، الترمذي: 3109، ابن ماجه: 182، ضعفه الألباني.}  ولكن هذا الحديث ضعفه الشيخ الألباني وقال فيه نظر؛ لأن وكيعاً هذا مجهول \href{https://shamela.ws/book/9442/5278#p10}{\faExternalLink} \cite{albani_Sahiha}، وقال شعيب الأرناؤوط في تخريج مسند الإمام أحمد إسناده ضعيف لنفس السبب.
\subsection{أقوال أهل العلم}

قال شيخ الإسلام ابن تيمية في مجموع الفتاوى \href{https://shamela.ws/book/7289/9349#p1}{\faExternalLink}:

وَمِنْ هَذَا: الْحَدِيثُ الَّذِي رَوَاهُ أَبُو دَاوُد وَالتِّرْمِذِي وَغَيْرُهُمَا عَنْ عبادة بْنِ الصَّامِتِ عَنْ النَّبِيِّ صَلَّى اللَّهُ عَلَيْهِ وَسَلَّمَ أَنَّهُ قَالَ: "{أَوَّلُ مَا خَلَقَ اللَّهُ الْقَلَمَ فَقَالَ لَهُ: اُكْتُبْ قَالَ: وَمَا أَكْتُبُ. قَالَ: مَا هُوَ كَائِنٌ إلَى يَوْمِ الْقِيَامَةِ}" (\fullautoref{sec:app_first_creation_hadith_3}) فَهَذَا الْقَلَمُ خَلَقَهُ لِمَا أَمَرَهُ بِالتَّقْدِيرِ الْمَكْتُوبِ قَبْلَ خَلْقِ السَّمَوَاتِ وَالْأَرْضِ بِخَمْسِينَ أَلْفَ سَنَةٍ وَكَانَ مَخْلُوقًا قَبْلَ خَلْقِ السَّمَوَاتِ وَالْأَرْضِ وَهُوَ أَوَّلُ مَا خُلِقَ مِنْ هَذَا الْعَالَمِ وَخَلَقَهُ بَعْدَ الْعَرْشِ كَمَا دَلَّتْ عَلَيْهِ النُّصُوصُ (\fullautoref{sec:app_first_creation_hadith_1}، \fullautoref{sec:app_first_creation_hadith_2}) وَهُوَ قَوْلُ جُمْهُورِ السَّلَفِ كَمَا ذَكَرْتُ أَقْوَالَ السَّلَفِ فِي غَيْرِ هَذَا الْمَوْضِعِ.

وأورد شيخ الإسلام خمسة عشرة وجها، وقال في الوجه الرابع عشر: 

مِثْلُ قَوْلِهِ فِي الْحَدِيثِ الْآخَرِ: "{قَدَّرَ اللَّهُ مَقَادِيرَ الْخَلَائِقِ قَبْلَ أَنْ يَخْلُقَ السَّمَوَاتِ وَالْأَرْضَ بِخَمْسِينَ أَلْفَ سَنَةٍ}" (\fullautoref{sec:app_first_creation_hadith_1}) فَإِنَّ الْخَلَائِقَ هُنَا الْمُرَادُ بِهَا الْخَلَائِقُ الْمَعْرُوفَةُ الْمَخْلُوقَةُ بَعْدَ خَلْقِ الْعَرْشِ وَكَوْنِهِ عَلَى الْمَاءِ. وَلِهَذَا كَانَ التَّقْدِيرُ لِلْمَخْلُوقَاتِ هُوَ التَّقْدِيرُ لِخَلْقِ هَذَا الْعَالَمِ كَمَا فِي حَدِيثِ الْقَلَمِ: (إنَّ اللَّهَ لَمَّا خَلَقَهُ قَالَ: اُكْتُبْ قَالَ: وَمَاذَا أَكْتُبُ؟ قَالَ: اُكْتُبْ مَا هُوَ كَائِنٌ إلَى يَوْمِ الْقِيَامَةِ) (\fullautoref{sec:app_first_creation_hadith_3}). وَكَذَلِكَ فِي الْحَدِيثِ الصَّحِيحِ: "{إنَّ اللَّهَ قَدَّرَ مَقَادِيرَ الْخَلَائِقِ قَبْلَ أَنْ يَخْلُقَ السَّمَوَاتِ وَالْأَرْضَ بِخَمْسِينَ أَلْفَ سَنَةٍ وَكَانَ عَرْشُهُ عَلَى الْمَاءِ}" (\fullautoref{sec:app_first_creation_hadith_1}) وَقَوْلُهُ فِي الْحَدِيثِ الْآخَرِ الصَّحِيحِ: "{كَانَ اللَّهُ وَلَا شَيْءَ قَبْلَهُ وَكَانَ عَرْشُهُ عَلَى الْمَاءِ وَكَتَبَ فِي الذِّكْرِ كُلَّ شَيْءٍ ثُمَّ خَلَقَ السَّمَوَاتِ وَالْأَرْضَ}" (\fullautoref{sec:app_first_creation_hadith_2}) يُرَادُ بِهِ أَنَّهُ كَتَبَ كُلَّ مَا أَرَادَ خَلْقَهُ مِنْ ذَلِكَ.

\comment{
قال شيخ الإسلام ابن تيمية في مجموع الفتاوى \href{https://shamela.ws/book/7289/2988#p1}{\faExternalLink}:

فَقَدْ أَخْبَرَ أَنَّ عَرْشَهُ كَانَ عَلَى الْمَاءِ قَبْلَ أَنْ يَخْلُقَ السَّمَوَاتِ وَالْأَرْضَ كَمَا قَالَ تَعَالَى: (وَهُوَ الَّذِي خَلَقَ السَّمَاوَاتِ وَالْأَرْضَ فِي سِتَّةِ أَيَّامٍ وَكَانَ عَرْشُهُ عَلَى الْمَاءِ). وَقَدْ ثَبَتَ فِي صَحِيحِ الْبُخَارِيِّ وَغَيْرِهِ عَنْ عِمْرَانَ بْنِ حُصَيْنٍ عَنْ النَّبِيِّ صَلَّى اللَّهُ عَلَيْهِ وَسَلَّمَ أَنَّهُ قَالَ: (كَانَ اللَّهُ وَلَمْ يَكُنْ شَيْءٌ غَيْرُهُ وَكَانَ عَرْشُهُ عَلَى الْمَاءِ وَكَتَبَ فِي الذِّكْرِ كُلَّ شَيْءٍ وَخَلَقَ السَّمَوَاتِ وَالْأَرْضَ) وَفِي رِوَايَةٍ لَهُ (كَانَ اللَّهُ وَلَمْ يَكُنْ شَيْءٌ قَبْلَهُ وَكَانَ عَرْشُهُ عَلَى الْمَاءِ ثُمَّ خَلَقَ السَّمَوَاتِ وَالْأَرْضَ وَكَتَبَ فِي الذِّكْرِ كُلَّ شَيْءٍ) وَفِي رِوَايَةٍ لِغَيْرِهِ صَحِيحَةٍ: (كَانَ اللَّهُ وَلَمْ يَكُنْ شَيْءٌ مَعَهُ وَكَانَ عَرْشُهُ عَلَى الْمَاءِ ثُمَّ كَتَبَ فِي الذِّكْرِ كُلَّ شَيْءٍ) (\fullautoref{sec:app_first_creation_hadith_2}) وَثَبَتَ فِي صَحِيحِ مُسْلِمٍ عَنْ عَبْدِ اللَّهِ بْنِ عَمْرٍو عَنْ النَّبِيِّ صَلَّى اللَّهُ عَلَيْهِ وَسَلَّمَ أَنَّهُ قَالَ: (إنَّ اللَّهَ قَدَّرَ مَقَادِيرَ الْخَلَائِقِ قَبْلَ أَنْ يَخْلُقَ السَّمَوَاتِ وَالْأَرْضَ بِخَمْسِينَ أَلْفِ سَنَةٍ وَكَانَ عَرْشُهُ عَلَى الْمَاءِ) (\fullautoref{sec:app_first_creation_hadith_1}). وَهَذَا التَّقْدِيرُ بَعْدَ وُجُودِ الْعَرْشِ وَقَبْلَ خَلْقِ السَّمَوَاتِ وَالْأَرْضِ بِخَمْسِينَ أَلْفِ سَنَةٍ.
}

قال ابن القيم في نونية ابن القيم الكافية الشافية \href{https://shamela.ws/book/11375/70#p2}{\faExternalLink}: 

\begin{center}
\begin{tabular}{ccc}
    هذا وعرش الرب فوق الماء من & ... &
    قبل السنين بمدة وزمان  \\
    والناس مختلفون في القلم الذي & ... &
    كتب القضاء به من الديان \\
    هل كان قبل العرش أو هو بعده & ... &
    قولان عند أبي العلا الهمداني \\
    والحق أن العرش قبل لأنه & ... &
    قبل الكتابة كان ذا أركان \\
    وكتابة القلم الشريف تعقبت & ... &
    إيجاده من غير فصل زمان \\
    لما براه الله قال اكتب كذا & ... &
    فغدا بأمر الله ذا جريان \\
    فجرى بما هو كائن أبدا إلى & ... &
    يوم المعاد بقدرة الرحمن \\
\end{tabular}
\end{center}
\bigbreak

قال ابن كثير في البداية والنهاية \href{https://shamela.ws/book/30097/106#p7}{\faExternalLink}:

واختلف هؤلاء في أيها خُلِق أولًا؟ فقال قائلون: خلق القلم قبل هذه الأشياء كلها، وهذا هو اختيار ابن جرير، وابن الجوزي، وغيرهما. قال ابن جرير: وبعد القلم السحاب الرقيق، وبعده العرش. واحتجوا بالحديث الذي رواه الإمام أحمد، وأبو داود والترمذي، عن عُبادة بن الصامت رضي الله عنه قال: قال رسولُ اللَّه ﷺ: "إنَّ أوَّلَ ما خَلَقَ اللَّهُ القَلَمُ. ثُمَّ قالَ لَهُ اكْتُبْ، فجرى في تلك السَّاعة بما هُوَ كائنٌ إلى يَوْمِ القيَامَةِ" لفظ أحمد. وقال الترمذي: حسن صحيح غريب (\fullautoref{sec:app_first_creation_hadith_3}).

والذي عليه الجمهور، فيما نقله الحافظ أبو العلاء الهَمَذَاني وغيره: أنَّ العرش مخلوق قبل ذلك، وهذا هو الذي رواه ابن جرير من طريق الضحاك عن ابن عباس، كما دلَّ على ذلك الحديث الذي رواه مسلم في "صحيحه" حيث قال: حدّثني أبو الطاهر أحمدُ بْن عمرو بن السَّرْح، حدَّثنا ابن وهب، أخبرني أبو هانئ الخَوْلاني، عن أبي عبد الرحمن الحُبُلي، عن عبد اللَّه بن عمرو بن العاص قال: سمعتُ رسولَ اللَّه ﷺ يقول: "كَتَبَ اللَّهُ مَقَاديْرَ الخَلائِقِ قَبْلَ أنْ يَخْلُقَ السَّموَاتِ وَالأرْضَ بخَمْسين ألْفَ سَنةٍ، قالَ: وعَرْشُهُ على المَاءِ" (\fullautoref{sec:app_first_creation_hadith_1})، قالوا: فهذا التقدير هو كتابته بالقلم المقاديرَ. وقد دلَّ هذا الحديثُ أنَّ ذلك بعد خلق العرش، فثبتَ تقدُّم العرش على القلم الذي كتبت به المقادير كما ذهب إلى ذلك الجماهير. ويُحمل حديثُ القلم على أنَّه أوَّلُ المخلوقات من هذا العالم.

ويؤيد هذا ما رواه البخارى، عن عِمْران بن حصين: قال: قال أهلُ اليمن لرسول اللَّه ﷺ: جِئْنَاكَ لنَتَفقَّه في الدِّيْن وَلنَسْألَكَ عَنْ أوَّلِ الأمْرِ. فَقَالَ: "كانَ اللَّهُ وَلَمْ يَكُنْ شَيْءٌ قَبْلَهُ -وفي رواية: معه، وفي رواية: غيره- وكَان عَرْشُهُ عَلَى الماءِ، وكَتَبَ في الذِّكْرِ كُلَّ شَيْءٍ، وخَلَقَ السَّمواتِ والأرْضَ"، وفي لفظ: " ثمَّ خَلقَ السَّمواتِ والأرْضَ" (\fullautoref{sec:app_first_creation_hadith_2}). فسألوه عن ابتداء خلق السَّمواتِ والأرض، ولهذا قالوا: جئناك نسألك عن أول هذا الأمر، فأجابهم عمَّا سألوا فقط. ولهذا لم يخبرهم بخلق العرش كما أخبر به في حديث أبي رَزين المتقدم.

قال ابن جرير: وقال آخرون: بل خلقَ اللَّهُ عز وجل الماءَ قَبْلَ العَرْشِ. رواه السُّدّي عن أبي مالك، وعن أبي صالح عن ابن عباس، وعن مُرَّة عن ابن مسعود، وعن ناس من أصحاب النبي ﷺ قالوا: إنَّ اللَّه كان عرشُه على الماء، ولم يخلق شيئًا غير ما خلق قبل الماء.

وقال أبو حجر العسقلاني في كتاب فتح البارئ بشرح صحيح البخاري \href{https://shamela.ws/book/1673/3483#p2}{\faExternalLink} تعليقا على (\fullautoref{sec:app_first_creation_hadith_2}): 

قَوْلُهُ: (كَانَ اللَّهُ وَلَمْ يَكُنْ شَيْءٌ غَيْرُهُ) [.] وَفِيهِ دَلَالَةٌ عَلَى أَنَّهُ لَمْ يَكُنْ شَيْءٌ غَيْرُهُ لَا الْمَاءُ وَلَا الْعَرْشُ وَلَا غَيْرُهُمَا؛ لِأَنَّ كُلَّ ذَلِكَ غَيْرُ اللَّهِ تَعَالَى، وَيَكُونُ قَولهُ وَكَانَ عَرْشُهُ عَلَى الْمَاءِ مَعْنَاهُ أَنَّهُ خَلَقَ الْمَاءَ سَابِقًا ثُمَّ خَلَقَ الْعَرْشَ عَلَى الْمَاءِ، وَقَدْ وَقَعَ فِي قِصَّةِ نَافِعِ بْنِ زَيْدٍ الْحِمْيَرِيِّ بِلَفْظِ: كَانَ عَرْشُهُ عَلَى الْمَاءِ ثُمَّ خَلَقَ الْقَلَمَ، فَقَالَ: اكْتُبْ مَا هُوَ كَائِنٌ، ثُمَّ خَلَقَ السَّمَاوَاتِ وَالْأَرْضَ وَمَا فِيهِنَّ فَصَرَّحَ بِتَرْتِيبِ الْمَخْلُوقَاتِ بَعْدَ الْمَاءِ وَالْعَرْشِ.

قَوْلُهُ: (وَكَانَ عَرْشُهُ عَلَى الْمَاءِ، وَكَتَبَ فِي الذِّكْرِ كُلَّ شَيْءٍ، وَخَلَقَ السَّمَاوَاتِ وَالْأَرْضَ) هَكَذَا جَاءَتْ هَذِهِ الْأُمُورُ الثَّلَاثَةُ مَعْطُوفَةً بِالْوَاوِ، وَوَقَعَ فِي الرِّوَايَةِ الَّتِي فِي التَّوْحِيدِ: ثُمَّ خَلَقَ السَّمَاوَاتِ وَالْأَرْضَ وَلَمْ يَقَعْ بِلَفْظِ ثُمَّ إِلَّا فِي ذِكْرِ خَلْقِ السَّمَاوَاتِ وَالْأَرْضِ، وَقَدْ رَوَى مُسْلِمٌ مِنْ حَدِيثِ عَبْدِ اللَّهِ بْنِ عَمْرٍو مَرْفُوعًا: (أَنَّ اللَّهَ قَدَّرَ مَقَادِيرَ الْخَلَائِقِ قَبْلَ أَنْ يَخْلُقَ السَّمَاوَاتِ وَالْأَرْضَ بِخَمْسِينَ أَلْفَ سَنَةٍ وَكَانَ عَرْشُهُ عَلَى الْمَاءِ) (\fullautoref{sec:app_first_creation_hadith_1}) وَهَذَا الْحَدِيثُ يُؤَيِّدُ رِوَايَةَ مَنْ رَوَى: ثُمَّ خَلَقَ السَّمَاوَاتِ وَالْأَرْضَ بِاللَّفْظِ الدَّالِّ عَلَى التَّرْتِيبِ.

قَوْلُهُ: (وَكَانَ عَرْشُهُ عَلَى الْمَاءِ) قَالَ الطِّيبِيُّ: هُوَ فَصْلٌ مُسْتَقِلٌّ؛ لِأَنَّ الْقَدِيمَ مَنْ لَمْ يَسْبِقْهُ شَيْءٌ، وَلَمْ يُعَارِضْهُ فِي الْأَوَّلِيَّةِ، لَكِنْ أَشَارَ بِقَوْلِهِ: وَكَانَ عَرْشُهُ عَلَى الْمَاءِ إِلَى أَنَّ الْمَاءَ وَالْعَرْشَ كَانَا مَبْدَأُ هَذَا الْعَالَمِ لِكَوْنِهِمَا خُلِقَا قَبْلَ خَلْقِ السَّمَاوَاتِ وَالْأَرْضِ، وَلَمْ يَكُنْ تَحْتَ الْعَرْشِ إِذْ ذَاكَ إِلَّا الْمَاءُ. وَمُحَصَّلُ الْحَدِيثِ أَنَّ مُطْلَقَ قَوْلِهِ وَكَانَ عَرْشُهُ عَلَى الْمَاءِ مُقَيَّدٌ بِقَوْلِهِ وَلَمْ يَكُنْ شَيْءٌ غَيْرُهُ، وَالْمُرَادُ بـ كَانَ فِي الْأَوَّلِ الْأَزَلِيَّةَ وَفِي الثَّانِيِ الْحُدُوثَ بَعْدَ الْعَدَمِ. وَقَدْ رَوَى أَحْمَدُ، وَالتِّرْمِذِيُّ وَصَحَّحَهُ مِنْ حَدِيثِ أَبِي رَزِينٍ الْعُقَيْلِيِّ مَرْفُوعًا: أَنَّ الْمَاءَ خُلِقَ قَبْلَ الْعَرْشِ، وَرَوَى السُّدِّيُّ فِي تَفْسِيرِهِ بِأَسَانِيدَ مُتَعَدِّدَةٍ: أَنَّ اللَّهَ لَمْ يَخْلُقْ شَيْئًا مِمَّا خَلَقَ قَبْلَ الْمَاءِ، وَأَمَّا مَا رَوَاهُ أَحْمَدُ، وَالتِّرْمِذِيُّ وَصَحَّحَهُ مِنْ حَدِيثِ عُبَادَةَ بْنِ الصَّامِتِ مَرْفُوعًا: (أَوَّلُ مَا خَلَقَ اللَّهُ الْقَلَمَ، ثُمَّ قَالَ: اكْتُبْ، فَجَرَى بِمَا هُوَ كَائِنٌ إِلَى يَوْمِ الْقِيَامَةِ) (\fullautoref{sec:app_first_creation_hadith_3}) فَيُجْمَعُ بَيْنَهُ وَبَيْنَ مَا قَبْلَهُ بِأَنَّ أَوَّلِيَّةَ الْقَلَمِ بِالنِّسْبَةِ إِلَى مَا عَدَا الْمَاءَ وَالْعَرْشَ أَوْ بِالنِّسْبَةِ إِلَى مَا مِنْهُ صَدَرَ مِنَ الْكِتَابَةِ، أَيْ أَنَّهُ قِيلَ لَهُ اكْتُبْ أَوَّلَ مَا خُلِقَ [.].

وَحَكَى أَبُو الْعَلَاءِ الْهَمْدَانِيُّ أَنَّ لِلْعُلَمَاءِ قَوْلَيْنِ فِي أَيِّهِمَا خُلِقَ أَوَّلًا الْعَرْشُ أَوِ الْقَلَمُ؟ قَالَ: وَالْأَكْثَرُ عَلَى سَبْقِ خَلْقِ الْعَرْشِ، وَاخْتَارَ ابْنُ جَرِيرٍ وَمَنْ تَبِعَهُ الثَّانِي، وَرَوَى ابْنُ أَبِي حَازِمٍ مِنْ طَرِيقِ سَعِيدِ بْنِ جُبَيْرٍ، عَنِ ابْنِ عَبَّاسٍ قَالَ: خَلَقَ اللَّهُ اللَّوْحَ الْمَحْفُوظَ مَسِيرَةَ خَمْسِمِائَةِ عَامٍ، فَقَالَ لِلْقَلَمِ قَبْلَ أَنْ يَخْلُقَ الْخَلْقَ وَهُوَ عَلَى الْعَرْشِ: اكْتُبْ، فَقَالَ: وَمَا أَكْتُبُ؟ قَالَ: عِلْمِي فِي خَلْقِي إِلَى يَوْمِ الْقِيَامَةِ، ذَكَرَهُ فِي تَفْسِيرِ سُورَةِ سُبْحَانَ، وَلَيْسَ فِيهِ سَبْقُ خَلْقِ الْقَلَمِ عَلَى الْعَرْشِ، بَلْ فِيهِ سَبْقُ الْعَرْشِ. وَأَخْرَجَ الْبَيْهَقِيُّ فِي الْأَسْمَاءِ وَالصِّفَاتِ مِنْ طَرِيقِ الْأَعْمَشِ، عَنْ أَبِي ظَبْيَانَ، عَنِ ابْنِ عَبَّاسٍ قَالَ: أَوَّلُ مَا خَلَقَ اللَّهُ الْقَلَمَ، فَقَالَ لَهُ: اكْتُبْ، فَقَالَ: يَا رَبِّ وَمَا أَكْتُبُ؟ قَالَ: اكْتُبِ الْقَدَرَ، فَجَرَى بِمَا هُوَ كَائِنٌ مِنْ ذَلِكَ الْيَوْمِ إِلَى قِيَامِ السَّاعَةِ (\fullautoref{sec:app_first_creation_hadith_3}) وَأَخْرَجَ سَعِيدُ بْنُ مَنْصُورٍ، عَنْ أَبِي عَوَانَةَ، عَنْ أَبِي بِشْرٍ عَنْ مُجَاهِدٍ قَالَ: بَدْءُ الْخَلْقِ الْعَرْشُ وَالْمَاءُ وَالْهَوَاءُ، وَخُلِقَتِ الْأَرْضُ مِنَ الْمَاءِ وَالْجَمْعُ بَيْنَ هَذِهِ الْآثَارِ وَاضِحٌ.

قَوْلُهُ: (وَكَتَبَ) أَيْ قَدَّرَ (فِي الذِّكْرِ) أَيْ فِي مَحَلِّ الذِّكْرِ، أَيْ فِي اللَّوْحِ الْمَحْفُوظِ (كُلَّ شَيْءٍ) أَيْ مِنَ الْكَائِنَاتِ، وَفِي الْحَدِيثِ جَوَازُ السُّؤَالِ عَنْ مَبْدَإِ الْأَشْيَاءِ وَالْبَحْثُ عَنْ ذَلِكَ وَجَوَازُ جَوَابِ الْعَالِمِ بِمَا يَسْتَحْضِرُهُ مِنْ ذَلِكَ، وَعَلَيْهِ الْكَفُّ إِنْ خَشِيَ عَلَى السَّائِلِ مَا يَدْخُلُ عَلَى مُعْتَقِدِهِ. وَفِيهِ أَنَّ جِنْسَ الزَّمَانِ وَنَوْعَهُ حَادِثٌ، وَأَنَّ اللَّهَ أَوْجَدَ هَذِهِ الْمَخْلُوقَاتِ بَعْدَ أَنْ لَمْ تَكُنْ، لَا عَنْ عَجْزٍ عَنْ ذَلِكَ بَلْ مَعَ الْقُدْرَةِ.

\section{مسألة يدين الله}

يَطْوِي اللَّهُ عزَّ وجلَّ السَّمَواتِ يَومَ القِيامَةِ، ثُمَّ يَأْخُذُهُنَّ بيَدِهِ اليُمْنَى، ثُمَّ يقولُ: أنا المَلِكُ، أيْنَ الجَبَّارُونَ؟ أيْنَ المُتَكَبِّرُونَ. ثُمَّ يَطْوِي الأرَضِينَ بشِمالِهِ، ثُمَّ يقولُ: أنا المَلِكُ أيْنَ الجَبَّارُونَ؟ أيْنَ المُتَكَبِّرُونَ؟
صحيح مسلم

قال الشيخ ابن باز رحمه الله مجيبا على: ما معنى حديث "وكلتا يدي الرحمن يمين"؟

الحديث ثابتٌ، ورواه مسلم، ومسلم رحمه الله توخَّى الأحاديث الصَّحيحة، وإذا كان جرح عمر بن حمزة بعض الناس فمُسلم لم يجرحه، وروى عنه، ووثَّقه ابنُ حبان، وصحح له الحاكم.
فالمقصود أن الحديث لا بأس به، وهي شمال في الاسم، وأما في الفضل فهي يمين، ولهذا في الحديث الصحيح: كلتا يدي ربي يمين مباركة، فكلاهما يمين مباركة في الشرف والفضل، وتُسمَّى إحداهما: يمينًا، كما قال تعالى: وَالسَّمَاوَاتُ مَطْوِيَّاتٌ بِيَمِينِهِ [الزمر:67]، وتُسمَّى الأخرى: شمالًا، وهي يمينٌ في الفضل والبركة والشرف، وإن سُمِّيَتْ شمالًا، لكنها في الفضل والشرف لها ما لليمين باليُمن والخير والبركة والشرف، ولا منافاة، فالحديث كلتا يدي ربي يمين مباركة، يُبين فضلها وشرفها، وأنه لا نقصَ فيها، والتَّسمية بتسميتها شمالًا لا يدل على النقص، بل إنما هي مجرد أسماء فقط، كما أن تسمية يده: يد، وتسميته قدمه: قدم، وعين، وسمع، وبصر، كل هذا لا يتضمن المشابهة والتَّمثيل، فكلها صفات تليق بالله، وكلها كاملة، ليس فيها نقصٌ، تليق بالله جلَّ وعلا، لا يُماثل فيها خلقه .

\section{مسألة أثقل المخلوقات}

قال شيخ الإسلام ابن تيمية في مجموع الفتاوى \href{https://shamela.ws/book/7289/2991#p1}{\faExternalLink}:

وَقَدْ ثَبَتَ فِي صَحِيحِ مُسْلِمٍ {عَنْ جُوَيْرِيَّةَ بِنْتِ الْحَارِثِ: أَنَّ النَّبِيَّ صَلَّى اللَّهُ عَلَيْهِ وَسَلَّمَ دَخَلَ عَلَيْهَا وَكَانَتْ تُسَبِّحُ بِالْحَصَى مِنْ صَلَاةِ الصُّبْحِ إلَى وَقْتِ الضُّحَى فَقَالَ: لَقَدْ قُلْت بَعْدَك أَرْبَعَ كَلِمَاتٍ لَوْ وُزِنَتْ بِمَا قلتيه لَوَزَنَتْهُنَّ: سُبْحَانَ اللَّهِ عَدَدَ خَلْقِهِ سُبْحَانَ اللَّهِ زِنَةَ عَرْشِهِ سُبْحَانَ اللَّهِ رِضَى نَفْسِهِ سُبْحَانَ اللَّهِ مِدَادَ كَلِمَاتِهِ} . فَهَذَا يُبَيِّنُ أَنَّ زِنَةَ الْعَرْشِ أَثْقَلُ الْأَوْزَانِ.

\section{مسألة تفاوت الزمان}


وفي تفاوت الزمان، يقول شيخ الإسلام ابن تيمية:

وَالرُّسُلُ أَخْبَرَتْ بِخَلْقِ الْأَفْلَاكِ وَخَلْقِ الزَّمَانِ الَّذِي هُوَ مِقْدَارُ حَرَكَتِهَا (أي حركة الأفلاك) مَعَ إخْبَارِهَا بِأَنَّهَا خُلِقَتْ مِنْ مَادَّةٍ قَبْلَ ذَلِكَ وَفِي زَمَانٍ قَبْلَ هَذَا الزَّمَانِ؛ فَإِنَّهُ سُبْحَانَهُ أَخْبَرَ أَنَّهُ خَلَقَ السَّمَوَاتِ وَالْأَرْضَ فِي سِتَّةِ أَيَّامٍ وَسَوَاءٍ قِيلَ: أَنَّ تِلْكَ الْأَيَّامَ بِمِقْدَارِ هَذِهِ الْأَيَّامِ الْمُقَدَّرَةِ بِطُلُوعِ الشَّمْسِ وَغُرُوبِهَا؛ أَوْ قِيلَ: إنَّهَا أَكْبَرُ مِنْهَا كَمَا قَالَ بَعْضُهُمْ: إنَّ كُلَّ يَوْمٍ قَدْرُهُ أَلْفُ سَنَةٍ فَلَا رَيْبَ أَنَّ تِلْكَ الْأَيَّامَ الَّتِي خُلِقَتْ فِيهَا السَّمَوَاتُ وَالْأَرْضُ غَيْرُ هَذِهِ الْأَيَّامِ وَغَيْرُ الزَّمَانِ الَّذِي هُوَ مِقْدَارُ حَرَكَةِ هَذِهِ الْأَفْلَاكِ. وَتِلْكَ الْأَيَّامُ مُقَدَّرَةٌ بِحَرَكَةِ أَجْسَامٍ مَوْجُودَةٍ قَبْلَ خَلْقِ السَّمَوَاتِ وَالْأَرْضِ. وَقَدْ أَخْبَرَ سُبْحَانَهُ أَنَّهُ {اسْتَوَى إلَى السَّمَاءِ وَهِيَ دُخَانٌ فَقَالَ لَهَا وَلِلْأَرْضِ ائْتِيَا طَوْعًا أَوْ كَرْهًا قَالَتَا أَتَيْنَا طَائِعِينَ} فَخُلِقَتْ مِنْ الدُّخَانِ وَقَدْ جَاءَتْ الْآثَارُ عَنْ السَّلَفِ إنَّهَا خُلِقَتْ مِنْ بُخَارِ الْمَاءِ؛ وَهُوَ الْمَاءُ الَّذِي كَانَ الْعَرْشُ عَلَيْهِ الْمَذْكُورُ فِي قَوْلِهِ: {وَهُوَ الَّذِي خَلَقَ السَّمَاوَاتِ وَالْأَرْضَ فِي سِتَّةِ أَيَّامٍ وَكَانَ عَرْشُهُ عَلَى الْمَاءِ}.

\section{مسألة فناء النار}

راجع كافة التفاسير

\quranayah*[11][106-108]{\footnotesize \surahname*[11]}.


\section{مسألة العدل مع الكفار}
\label{sec:app_justice}

الدولة الكافرة العادلة لها وعليها، فيذم كفرها ويحمد عدلها، ولا يرد عليها كل أمرها، بل يحمد ما فيها من العدل والإنصاف والمحاسن الإنسانية الموافقة للفطرة، ويذم ما فيها من كفر وفسق وعدوان على دين الله ورسله وهذا ما أوصانا به جل جلاله في كتابه العظيم فقال: 
\quranayah*[5][8]{\footnotesize \surahname*[5]}. وقال القرطبي في تفسيره: ودلت الآية أيضا على أن كفر الكافر لا يمنع من العدل عليه. وقال ابن كثير في تفسيره: وقوله: (ولا يجرمنكم شنآن قوم على ألا تعدلوا) أي: لا يحملنكم بغض قوم على ترك العدل فيهم، بل استعملوا العدل في كل أحد، صديقا كان أو عدوا [هـ]. 


وقول شهادة الحق في الدولة الكافرة لا يعني موالاتها وإن كانت عادلة، بل هذا ما هوا إلا شهادة الحق وقد تقدم بيان ذم ما فيها من كفر وفسق وعصيان لدين الله ورسله. وهذا لأن الله جل جلاله أمرنا بالعدل في القول ولو على أنفسنا فقال جل في علاه:
\quranayah*[4][135]{\footnotesize \surahname*[4]}. وقد جاء في تفسير ابن كثير: وقوله (فلا تتبعوا الهوى أن تعدلوا) أي : فلا يحملنكم الهوى والعصبية وبغضة الناس إليكم، على ترك العدل في أموركم وشؤونكم، بل الزموا العدل على أي حال كان، كما قال تعالى : (ولا يجرمنكم شنآن قوم على ألا تعدلوا اعدلوا هو أقرب للتقوى) [المائدة: 8] [هـ]. ومن ذلك ما صح عن جابِرُ بنُ عبدِ اللهِ رضِيَ اللهُ عنهما أنه قال: أفاءَ اللهُ عزَّ وجلَّ خَيبرَ على رسولِ اللهِ صلَّى اللهُ عليه وسَلَّم، فأقَرَّهُم رسولُ اللهِ صلَّى اللهُ عليه وسَلَّم كما كانوا، وجَعَلَها بَينَه وبَينَهُم، فبَعَثَ عَبدَ اللهِ بنَ رَواحةَ، فخَرَصَها عليهم، ثُمَّ قال لهم: يا مَعشَرَ اليَهودِ، أنتُم أبغَضُ الخَلْقِ إليَّ، قتَلتُم أنبياءَ اللهِ عزَّ وجلَّ، وكَذَبتُم على اللهِ، وليس يَحمِلُني بُغْضي إيَّاكم على أنْ أَحيفَ عليكم، قد خَرَصتُ عِشرينَ ألْفَ وَسْقٍ مِن تَمرٍ، فإنْ شِئتُم فلكُم، وإنْ أبَيتُم فلي، فقالوا: بهذا قامَتِ السَّمَواتُ والأرضُ، قد أخَذْنا، فاخْرُجوا عنَّا.{\footnotesize (صحيح على شرط مسلم، تخريج المسند لشعيب، تخريج سنن الدارقطني)}. وهذا فيه أن اليهود عرفوا أنه بالعدل قامت السموات والأرض وهذا ما سبق بيانه في الميزان الكوني، وأن عبد الله بن رواحة رضي الله عنه أقام فيهم الميزان الشرعي وأقر لهم بذلك بعدله معهم. 

وقد جاء في تفسير الطبري عن ابن عباس قوله: "كونوا قوامين بالقسط شهداء لله ولو على أنفسكم أو الوالدين والأقربين"، قال: أمر الله المؤمنين أن يقولوا الحقَّ ولو على أنفسهم أو آبائهم أو أبنائهم، ولا يحابوا غنيًّا لغناه، ولا يرحموا مسكينًا لمسكنته، وذلك قوله: "إن يكن غنيًّا أو فقيرًا فالله أولى بهما فلا تتبعوا الهوى أن تعدلوا "، فتذروا الحق، فتجوروا [هـ]. وأيضا جاء في تفسير الطبري: حدثنا سعيد، عن قتادة: "يا أيها الذين آمنوا كونوا قوامين بالقسط شهداء لله" الآية، هذا في الشهادة. فأقم الشهادة، يا ابن آدم، ولو على نفسك، أو الوالدين، أو على ذوي قرابتك، أو شَرَفِ قومك. فإنما الشهادة لله وليست للناس، وإن الله رضي العدل لنفسه، والإقساط والعدل ميزانُ الله في الأرض، به يردُّ الله من الشديد على الضعيف، ومن الكاذب على الصادق، ومن المبطل على المحق. وبالعدل يصدِّق الصادقَ، ويكذِّب الكاذبَ، ويردُّ المعتدي ويُرَنِّخُه، تعالى ربنا وتبارك. وبالعدل يصلح الناس، يا ابن آدم "إن يكن غنيًّا أو فقيرًا فالله أولى بهما"، يقول: أولى بغنيكم وفقيركم. قال: وذكر لنا أن نبيَّ الله موسى عليه السلام قال: "يا ربِّ، أي شيء وضعت في الأرض أقلّ؟"، قال: " العدلُ أقلُّ ما وضعت في الأرض". فلا يمنعك غِنى غنيّ ولا فقر فقير أن تشهد عليه بما تعلم، فإن ذلك عليك من الحق، وقال جل ثناؤه: " فالله أولى بهما " [هـ].



\href{https://shamela.ws/book/592/187#p5}{\faExternalLink} \cite{bukhari}.\footnote{صحيح البخاري: 3199، أورده الألباني في صحيح السيرة النبوية وفي السلسلة الصحيحة، وصححه الأرناؤوط في تخريج سير أعلام النبلاء}


وقال ابن إسحاق: حدثني الزهري عن أبي بكر بن عبد الرحمن بن حارث بن هشام عن أم سلمة رضي الله عن ها قالت:



لما ضاقت (مكة) وأوذي أصحاب رسول الله صلى الله عليه وسلم وفتنوا ورأوا ما يصيبهم من البلاء والفتنة في دينهم وأن رسول الله صلى الله عليه وسلم لا يستطيع دفع ذلك عنهم وكان رسول الله صلى الله عليه وسلم في منعة من قومه ومن عمه لا يصل إليه شيء مما يكره ومما ينال أصحابه فقال لهم رسول الله صلى الله عليه وسلم: (إن بأرض الحبشة ملكا لا يظلم أحد عنده فالحقوا ببلاده حتى يجعل الله لكم فرجا ومخرجا مما أنتم فيه) فخرجنا إليها أرسالا حتى اجتمعنا بها فنزلنا بخير دار إلى خير جار آمنين على ديننا ولم نخش فيها ظلما


لمَّا ضاقتْ علينا مكَّةُ، وأُوذيَ أصحابُ رسولِ اللهِ صلَّى اللهُ عليه وسلَّمَ وفُتِنوا، ورأَوْا ما يُصيبُهم مِن البَلاءِ، وأنَّ رسولَ اللهِ لا يَستطيعُ دَفْعَ ذلك عنهم، وكان هو في مَنَعةٍ مِن قَومِه وعَمِّه، لا يَصِلُ إليه شيءٌ ممَّا يَكرَهُ ممَّا يَنالُ أصحابَه. فقال لهم رسولُ اللهِ صلَّى اللهُ عليه وسلَّمَ: إنَّ بأرضِ الحَبَشةِ ملِكًا لا يُظلَمُ أحدٌ عندَه؛ فالحَقوا ببِلادِه حتى يَجعَلَ اللهُ لكم فَرَجًا ومَخرجًا. فخرَجْنا إليه أرسالًا ، حتى اجتَمَعْنا، فنزَلْنا بخيرِ دارٍ إلى خيرِ جارٍ، أمِنَّا على دِينِنا.

\section{مسألة الخروج على ولي أمر المسلمين}
\label{sec:app_rebellion}

إن من المسائل المهمة لأمة الإسلام بالعموم هي مسئلة الخروج على ولي أمر المسلمين. فهذه مسألة خطيرة وعظيمة يجب ألا يتكلم فيها إلا بعلم. 
وقد نهى النبي ﷺ عن الخروج على الدولة المسلمة الظالمة وبالأخص لما يترتب على ذلك من ظلم الذي يخالف الميزان الفطري والذي به يكون فساد المصالح العامة في الدنيا كسفك الدماء ونهب الأموال وهتك الأعراض، والتي هي أشد ظلما في الدنيا من الظلم الذي يكون بمخالفة الميزان الديني كمنع الزكاة أو الحكم بغير ما أنزل الله من باب الهوى. وقد تقدم معنا أن الله جل جلاله قدم في الدنيا إقامة الميزان بين الناس بالعدل على إقامة الحق في نفوس الناس لتقديم المصلحة العامة على الخاصة. ولذلك فقد نهى النبي ﷺ عن الخروج على ولاة الأمر المسلمين ولو كانوا ظالمين وعاصين لله ولرسوله  ما أقاموا فينا الصلاة وكفى بنا أن نبغضهم في الله على ما عصوا به الله ورسوله كما جاء عن عوف بن مالك الأشجعي أن النبي ﷺ قال: خِيارُ أئِمَّتِكُمُ الَّذِينَ تُحِبُّونَهُمْ ويُحِبُّونَكُمْ، ويُصَلُّونَ علَيْكُم وتُصَلُّونَ عليهم، وشِرارُ أئِمَّتِكُمُ الَّذِينَ تُبْغِضُونَهُمْ ويُبْغِضُونَكُمْ، وتَلْعَنُونَهُمْ ويَلْعَنُونَكُمْ، قيلَ: يا رَسولَ اللهِ، أفَلا نُنابِذُهُمْ بالسَّيْفِ؟ فقالَ: لا، ما أقامُوا فِيكُمُ الصَّلاةَ، وإذا رَأَيْتُمْ مِن وُلاتِكُمْ شيئًا تَكْرَهُونَهُ، فاكْرَهُوا عَمَلَهُ، ولا تَنْزِعُوا يَدًا مِن طاعَةٍ ، وفي رواية أخرى، قالَ: لا، ما أقامُوا فِيكُمُ الصَّلاةَ، لا، ما أقامُوا فِيكُمُ الصَّلاةَ، ألا مَن ولِيَ عليه والٍ، فَرَآهُ يَأْتي شيئًا مِن مَعْصِيَةِ اللهِ، فَلْيَكْرَهْ ما يَأْتي مِن مَعْصِيَةِ اللهِ، ولا يَنْزِعَنَّ يَدًا مِن طاعَةٍ {\footnotesize (صحيح مسلم، وصححه الألباني في تخريج كتاب السنة)}. 


وعن حذيفة بن اليمان رضي الله أن النبي ﷺ قال: يكونُ بَعْدِي أَئِمَّةٌ لا يَهْتَدُونَ بهُدَايَ، وَلَا يَسْتَنُّونَ بسُنَّتِي، وَسَيَقُومُ فيهم رِجَالٌ قُلُوبُهُمْ قُلُوبُ الشَّيَاطِينِ في جُثْمَانِ إنْسٍ، قُلتُ: كيفَ أَصْنَعُ يا رَسولَ اللهِ، إنْ أَدْرَكْتُ ذلكَ؟ قالَ: تَسْمَعُ وَتُطِيعُ لِلأَمِيرِ، وإنْ ضُرِبَ ظَهْرُكَ، وَأُخِذَ مَالُكَ، فَاسْمَعْ وَأَطِعْ {\footnotesize (صحيح مسلم)}. وقد أوصى بذلك النبي ﷺ في حجة الوداع فعن أم الحصين الأحمسية أنها قالت: سمعتُ رسولَ اللَّهِ صلَّى اللَّهُ عليْهِ وسلَّمَ يخطبُ في حجَّةِ الوداعِ  \comment{، وعليْهِ بردٌ قدِ التفعَ بِهِ من تحتِ إبطِهِ قالت فأنا أنظرُ إلى عضلةِ عضدِهِ تزتَجُّ، سمعتُهُ} يقولُ: يا أيُّها النَّاسُ اتَّقوا اللَّهَ وإن أمِّرَ عليْكم عبدٌ حبشيٌّ مجدَّعٌ فاسمعوا لَهُ وأطيعوا ما أقامَ لَكم كتابَ اللَّهِ  {\footnotesize (صحيح الترمذي، وصححه الألباني)}. وأيضا حديث العرباض بن سارية رضي الله عنه أنه قال: وعظَنا رسولُ اللَّهِ صلَّى اللَّهُ عليْهِ وسلَّمَ يومًا بعدَ صلاةِ الغداةِ موعِظةً بليغةً ذرِفَت منْها العيونُ ووجِلَت منْها القلوبُ، فقالَ رجلٌ إنَّ هذِهِ موعظةُ مودِّعٍ فماذا تعْهدُ إلينا يا رسولَ اللَّهِ، قالَ أوصيكم بتقوى اللَّهِ والسَّمعِ والطَّاعةِ وإن عبدٌ حبشيٌّ فإنَّهُ من يعِش منْكم يرَ اختلافًا كثيرًا وإيَّاكم ومحدَثاتِ الأمورِ فإنَّها ضَلالةٌ فمن أدرَكَ ذلِكَ منْكم فعليْهِ بِسُنَّتي وسنَّةِ الخلفاءِ الرَّاشدينَ المَهديِّينَ عضُّوا عليْها بالنَّواجذِ {\footnotesize (صحيح الترمذي، وصححه الألباني)}.

فالأدلة في نهي الرسول ﷺ على الخروج على الولاة العصاة كثيرة جدا، فلا يسعنا الخروج على ولاة الأمر الظالمين والعاصين لله ورسوله ليس مجاملة أو حبا لهم ولا مداهنة في دين الله وإنما إلتزاما بأمر النبي ﷺ حقنا للدماء وتقديما للمصلحة العامة على الخاصة، ولكن نبغضهم في الله على ما عصوا به الله ورسوله ولا نصدقهم ولا نعينهم على ظلمهم كما صح ذلك عن النبي ﷺ أنه قال لكَعبِ بنِ عُجْرةَ: أعاذَكَ اللهُ من إمارةِ السُّفهاءِ، قال: وما إمارةُ السُّفهاءِ؟ قال: أُمراءُ يكونونَ بَعْدي، لا يَقتَدونَ بهَدْيي، ولا يَستَنُّونَ بسُنَّتي، فمَن صدَّقَهم بكذِبِهم، وأعانَهم على ظُلْمِهم، فأولئك ليسوا منِّي، ولستُ منهم، ولا يَرِدوا عليَّ حَوْضي، ومَن لم يُصدِّقْهم بكذِبِهم، ولم يُعِنْهم على ظُلْمِهم، فأولئك منِّي وأنا منهم، وسيَرِدوا عليَّ حَوْضي {\footnotesize (صحيح ابن حبان)}. ويكفي ولي الأمر المسلم الظالم ذلا وخسرانا أن أن النبي ﷺ قد تبرأ منه كما جاء في حديث سعد بن تميم أنه قيل: يا رسولَ اللهِ، ما للخليفةِ مِن بعدِك؟ قال: مِثلُ الذي لي، ما عدَلَ في الحُكمِ، وقسَطَ في القِسطِ، ورَحِمَ ذا الرَّحِمِ، فمَن فعَلَ غيرَ ذلك فليس منِّي ولستُ منه {\footnotesize (صحيح، تخريج سنن أبي داود، وصححه الألباني)}.


\comment{
فلا يسعنا الخروج على ولاة الأمر وإن ظلموا ولكن لا نصدقهم ولا نعينهم على ظلمهم كما أخبر بذلك جابر بن عبد الله أن النبي ﷺ قال لكَعبِ بنِ عُجْرةَ: أعاذَكَ اللهُ من إمارةِ السُّفهاءِ، قال: وما إمارةُ السُّفهاءِ؟ قال: أُمراءُ يكونونَ بَعْدي، لا يَقتَدونَ بهَدْيي، ولا يَستَنُّونَ بسُنَّتي، فمَن صدَّقَهم بكذِبِهم، وأعانَهم على ظُلْمِهم، فأولئك ليسوا منِّي، ولستُ منهم، ولا يَرِدوا عليَّ حَوْضي، ومَن لم يُصدِّقْهم بكذِبِهم، ولم يُعِنْهم على ظُلْمِهم، فأولئك منِّي وأنا منهم، وسيَرِدوا عليَّ حَوْضي، يا كَعبُ بنَ عُجْرةَ، الصومُ جُنَّةٌ، والصَّدقةُ تُطفِئُ الخَطيئةُ، والصَّلاةُ قُربانٌ -أو قال: بُرهانٌ- يا كَعبَ بنَ عُجْرةَ، إنَّه لا يدخُلُ الجَنَّةَ لَحمٌ نبَتَ من سُحتٍ، النَّارُ أَوْلى به، يا كَعبُ بنَ عُجْرةَ، النَّاسُ غَاديانِ: فمُبتاعٌ نفْسَه فمُعتِقُها، وبائعٌ نفْسَه فموبِقُها {\footnotesize (صحيح ابن حبان)}. \comment{{\footnotesize (صحيح ابن حبان، حسنه الوادعي في الصحيح المسند، والمنذري في الترغيب والترهيب، وقال شعيب الأرناؤوط إسناده قوي على شرط مسلم ف تخريج المسند لشعيب، وقال غيرهم رجاله رجال الصحيح)}. }
}
ويفرق بين النصح لولي الأمر الظالم وبين إنكار المنكر بالعموم، فإنكار المنكر بالعموم واجب على كل مسلم، وبالأخص رد الظالمين لمن استطاع أن يغير ويصلح بدون أن يترتب على ذلك مفسدة أعظم، فقد جاء عن أبوبكر الصديق أنه قال بعد أن حمِد اللهَ وأثنَى عليه: يا أيُّها النَّاسُ، إنَّكم تقرءون هذه الآيةَ، وتضعونها على غيرِ موضعِها (عَلَيْكُمْ أَنْفُسَكُمْ لَا يَضُرُّكُمْ مَنْ ضَلَّ إِذَا اهْتَدَيْتُمْ)، وإنَّا سمِعنا النَّبيَّ صلَّى اللهُ عليه وسلَّم يقولُ: إنَّ النَّاسَ إذا رأَوُا الظَّالمَ فلم يأخُذوا على يدَيْه أوشك أن يعُمَّهم اللهُ بعقابٍ وإنِّي سمِعتُ رسولَ اللهِ صلَّى اللهُ عليه وسلَّم يقولُ: ما من قومٍ يُعمَلُ فيهم بالمعاصي، ثمَّ يقدِرون على أن يُغيِّروا، ثمَّ لا يُغيِّروا إلَّا يوشِكُ أن يعُمَّهم اللهُ منه بعقابٍ {\footnotesize (صحيح أبي داود، وصححه الألباني)}. ولقد بايع النبي ﷺ أصحابه على السمع والطاعة والنصح لكل مسلم كما جاء ذلك عن جرير بن عبدالله أنه قال: بايَعْتُ النبيَّ صَلَّى اللَّهُ عليه وسلَّمَ علَى السَّمْعِ والطَّاعَةِ، فَلَقَّنَنِي، فِيما اسْتَطَعْتُ، والنُّصْحِ لِكُلِّ مُسْلِمٍ {\footnotesize (صحيح مسلم)}. فالنصح يكون لكل مسلم سواء كان ولي الأمر وغير ولي الأمر ويكون بحسب الحاجة وبالحكمة والموعظة الحسنة. ومن المصلحة في أغلب الأحوال أن تكون النصيحة لولي الأمر بالسر لما قد يترتب على الجهر بها من الفتن أو التحريض. ولهذا فقد قال النبي ﷺ: مَن أرادَ أن ينصحَ لذي سلطانٍ في أمرٍ فلا يُبدِهِ عَلانيةً ولَكِن ليأخذْ بيدِهِ فيَخلوَ بهِ فإن قبِلَ منهُ فذاكَ وإلَّا كانَ قد أدَّى الَّذي علَيهِ لَهُ {\footnotesize (صححه الألباني في تخريج كتاب السنة)}. فلو كانت هذه النصيحة لسلطان ظالم فهذا من أفضل الجهاد كما جاء ذلك عن أبو سعيد الخدري أن النبي ﷺ قال: أفضَلُ الجِهادِ كلمةُ عدلٍ، وفي رواية: كلمة حق، عندَ سُلطانٍ جائرٍ {\footnotesize (صحيح ابن ماجه، وصححه الألباني)}.

وأما الخروج على ولاة الأمور فشرطه أن يكون عندهم كفرا بواحا ظاهرا لا شك فيه. فقد جاء عن عبادة بن الصامت أنه قال: دَعَانَا رَسولُ اللهِ صَلَّى اللَّهُ عليه وسلَّمَ فَبَايَعْنَاهُ، فَكانَ فِيما أَخَذَ عَلَيْنَا: أَنْ بَايَعَنَا علَى السَّمْعِ وَالطَّاعَةِ في مَنْشَطِنَا وَمَكْرَهِنَا، وَعُسْرِنَا وَيُسْرِنَا، وَأَثَرَةٍ عَلَيْنَا، وَأَنْ لا نُنَازِعَ الأمْرَ أَهْلَهُ، قالَ: إلَّا أَنْ تَرَوْا كُفْرًا بَوَاحًا عِنْدَكُمْ مِنَ اللهِ فيه بُرْهَانٌ {\footnotesize (صحيح مسلم)}. ومن المعلوم أنه ليس كل حكم بغير ما أنزل الله كفر ومن ذلك بلا شك القوانين الوضعية التي لا تعارض كتاب الله وسنة نبيه ﷺ. ومثال ذلك الأمور التي فيها مصالح الناس كالأمور التنظيمة المباحة فهذا أمر مطلوب ولازم وبه يؤجر ولي الأمر لما في ذلك من نفع عام لجميع المسلمين، كتنظيم طرق سير السيارات، وقوانين حماية البيانات، وغيرها من القوانين التي بها تحفظ الدماء، والأموال، والأعراض. والكفر البواح لا يكون بالحكم بغير ما أنزل الله مع الإقرار بالذنب دون الإعتقاد بجواز ذلك والجهر به كمن يفعل ذلك من باب الهوى. وإنما الكفر البواح هو الإعتقاد مع الجهر أن الحكم المخالف لشرع الله وكتابه هو حكم جائز على وجه التفضيل أو المساواة أو الرد أو غير ذلك. ومن ذلك من يعتقد بأفضلية حكم غير الله على حكم الله أو مساواة حكم غير الله مع حكم الله أو جواز حكم غير الله أو رد حكم الله، المخالف لشرع الله وكتابه والجهر بذلك. ولقد بين ذلك الشيخ ابن باز رحمه الله في بيان القوانين الوضعية والآراء البشرية \textbf{التي تخالف شرع الله} فقال: الحكم بغير ما أنزل الله [بالقوانين التي تخالف شرع الله] أقسام، تختلف أحكامهم بحسب اعتقادهم وأعمالهم، فمن حكم بغير ما أنزل الله يرى أن ذلك أحسن من شرع الله فهو كافر عند جميع المسلمين، وهكذا من يحكّم القوانين الوضعية بدلا من شرع الله ويرى أن ذلك جائز، ولو قال: إن تحكيم الشريعة أفضل فهو كافر لكونه استحل ما حرم الله. أما من حكم بغير ما أنزل الله اتباعا للهوى أو لرشوة أو لعداوة بينه وبين المحكوم عليه أو لأسباب أخرى وهو يعلم أنه عاص لله بذلك وأن الواجب عليه تحكيم شرع الله [وإنما خالفها فعلا لا عقيدة لهوى] فهذا يعتبر من أهل المعاصي والكبائر ويعتبر قد أتى كفرا أصغر وظلما أصغر وفسقا أصغر كما جاء هذا المعنى عن ابن عباس رضي الله عنهما وعن طاووس وجماعة من السلف الصالح وهو المعروف عند أهل العلم. والله ولي التوفيق {\footnotesize (مجموع فتاوى ومقالات الشيخ ابن باز: 4/416)}. 

\comment{
ومن ذلك ما أخبر به حذيفة بن اليمان رضي الله أن النبي ﷺ قال:
كان الناسُ يسألون رسولَ اللهِ صلَّى اللهُ عليهِ وسلَّم عن الخيرِ، وكنتُ أسألُه عن الشرِّ، مخافةَ أن يُدركني، فقلتُ يارسولَ اللهِ، إنَّا كنا في جاهليةٍ وشرٍّ، فجاءنا اللهُ بهذا الخيرِ [فنحنُ فيه]، [وجاء بك]، فهل بعد هذا الخيرِ من شرٍّ [كما كان قبلَه؟]، [قال ياحذيفةُ تعلَّمْ كتابَ اللهِ، واتَّبِعْ ما فيه، (ثلاثَ مراتٍ). قال: قلتُ: يارسولَ اللهِ أبعد هذا الشرِّ من خيرٍ؟] قال نعم، [قلتُ: فما العِصمةُ منه؟ قال: السيفُ]، قلتُ وهل بعد ذلك الشرِّ من خيرٍ؟(وفي طريقٍ: قلتُ: وهل بعدَ السيفِ بقيَّةٌ؟) قال: نعم، وفيه (وفي طريقٍ: تكونُ إمارةٌ (وفي لفظٍ: جماعةٌ) على أقذاءٍ، وهُدنةٍ على) دَخَنٍ، قال: قلتُ: وما دَخَنُه؟ قال: قومٌ، وفي طريقٍ أخرى: يكونُ بعدي أئمةٌ [يستنُّونَ بغيرِ سُنَّتِي، و] يَهدُون بغيرِ هديِي، تعرفُ منهم وتُنكرُ، [وسيقومُ فيهم رجالٌ قلوبُهم قلوبُ الشياطينِ، في جثمانِ إنسٍ] (وفي أخرى: الهُدنةُ على دَخَنٍ ما هيَ؟قال: لا ترجعُ قلوبُ أقوامٍ على الذي كانت عليه) فقلتُ: هل بعد ذلك الخيرِ من شرٍّ؟قال: نعم، [فتنةٌ عمياءُ صمَّاءُ عليها] دعاةٌ على أبوابِ جهنمَ، من أجابَهُم إليها قذفوهُ فيها فقلتُ: يا رسولَ اللهِ، صِفْهُمْ لنا؟قال: هم من جِلدَتِنا، و يتكلمون بألسِنَتِنا، قلتُ: [يا رسولَ اللهِ]، فما تأمُرني إذا أدركني ذلك؟قال: تلزمُ جماعةَ المسلمين، وإمامَهم [تسمعُ وتُطيعُ الأميرَ، وإن ضرب ظهرَك، وأخذ مالَك، فاسمع وأطِعْ] فقلتُ: فإن لم يكن لهم جماعةٌ ولا إمامٌ؟ قال: فاعتزل تلك الفِرقَ كلَّها، ولو أن تعضَّ على أصلِ شجرةٍ، حتى يُدرِكَك الموتُ وأنت على ذلك. (وفي طريقٍ) فإن تَمُتْ يا حذيفةُ وأنت عاضٌّ على جذلٍ خيرٌ لك من أن تتبعَ أحدًا منهم. (وفي أخرى) فإن رأيتَ يومئذٍ للهِ عزَّ وجلَّ في الأرضِ خليفةً، فالزَمْهُ وإن ضرب ظهرَك وأخذ مالَك، فإن لم ترَ خليفةً فاهرب [في الأرضِ] حتى يُدرِكَك الموتُ وأنت عاضٌّ على جذلِ شجرةٍ. قال: قلتُ: ثم ماذا؟ قال: ثم يخرجُ الدجالُ. قال: قلتُ: فبم يجيءُ؟ قال: بنهرٍ أو قال: ماءٍ ونارٍ فمن دخل نهرَه حُطَّ أجرُه ووجب وِزْرُه، ومن دخل نارَه وجب أجرُه، وحُطَّ وِزْرُه . [قلتُ: يا رسولَ اللهِ: فما بعد الدجالِ؟ قال: عيسى بنُ مريمَ] قال: قلتُ: ثم ماذا؟ قال: لو أنتجتَ فرسًا لم تركب فَلُوَّها حتى تقومَ الساعةُ. {\footnotesize (جمعه الألباني من عدة طرق في صحيح البخاري وأورده في السلسلة الصحيحة)}.
}


\section{مسألة التفرق في الدين }
\label{sec:app_division}




\section{مسألة تجريح الأعيان}
\label{sec:app_division}

يقول شيخ الإسلام بن تيمية: وليعلم أن المؤمن تجب موالاته، وإن ظلمك واعتدى عليك، والكافر تجب معاداته، وإن أعطاك وأحسن إليك؛ فإن الله سبحانه بعث الرسل وأنزل الكتب؛ ليكون الدين كله لله، فيكون الحب لأوليائه، والبغض لأعدائه، والإكرام لأوليائه، والإهانة لأعدائه، والثواب لأوليائه، والعقاب لأعدائه. وإذا اجتمع في الرجل الواحد خير وشر وفجور، وطاعة ومعصية، وسنة وبدعة: استحق من الموالاة والثواب بقدر ما فيه من الخير، واستحق من المعاداة والعقاب بحسب ما فيه من الشر، فيجتمع في الشخص الواحد موجبات الإكرام والإهانة، فيجتمع له من هذا وهذا، كاللص الفقير تقطع يده لسرقته، ويعطى من بيت المال ما يكفيه لحاجته. هذا هو الأصل الذي اتفق عليه أهل السنة والجماعة، وخالفهم الخوارج، والمعتزلة، ومن وافقهم عليه، فلم يجعلوا الناس لا مستحقًّا للثواب فقط، ولا مستحقًّا للعقاب فقط. 

وقال أيضًا: معلوم أنه في كل طائفة بر، وفاجر، وصديق، وزنديق، والواجب موالاة أولياء الله المتقين من جميع الأصناف، وبعض الكفار والمنافقين من جميع الأصناف، والفاسق الِملِّيُّ يعطى من الموالاة بقدر إيمانه، ويعطى من المعاداة بقدر فسقه، فإن مذهب أهل السنة والجماعة أن الفاسق الملي له الثواب والعقاب إذا لم يعف الله عنه، وإنه لا بد أن يدخل النار من الفساق من شاء الله، وإن كان لا يخلد في النار أحد من أهل الإيمان، بل يخلد فيها المنافقون كما يخلد فيها المتظاهرون بالكفر. اهـ.

وراجع للفائدة الفتوى رقم: 113503.


\section{دعاء النبي ﷺ}


اسئل الله العلي العظيم أن يجعل هذا العمل خالصا لوجهه الكريم وأن يبارك فيه ويجعله سببا لعودة أمة الإسلام إلى الطريق المستقيم، وأن يجعل دعوتنا دعوة الراسخين في العلم كما في قوله تعالى:
\quranayah*[3][7][32]\quranayah*[3][8]{\footnotesize \surahname*[3]}. اللهم اجعل دعائنا كدعاء نبينا ﷺ كما جاء عن عائشة أم المؤمنين أن النبي ﷺ كان إذَا قَامَ مِنَ اللَّيْلِ افْتَتَحَ صَلَاتَهُ: اللَّهُمَّ رَبَّ جِبْرَائِيلَ، وَمِيكَائِيلَ، وإسْرَافِيلَ، فَاطِرَ السَّمَوَاتِ وَالأرْضِ، عَالِمَ الغَيْبِ وَالشَّهَادَةِ، أَنْتَ تَحْكُمُ بيْنَ عِبَادِكَ فِيما كَانُوا فيه يَخْتَلِفُونَ، اهْدِنِي لِما اخْتُلِفَ فيه مِنَ الحَقِّ بإذْنِكَ؛ إنَّكَ تَهْدِي مَن تَشَاءُ إلى صِرَاطٍ مُسْتَقِيمٍ {\footnotesize (صحيح مسلم)}. وكما جاء أيضا عن عائشة أم المؤمنين أن النبي ﷺ علمها هذا الدعاء: 
للَّهمَّ إنِّي أسألُكَ مِنَ الخيرِ كلِّهِ عاجلِهِ وآجلِهِ، ما عَلِمْتُ منهُ وما لم أعلَمْ، وأعوذُ بِكَ منَ الشَّرِّ كلِّهِ عاجلِهِ وآجلِهِ، ما عَلِمْتُ منهُ وما لم أعلَمْ، اللَّهمَّ إنِّي أسألُكَ من خيرِ ما سألَكَ عبدُكَ ونبيُّكَ، وأعوذُ بِكَ من شرِّ ما عاذَ بِهِ عبدُكَ ونبيُّكَ، اللَّهمَّ إنِّي أسألُكَ الجنَّةَ وما قرَّبَ إليها من قَولٍ أو عملٍ، وأعوذُ بِكَ منَ النَّارِ وما قرَّبَ إليها من قولٍ أو عملٍ، وأسألُكَ أن تجعلَ كلَّ قَضاءٍ قضيتَهُ لي خيرًا {\footnotesize (صحيح ابن ماجه وصححه الألباني)}. وكما جاء عن أم المؤمنين أم سلمة أن أكثر دعاء نبينا ﷺ كان: اللهم يا مقلب القلوب ثبت قلبي على دينك {\footnotesize (صحيح الترمذي وصححه الألباني)}.  وكما جاء عن عبدالله بن عمر أن نبينا ﷺ قلَّما يقوم من مجلس حتى يدعو بهؤلاء الدعوات لأصحابه: اللهمَّ اقسِمْ لنا مِنْ خشيَتِكَ ما تحولُ بِهِ بينَنَا وبينَ معاصيكَ، ومِنْ طاعَتِكَ ما تُبَلِّغُنَا بِهِ جنتَكَ، ومِنَ اليقينِ ما تُهَوِّنُ بِهِ علَيْنَا مصائِبَ الدُّنيا، اللهمَّ متِّعْنَا بأسماعِنا، وأبصارِنا، وقوَّتِنا ما أحْيَيْتَنا، واجعلْهُ الوارِثَ مِنَّا، واجعَلْ ثَأْرَنا عَلَى مَنْ ظلَمَنا، وانصرْنا عَلَى مَنْ عادَانا، ولا تَجْعَلِ مُصِيبَتَنا في دينِنِا، ولَا تَجْعَلْ الدنيا أكبرَ هَمِّنَا، ولَا مَبْلَغَ عِلْمِنا، ولَا تُسَلِّطْ عَلَيْنا مَنْ لَا يرْحَمُنا {\footnotesize (صحيح الترمذي)}. وكما جاء عن زيد بن أرقم رضي الله عنه أنه قال: لا أُعلِّمُكم إلا ما كان رسولُ اللهِ ﷺ يُعلِّمُنا: اللَّهمَّ إني أعوذُ بك من العجزِ والكسلِ، والبخلِ والجبنِ، والهَرَمِ وعذابِ القبرِ، اللَّهمَّ آتِ نفسي تقْوَاها وزكِّها أنت خيرُ من زكَّاها أنت ولِيُّها ومولاها، اللَّهمَّ إني أعوذُ بك من قلبٍ لا يخشعُ ومن نفسٍ لا تشبعُ وعلمٍ لا ينفعُ ودعوةٌ لا يُستجابُ لها {\footnotesize (صحيح النسائي)}. وعن أنس ابن مالك أنه قال: كثيرًا ما كُنتُ أسمعُ النَّبيَّ صلَّى اللَّهُ علَيهِ وسلَّمَ يدعو بِهَؤلاءِ الكلِماتِ (وفي رواية في صحيح النسائي: لا يدعهُنَّ): اللَّهمَّ إنِّي أعوذُ بِكَ منَ الهمِّ والحزنِ والعَجزِ والكَسلِ والبُخلِ وضَلَعِ الدَّينِ وغلبةِ الرِّجالِ {\footnotesize (صحيح الترمذي، صححه الألباني)}.
