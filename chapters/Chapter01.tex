
\chapter{الحساب في زمن الحكم الرشيد}

\section{مقدمة}
كان البحث والعناية بعلم الحساب هو غاية جليلة ومهمة عظيمة أعتنى بها المسلمون اللاحقون في زمن الخليفة الراشد هارون الرشيد التي أسس دار الحكمة في بغداد العراق حتى أصبح المسلمين في ذلك الوقت روادا في علم الحساب والذى كان مفتاحا لهم لشتى العلوم الأخرى حتى عرف ذلك الزمان بالعصر الإسلامي الذهبي. ومن أبرز من بحث وألف في علم الحساب هو العالم الفذ محمد بن موسى الخوارزمي رحمه الله تعالى والذي وصل صيته أقطاب الأرض حتى دخل أسمه معاجم وقواميس كافة اللغات الأخرى. فاللوغرتميات جاءت من الترجمة اللاتينية لإسمه وهو ما عرف عند العرب المتأخرين بالخوارزميات. وهذا مفهوم يبنى عليه كافة الحسابات المركبة والمعقدة التي نراها اليوم من انظمة الحساب والمنطق بشتى أنواعها بما فيها أنظمة الصواريخ والطيران وحتى انظمة الذكاء الإصطناعي. وقد ألف الخوارزمي كتابه "المختصر في الجبر والمقابلة" وكان هذا الكتاب نافعا للمسلمين وغيرهم وهو أساس تقدم البشرية في شتى المجالا إلى يومنا هذا. ولهذا سمي علم الموازنة والمقابلة بعلم "الجبر" كما سماه الخوارزمي بذلك وتمت اضافة كلمة "الجبر" أيضا إلى كافة معاجم اللغات الأخرى. ويعتبر الخوارزمي إلى يومنا هو مؤسس علم الجبر والحساب والخوارزميات ومن أهم علماء الحساب في تاريخ البشرية.

وللأسف فقد غاب وغيب على أغلب المسلمين في زماننا هذا أهمية ميراث الخوارزمي في علم الجبر والحساب. وهو ميراث حري بنا جمع شتاته وإعادة بناء أركانه لتقوم الأمة بالميزان الذي أمرنا الله به. فقد جهل الكثير من المسليمن ميراث الخوارزمي حتى بخس قدره ونسي علمه فكان بين مفرط أو مدلس. ومن ذلك ضياع كتابه في الجبر والمقابلة من المسلمين حتى تمت طباعة أول نسحة عربية منه في عام 1939م (1357هـ) بناء على النسخة الأصلية الوحيدة التي سرقت من مصر ونقلت إلى بريطانيا والتي يرجع تاريخها إلى عام 1439م (843هـ) أي بعد وفاة الخوارزمي بحوالي 500 عام شمسية.
ليرجع لنا كتاب الخوارزمي بعد حوالي ألف عام من تأليفه.
وفي كل هذه الأعوام ترجم كتابه إلى شتى اللغات ومنها الأنجليزية والألمانية والفرنسية وأصبحت مرجعا لجميع الحضارات الأوروبية وغيرها.
ليتفاجأ المسلمين بوجود كلمات عربية في هذه الثقافات ومنها algorithms والتي تعني الخوارزميات وكلمة algebra وهي الجبر في معجم اللغة الانجليزية على سبيل المثال لا الحصر.

ومن التدليس الذي تعرض له الخوارزمي في تقديم كتابه هو نسبة عمله إلى الحضارة المصرية في طرح مخالف للطرح الذي وضعه الخوارزمي في كتابه. وهذا ليس إلا إحقاقا للحق ولا يجب أن يحمل هذا على محمل الإستنقاص لمن نقل هذا العمل لنا تقديما وتعليقا فجزاهم الله خير الجزاء. ومن التدليس أيضا طرح كتابه في الحساب مجردا من الغاية التي كتب لها ومنه عدم ذكر سبب تأليف كتابه في الجبر والذي كان في الأساس سعيا منه رحمه الله لتحقيق الحكم الرشيد بناء على الحساب الصحيح في الميراث والبيع والشراء والكراء وما بتعلق بذلك من حساب المسافات والأرض. وليتبين طرح الخوارزمي نضع مقدمة كتابه رحمه الله والتي جاء فيها:\footnote{مع تصرف يسير من حذف لكلمات التي تخالف السياق وفي الغالب قد يظن انها أخطاء خلال النسخ.}

%\begin{mdframed}
\newpage
\fbox{\small
    \begin{minipage}{34em}
        \begin{center}
            بسم الله الرحمن الرحيم
        \end{center}
        هذا كتاب وضعه محمد بن موسى الخوارزمي افتتحه بأن قال:

        الحمد الله على نعمه بما هو أهله من محامده التي بأداء ما افترض منها على من يعبده من خلقه يقع اسم الشكر ويستوجب المزيد إقرارا بروبويته وتذللا لعزته وخشوعا لعظمته. بعث محمدا صلى الله عليه وعلى آله وسلم بالنبوة على حين فترة من الرسل نورا من الحق ودروس من الهدي فبصر به من العمى واستنقذ به من الهلكة وكثر به بعد قلة وألف به بعد الشتات.

        تبارك الله ربنا وتعالى جده وتقدست أسماؤه ولا إله غيره, وصلى الله على محمد النبي وآله وسلم. ولم تزل العلماء في الأزمنة الخالية والأمم الماضية يكتبون الكتب مما يصنفون من صنوف العلم ووجوه الحكمة نظرا لمن بعدهم واحتسابا للأجر بقدر الطاقة ورجاء أن يلحقهم من أجر ذلك وذخره وذكره ويبقى لهم من لسان صدق ما يصغر في جنبه كثير مما كانوا يتكفلونه من المؤونة ويحملونه على أنفسهم من المشقة في كشف أسرار العلم وغامضه. إما رجل سبق إلى مالم يكن مستخرجا قبله فورثه من بعده. وإما رجل شرح مما أبقى الأولون ما كان مستغلقا فأوضح طريقه وسهل مسلكه وقرب مأخذه. وإما رجل وجد في بعض الكتب خللا فلم شعثه وأقام أوده وأحسن الظن بصاحبه غير راد عليه ولا مفتخر بذلك من فعل نفسه.

        وقد شجعني ما فضل الله به الامام المأمون أمير المؤمنين مع الخلافة التي حاز له إرثها وأكرمه بلباسها وحلاه بزينتها, من الرغبة في الأدب وتقريب أهله وإدنائهم وبسط كنفه لهم ومعونته إياهم على إيضاح ما كان مستبهما وتسهيل ما كان مستوعرا. على أن ألفت من كتاب الجبر والمقابلة كتابا مختصرا حاصرا للطيف الحساب وجليله لما يلزم الناس من الحاجة إليه في مواريثهم ووصياهم وفي مقاسمتهم وأحكامهم وتجارتهم, وفي جميع ما بتعاملون به بينهم من مساحة الأرضين وكرى الأنهار والهندسة وغير ذلك من وجوهه وفنونه, مقدما لحسن النية فيه وراجيا لأن ينزله أهل الأدب بفضل ما استودعوا من نعم الله تعالى وجليل آلائه وجميل بلائه عندهم منزلته وبالله توفيقي في هذا عليه توكلت وهو رب العرش العظيم وصلى الله على جميع الأنبياء والمرسلين.
        [هـ].
    \end{minipage}
}
%\end{mdframed}

وعليه يلعم أن الخوارزمي رحمه الله إنما ألف كتابه هذا لتوضيح علم الحساب الصحيح الذي يحتاج إليه الناس في أمور دينهم ودنياهم. فقد إفتتح الخوارزمي رحمه الله كتابه بالبسملة متبعا سنة الأنبياء في ذلك. وكان رحمه الله حريصا وراجيا بأن يعتنى أهل الأدب بهذا الكتاب ويعطونه حقه وينزلونه منزلته لما علم ما فيه من أسس وقواعد لا غنى عنها في علم الحساب الصحيح. وختم مقدمته سائلا الله التوفيق في ذلك ومتوكلا عليه. وبهذا يتبين حسن مقصد الخوارزمي في تأليف كتابه فنسأل الله العلي العظيم أن يرحمه رحمة واسعة وأن يرفغ قدره في الجنة وأن يجزيه عنا خير الجزاء.

فبدأو بالعناية بعلم الحساب وجمع مؤلفاته من كافة أقطاب الدنيا فعكفوا على ترجمتها حتى فهموها وعقلوها وعرفوا ما شابها من خطأ ونقصان. فأسسوا نظام الأرقام الذي نعرفه اليوم فقسموا الأرقام إلى ارقام فردية وأسسوا علم الجبر وحساب المثلثات وغيرها من علوم الحساب بشكل لم تعرفه البشرية من قبل. وكان ذلك سببا في تحقيق الحكم الرشيد في المعاملات والبيع والشراء والكراء. فكان علم الحساب مفتاحا في تطور المسلمين في شتى مجالات الدنيا ومنها مجال الهندسة والطب في العصر الإسلامي الذهبي.


\section{معادلات}
فيما يلي مثال على معادلة رياضية:

\begin{equation}
    E = mc^2
\end{equation}

ومثال آخر على معادلة معقدة:

\begin{equation}
    \int_0^\infty e^{-x^2} \, dx = \frac{\sqrt{\pi}}{2}
\end{equation}

\newpage

\section{نص الفصل الأول - الصفحة الثانية}

هذه الصفحة الثانية للفصل الأول تحتوي على نص إضافي لتوضيح كيفية تنسيق النصوص في كتب اللاتكس باللغة العربية.

\newpage

\section{نص الفصل الأول - الصفحة الثالثة}

هذه الصفحة الثالثة للفصل الأول تحتوي على المزيد من النصوص لاختبار تقسيم الصفحات وظهور الرؤوس والأقدام بشكل صحيح في النصوص العربية.
