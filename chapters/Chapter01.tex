\chapter{علم الحساب والغاية منه}

\section{المقدمة}

الحساب من حيث اللغة له العديد من المعاني وهذا البحث يتناول المعنى اللغوى الذي يشمل العدُّ والإحصاء والتقدير والتفصيل، ويمتد هذا المعنى ليشمل أيضا المكافأة والجزاء والحفظ بعلم وعدل. ومن معاني الحساب أيضا الإتقان والضبط. ومصدره حَسَبَ مثل حَسَبَ الشيء أي عدَّه وأحصاه وقدَّره. وحِسَاب على وزن فِعَال بمعنى محسوب أي مفعول مثل كتاب أي مكتوب. وجمع حساب حسبان مثل شهاب شهبان ومن معانيه العذاب. والذي يَحْسُب الشيء يقال له حَاسِب\comment{(على وزن فَاعِل)} وجمعه حَاسِبُون عند الرفع\comment{(على وزن فَاعِلُون)} وحَاسِبِين عند النصب والجر\comment{(على وزن فَاعِلِين)}. ويأتي اسم الحسيب  على وزن فعيل بمعنى الكافئ وبمعنى الحاسب. والحسيب هي صيغة مبالغة لإسم الفاعل حاسب مثل عليم لإسم الفاعل عالم. ويقال أيضا على الحَاسِب والحسيب أنه مُحَاسِب على وزن مفاعل وجمعه مُحَاسِبون عند الرفع ومُحَاسِبين عند النصب والجر. والذي أُجْرِيَ له الحساب يقال له مُحَاسَب (أي مفعول به أول) وجمعه مُحَاسَبون عند الرفع ومُحَاسَبين عند النصب والجر. والشيء الذي حُسِبَ يقال له مَحْسُوب (أي مفعول به ثاني) وجمعه مَحْسُوبُون عند الرفع ومَحْسُوبِين عند النصب والجر. 

وأما في الشرع فمدار الدين كله على الحساب أي أن الله عز وجل خلق الناس لحكمة عظيمة وهي عبادته وحده لا شريك له وإنه سبحانه لم يخلقهم عبثا بل إنهم مكلفون ومُحَاسَبون على أعمالهم كبيرها وصغيرها ظاهرها وباطنها. وهذا الحساب يترتب عليه الجزاء العاجل في الدنيا والجزاء الآجل في الآخرة. ويدخل الحساب أيضا في المعاملات بين الناس التي بينها الشرع كالزكاة والمواريث والبيع والشراء وغير ذلك. والله جل جلاله هو الحَاسِب وهو المُحَاسِب الذي يُحَاسِب المكلفين بعلم وعدل وهو الحسيب الكافئ لعباده الصالحين. والمكلفون من الجن والإنس مُحَاسَبُون على أعمالهم ومُحَاسِبُون لأنفسهم. وجميع الأعمال مَكْتُوبَة ومَحْسُوبَة لا يخفى منها شيئ عند الله تبارك وتعالى. والحساب يوم القيامة يشمل الجزاء الذي يكون إما جنة أو نار ولا يكتمل الحساب إلا بذلك. ولذلك فقد سمى الله جل جلاله يوم القيامة بيوم الحساب وهو يوم الدين\comment{الذي فيه يُحُاسِب الله جل جلاله المكلفين ويجازيهم على أعمالهم برحمته لمن إتبع أمره ورضى عليه وبعدله لمن خالف أمره وغضب عليه}. ويكون الحساب يوم القيامة عند الميزان وهذا فيه علاقة الحساب بالميزان وأن الميزان هو الميزان الشرعي الذي كلفنا الله به في صورته الحسية. 

وفي الإصطلاح فإن الحساب هو ميزان معنوى لتقدير وتفصيل الأشياء بالأعداد (أو الأرقام) وإحصائها بالجمع (أي الزيادة) والطرح (أي النقصان) والضرب (أي المضاعفة) والقسمة (أي التقسيم) وغير ذلك من العمليات الحسابية والتي بها يضبط الميزان بالعدل وهو التساوي (أي المساواة) والذي به تعرف المقادير المتغيرة والثابتة والعلاقات بينها. وعرف الحساب قديما وحديثا وله العديد من المسميات والأنواع وهو الرياضيات وهو علم الدقة وهو الأساس الذي تبنى عليه الهندسة بكافة أنواعها وفروعها. فالهندسة لا تقوم إلا بالحساب الصحيح الذي به يمكن بناء معادلات ونماذج رياضية للتعبير بدقة عن العلاقات البسيطة والمعقدة والعوامل المؤثرة فيها. وهذا لأنه بالحساب تعرف المجاهيل بعدة طرق منها ما هو سهل وبسيط ويحسب ذهنيا ومنها ما هو صعب ومعقد ويحسب كتابيا أو بإستخدام الطرق الحديثة. والحساب حاله كحال الميزان الحسي يبنى على الوزن ولهذا فقد سماه الخوارزمي رحمه الله بالجبر والمقابلة بحيث يبنى الحساب على التساوي بين المتغيرات لجبر ما اختل من الميزان وعليه يمكن حساب ما جهل منها وهو ما نعرفه اليوم بالتساوي (أي علامة =) في المعادلات الرياضية بجميع أشكالها. فيكون بذلك الطرف الأيمن والأيسر بالنسية لعلامة التساوي في المعادلات الرياضية ككفتان الميزان تماما ولا يستقيم ذلك إلا بالحساب الصحيح الذي به يمكن تحقيق العدل في المعاملات.

ولهذا فإن الآيات الشرعية التي تأمر بالوفاء في الكيل والميزان بالقسط في المعاملات بين الناس هي بلا شك تشمل الحساب وهذا لأن الحساب في أصله داخل في معنى الميزان. ومن الأمثلة على تطبيقات علم الحساب في اتباع آيات الله الشرعية كحساب الزكاة والصدقات والمواقيت والمواريث والبيع والشراء وغيرها من المعاملات التي يحتاج إليها الناس.  ومن الأمثلة على تطبيقات علم الحساب في فهم آيات الله الكونية كمعرفة حركة الشمس والقمر والأرض وغيرها من الظواهر الطبيعية التي خلقها الله لما في ذلك من مصالح دينية مثل التفكر في عظمة الله والزيادة في الإيمان ومصالح دنيوية مثل حساب الوقت والمواسم. فعلم الحساب من الضروريات التي يحتاج إليها الناس في أمور دينهم ودنياهم وهو مفتاح جميع العلوم الكونية السببية الظاهرة وهو الوسيلة لتحقيق الغاية العظيمة التي أمر الله بها وهي إقامة الميزان الشرعي. 

ولهذا كان البحث في علم الحساب من الأمور التي حث الله تعالى عليها في أكثر من موضع في كتابه العظيم. قال تعالى: \quranayah*[17][12]{\footnotesize \surahname*[17]}. يقول السعدي رحمه الله في تفسيره: (وَلِتَعْلَمُوا) بتوالي الليل والنهار واختلاف القمر (عَدَدَ السِّنِينَ وَالْحِسَابَ) فتبنون عليها ما تشاءون من مصالحكم. (وَكُلَّ شَيْءٍ فَصَّلْنَاهُ تَفْصِيلًا) أي: بينا الآيات وصرفناه لتتميز الأشياء ويستبين الحق من الباطل \href{https://shamela.ws/book/42/1001#p1}{\faExternalLink} \cite{tafsir_Saadi}. وقال تعالى: \quranayah*[10][5]{\footnotesize \surahname*[10]}. يقول السعدي رحمه الله في تفسيره: وفي هذه الآيات الحث والترغيب على التفكر في مخلوقات الله، والنظر فيها بعين الاعتبار، فإن بذلك تنفتح البصيرة، ويزداد الإيمان والعقل، وتقوى القريحة، وفي إهمال ذلك، تهاون بما أمر الله به، وإغلاق لزيادة الإيمان، وجمود للذهن والقريحة \href{https://shamela.ws/book/42/729#p6}{\faExternalLink} \cite{tafsir_Saadi}. وقال تعالى: \quranayah*[55][5]{\footnotesize \surahname*[55]}. يقول السعدي في تفسيره: أي: خلق الله الشمس والقمر، وسخرهما يجريان بحساب مقنن، وتقدير مقدر، رحمة بالعباد، وعناية بهم، وليقوم بذلك من مصالحهم ما يقوم، وليعرف العباد عدد السنين والحساب \href{https://shamela.ws/book/42/1887#p6}{\faExternalLink} \cite{tafsir_Saadi}. وقال تعالى: \quranayah*[6][96]{\footnotesize \surahname*[6]}. يقول القرطبي رحمه الله: ومعنى حسبانا أي بحساب يتعلق به مصالح العباد. وقال ابن عباس في قوله جل وعز: والشمس والقمر حسبانا أي بحساب. قال الأخفش: حسبان جمع حساب; مثل شهاب وشهبان \href{https://shamela.ws/book/20855/2582#p1}{\faExternalLink} \cite{tafsir_Qurtubi}.

وفي هذه الآيات الحث والترغيب في علم الحساب والأخذ والإستدلال به وهذا فيه الحكمة البالغة من الله جل جلاله. ومن ذلك حتى يبني عليه الناس مصالحهم الدينية والدنيوية ويقوموا بالقسط في جميع عباداتهم ومعاملاتهم وعلى الوجه المطلوب. ومن حكمة الله تعالى أنه جعل علم الحساب موافق للفطرة والعقل وجعله مفتاح جميع العلوم الكونية السببية الظاهرة\footnote{ظاهرة بالحجة العقلية والقياس في العلم التجريبي ولا يشترط أن تكون ظاهرة للحواس. فالعديد من الظواهر يمكن قياسها ولا يمكن رؤيتها على سبيل المثال.} التي يمكن فيها العد والقياس والتي لا يمكن في الغالب فهمها فهما صحيحا من دون الحساب. ومن ذلك الظواهر الطبيعية كحركة الأفلاك وغيرها والتي جعل الله لها أقدارها بحيث تتبع حساب متقن لا تحيد عنه ولا تميل. فعلم الحساب به يفهم الميزان الكوني من آيات الله الكونية وبه يقام الميزان الشرعي باتباع آيات الله الشرعية. وهذا لأن علم الحساب ما هو إلا صورة من صور الميزان ولكن ميزان معنوى، وأما الميزان المعروف الذي توزن به الأشياء فهو ميزان حسي. فالميزان تقدّر به الأشياء بالوزن، والمكيال تقاس به الأشياء بالحجم، وأما الحساب فتحسب به الأشياء بالعدد لكل ما هو مقاس وزنا أو حجما أو ثمنا أو غير ذلك من وحدات القياس الأخرى. ولهذا فقد قرن الله جل جلاله الحساب مع العدد فلا حساب دون عد بالإحصاء أو القياس ولو على وجه التقريب. كما أنه أيضا لا هندسة دون حساب.

ومن المعلوم أن الناس يتفاوتون في قدراتهم الحسابية كل بحسب إجتهاده وبما أودعه الله جل جلاله فيهم من إدراك وعقل وعلم. والأشياء أيضا تتفاوت في إمكانية ودقة حسابها، فمنها السهل والمعروف كعدد ساعات اليوم، ومنها الصعب والمعقد كحركة السحاب والمطر، ومنها المستحيل من الأمور الغيبية كموعد الساعة الكبرى أو الصغرى. وهذا فيه أن الإنسان قد يخفى عليه العديد من الأمور وقد يتخلل حسابه النقص وبالأخص ما يصعب إدراكه أو إدراك أسبابه. ومتى أدرك الإنسان أسباب الأشياء، كانت له القدرة على قياسها وتقديرها. فإن كان لديه العقل الراجح والعلم النافع، إستطاع الإستدلال بالحساب والإسترشاد به في تقرير حقائق الأشياء ولو على وجه التقريب. وكلما زاد إدراكه للأشياء، زادت بصيرته وزاد معها تعقله لها وصح حسابه حتى كان حسابه صحيحا وموافقا للحق بإذن الله تعالى.

ومن عظيم آيات الله جل جلاله أنه جعل الناس متوافقين في أصل العدد والحساب بفطرتهم. ولهذا فإن العدد والحساب من الأمور التي توافقت عليه جميع الحضارات البشرية ولم تختلف في أصل هذا العلم إلا في طرق التعبير عنه كالرموز واللغات المستخدمة في ذلك. والله جل جلاله له كمال العدل في حسابه وهذا لأن حسابه وتقديره كامل لا نقص فيه لأن الله هو العليم بكل شيء والقادر على كل شيء. وفي معرفة حساب الله العديد من المصالح منها أن الله قائم بالقسط في الدنيا والآخرة، وفي الميزان الكوني والميزان الشرعي. وفي معرفة ذلك الإستعداد ليوم الحساب الذي فيه يحاسب الله المكلفين. وفي هذه المعرفة أيضا الترغيب في إقامة الحق والميزان مع الأخذ بالأسباب والتي بها يمكن تحقيق الحكم الرشيد. وغير ذلك من المصالح الدينية والدنيوية.

\section{الحساب داخل في أسماء الله وصفاته وأفعاله}

العد والإحصاء والحساب كلها داخلة في أسماء الله جل جلاله وصفاته وأفعاله. ومن ذلك ما دلت عليه أسماءه الحسنى سبحانه وتعالى ومنها العليم والحكيم والحسيب والحفيظ واللطيف والخبير والحي والقيوم والرقيب والمحيط والشهيد والحكم والوكيل والعدل. وهذا من كمال عدله وحكمته سبحانه ومن ذلك حتى يعطي لكي ذي حق حقه يوم الحساب. ويستثنى من ذلك أنه سبحانه يرزق من يشاء بغير حساب كما أنه يوفي للصابرين أجرهم بغير حساب. ولقد أثبت سبحانه في مواضع كثيره في كتابه العظيم أنه سبحانه أحصى كل شيء عددا، وأنه على كل شيء حسيبا، وأن كل شيء عنده بمقدار، وأنه سبحانه أتقن كل شيء، وأنه سبحانه فصَّل كل شيء تفصيلا، وأنه سبحانه سريع الحساب وأسرع الحاسبين كما يليق بجلاله بدون تشبيه أو تكييف أو تعطيل أو تمثيل. وهذا كله فيه إثبات كمال الحساب لله جل جلاله لإجتماع معاني الحساب كلها في صفاته وأسمائه وأفعاله.

فالله جل جلاله قدر المقادير كلها وأحصاها عددا، وسبق علمه بذلك وكتبه في اللوح المحفوظ قبل خلق السموات والأرض. ومن كمال عدله سبحانه أنه يقوم بمحاسبة كل نفس يوم الحساب وهو سبحانه أسرع الحاسبين كما قال تعالى: \quranayah*[6][62]{\footnotesize \surahname*[6]}. وجاء في تفسير الطبري رحمه الله: وهو أسرع من حسب عددكم وأعمالكم وآجالكم وغير ذلك من أموركم أيها الناس وأحصاها وعرف مقاديرها ومبالغها لأنه لا يحسب بعقد يد ولكنه يعلم ذلك ولا يخفى عليه منه خافية، \quranayah*[34][3][13]{\footnotesize \surahname*[34]} \href{https://shamela.ws/book/7798/6327#p1}{\faExternalLink} \cite{tafsir_Tabari}.

فالعد والإحصاء والحساب من كمال حكمة الله سبحانه وتعالى وعدله. ومن أعظم ذلك أن الله عز وجل قائم بالقسط وهو العدل الظاهر في حكمه الكوني\footnote{حكم الله الكوني هو حكم الله القدري التابع لإرادته والقضاءي النافذ بقدرته والذي فيه شأن الله عز وجل وتدبيره لهذا الكون بحكمته ورحمته وعدله، والذي سبق علم الله به أزلا وكتابته له في اللوح المحفوظ قبل خلق السموات والأرض.} والشرعي والجزائي. فيكون عدله سبحانه ظاهر لنفسه في حكمه الكوني بالميزان الكوني الذي وضعه في يده، وظاهر لخلقه في حكمه الشرعي بالميزان الشرعي الذي كلفنا به في الدنيا، وظاهر لخلقه في حكمه الجزائي بالميزان الشرعي الذي يحاسبنا به في الآخرة. ولهذا فقد شهد الله جل جلال لنفسه بالقسط في حكمه كله الكوني والشرعي والجزائي في أعظم شهادة في كتابه العظيم فقال جل في علاه: \quranayah*[3][18]{\footnotesize \surahname*[3]}. وأخبر سبحانه أنه أرسل الرسل بالحق والميزان الشرعي ليقوم الناس بالقسط بحسب ما كلفهم به فقال تعالى: \quranayah*[57][25][1-11]{\footnotesize (\surahname*[57])}. وأخبر أنه سبحانه أنزل الكتب بالحق والميزان الشرعي في قوله تعالى: \quranayah*[42][17]{\footnotesize (\surahname*[42])}. وأخبر سبحانه أنه سيضع موازين القسط يوم القيامة لحساب المكلفين في حكمه الجزائي المترتب على حكمه الشرعي الذي كلفهم به فقال تعالى: \quranayah*[21][47]{\footnotesize \surahname*[21]}. وجعل الرسل عليهم الصلاة والسلام شاهدين على ذلك يوم الحساب فقال سبحانه: \quranayah*[10][47]{\footnotesize \surahname*[10]}.

فتأمل كيف أن الله عز وجل وصف نفسه بأنه قائم بالقسط في حكمه الكوني، وأمر عباده بالقسط في حكمه الشرعي، وأنه سبحانه سيحاسبهم بالقسط يوم الحساب في حكمه الجزائي. وهذا كله من كمال عدله وحكمته سبحانه وتعالى. ومن تمام عدله أنه سبحانه حرم الظلم على نفسه في حكمه الكوني والجزائي وجعله محرما على عباده في حكمه الشرعي كما أخبر النبي ﷺ في الحديث القدسي فيما روى عن الله تبارك وتعالى أنه قال: يا عبادي إني حرمت الظلم على نفسي وجعلته بينكم محرما. فلا تظالموا \href{https://shamela.ws/book/1727/6507#p3}{\faExternalLink} \cite{muslim}.\footnote{صحيح مسلم: 2577.}

\section{الحساب داخل في أمر الله الشرعي}

أرسل الله عز وجل الرسل وأنزل الكتب أولا لإقامة الحق ومن أعظم ذلك عبادة الله وحده لا شريك له، وثانيا لإقامة الميزان الشرعي ليقوم الناس بالقسط فيما بينهم. ولقد كان الأنبياء والرسل عليهم الصلاة والسلام أكثر الناس إعتبارا بآيات الله الكونية وأعلم الناس بأحكام الله الشرعية والجزائية وأكملهم إدراكا وعقلا. ولهذا فقد سعى الرسل إلى إقامة الحق أولا بالدعوة إلى توحيد الله في عبادته، وإلى إقامة الميزان الشرعي ثانيا بالدعوة إلى القسط والعدل بين الناس. فلزم بذلك معرفتهم للكيل والميزان وما يدخل في ذلك من العد والحساب الذي به تحفظ الأموال والحقوق في المعاملات بين الناس. فالحساب يمكن حفظه أو كتابته وفي صحته وبيانه القسط الذي أمر الله به. والغش في الحساب أو الجهل به من مخالفة الميزان الشرعي ومن الظلم والفساد الذي لا يرضى الله به ومن أسباب تعجيل سخط الله في الدنيا قبل الأخرة. 

وقد كان يوسف عليه السلام عالما بالحساب حافظا وكاتبا له كما في قوله تعالى: \quranayah*[12][55]{\footnotesize \surahname*[12]}. فقد جاء في تفسير الطبري والبغوى معنى (إِنِّي حَفِيظٌ عَلِيمٌ): أي كاتب حاسب حفيظ للخزائن عليم بوجوه مصالحها، وقيل: أي حفيظ للحساب عليم بالألسن أعلم لغة كل من يأتيني \cite{tafsir_Tabari}\cite{tafsir_Baghawi}. وجاء في تفسير القرطبي: إني حاسب كاتب، وأنه أول من كتب في القراطيس \cite{tafsir_Qurtubi}. ويقول السعدي في تفسير ذلك: (إِنِّي حَفِيظٌ عَلِيمٌ) أي: حفيظ للذي أتولاه، فلا يضيع منه شيء في غير محله، وضابط للداخل والخارج، عليم بكيفية التدبير والإعطاء والمنع، والتصرف في جميع أنواع التصرفات، وليس ذلك حرصا من يوسف على الولاية، وإنما هو رغبة منه في النفع العام، وقد عرف من نفسه من الكفاءة والأمانة والحفظ ما لم يكونوا يعرفونه. فلذلك طلب من الملك أن يجعله على خزائن الأرض، فجعله الملك على خزائن الأرض وولاه إياها \href{https://shamela.ws/book/42/841#p5}{\faExternalLink} \cite{tafsir_Saadi}. وهذا فيه أنه بالحساب الصحيح تحفظ الأموال وتعطى الحقوق بالعدل وبه يقام الميزان الشرعي لتحقيق العدل بين الناس.

ولقد كان النبي ﷺ يرشد الناس إلى الحساب الصحيح ويحثهم عليه بحسب ما أوحي له وفي هذا دليل نبوته فهو نبي أمي لا يحسب فعَنْ أَبِي سَعِيدٍ وَأَبِي هُرَيْرَةَ: أَنَّ رَسُولَ اللَّهِ ﷺ اسْتَعْمَلَ رَجُلًا عَلَى خَيْبَرَ فَجَاءَهُ بِتَمْرٍ جَنِيبٍ فَقَالَ: «أَكُلُّ تَمْرِ خَيْبَرَ هَكَذَا؟» قَالَ: لَا وَاللَّهِ يَا رَسُولَ اللَّهِ، إِنَّا لَنَأْخُذُ الصَّاعَ مِنْ هَذَا بِالصَّاعَيْنِ وَالصَّاعَيْنِ بِالثَّلَاثِ فَقَالَ: «لَا تَفْعَلْ بِعِ الْجَمْعَ بِالدَّرَاهِمِ ثُمَّ ابْتَعْ بِالدَّرَاهِمِ جَنِيبًا». وَقَالَ: «فِي الْمِيزَانِ مِثْلَ ذَلِكَ» {\footnotesize (مُتَّفَقٌ عَلَيْهِ)}.
وهذا فيه حرص النبي ﷺ حيث علم أنه بزيادة الصاع لا تثبت قيمة البيع لا حجما ولا وزنا ولا ثمنا. فيكون من أخذ صاعين بدل صاع فقد اشترى بنصف قيمة ما باع، بينما من أخذ ثلاثة بدل اثنين فقد اشترى بثلثي قيمة ما باع، وهكذا. وهذا من الظلم في المعاملات الذي لا يقع إلا خطأ أو جهلا أو غشا. فأخبر النبي ﷺ أن هذا بخلاف الميزان وهو الحساب الصحيح في البيع والشراء، بل ونهى عن ذلك وأمر بأخذ القيمة عند البيع ومن ثم الشراء حتى تثبت القيمة بالنسبة للنوع وزنا أو حجما. وفيه أيضا أن الرسول ﷺ سمى الحساب الصحيح ميزانا في قوله "في الميزان مثل ذلك". ولهذا فإن الحساب في أصله ما هو إلا صورة معنوية للميزان الحسي.

وغير ذلك من المواقف الأخرى التي كان النبي ﷺ يبين فيها الحساب الصحيح للناس كبيان عدد ساعات اليوم وأيام الشهر وغير ذلك من الأمور التي تنفع الناس في دينهم ودنياهم. ومن ذلك بيان أحب الصلاة والصيام عند الله تبارك وتعالى فعن عبد الله بن عمرو أنَّ رَسولَ اللَّهِ ﷺ قالَ له: أَحَبُّ الصَّلَاةِ إلى اللَّهِ صَلَاةُ دَاوُدَ عليه السَّلَامُ، وأَحَبُّ الصِّيَامِ إلى اللَّهِ صِيَامُ دَاوُدَ، وكانَ يَنَامُ نِصْفَ اللَّيْلِ ويقومُ ثُلُثَهُ، ويَنَامُ سُدُسَهُ، ويَصُومُ يَوْمًا، ويُفْطِرُ يَوْمًا. ومجموع النصف مع الثلث والسدس هو واحد وهو مجموع ساعات الليلة الواحدة من بعد صلاة العشاء إلى أذان صلاة الفجر. ولا يحسب ذلك من غروب الشمس لأن النبي ﷺ أولا بدأ بالنوم وثانيا جعل الثلث لصلاة قيام الليل والثلثين للنوم ولم يدخل صلاتي المغرب والعشاء في ذلك، فدل على أن ساعات الليل المذكورة في هذا الحديث تبدأ من بعد صلاة العشاء مباشرة. فلو إنتهت صلاة العشاء عند التاسعة مساءا وكانت صلاة الفجر عند السادسة صباحا، كان مجموع ساعات الليل الخاصة بالقيام والنوم معا 9 ساعات. وبهذا تكون ساعات النوم الأولى قبل قيام الليل أربعة ساعات ونصف (أي من 9:00 م إلى 01:30 ص)، وتكون ساعات القيام ثلاث ساعات (من 01:30 ص إلى 4:30 ص)، وساعة ونصف من النوم قبل الفجر (من 4:30 ص إلى 6:00 ص). ولو أدخلت صلاتي المغرب والعشاء وما بينهما في ذلك للزم دخولهم في ثلث القيام وبالتالي يكون المغرب أول قيام الليل، والعشاء منتصف قيام الليل، وقيام الليل آخر قيام الليل وكلها جميعا ثلث الليل وكانت باقي الثلثين للنوم وهذا غير راجح لأن النبي ﷺ بدأ بالنوم فدل على أن ذلك إنما يكون من بعد صلاة العشاء، والله فوق كل ذي علم عليم.

ومن ذلك أن النبي ﷺ  بين للناس حساب العشرة الأضعاف من الحسنات خلال اليوم والليلة في الذكر وأنها تذهب مثلها من السيئات في العدد فقال: خَصْلتانِ لا يُحافِظُ عليهِما عبدٌ مُسلمٌ إلا دخل الجنةَ، ألا وهُما يَسِيرٌ، ومَن يعملْ بِهِما قَليلٌ، يُسَبِّحُ اللهَ في دُبُرِ كُلِّ صلاةٍ عَشْرًا (10)، ويَحمدُه عشْرًا (10)، ويُكبِّرُه عشْرًا (10)، (ومجموع ذلك 30 في الصلاة الواحدة)، فذلِكَ خَمسُونَ ومِائَةٌ باللِسانِ (150 = 30 × 5 أي في الصلوات الخمس)، وألفٌ وخَمسُمِائةٌ في المِيزانِ (1500 = 150 × 10 أي عشرة أضعافها). ويُكبِّرُ أربعًا وثلاثِينَ إذا أخَذَ مَضْجَعَهُ (34)، ويَحمدُه ثلاثًا وثلاثِين (33)، ويُسَبِّحُ ثلاثًا وثلاثِينَ (33)، فتِلكَ مائةٌ باللِسانِ (100 = 34 + 33 + 33 أي حاصل جمعها)، وألْفٌ في المِيزانِ (1000 = 100 × 10 أي عشرة أضعافها)، فأيُّكمْ يَعْملُ في اليومِ والليلةِ ألْفينِ وخَمسَمائةِ سَيِّئَةٍ (2500 = 1500 + 1000 أي عدد الحسنات الكلي المترتب على هذا الذكر خلال اليوم والليلة) {\footnotesize (صحيح الجامع، وصححه الألباني)}. وكل ذلك فيه دليل على نبوته ﷺ فهو أمي لا يحسب ولكن لا ينطق إلا بالحق كما أخبر ذلك الله عز وجل في كتابه العظيم:
\quranayah*[53][3-4]{\footnotesize \surahname*[53]}. فلك أن تتأمل في حكمة الله عز وجل في الحث على تعلم العدد والحساب وكيف أن النبي ﷺ كان يبين للناس الحساب الصحيح في الأعمال الصالحة للمحافظة عليها طلبا لرضوان الله جل جلاله وجنته وعلما بأن الله كان على كل شيء حسيبا كما في قوله تعالى: \quranayah*[4][86-87]{\footnotesize \surahname*[4]}.

ولقد جاء عن النبي ﷺ أنه دعى لمعاوية تعلم الحساب مع كتاب الله عز وجل في شهر رمضان المبارك فعن العرباض بن سارية أنه قال سمِعتُ النَّبيَّ ﷺ وهوَ يَدعو إلى السَّحورِ في شهرِ رمضانَ: هَلُمَّ إلى الغداءِ المبارَكِ ثمَّ سمعتُه يقولُ: اللَّهمَّ علِّمْ مُعاويةَ الكِتابَ، والحِسابَ ، وقِهِ العَذابَ. وفي رواية أخرى: اللهم علمه الكتاب ومكن له في البلاد وقه العذاب \href{https://shamela.ws/book/25794/13683#p2}{\faExternalLink}\href{https://shamela.ws/book/22669/1909#p2}{\faExternalLink}\href{https://shamela.ws/book/4445/6628#p1}{\faExternalLink}\href{https://shamela.ws/book/9442/5496#p12}{\faExternalLink} \cite{ahmid}\cite{dahabi_Siyar}\cite{ibnKathir_AlBidayah}\cite{albani_Sahiha}.\footnote{أحمد: 17152، وأورده الذهبي في سير أعلام النبلاء وبن كثير في البداية والنهاية، والإمام الألباني في السلسلة الصحيحة.} وفي رواية أخرى بتقديم الحساب على الكتاب: اللهم علم معاوية الحساب والكتاب، وقه العذاب \href{https://shamela.ws/book/20879/1060#p1}{\faExternalLink}\href{https://shamela.ws/book/23100/2493#p1}{\faExternalLink}. ولعل هذا فيه الإشارة الكافية لأهمية علم الحساب وأنه من الدين وأنه من أسباب التمكين لأن به يقام العدل في الحكم كما سيأتي بيانه في فصل الحكم الرشيد. وفي هذا الحديث العديد من الأسرار والحكم. ومن ذلك أن معاوية رضي الله عنه كان من كتبة الرسول ﷺ فكان يعرف الكتابة والقراءة ويكتب الوحي. فدعى له النبي ﷺ تعلم الحساب مع الكتاب حتى تزال عنه الأمية بكاملها. فقد صح عن النبي ﷺ أنه قال: إنَّا أُمَّةٌ أُمِّيَّةٌ، لا نَكْتُبُ ولَا نَحْسُبُ {\footnotesize (صحيح البخاري)} فجعل الجهل بالحساب من الأمية. ولهذا فإن الأمية تحققت في حق معاوية من جهة الجهل بالحساب فقط. ولقد كان معاوية رضي الله عنه أرجى من غيره في ذلك لذهاءه فقد كان من أصحاب العقول الذكية حتى قال عنه عمر بن الخطاب في مجلسه: تذكرون كسرى وقيصر ودهاءهما؛ وعندكم معاوية! \href{https://shamela.ws/book/9783/2790#p3}{\faExternalLink} \cite{ibnJareerTabari_Tareekh}. وقد جاء في سيرته أنه تعلم وأتقن الحساب حتى كان من الكتبة الحسبة الفصحاء \href{https://shamela.ws/book/12286/6715#p6}{\faExternalLink} \href{https://shamela.ws/book/7299/5215#p3}{\faExternalLink}. 

لعل من الحكم أيضا أن النبي ﷺ علم أن معاوية رضي الله عنه سيكون ملكا على الشام فدعى له تعلم الحساب حتى يقيم العدل في حكمه فقد صح عن النبي ﷺ أنه قال: أوَّلُ هذا الأمرِ نُبوَّةٌ ورحمةٌ، ثمَّ يكونُ خلافةً ورحمةً، ثمَّ يكونُ مُلكًا ورحمةً \href{https://shamela.ws/book/9442/5496#p12}{\faExternalLink} \cite{albani_Sahiha}.\footnote{صححه الألباني في السلسلة الصحيحة: 3270.} ولقد إستجاب الله جل جلاله هذا الدعاء لنبيه ﷺ فقد كان معاوية رضي الله عنه أول ملوك المسلمين بعد الخلفاء الراشدين وكان حكما عدلا حتى عرف بالمهدي عند أئمة التابعين فقد جاء عن مجاهد رحمه الله أنه قال: "لو رأيتم معاوية لقلتم هذا المهدي" \href{https://shamela.ws/book/1077/702#p1}{\faExternalLink}. وقال قتادة رحمه الله: "لو أصبحتم في مثل عمل معاوية لقال أكثركم: هذا المهدي" \href{https://shamela.ws/book/1077/701#p1}{\faExternalLink}. وقد ذكر عند الأعمش عمر بن عبد العزيز وعدله، فقال الأعمش: "فكيف لو أدركتم معاوية؟ قالوا: يا أبا محمد، يعني في حلمه؟ قال: لا والله، ألا بل في عدله" \href{https://shamela.ws/book/1077/700#p1}{\faExternalLink} \href{https://shamela.ws/book/927/3063#p3}{\faExternalLink}. ولهذا يقول شيخ الإسلام بن تيمية: اتفق العلماء على أن معاوية أفضل ملوك هذه الأمة فإن الأربعة قبله كانوا خلفاء نبوة وهو أول الملوك \href{https://shamela.ws/book/7289/1800#p1}{\faExternalLink} \cite{ibnTaimia_Majmoo}.\footnote{محموع الفتاوى: 4/478.} 

ومن هذا يعرف أن علم الحساب هو علم صادق وشريف ودقيق به يقام العدل والقسط في الحكم والمعاملات وهو من بنيان الحكم الرشيد. يقول الإمام الشافعي رحمه الله: من تعلم القرآن عظمت قيمته، ومن تكلم في الفقه نما قدره، ومن كتب الحديث قويت حجته، ومن نظر في اللغة رق طبعه، ومن نظر في الحساب جزل رأيه،\footnote{جزل أي عظم وغلظ. جزل رأيه أي صار ذا رأي فصيح محكم. الجزل من الكلام أي القوي الفصيح الجامع.} ومن لم يصن نفسه لم ينفعه علمه \href{https://shamela.ws/book/22669/4471#p4}{\faExternalLink} \href{https://shamela.ws/book/10495/14224#p1}{\faExternalLink} \cite{dahabi_Siyar}. فإضافة علم الحساب مع القرآن والحديث والفقه واللغة فيه بيان شرف هذا العلم وأهميته في إقامة الحق والميزان معا. ولهذا فقد نقل شيخ الإسلام بن تيمية رحمه الله إعتناء أهل السنة في زمانه بالعلوم الصادقة ومنها الحساب وخص الخوارزمي في ذلك فقال: ولهذا يرغب كثير من علماء السنة في النظر في العلوم الصادقة الدقيقة كالجبر والمقابلة وعويص الفرائض والوصايا والدور وهو علم صحيح في نفسه [.] وأما "حساب الفرائض" فمعرفة أصول المسائل وتصحيحها والمناسخات وقسمة التركات. وهذا الثاني كله علم معقول يعلم بالعقل كسائر حساب المعاملات وغير ذلك من الأنواع التي يحتاج إليها الناس. ثم قد ذكروا حساب المجهول الملقب بحساب الجبر والمقابلة في ذلك وهو علم قديم لكن إدخاله في الوصايا والدور ونحو ذلك أول من عرف أنه أدخله فيها محمد بن موسى الخوارزمي. وبعض الناس يذكر عن علي بن أبي طالب أنه تكلم فيه وأنه تعلم ذلك من يهودي وهذا كذب على علي
\href{https://shamela.ws/book/7289/4478#p2}{\faExternalLink} \cite{ibnTaimia_Majmoo}.\footnote{مجموع الفتاوى 9/214.} 

\section{الجهل بالحساب من الأمية}

ومن البلايا في زماننا هذا أن المسلمين قد غفلوا عن أهمية علم الحساب والإستدلال به حتى ظن الكثير منهم أنه ليس من الدين في شيء فضيعوه وتأخروا فيه علما وتهاونوا فيه عملا بعد أن كانوا روادا فيه ووضعوا أسسه بالإيمان الصادق للبشرية جمعاء. فقد تمكن المسلمون  بناءا على آيات القرآن الكريم التي ترشد إلى الإستدلال بالحساب والآيات التي تأمر بالقسط والعدل في الميزان من وضع أسس وقواعد علم الجبر والمقابلة في زمن هارون الرشيد على يدي العالم الجليل محمد بن موسى الخوارزمي رحمه الله تعالى. فسبقوا بذلك كل الأمم الأخرى كما سيأتي وكان الخوارزمي أول من كتب المعادلات الجبرية وأول من حلها بطريقته التي عرفت بإسمه "الخوارزميات" حتى أطلق عليه أبو الحساب، وأبو الجبر، وأبو الخوارزميات. فاعتنت وتسابقت وتهالفت على علمه الأمم الأخرى وكان سببا في نهوضها وإزدهارها بل وأيضا تسلطها على أمة الإسلام. وفي المقابل تأخر المسلمون في علم الحساب وفُقِدَ كتاب الخوارزمي وسرق وما طبع إلا بعد ألف عام من تأليفه بعد أن تشبعت الحضارات الأخرى من علمه وبالأخص الحضارات الغربية. فقد ترجم كتابه إلى أغلب اللغات الأوروبية وأعتني به في جامعاتهم وأدخلت مصطلحاته إلى معاجمهم وكان هذا هو المفتاح والسر وراء نهضتهم العلمية في كافة العلوم الكونية السببية النافعة. فتقدموا في كافة المجالات ولا زلنا نرى إلى يومنا هذا تأثير علم الجبر والخوارزميات في جميع علوم التكنولوجيا وبالأخص في مجال الحاسوب والذكاء الإصطناعي بجميع أنواعه. وأصبحنا نسعى لتعلم هذه العلوم منهم بعد أن تعلموها منا. وما هذا إلا لأن تضيع علم الحساب والجهل به من الأمية التي جاء الإسلام بالحث على خلافها من طلب العلم ونشره وإقامة الحق والميزان. 

فالأمية لا تكون فقط بعدم القدرة على القراءة والكتابة كما هو شائع، وإنما ايضا بعدم القدرة على الحساب الصحيح وعلى الوجه المطلوب. ومما يأكد هذا قوله ﷺ عندما سأل عن عدد الأيام في الشهر فقال ﷺ: إنَّا أُمَّةٌ أُمِّيَّةٌ، لا نَكْتُبُ ولَا نَحْسُبُ، الشَّهْرُ هَكَذَا وهَكَذَا. يَعْنِي مَرَّةً تِسْعَةً وعِشْرِينَ، ومَرَّةً ثَلَاثِينَ" {\footnotesize (صحيح البخاري)}. فجعل ﷺ الجهل بعلم الحساب في زمانه من الأمية. ولا ينبغي أن يستدل بهذا على عدم تعلم الحساب والأخذ به فهذا الإستدلال باطل يخالف كتاب الله عز وجل وسنة نبيه ﷺ كما تقدم. وإنما المقصود بالأمية في هذا الحديث هو الرسول ﷺ وأصحابه رضوان الله عليهم أجمعين الذين وصفهم الله جل جلاله بالأمية، أي أنهم في غالبهم لا يكتبون ولا يحسبون. وقد بين الله جل جلاله الأمية في حق نبيه في قوله تعالى: \quranayah*[7][158][21]{\footnotesize \surahname*[7]}. والأمية في القراءة والكتابة والحساب لا تعارض النبوة في شيء بل هي دليل على صدق النبي ﷺ. وقد بين الله جل جلاله الأمية في حق الصحابة رضوان الله عليهم في قوله تعالى: \quranayah*[62][2]{\footnotesize \surahname*[62]}. وهذا هو المعنى الصحيح كما بين ذلك الشيخ ابن باز رحمه الله حيث قال: فكل إنسان لم يتعلم ولم يكتب يقال له: أمي، والأمة العربية هكذا كان الغالب عليها أنها أمية لا تكتب ولا تقرأ، هذا الغالب على أمة محمد ﷺ [هـ]. ومن ذلك أيضا أنها كانت أمة لا تحسب كما بين ذلك النبي ﷺ.

وقوله ﷺ: "إنَّا أُمَّةٌ أُمِّيَّةٌ، لا نَكْتُبُ ولَا نَحْسُبُ" لا يعنى بحال من الأحوال أن تبقى أمة الإسلام أمة أمية لا تكتب ولا تحسب.  فقد جاءت الشريعة أولا بالحث على القراءة والكتابة، وثانيا بالحث على الحساب لرفع الأمية عن أمة الإسلام. فكان أول ما أنزل الله "إقرا" وفيه الحث لأمة الإسلام على تعلم القراءة والكتابة وطلب العلم ونشره. يقول تعالى:
\quranayah*[96][1-5]{\footnotesize \surahname*[96]}.
ومن ثم جاء الحث على التأمل في آيات الله الكونية في مواضع كثيرة ليس فقط لمجرد النظر فيها ولكن للتفكر فيها وتعقلها ولتعلم العدد والحساب والإستدلال به كما جاء في قوله تعالى: \quranayah*[10][5]{\footnotesize \surahname*[10]}. وكان هذا الحث على القراءة والكتابة والحساب منذ بداية فجر الإسلام في العهد المكي الذي اتسم بالدعوة إلى التوحيد الخالص. فدل ذلك على أن دين الإسلام هو دين الحضارة التي تبنى على الحق والعدل معا. ولهذا فإن كل إنسان لم يتعلم الحساب مع القراءة والكتابة على الوجه المطلوب يكون أميا كما بين ذلك النبي ﷺ. والله حث هذه الأمة في كتابه العظيم على العلم الذي يتأتى بالقراءة والكتابة حتى تقيم الحق، وعلى تعلم العدد والحساب حتى تقيم الميزان الشرعي بالعدل والقسط في جميع المعاملات بين الناس. 

وفي سيرة النبي ﷺ وأصحابه رضوان الله عليهم العديد من المواقف والتي فيها إهتمامهم برفع الأمية عن أمة الإسلام ومن ذلك ما جاء في مسند الإمام أحمد عن ابن عباس قال: "كان ناس من الأسرى يوم بدر لم يكن لهم فداء، فجعل رسول الله صلى الله عليه وسلم، فداءهم أن يعلموا أولاد الأنصار الكتابة" \href{https://shamela.ws/book/25794/1602#p1}{\faExternalLink} \cite{ahmid}.\footnote{مسند الإمام أحمد: 2216.}  وهذا فيه إهتمام النبي ﷺ بتعليم أبناء المسلمين القراءة والكتابة. وكان زيد بن ثابت رضي الله عنه ممن تعلم القراءة والكتابة من الأسرى حتى أن النبي ﷺ قدمه على غيره من الصحابة حتى صار من كتبة الوحي والرسائل. ولقد كان أبو هريرة رضي الله عنه أحفظ الناس لحديث النبي ﷺ إلا أنه كان لا يكتب وجاء في صحيح البخاري أنه قال: ما من أصحاب النبي ﷺ أحد أكثر حديثا عنه مني، إلا ما كان من عبد الله بن عمرو؛ فإنه كان يكتب ولا أكتب\href{https://shamela.ws/book/1284/250#p1}{\faExternalLink} \cite{ahmid}.\footnote{صحيح البخاري: 116.} وهذا فيه بركة الكتابة في حفظ العلم حتى بوب لذلك البخاري: باب كتابة العلم. 

وفي ذلك الحكمة البالغة من الله جل جلاله حتى يكون لأمة الإسلام دعوة الحق والميزان ويكون الحساب مفتاحا للعلوم الكونية السببية النافعة، فتأخذ بالأسباب وتتقدم في كافة المجالات على سائر الأمم الأخرى كما كان الحال في زمن هارون الرشيد في بداية العصر الإسلامي الذهبي قبل أن تعصف به رياح البدع والأهواء. وبالأخص فتنة خلق القرآن في زمن الإمام أحمد رحمه الله وما جاء بعد ذلك في زمن شيخ الإسلام بن تيمية رحمه الله. ورغم ذلك فقد جعل أهل السنة أوقات فراغهم لتعلم الحساب والهندسة كما نقل ذلك شيخ الإسلام بن تيمية في قوله: وكذلك كثير من متأخري أصحابنا يشتغلون وقت بطالتهم بعلم الفرائض والحساب والجبر والمقابلة والهندسة ونحو ذلك؛ لأن فيه تفريحا للنفس وهو علم صحيح لا يدخل فيه غلط. وقد جاء عن عمر بن الخطاب أنه قال: "إذا لهوتم فالهوا بالرمي وإذا تحدثتم فتحدثوا بالفرائض". فإن حساب الفرائض علم معقول مبني على أصل مشروع فتبقى فيه رياضة العقل وحفظ الشرع. لكن ليس هو علما يطلب لذاته ولا تكمل به النفس \href{https://shamela.ws/book/7289/4394#p1}{\faExternalLink} \cite{ibnTaimia_Majmoo}.\footnote{مجموع الفتاوى 9/129.} فتأمل في وصف شيخ الإسلام لعلم الحساب أن فيه حفظا للشرع وهذا من المصالح الدينية العظيمة.

وكل هذا فيه اهتمام السلف بعلم الحساب والهندسة في أوقات فراغهم رغم إنشغالهم بالأمور العظيمة الأخرى في بيان الحق ورد البدع والشبهات التي عصفت بذلك الزمان كعلم الكلام أي الفلفسة في الدين والعقائد التي تخالف صريح كتاب الله جل جلاله وسنة نبيه ﷺ. وهذا لأن الجهل بعلم الحساب يترتب عليه العديد من المفاسد الدينية والدنيوية. فمن المفاسد الدينية أن التفريط بالحساب الصحيح يؤدي إلى الظلم والفساد في المعاملات والمواريث والزكاة والصدقات وغيرها. ومن المفاسد الدنيوية أن الجهل بالحساب يؤدي إلى الجهل بالعلوم الكونية السببية النافعة والتي لا سبيل لفهمها فهما صحيحا إلا بالحساب كالهندسة والطب وما يترتب على ذلك من التخلف الحضاري عن سائر الأمم الأخرى في كافة الصناعات والمجالات الأخرى. وقد كان الإمام الشافعي رحمه الله يتحسر على تأخر المسلمين في علم الطب في زمانه كما نقل ذلك الإمام الذهبي رحمه الله في سير أعلام النبلاء أن الإمام الشافعي قال: لا أعلم علمًا بعد الحلال والحرام أنبل من الطب إلَّا أن أهل الكتاب قد غلبونا عليه. وقال حرملة: كان الشافعي يتلهف على ما ضيع المسلمون من الطب ويقول: ضيعوا ثلث العلم ووكلوه إلى اليهود والنصارى \href{https://shamela.ws/book/22669/4486#p7}{\faExternalLink} \cite{dahabi_Siyar}. 

ولقد كان النبي ﷺ كان يأمر أصحابه بالأخذ بالعلم السببي النافع مثل الرمي لحاجة المسلمين له فقال: مَن عَلِمَ الرَّمْيَ، ثُمَّ تَرَكَهُ، فليسَ مِنَّا، أوْ قدْ عَصَى {\footnotesize (صحيح مسلم، وصححه الألباني)}. ولهذا فإن العلوم الكونية السببية النافعة كالهندسة والطب وغيرها، التي يحتاج إليها المسلمين والتي هي من فروض الكفاية، تكون واجبة عند حاجة المسلمين لها في دينهم ودنياهم وفي تضييعها ضعف المسلمين والحاجة إلى الكفار والإعتماد عليهم والإنهزام أمامهم. وفي الأخذ بها قوة للمسلمين وكفايتهم عن غيرهم. فكيف بتضييع مفتاحها وهو علم الحساب الذي جاءت الشريعة بالحث عليه لإقامة العدل بين الناس. فالحساب هو علم الإتقان ولقد كان النبي ﷺ يحث أمته على الإتقان في كل شئ فعن عائشة أم المؤمنين أن النبي ﷺ قال: إنَّ اللهَ تعالى يُحِبُّ إذا عمِلَ أحدُكمْ عملًا أنْ يُتقِنَهُ \href{https://shamela.ws/book/21659/2761#p1}{\faExternalLink} \cite{jamaaSagheer}.\footnote{صحيح الجامع: 1880، وقال الألباني حديث حسن.} فالإتقان في علم الحساب الذي به يحصل القسط في المعاملات واجب لا ينبغى أن يتهاون فيه لأن خلاف ذلك يترتب عليه الظلم في المعاملات.

ولا يزال علماء السنة يحثون على هذه العلوم النافعة إلى زماننا هذا. \comment{ويقول الشيخ ابن باز رحمه الله: ولكن على الأمة أن تتعلم أيضًا ما ينفعها في دنياها: من الصناعات النافعة، ومن الاستعانة بها على قتال الأعداء وجهاد الأعداء، فيتعلم شؤون الزراعة، ويتعلم شؤون استخراج خزائن الأرض: من البترول والمعادن وغير ذلك؛ حتى تستغني عن أعداء الله، وتستخرج من بطون الأرض ومن خزائن الأرض ما ينفعها {\footnotesize (فتاوى الدروس، ما حكم من ينكر تعلم العلوم الدنيوية؟)}.} يقول الشيخ ابن باز رحمه الله: لا شكَّ أن الدراسة للعلوم الدنيوية أمرٌ مطلوبٌ، والله يقول سبحانه: وَأَعِدُّوا لَهُمْ مَا اسْتَطَعْتُمْ مِنْ قُوَّةٍ [الأنفال:60]، فالمسلمون بحاجةٍ إلى العلوم الدنيوية حتى يستعينوا بها على طاعة الله، وعلى الاستغناء عمَّا في أيدي الناس، وعلى جهاد أعداء الله، مع كون المسلمين يتعلَّمون الجيولوجيا والهندسة والطب وغير ذلك مما يُعينهم، وكذلك ما يخترعون في القوة التي يُجاهد بها الأعداء؛ كل ما يُعينهم على جهاد الأعداء واتِّقاء شرِّ الأعداء ويغنيهم عن الأعداء فهو أمرٌ مطلوبٌ: يَا أَيُّهَا الَّذِينَ آمَنُوا خُذُوا حِذْرَكُمْ [النساء:71] [.] فالاشتغال بالعلوم الدنيوية التي تنفع المسلمين إن كان لله؛ أُجِرَ عليها، مع فائدتها العظيمة، وإن كان يتعلَّمها للدنيا ليستفيد في دنياه فهذا مباحٌ ولا يضرُّه ذلك. لكن العلوم الدينية أهم، فليأخذ منها بنصيبٍ، ويجتهد في تعلم دينه، والتَّفقه في دينه، ثم مع ذلك يتعلم ما ينفعه في دنياه إذا استطاع ذلك، وإذا جمع بين الأمرين فهو خيرٌ إلى خيرٍ، يقول النبيُّ ﷺ: مَن يُرد الله به خيرًا يُفقهه في الدين، فإذا تفقَّه في دينه واستفاد مع ذلك في دنياه طبًّا أو صنعةً أخرى تنفعه أو أشياء مما ينفع العبدَ في هذه الدنيا؛ فذلك خيرٌ إلى خيرٍ \comment{وقد قال عليه الصلاة والسلام في الحديث الصحيح لما سُئل: أي الكسب أطيب؟ قال: عمل الرجل بيده، وكل بيع مبرور، وقال عليه الصلاة والسلام: ما أكل أحد طعامًا أفضل من أن يأكل من عمل يده، وإن نبي الله داود كان يأكل من عمل يده، كان صاحب حدادةٍ، كان يصنع الدروع عليه الصلاة والسلام، وهو نبي الله، يصنع الدروع التي يُتَّقى بها في الحرب، تُستعمل في الحرب، وكان زكريا نجارًا، وهو نبيٌّ من أنبياء الله عليه الصلاة والسلام [.] فتعلم العلوم الدنيوية أمرٌ مفيدٌ ونافعٌ، بشرط ألا يشغل عن علم الآخرة وعمَّا ينفعه في الآخرة، فإن جمع بينهما فقد جمع خيرًا إلى خيرٍ، وإن صلحت نيته في علوم الدنيا كانت عبادةً، وإذا تعلَّمها للدنيا فليس له ولا عليه، تعلم شيئًا مباحًا، لا حرج عليه، لكن متى صلحت نيته وأراد بهذا نفع المسلمين وتقويتهم ضدّ عدوهم؛ جمع الله له الخيرين: الأجر، ومع ذلك النَّفع بهذا المُتعلَّم} \href{https://www.youtube.com/watch?v=soOBNJPR0E8}{\faExternalLink}.\footnote{فتاوى الدروس، ما حكم وأهمية دراسة العلوم الدنيوية؟.}

فالحساب وما يبنى عليه من شتى العلوم الأخرى وبالأخص الهندسة فيها العديد من المنافع المعرفية التي تطور الإدراك ومن أهم ذلك قوة الإستدلال العقلى في فهم الأسباب والبحث فيها ومناقشتها وإثباتها بالحجة والبرهان وبالحقائق وليس بمجرد الآراء. وقلما وجدت مهندسا بارعا في مجاله إلا ورأيت فيه قوة الحجة والإستدلال العقلي. فإن إجتمع ذلك مع الصدق في النظر في مختلف المعطيات كان ذلك سببا في إنجاز ما يصعب إنجازه. ولقد تقدم قول الإمام الشافعي: "ومن نظر في الحساب جزل رأيه". ولعل من ذلك ما ذكره الشيخ صالح آل الشيخ حفظه الله من خلال دراسته للرياضيات والفيزياء والكيمياء في كلية الهندسة فقال: في كلية الهندسة [.] وجدت البيئة التي فيها مناقشة عقلية قوية، فيها بحث عن البرهان، فيها بحث عن المشكلة وحلها، فيها فهم حقائق الأشياء [.] الهندسة تخصصات، فيها تخصص العمارة، وفيها تخصص الهندسة المدنية، فيها تخصص الهندسة الميكانيكية والهندسة الكيميائية، وفيها تخصص الهندسة الكهريائية [.] وفي كلية الهندسة كان يقال لنا كلمة آنشتاين المشهورة: "الجنون هو أن تفعل نفس الشيء مرارا وتكرارا وتتوقع نتائج مختلفة" [.] إذا تخرجت بصير مهندس هذا شرف ما في شيء، أفتخر يعني لو كنت مهندسا لما عابني ذلك بالعكس، وأحترم كل المهندسين لأنهم عقول نافعة والبشرية كلها بنيت بالهندسة [.] الذين أسدوا للبشرية نفعا في حياتهم هم المهندسين [.] هم الذين بنوا الطرق والمباني، هم الذين مدوا جسور الحياة. التقنية الآن المفيدة كلها عمل مهندسين، فثلث أرباع الحياة عمل مهندسين، والطب جزء والباقي منظمون ينظمون الحياة \href{https://www.youtube.com/watch?v=kw_s0K_zbqc}{\faExternalLink}. ولقد نوه الشيخ حفظه الله التشابه بين علم الحديث وعلم الهندسة وأن دراسته للهندسة كانت باعثا له للإهتمام بعلم الحديث لما فيه من قواعد وأصول في تقرير أسانيد الأحاديث كما في الهندسة من قواعد وقوانين في تقرير العمليات الحسابية.

\section{نهج القرآن في تعلم الحساب}

إن أفضل طريقة لفهم علم الحساب وتعلمه والبحث فيه هي التأمل والتفكر في آيات الله الكونية وهي الظواهر الطبيعية التي خلقها الله وجعل لها الميزان الكوني. فمتى أدرك الإنسان هذه الآيات العظيمة وتبصر فيها وتعقلها وفهم حركاتها وسكناتها وأسبابها كان ذلك باعثا له على تعلمها وحسابها. وهذا لأن الآيات الكونية هي المرجع لنا لتعلم الحساب والتحقق من صحته. وهذا النهج هو نهج القرآن وهو أفضل الطرق وأحسنها. فجميع الآيات الشرعية التي تدل على تعلم العدد والحساب جائت مقرونة بالآيات الكونية كالشمس والقمر والليل والنهار. ويمكن أيضا دراسة علم الحساب مجردا من أي تطبيقات وهذا نهج معروف.\footnote{يعرف هذا المجال اليوم بالرياضيات البحثة وقد وجد فيه أن أغلب المفاهيم الرياضية لها تطبييقاتها.} ولكن الجمع بين العلوم الطبيعية كعلم الفيزياء والحساب لمحاولة محاكاة الظواهر الطبيعية هي الطريق الأمثل لتعلم وتطوير علم الحساب وهذا معروف لأهل هذا العلم.\footnote{يعرف هذا المجال اليوم بالرياضيات الفيزيائية وهو من أهم مجالات الرياضيات التطبيقية والهندسة.} وبهذا يكون الميزان الكوني طريقا لتعلم الحساب الصحيح ومن ثم يكون الحساب الصحيح وسيلة لإقامة الميزان الشرعي الذي أمرنا الله به. وهذا النهج فيه العديد من المصالح الدينية والدنيوية ومنها الزيادة في الإيمان ومعرفة قدرة الله جل جلاله ومعرفة فضله علينا والرقي بالحضارة في تطوير الطرق الحسابية وأدائها على الوجه المطلوب.

\comment{
فكل الظواهر الطبيعية التي خلقها الله جل جلاله هي تذكير لعباده حتى يتفكروا فيها ويعقلوها ومن ذلك قوله تعالى: \quranayah*[30][24]{\footnotesize \surahname*[30]}. وقوله تعالى: \quranayah*[45][13]{\footnotesize \surahname*[45]}. ووصف الله جل جلاله من أعتبر بهذه الآيات بأولي الألباب وذكر خصالهم في قوله تعالى: \quranayah*[3][190-191]{\footnotesize \surahname*[3]}. وهذا كله فيه الحث من الله تبارك وتعالى على التفكر والتأمل في آياته الكونية والتي بها يعرف العبد قدرة الله جل جلاله وفضله علينا فيرجوا رحمته ويخاف عذابه فيزداد إيمانا ويكون ذلك باعثا له على قيام الليل كما في قوله تعالى: \quranayah*[39][9]{\footnotesize \surahname*[3]}.
}

ومن أهم الظواهر الطبيعية التي يزداد بها الإيمان ويتعلم منها الحساب هي حركة الأفلاك من الكواكب والنجوم وبالأخص الأرض\footnote{من أهم دلائل هذا البحث هو الإستدلال بالليل والنهار على حركة الأرض وهذا لكون أن هذه الحركة جائت مستقلة عن حركة الشمس والقمر في جميع الآيات الشرعية فدل ذلك على حركة الأرض. ولا ينبغى أن يترك هذا لما يترتب عليه من الإستدلال الخاطئ الذي يخالف آيات الله الشرعية والكونية. وهذا بخلاف ما قرره العديد من أهل العلم أن الليل والنهار يكون بحركة الشمس حول الأرض وهذا غير صحيح ولا يثبت شرعا ولا كونا.} والشمس والقمر التي بها يعرف الوقت كما قال تعالى: \quranayah*[16][12]{\footnotesize \surahname*[16]}. وهذا لأن الله جل جلاله جعل حركة هذه الأفلاك في غاية الإنتظام والتناسق كما في قوله تعالى: \quranayah*[21][33]{\footnotesize \surahname*[21]}. وقال تعالى: \quranayah*[36][40]{\footnotesize \surahname*[36]}. وبمعرفة ذلك يزداد الإيمان ويعرف فضل الله علينا ودقة خلقه وأنه سبحانه هو بديع السماوات والأرض خلق كل شيء بقدر معلوم فقدره تقديرا. ولهذا فقد حثنا وأوصانا الله عز وجل بالتدبر والنظر في هذه الآيات العظيمة وذكرنا بذلك في غالب آيات القرآن حتى نعرف فضل الله علينا ونتفع بذلك ونوقن بلقائه كما في قوله تعالى: \quranayah*[13][2]{\footnotesize \surahname*[13]}.

\section{معرفة الوقت بحساب حركة الأفلاك}

ومن أعظم النعم التي بها تقوم مصالح الناس الدينية والدنيوية هي معرفة الوقت الذي جعله الله جل جلاله بحسب حركة الأفلاك كما أرشد في كتابه العظيم. ولهذا يعرف الوقت بالعد والحساب لحركة الأفلاك المتناسقة والمتزنة وهذا من فضل الله جل جلاله على الناس أجمعين ومن ذلك قوله تعالى: 
\quranayah*[6][96]{\footnotesize \surahname*[6]}. يقول السعدي رحمه الله في تفسيره:  جعل تعالى الشمس وَالْقَمَرَ حُسْبَانًا بهما تعرف الأزمنة والأوقات، فتنضبط بذلك أوقات العبادات، وآجال المعاملات، ويعرف بها مدة ما مضى من الأوقات التي لولا وجود الشمس والقمر، وتناوبهما واختلافهما - لما عرف ذلك عامة الناس، واشتركوا في علمه، بل كان لا يعرفه إلا أفراد من الناس، بعد الاجتهاد، وبذلك يفوت من المصالح الضرورية ما يفوت \cite{tafsir_Saadi}.

بدوران الأرض حول نفسها وحول الشمس يعرف تتابع الليل والنهار حيث قال تعالى: \quranayah*[25][62]{\footnotesize \surahname*[25]}، ومقدار كل منهما 12 ساعة على خط الإستواء كما بين ذلك النبي ﷺ في قوله: يَومُ الجُمُعةِ ثِنْتا عَشرةَ ساعةً. ودوران القمر حول الأرض ومعها حول الشمس يعرف به منازل القمر والتي بها تعرف أيام الشهر حيث قال تعالى: \quranayah*[36][39]{\footnotesize \surahname*[36]}، ومقداره 29 أو 30 يوما كما بين ذلك النبي ﷺ في قوله: الشَّهْرُ هَكَذَا وهَكَذَا. يَعْنِي مَرَّةً تِسْعَةً وعِشْرِينَ، ومَرَّةً ثَلَاثِينَ.\footnote{ صحيح البخاري: الرقم.} ودوران الأرض والقمر جميعا حول الشمس يعرف به السنة ومقداره 12 شهر منها أربعة حرم كما بين ذلك ربنا سبحانه وتعالى في قوله: \quranayah*[9][36]{\footnotesize \surahname*[9]}. 

ويعتدل طول الليل والنهار في الأرض على خط الإستواء طول العام فيكون عنده الليل 12 ساعة والنهار (أو اليوم) 12 ساعة وبهذا يكون اليوم كاملا 24 ساعة. ولا يزيد طول اليوم الكامل عن 24 ساعة ولكن يزداد طول النهار ويقصر فوق وأسفل خط الإستواء بحسب أوقات السنة. فإذا قصر النهار طال الليل وإذا طال النهار قصر الليل بنفس ذلك المقدار. وقد جاء عن جابر بن عبد الله أن النبي قال: يَومُ الجُمُعةِ ثِنْتا عَشرةَ ساعةً. فإن كان المراد الساعة التي هي 60 دقيقة، لا يكون ذلك على طول العام إلا على خط الإستواء. ويمكن معرفة مقدار هذه الساعة بطول النهار وقصره بتقسيم عدد ساعات النهار على 12. فمثلا لو قصر طول النهار من 12 ساعة إلى 10 ساعات يكون مقدار الساعة 50 دقيقة. ولو زاد طول النهار إلى 14 ساعة يكون مقدار الساعة 70 دقيقة، وهكذا. وتأمل في أن النبي ﷺ  قد دل على عدد ساعات الليل والنهار بالمتوسط الذي لا يعتدل طول العام إلا على خط الإستواء، ولم يقيد ذلك بالمكان أو أوقات السنة فدل هذا على العموم وهذا بلا شك صحيح ومن دلائل نبوته ﷺ.

وقد ذكر سبحانه وتعالى في كتابه الكريم أنه جعل الليل والنهار خلفة، وأنه سبحانه يكور الليل على النهار ويكور النهار على الليل، وأنه سبحانه يولج الليل في النهار ويولج النهار في الليل. فلما كانا الليل والنهار في الأرض، وأن الله جل جلاله وصف حركة الليل والنهار بحركة مستقلة غير حركة الشمس والقمر، دل ذلك على أن الليل والنهار يكون بحركة الأرض وأن الأرض كروية تدور حول نفسها بزاوية مائلة بالنسبة لمدارها حول الشمس. وهذا فيه أن الله يكور الليل على النهار بدوران الأرض حول نفسها وليس بدوران الشمس عليها. فمن معنى التكوير تدوير الشئ نفسه وهذا لا يكون بتدوير الشمس على الأرض. وقد جاء في تفسير القرطبي: معنى التكوير في اللغة، وهو طرح الشيء بعضه على بعض، يقال كور المتاع أي: ألقى بعضه على بعض، ومنه كور العمامة 
\cite{tafsir_Qurtubi}. وبهذا يخلف الليل والنهار كل منهما الأخر، ويلج أحدهما في الأخر بإستمرار، فما طال من الليل يدخل في النهار والعكس بحسب مكان الأرض في مدارها وإتجاء درجة ميلانها بالنسبة للشمس. وهذا لا يتعارض بأي حال من الأحوال مع حركة الشمس. ولا يلزم من إثبات حركة الشمس هو دورانها حول الأرض فهذا إستدلال باطل فالله جل جلاله أثبت جريان الشمس بالعموم ولم يجعل ذلك دليلا على دورانها حول الأرض. كما أنه لا يلزم بالقول بدوران الأرض حول الشمس بثبات الشمس بالكلية، فهذا ينافي صريح كتاب الله جل جلاله.\footnote{وكل هذا فيه أن الأرض هي التي تدور حول الشمس وليس العكس بخلاف ما قرره العديد من أهل العلم الشرعي أخذا بظاهر الآيات مع المشاهدة فقط دون الرجوع إلى القياس أو الحساب.}

ولهذا تتابع الفصول الأربعة وهي الشتاء والربيع والصيف والخريف خلال السنة الواحدة. وهذا لأن الأرض تدور حول نفسها بزاوية دوران مائلة تقدر 5.23 درجة بالنسبة لمحور دورانها حول الشمس. وهذه الزاوية هي التي تتسبب في طول النهار وقصره فوق وأسفل خط الإستواء خلال السنة الواحدة. فلو كانت الأرض تدور حول نفسها بشكل عمودي بالنسبة للشمس لما كانت الفصول الأربعة ولما قصر أو طال النهار خلال العام الواحد في نفس البلد. ولو كانت الأرض تدور حول نفسها بشكل أفقي بالنسبة للشمس لما كانت الفصول الأربعة ولأستغرق الليل والنهار عاما كاملا. ولك أن تتصور حكمة الله البالغة في ميل زاوية دوران الإرض بالنسبة للشمس بهذه الدرجة الدقيقة لكي تتابع كل المواسم الأربعة خلال العام الواحد فتكون بذلك فترة كل موسم 3 شهور وتتابع جميعها خلال 12 شهرا لتتم العام الواحد كما يتتابع الليل والنهار خلال 24 ساعة ليتم اليوم الكامل. وتتباع هذه المواسم ابتداءا من الشتاء الذي يقصر فيه النهار فيبرد فيه الجو وتكثر فيه الأمطار، والربيع الذي يتوسط فيه النهار بعد قصره فيعتدل فيه الجو بعد برودته فتتلون فيه الأشجار، والصيف الذي يطول فيه النهار بعد توسطه فيسخن فيه الجو بعد إعتداله، إلى الخريف الذي يتوسط فيه النهار بعد طوله فيعتدل فيه الجو بعد سخونته فتسقط الأشجار أوراقها. وتتفاوت الأشجار والحيونات ومصالح الناس في هذه الفصول والمواسم كل بحسب ما قدر الله جل جلاله له وهذا من فضل الله ورحمته.

وقد ذكر الله جل جلاله هذه الفصول والمواسم التي بتتابعها تسقط الأمطار فتنموا الأشجار وتتلون ثم تصفر فتسقط أوراقها كما في قوله تعالى: \quranayah*[39][21]{\footnotesize \surahname*[39]}. يقول ابن كثير في تفسيره: وكثيرا ما يضرب الله تعالى مثل الحياة الدنيا بما ينزل الله من السماء من ماء، وينبت به زروعا وثمارا، ثم يكون بعد ذلك حطاما، كما قال تعالى: \quranayah*[18][45]{\footnotesize \surahname*[18]} \href{https://shamela.ws/book/8473/3591#p4}{\faExternalLink} \cite{tafsir_ibnKathir}. وكل هذا فيه تذكير من الله لعباده لكي يعتبروا بذلك ويستحضروا أن الله قادر على بعثهم وحسابهم كما قال تعالى: \quranayah*[30][50]{\footnotesize \surahname*[9]}. \comment{وجعل سبحانه ذلك مثلا على الحياة الدنيا كما في قوله تعالى: \quranayah*[10][24]{\footnotesize \surahname*[10]}.}

\section{طرق الإستدلال وبطلان الأخذ بالمشاهدة فقط}

ولتقرير حقيقة أن الليل والنهار يكون بدوران الأرض حول نفسها وليس بدوران الشمس عليها، وجب بيان طرق الإستدلال لإقامة الحجة الشرعية والكونية على هذا الأمر. فمن المعلوم أن الأشياء الظاهرة في العلم الكوني السببي تكون قابلة للقياس والحساب ولهذا يمكن إثباتها بالعلم التجريبي بما تقوم به الحجة العقلية. وهذا أنما يكون في العلم الكوني السببي الظاهر وليس في العلم الشرعي أو العلم الغير ظاهر كعلم الغيب الذي لا يدرك إلا بالوحي من عند الله تبارك وتعالى. ولكن الأخذ بالمشاهدة فقط في تقرير الحقائق الكونية دون الإعتبار بطرق الإستدلال الأخرى يؤدي إلى الخطأ في وصفها على الوجه الصحيح. فالعديد من الظواهر الكونية السببية القابلة للقياس لا تظهر بوضوح للمشاهد، وقد تظهر بشكل مغاير تماما للواقع، وقد لا تظهر بالكلية. فعدم مشاهدتها لا يلزم عدم وجودها، كما أن مشاهدتها على وجه معين لا يلزم حقيقتها إلا إذا توفرت الحجة في إثبات ذلك مع عدم وجود الموانع بطرق الإستدلال الأخرى. ولهذا لا يمكن الإعتماد على الإستدلال بالمشاهدة فقط في تقرير حقائق الأشياء. بل إن ذلك قد يكون مضللا في العديد من الأحيان.

إن الفرق بين الإستدلال بالمشاهدة فقط والإستدلال لمعرفة حقائق الأشياء كالفرق بين النظر والبصر. فالنظر فيه رؤية الشئ على ظاهره وأما البصر فيه إدراك الشئ على حقيقته. فلا يلزم أن يكون ظاهر الأشياء موافق لحقيقتها على الإطلاق. فالعديد من الأشياء قد لا تدرك على حقيقتها بمجرد النظر فقط كما هو معروف مع السراب. ولقد ضرب الله جل جلاله السراب مثلا على أعمال الكفار في قوله تعالى: \quranayah*[24][39]{\footnotesize \surahname*[24]}. وقد جاء في تفسير البغوي: "السراب" الشعاع الذي يرى نصف النهار عند شدة الحر في البراري ، يشبه الماء الجاري على الأرض يظنه من رآه ماء ، فإذا قرب منه انفش فلم ير شيئا \cite{tafsir_Baghawi}. وهذا فيه أن الإستدلال بالمشاهدة فقط على وجود الماء من بعيد في وضح النهار دون تأكيد ذلك بطرق الإستدلال الأخرى يكون غير صحيح وغير موافق للواقع. 


وبهذا يتبين بطلان الإستدلال بظاهر الأشياء فقط في تقرير حقائقها دون إتباع طرق الإستدلال الأخرى التي أرشدنا الله تعالى إليها. ولأن الأخذ بالنظر فقط لا يلزم بذلك إدراك حقيقة الأشياء كما هو الحال مع البصر، فإن البصر أكمل من النظر. فالبصر هو النظر مع إدراك الأشياء على حقيقتها بقدر المستطاع مع معرفة مراد الله منها والإعتبار بها. ولهذا تكرر في كتاب الله جل جلاله أن العبرة تكون لأولي الأبصار ولم يأتي لفظ أولي الأنظار ولو لمرة واحدة. فدل ذلك على أن العبرة في معرفة الحق وإتباعه والإنتفاع به وليس فقط مجرد النظر في الأشياء دون تعقلها والتفكر فيها لإدراك حقيقتها والإعتبار بها.
وقد جاء هذا المعنى في قوله تعالى: \quranayah*[7][198]{\footnotesize \surahname*[7]}. فقد أثبت الله جل جلاله مجرد النظر لهم على الظاهر ونفى عنهم البصر وهو إدراك الشئ على حقيقيته. وجاء في تفسير السعدي رحمه الله معنى ذلك: فتراهم ينظرون إليك، وهم لا يبصرون حقيقة [.] وقيل: إن الضمير يعود إلى المشركين المكذبين لرسول اللّه صلى الله عليه وسلم، فتحسبهم ينظرون إليك يا رسول اللّه نظر اعتبار يتبين به الصادق من الكاذب، ولكنهم لا يبصرون حقيقتك وما يتوسمه المتوسمون فيك من الجمال والكمال والصدق \cite{tafsir_Saadi}.

\comment{
ومن ذلك أن الله عز وجل نفى إدراك الأبصار له على الإطلاق في قوله تعالى: \quranayah*[6][103]{\footnotesize \surahname*[6]}. مع إثبات النظر إليه في الجنة كما في قوله تعالى: \quranayah*[75][22-23]{\footnotesize \surahname*[75]}. وجاء بيان ذلك في تفسير السعدي رحمه الله: (لَا تُدْرِكُهُ الْأَبْصَارُ) لعظمته، وجلاله وكماله، أي: لا تحيط به الأبصار، وإن كانت تراه، وتفرح بالنظر إلى وجهه الكريم، فنفي الإدراك لا ينفي الرؤية، بل يثبتها بالمفهوم. فإنه إذا نفى الإدراك، الذي هو أخص أوصاف الرؤية، دل على أن الرؤية ثابتة. فإنه لو أراد نفي الرؤية، لقال "لا تراه الأبصار" ونحو ذلك \cite{tafsir_Saadi}. 

 كما أن الله جل جلاله وصف نفسه بالبصير وهو الخبير المطلع على كل من في السموات والأرض كما في قوله تعالى:  
\quranayah*[49][18]{\footnotesize \surahname*[49]}. \comment{كما أنه سبحانه مطلع على أعمال العباد على حقيقتها ولا يخفى منها شئ كما في قوله تعالى:
\quranayah*[2][110]{\footnotesize \surahname*[2]}.}

ولهذا فإن جميع الأنبياء عليهم الصلاة والسلام وصفوا بالبصر لإعتبارهم بآيات الله على وجهها الصحيح كما في قوله تعالى عن يعقوب عليه السلام: \quranayah*[12][96]{\footnotesize \surahname*[12]}. ولهذا أثبت الله سبحانه وتعالى صفة البصر له ولأبويه إبراهيم وإسحاق عليهم السلام في قوله تعالى: \quranayah*[38][45]{\footnotesize \surahname*[38]}. وقال تعالى في حق من يدعوا الناس لدعوة الحق: \quranayah*[12][108]{\footnotesize \surahname*[12]}.
}

وطرق الإستدلال التي هي من أسباب إدراك حقائق الأشياء الظاهرة ثلاثة: السمع والبصر والعقل. وقد جائت مجتمعة في قوله تعالى: \quranayah*[67][23]{\footnotesize \surahname*[67]}. ولكن الله جل جلاله جعل هذه الطرق على مراتب وأساس ذلك العقل، فمتى أغلق العقل إنسد طريق الإستدلال بالسمع والبصر. قال تعالى: 
\quranayah*[2][171]{\footnotesize \surahname*[2]}. قال السعدي رحمه الله في تفسيره: فهم يسمعون مجرد الصوت، الذي تقوم به عليهم الحجة، ولكنهم لا يفقهونه فقها ينفعهم، فلهذا كانوا صما، لا يسمعون الحق سماع فهم وقبول، عميا، لا ينظرون نظر اعتبار، بكما، فلا ينطقون بما فيه خير لهم. والسبب الموجب لذلك كله، أنه ليس لهم عقل صحيح، بل هم أسفه السفهاء، وأجهل الجهلاء \cite{tafsir_Saadi}.\comment{وقال تعالى: 
\quranayah*[8][22]{\footnotesize \surahname*[8]}. وقال تعالى: 
\quranayah*[25][44]{\footnotesize \surahname*[25]}.} وقال تعالى: 
\quranayah*[22][46]{\footnotesize \surahname*[22]}. فدل على أن العمي لا يكون بمجرد إنتفاء النظر ولكن إنتفاء البصيرة والتي محلها القلب. وجاء في تفسير ابن كثير رحمه الله: أي: ليس العمى عمى البصر، وإنما العمى عمى البصيرة، وإن كانت القوة الباصرة سليمة فإنها لا تنفذ إلى العبر، ولا تدري ما الخبر \cite{tafsir_ibnKathir}. وجاء في تفسير البغوي: معناه أن العمى الضار هو عمى القلب، فأما عمى البصر فليس بضار في أمر الدين \cite{tafsir_Baghawi}. 

ولقد ذم الله تبارك وتعالى الكافرين في عدم قدرتهم على الإنتفاع بطرق الإستدلال رغم تعددها وتنوعها والتي هي من أعظم أسباب معرفة الحق وإتباعه. ولم يذكر سبحانه وتعالى النظر فقط في طرق الإستدلال بل أضاف السمع والعقل مع البصر كما في قوله تعالى: \quranayah*[10][42-43]{\footnotesize \surahname*[10]}. يقول السعدي في تفسيره لهذه الآيات: وهذا الاستفهام، بمعنى النفي المتقرر، أي: لا تسمع الصم الذين لا يستمعون القول ولو جهرت به، وخصوصًا إذا كان عقلهم معدومًا [.] وأما سماع الحجة، فقد سمعوا ما تقوم عليهم به حجة الله البالغة، فهذا طريق عظيم من طرق العلم قد انسد عليهم، وهو طريق المسموعات المتعلقة بالخير. ثم ذكر انسداد الطريق الثاني، وهو: طريق النظر فقال: {وَمِنْهُمْ مَنْ يَنْظُرُ إِلَيْكَ} فلا يفيده نظره إليك، ولا سبر أحوالك شيئًا، فكما أنك لا تهدي العمي ولو كانوا لا يبصرون، فكذلك لا تهدي هؤلاء. فإذا فسدت عقولهم وأسماعهم وأبصارهم التي هي الطرق الموصلة إلى العلم ومعرفة الحقائق، فأين الطريق الموصل لهم إلى الحق؟ \cite{tafsir_Saadi}. وهذا فيه بيان طرق الإستدلال الموصل إلى الحق وهذا يشمل السمع والبصر والعقل معا.

ولهذا فإن الأخذ بالنظر (أي المشاهدة) فقط دون الإعتبار بطرق الإستدلال الأخرى وخصوصا مع وجود الموانع يؤدي إلى الخطأ في الإستدلال كما هو الحال في القول بثبات الأرض بالكلية وبدوران الشمس حولها. وهذا مخالف لما أرشد الله تعالى إليه في كتابه العظيم ومن ذلك وجوب الإعتبار بالحجة العقلية والتي جاءت في هذا الأمر على وجه الخصوص في قوله تعالى: \quranayah*[3][190]{\footnotesize \surahname*[3]}. فدل ذلك على وجوب الإعتبار بالحجة العقلية مع النظر لتحقق البصيرة التي بها يمكن إدراك الحقيقة كما في قوله تعالى: \quranayah*[24][44]{\footnotesize \surahname*[24]}. ويقول السعدي رحمه الله في تفسيره: أي: لذوي البصائر، والعقول النافذة للأمور المطلوبة منها، كما تنفذ الأبصار إلى الأمور المشاهدة الحسية. فالبصير ينظر إلى هذه المخلوقات نظر اعتبار وتفكر وتدبر لما أريد بها ومنها، والمعرض الجاهل نظره إليها نظر غفلة، بمنزلة نظر البهائم \cite{tafsir_Saadi}. ومن طرق الإستدلال التي دل الله جل جلاله إليها في معرفة حقيقة الشمس والقمر والليل والنهار على وجه الخصوص العقل والحساب. فعدم الإعتبار بالعقل والحساب مخالفة واضحة وصريحة لكتاب الله جل جلاله. وعليه فإن تنوع طرق الإستدلال من سمع وبصر وعقل وغير ذلك من طرق القياس والحساب الموافقة للعقل يكون واجبا في تقرير الحقائق الكونية لمعرفة الحق بالحجة والبراهين الثابتة وليس مجرد الآراء والإجتهادات التي لا تقوم بها الحجة البالغة.

\section{مفاهيم خاطئة حول الأرض والشمس والقمر}

وأما من فسر كلام الله على أن الأرض في حقيقتها ثابتة بالكلية لا تدور حول نفسها وأن الشمس هي التي تدور عليها أخذا بالمشاهدة فقط ودون الإعتبار بطرق الإستدلال الأخرى فهذا كلام خاطئ لا يصح عقلا ولا يثبت شرعا. وإن فُهِمَ أن ظاهر الأدلة في ذلك وهذا لأن الله جل جلاله وصف هذه الآيات على وجهين. الوجه الأول وصفها بالنسبة لظاهرها مثل قوله تعالى: 
\quranayah*[2][258][24-35] {\footnotesize (\surahname*[2])}. وقوله تعالى: 
\quranayah*[18][17][1-14] {\footnotesize (\surahname*[18])}. وأما الوجه الثاني وصفها بالنسبة لحقيقتها مثل قوله تعالى: 
% \quranayah*[14][33] {\footnotesize (\surahname*[14])}. وقوله تعالى: 
\quranayah*[21][31-33] {\footnotesize (\surahname*[21])}. وقوله تعالى: 
\quranayah*[36][37-40] {\footnotesize (\surahname*[36])}. 

فالوجه الأول فيه بيان هذه الآيات على ظاهرها بحسب المشاهدة وبما تفهمه العقول.
فالله جل جلاله يخاطب الناس بما يناسب عقولهم كما في قوله تعالى: \quranayah*[30][27][7-9] {\footnotesize (\surahname*[30])} أي بالنسبة للعقول فالله على كل شئ قدير. وأما الوجه الثاني فيه بيان هذه الآيات على حقيقتها بالطريقة التي سخرها الله بها. ولا يلزم أن يكون ظاهر الأشياء موافق لحقيقتها وبالأخص فيما يتعلق بحركة الشمس والقمر والأرض. ولمعرفة الحق في ذلك لابد من الإستدلال بالطرق التي دل الله عليها ومن ذلك العقل مع السمع والبصر وغير ذلك من طرق القياس الأخرى. وقد أرشد سبحانه وتعالى إلى الحساب عند ذكره للشمس والقمر والليل والنهار في العديد من المواضع في كتابه العظيم فدل على وجوب الأخذ والإستدلال به لمعرفة هذه الحقائق الكونية. وأما الإستدلال بالمشاهدة فقط دون الإعتبار بطرق الإستدلال الأخرى التي أرشد الله تعالى إليها فهذا يخالف كتاب الله جل جلاله. ولهذا يجب أن يكون الإستدلال بالمشاهدة معتبرا في تقرير حقائق هذه الأشياء إذا كان مدعوما بالحجة البالغة والدليل القاطع ولا يوجد ما ينافي ذلك من طرق الإستدلال الأخرى.

وأما الإستدلال بثبات سطح الأرض أنها ثابتة بالكلية فهذا لا يقوم بالحجة العقلية. فثبات سطحها لا ينافي دورانها حول نفسها بالنسبة لمن هو عليها كالذي يثبت على سطح المركوب المنتظم في السير. كما أن ثبات سطح الأرض لا يلزم دوران الشمس عليها. وهذا لأنه بالنسبة لمن هو على الأرض فإنه يشاهد الشمس تدور عليه بتعاقب الليل والنهار. ولكن في حقيقة الأمر أنه ثابت على سطح الأرض التي تدور حول نفسها في غاية الإنتظام فلا يشعر بدورانها ويظهر له أن الشمس تدور عليه ولكن الأرض الكروية هي التي تدور حول نفسها وهذه الدورة الكاملة هي التي مقدارها 24 ساعة والتي فيها يتعاقب الليل والنهار. 

وقد ثبت بالقياس دوران الأرض حول نفسها بتجربة بندول فوكو.\footnote{ثبت بالتجربة أن البندول المتأرجح ثابت الإتجاه نتيجة لقانون حفظ الزخم الزاوي. وقد وجد أن البندول المتأرجح يدور دورة كاملة خلال 24 ساعة عند القطب الشمالي. كما أمكن أيضا حساب فترة دورانه بحسب موقع القياس. يعتبر هذا من أبسط وأقوى الأدلة العقلية على دوران الأرض حول نفسها.} وفي المقابل فإن القول بثبات الأرض بالكلية ودوران الشمس عليها لا يثبت حتى بالمشاهدة لأطوار الزهرة بل أن ذلك موافق لدوران الزهرة حول الشمس. كما أنه قد ثبت أن للمشتري أربعة أقمار كبيرة تدور عليه. كما أنه لا يمكن تفسير إختلاف مواقع النجوم في السماء خلال العام الواحد إلا بدوران الأرض حول الشمس وليس العكس. وكل هذا بخلاف القول أن كل شئ يدور حول الأرض بما في ذلك الشمس. وقد تبين بالحساب أن النماذج الرياضية على أساس دوران الكواكب في المجموعة الشمسية حول الشمس أعم وأشمل وأدق بكثير من النماذج التي تكون على أساس مركزية الأرض. وهذا لأن هذه الأخيرة لا يمكنها تفسير حركة الكواكب إلا بكثير من التعقيد والتعديلات التي تنافي الحركة الطبيعية لهذه الأفلاك.

ومن جعل الأرض ثابتة بالكلية لا تدور حول نفسها ولا تسير في فلكها فقد خالف النقل والعقل ونفي عنها السير  في الفلك والسباحة فيه وهي الدوران. بل ويترتب على هذا الفهم الخاطئ معارضة الآيات الشرعية التي تدل على تكوير الليل والنهار في الأرض بدورانها حول نفسها لأن هذه الحركة جائت مستقلة دون حركة الشمس والقمر كما في قوله تعالى: \quranayah*[21][33]{\footnotesize \surahname*[21]}\comment{\quranayah*[36][40]{\footnotesize (\surahname*[36])}}. ومن قال أن هذه الحركة لا تشمل الأرض فقد خالف سياق الآيات التي عممت هذا الحكم الكوني على هذه الأفلاك والتي إشتملت على الليل والنهار اللذان يكونان في الأرض. ويلزم بهذا النفي إنكار وجود الليل والنهار في الأرض وهذا لا ينكره عاقل. فجميع آيات "وكل في فلك يسبحون" إشتملت على الليل والنهار فدل بشكل لا لبس فيه أن الأرض في فلك تسبح أي تدور حول نفسها وتسير في فلكها حول الشمس. ومما يوضح هذا المعنى أن الله عز وجل وصف النهار بأنه يجلي الشمس وأن الليل يغشي الشمس في قوله تعالى: \quranayah*[91][1-4]{\footnotesize \surahname*[91]}، فدل هذا على أن الليل والنهار يكونان بحركة مستقلة وهي حركة الأرض وليس حركة الشمس. ولهذا لما كان سياق الآيات السابقة عن الشمس، جعل الله تعالى ظهور الشمس بالنهار وغيابها بالليل. وعندما كان سياق الآيات عن الليل والنهار لم يجعل ذلك مقيدا بحركة الشمس كما في قوله تعالى: \quranayah*[92][1-2]{\footnotesize \surahname*[92]}، فجعل سبحانه تجلي النهار وتغشى الليل مستقلان فدل على أن الليل والنهار يكونان بإستمرار دوران الأرض حول نفسها وأن الشمس ثابتة بالنسبة للأرض التي تدور عليها، والله فوق كل ذي علم عليم \href{https://www.youtube.com/watch?v=GnZ3dogED7w}{\faExternalLink}.

فدورة الأرض الواحدة حول نفسها تستغرق 24 ساعة، ودورة القمر حول نفسه مقيدة بدورانه حول الأرض فيتم دورة كاملة عند تمام دورته على الأرض ومقدارها 29 أو 30 يوما. ولهذا فإن وجه القمر ثابت لا يتغير بالنسبة لمن هو في الأرض. وقد ثبت أن الشمس تدور حول نفسها ومقدار ذلك 27 يوم وقد صور ذلك بالتلسكوبات. وقد ثبت أيضا أن الشمس وما معها من كواكب في هذا النظام الشمسي تدور جميعا حول مجرتنا كما تدور الأرض ومعها القمر حول الشمس. ويمكن معرفة ذلك بحركة النجوم في السماء والتي جعلها الله تبارك وتعالى مرجعا للأرض وللنظام الشمسي كله للإهتداء بها كما في قوله تعالى: \quranayah*[6][97]{\footnotesize (\surahname*[6])}. فدل هذا على ثباتها بالنسبة للمجموعة الشمسية وأنها تدور معها وهذا من فضل الله ورحمته بعباده. ولا يلزم بالقول بدوران الأرض حول نفسها أن الشمس ثابتة لا تتحرك بالكلية وإنما الشمس ثابتة بالنسبة لمن يدور عليها من الكواكب ولكنها تدور حول نفسها وتتحرك حول المجرة مع كواكبها في النظام الشمسي كما أن الأرض ثابتة بالنسبة للقمر وهم معا يدورون حول الشمس.

وقد أرشدنا الله جل جلاله أن كل الأفلاك في فلك يسبحون، وهذا يعنى أن الأرض لها فلك أي مدار تسير عليه كما للشمس والقمر وسائر الكواكب والنجوم الأخرى وأن كل منهم يسبح في هذه المدارات أي في حركة مستمرة يدور حول نفسه بحسب ما قدر الله عز وجل لها. فالفلك هو المدار والسباحة هي الدوران. وقد جاء في تفسير ابن كثير عن معنى قوله تعالى: \quranayah*[36][40][12] {\footnotesize (\surahname*[36])} أي: يدورون. قال ابن عباس: يدورون كما يدور المغزل في الفلكة. وكذا قال مجاهد: فلا يدور المغزل إلا بالفلكة، ولا الفلكة إلا بالمغزل، كذلك النجوم والشمس والقمر، لا يدورون إلا به، ولا يدور إلا بهن \href{https://shamela.ws/book/8473/2750#p4}{\faExternalLink} \cite{tafsir_ibnKathir}. فتأمل تشبيه الصحابة لحركة الأفلاك بحركة المغزل في الفلكة وهذا فيه أن الصحابة رضوان الله عليهم قد فهموا ذلك فهما صحيحا من النبي ﷺ. وهذا موافق للعلم الحديث المؤكد وفيه دليل نبوته ﷺ وأن هذا الوصف لا يأتي قبل 1400 عام إلا من العليم الخبير الذي خلق هذا النظام وأتقنه سبحانه وتعالى. فمن المعروف أن المغزل لا يغزل به إلا بتدويره حول نفسه وهذا فيه وصف السباحة وهي الدوران. ومن تمام معرفتهم أنهم شبهوا مدار الأفلاك بالمغزل الذي يسير في الفلكة أي في مداره وهذا فيه إثبات السير مع الدوران لكل الأفلاك. وهذا يدل على أن الأرض تدور حول نفسها كما يدور القمر والشمس وسائر الكواكب والنجوم الأخرى. وأنها جميعا تسير مع الدوران في مداراتها ومن ذلك حركة القمر حول الأرض وحركة الأرض حول الشمس وحركة الشمس حول المجرة وكذالك سائر النجوم والكواكب الأخرى تدور حول نفسها وتسير في أفلاكها وهذا معروف وثابت بالمشاهدة والقياس والحساب وبالأخص في النظام الشمسي. 

وبهذا يعلم أن الأفلاك الصغيرة تتبع الأفلاك الأكبر حجما والأثقل وزنا وكلها تتابع في سلسلة مقيدة بذلك لا تحيد عنه ولا تميل في نظام بديع ودقيق وجميل لا يدل إلا على عظمة الخالق وعظيم سلطانه وكمال علمه وواسع فضله. فكل الأفلاك تسير في حركة مستمرة ومتناسقة ومنتظمة بإستمرار تدور حول نفسها وتسير في مداراتها.\footnote{وهذا الدوران والسير يكون غالبا عكس عقارب الساعة بالنسبة للأقطاب الأفلاك الشمالية. والسير يكون في حركة دائرية تشبه حركة الطواف حول الكعبة. وكل ميسر لما خلق له بأمر الله تعالى.} ومن ذلك حركة الشمس والقمر والليل والنهار في الأرض كما في قوله تعالى: \quranayah*[14][33]{\footnotesize (\surahname*[14])}. يقول السعدي رحمه الله: (وَسَخَّرَ لَكُمُ الشَّمْسَ وَالْقَمَرَ دَائِبَيْنِ) لا يفتران، ولا ينيان، يسعيان لمصالحكم، من حساب أزمنتكم ومصالح أبدانكم، وحيواناتكم، وزروعكم، وثماركم \href{https://shamela.ws/book/42/905#p5}{\faExternalLink} \cite{tafsir_Saadi}. وتستمر هذه الحركة المنتظمة إلى أن يشاء الله كما في قوله تعالى: \quranayah*[39][5]{\footnotesize (\surahname*[39])}. \href{https://shamela.ws/book/42/905#p5}{\faExternalLink} \cite{tafsir_Saadi} يقول السعدي رحمه الله: (لِأَجَلٍ مُسَمًّى) وهو انقضاء هذه الدار وخرابها، فيخرب اللّه آلاتها وشمسها وقمرها، وينشيء الخلق نشأة جديدة ليستقروا في دار القرار، الجنة أو النار \href{https://shamela.ws/book/42/1623#p4}{\faExternalLink} \cite{tafsir_Saadi}.

وقد ظن الناس قديما أن الشمس بل والسماء كلها هي التي تدور حول الأرض أخذا بالمشاهدة فقط لقلة علمهم بالحساب وعدم توفر أدوات القياس وعدم الإعتبار بمنازل القمر وفصول السنة ومواقع النجوم في السماء وماذا يترتب على هذا القول الخاطئ. فلو كانت الشمس هي التي تدور حول الأرض بنفس زاوية الميل لتعاقبت علينا الفصول الأربعة خلال اليوم والليلة الواحدة وهذا لا يكون. ولو كانت الشمس هي التي تدور حول الأرض لوافقت منازل القمر تعاقب الليل والنهار وكان بذلك الشهر هو نفسه اليوم والليلة وهذا لا يكون. بل إن هذا يلزم أن القمر يتبع الشمس في دورتها اليومية حول الأرض وأن الشمس تدور على القمر مرة في الشهر رغم بعدها فلزم أن تكون الشمس في حركة سريعة جدا يتنافى مع طبيعة هذه الأفلاك. وهذا يلزم أيضا أن الشمس تدرك القمر مرة في الشهر خلال دورانهم حول الأرض والله جل جلاله نفى أن تدرك الشمس القمر على الإطلاق. والعديد من الظواهر الطبيعية الأخرى كظاهرة المد والجزر التي لا تستقيم بالقول بثبات الأرض ودوران الشمس عليها. وهذا الظن سببه أن الإنسان لا يشعر بدوران الأرض ولا بكرويتها. وهذا أولا لأن حركة الأرض في مدارها مع دورانها حول نفسها في غاية الإنتظام والتناسق، وثانيا لأن حجم الأرض حتى بالنسبة لنظر الإنسان كبير جدا فيصعب عليه إدراك كرويتها فضلا عن إدراك حركتها دون الإبتعاد عنها.

ولا يستقيم أن يقال الأرض مسطحة في شكلها بالكلية وإنما قد تبدوا مسطحة بالنسبة للمشاهد لكبر حجمها ولكنها كروية في شكلها الكلي. فلو كانت مسطحة في شكلها بالكلية لإختلت كل الظواهر الطبيعية الأخرى التي نعرفها. ومن ذلك تعطل منازل القمر وإنعدام ظاهرة الكسوف والخسوف وغيرها من الظواهر الأخرى. فمن قال أن الأرض مسطحة جعل الشمس والقمر على مسافة واحدة من الأرض ولو كان ذلك حقا لما حجب القمر الشمس في ظاهرة الكسوف ولضرب كل منهما الآخر ولختل هذا النظام كله ولما حجبت الأرض ضوء الشمس عن القمر في ظاهرة الخسوف. ولو كان القمر أبعد من الشمس لما حجب القمرُ الشمسَ ولما رأينا الخسوف على البدر في ظلمة الليل ولتعطلت منازل القمر فما رأينا إلا البدر. وأما من جعل الأرض مسطحة ولكن القمر أقرب من الشمس فقد عطل منازل القمر ولما رأينا البدر أبدا فضلا عن الخسوف في ظلمة الليل. 

ودوران الأرض حول نفسها مع كرويتها لا يتعارض مع كونها ممهدة ومسطحة بالنسبة لمن يعيش فيها ويمشي عليها. فمن المعروف أن السطح المكور لا يعرف تكويره بالنسبة لمن هو عليه إن كبر حجمه أو صغر جزء القياس والنظر فيه. ولهذا تعرف كروية الأرض بالإستدلال بكروية الشمس والقمر وبالمشاهدة كما هو معروف وثابت بالصور الملتقطة من الفضاء. وبهذا فإن طرق الإستدلال الكونية والشرعية توافقت على كروية الأرض ودورانها حول نفسها وحول الشمس وبطلان القول بدوران الشمس عليها.

\section{مناقشة علمية لأقوال أهل العلم في حقيقة حركة الأفلاك}

لا شك أن حب العماء والتأدب معهم ولزوم غرزهم من أعظم أسباب الصلاح في الدنيا والآخرة. ولكن ذلك لا يمنع مخالفتهم متى كان الحق في خلاف قولهم مع ثبوت الدليل بذلك. فأهل السنة والجماعة يدورون مع الحق أينما كان ويعرفون الرجال بالحق وليس الحق بالرجال. ولا ينقص من قدر أهل السنة والجماعة أن يخطئوا في بعض المسائل الإجتهادية التي لا تخالف أصول الدين ومنهج أهل الحق. ولهذا فإن دراسة أقوال أهل العلم لبيان الحق وليس لإستنقاصهم والبحث فيها لمعرفة الراجح منها يكون من شهادة القسط التي أوصى الله تعالى بها في قوله تعالى: 
\quranayah*[4][135]{\footnotesize (\surahname*[4])}. وقوله تعالى: \quranayah*[5][8]{\footnotesize (\surahname*[5])}. ولهذا كان من الواجب النظر في أقوال أهل العلم فيما يتعلق بحقيقة الشمس والقمر والأرض وعرض ذلك على ما تقدم من الأدلة.

نقل شيخ الإسلام بن تيمية رحمه الله عن السلف وكذلك اجماع أهل العلم في زمانه على أن كل الأفلاك كروية، راجع \fullautoref{sec:app_round_planets_teimia}. إلا أن شيخ الإسلام نقل أيضا أن الأرض في وسط السماء وأن السماء كالقبة تدور حول الأرض نظرا لأن جميع الكواكب تدور جميعا من المشرق إلى المغرب على ترتيب واحد. وهذا صحيح بالنسبة للمشاهد فقط كالذي يرى أن الشمس تدور عليه بتعاقب الليل والنهار. ولكن في حقيقة الأمر أن الأرض هي التي تدور حول نفسها فتبدو للمشاهد وكأن السماء بكاملها تدور عليه. ولقد تقدم بطلان الأخذ بالمشاهدة فقط في تقرير حقائق الأشياء وبالأخص فيما يتعلق بالشمس والقمر والأرض. فلو كانت السماء هي التي تدور حول الأرض لما تغيرت مواضع النجوم التي يهتدى بها خلال العام الواحد ولزم بهذا القول أن نجوم الإستدلال ثابتة في اليوم والليلة كما في العام الواحد وهذا غير صحيح. فلا يمكن تفسير حركة النجوم في السماء إلا بالقول بدوران الأرض حول الشمس وليس العكس. وبهذا لا يلزم أن تكون الأرض في وسط السماء ومحور الكون كله ومركزه بل هي في جزء صغير جدا منه تدور حول نفسها وتسير في مدارها حول الشمس.

ولقد قرر الشيخ العثيمين رحمه الله تعالى أن الشمس هي التي تدور حول الأرض واستدل بقوله تعالى: \quranayah*[18][17][1-14] {\footnotesize (\surahname*[18])}. وقال: فهذه أربعة أفعال كلها مضافة إلى الشمس، إذا طلعت تزاور، وإذا غربت تقرضهم. والأصل في إضافة الفعل إلى فاعله أنه قائم به فترى الشمس هي التي تطلع، وهي التي تزاور، وهي التي تغرب، وهي التي تقرض [.] فعلينا أن نؤمن بهذا الظاهر إلا إذا جاءنا أمر يقين مثل الشمس: أن الشمس ثابتة، وأن تعاقب الليل والنهار يكون بدوران الأرض فحينئذٍ نؤمن بهذا، ويمكن أن يؤول القرآن إلى أن معنى قوله: (إذا طلعت) أي: في رأي العين؛ لأن على من يقول: إن تعاقب الليل والنهار بسبب دوران الأرض: هل هي التي طلعت علينا؟ أم نحن الذين طلعنا عليها؟ نحن الذين طلعنا عليها؛ لأننا نحن الذين جئنا إليها، ما دام تعاقب الليل والنهار يكون بدوران الأرض معناه: نحن الذين جئنا إليها حتى رأيناها، وهي واقفة في مكان ثم دارت الأرض، نحن الذين جئنا ونحن الذين طلعنا عليها. وعلى كل حال: نحن لا نعلم الغيب إلا ما علمنا الله -عز وجل-، فنؤمن بظاهر القرآن، ونقول: إن الشمس والقمر يكون بسيرها تعاقب الليل والنهار، هذا هو الظاهر لنا، فلو قدرنا أننا علمنا علم اليقين بأن الأمر ليس على ظاهره، أمكن أن نصرف الآيات عن ظاهرها إلى معنىً لا ينافيه \href{https://binothaimeen.net/ar/voice_library/lessonDetails/%D8%A7%D9%84%D8%A8%D8%AD%D8%AB/%D9%85%D8%B3%D8%A3%D9%84%D8%A9%20%D8%AF%D9%88%D8%B1%D8%A7%D9%86%20%D8%A7%D9%84%D8%B4%D9%85%D8%B3%20%D8%AD%D9%88%D9%84%20%D8%A7%D9%84%D8%A3%D8%B1%D8%B6/0eff5ba6-e968-4953-9475-fe75d23ab744}{\faExternalLink}. ولقد أقر الشيخ العثيمين رحمه الله أن سبب هذا الترجيح هو الأخذ بالمشاهدة فقط كما أنه أقر القول الأخر عند ثبوت الدليل. وقد تقدم معنا عدم لزوم موافقة ظاهر الأشياء لحقيقتها وأن المشاهدة فقط لا تكفي. ولقد تحققت الحجة العقلية والشرعية في أن الليل والنهار لا يكونان بدوران الشمس حول الأرض وإنما بدوران الأرض حول نفسها وحول الشمس. ومن ذلك أن النص الشرعي  جعل الليل والنهار بحركة مستقلة غير حركة الشمس والقمر فدل على أن ذلك لا يكون إلا بحركة الأرض وليس العكس.

ولقد كان الشيخ ابن باز رحمه الله يرى بثبات الأرض بالكلية فقال: "وبين سبحانه وتعالى أنه ثبتها بالجبال وأرساها وجعلها لها أوتادا. فالواجب التمسك بهذا والأخذ بهذا وأنها لا تميد ولا تضطرب ولا تدور. ولو دارت لأحسوا بها العباد من أجل الزلازل. ولو زلازل قليلة عرفها الناس. وربما هلك من حولها إذا عظمت الزلزلة وتهدمت البيوت وسقطت الأشجار وهلك الناس بأقل زلزلة."، إلا أنه قال: "ومن شاهد أشياء وتيقنها يقينا وأن هناك حركة لا تمنع وصف الأرض بأنها غير مائدة وأنها قرار وأنه دوران خاص لا ينافي كونها قرارا ولا ينافي كونها قد أرسيت بالجبال ولا ينافي كونها لا تميد، من تيقن هذا وعرفه بقلبه وصدقه بعينه فلا لوم عليه إذا اعتقد ذلك [.] فأنا أعتقد، وقد كتبت هذا في كتابا من مدة سنوات، أعتقد أنها قارة كما قال الله وأنها لا تدور ولا تضطرب ولا تتحرك بل هي ثابتة" \href{https://www.youtube.com/watch?v=nbzh7p2ZlFQ}{\faExternalLink}. ولكن الشيخ ابن باز رحمه الله لم يغلق باب البحث والإجتهاد في هذه المسألة فقال رحمه الله: "ولا يمكن أن نسلم لهم ذلك إلا بدليل من كتاب الله وسنة رسوله عليه الصلاة والسلام أو شئ نلمسه بأيدينا ونراه بأبصارنا ونعقله لا شبهة فيه. فإذا وجد ذلك أمكن تأويل أن تميد بالإضطراب الذي يضر الناس وأن الحركة التي لا تضر الناس من دوران وغيره لا تخالف الميد الذي ذكره الله" \href{https://www.youtube.com/watch?v=nbzh7p2ZlFQ}{\faExternalLink}. وكل هذا فيه أن العبرة إنما تكون بالحجة والبرهان.

فلا يلزم القول بدوران الأرض حول نفسها بحدوث الإطراب لها فقد جعل الله جل جلاله هذه الحركة في غاية الإنتظام والإتقان وذلك صنع الله الذي أتقن كل شيء. ومن فسر أن الجبال أوتادا يعني ثبات الأرض بالكلية فهذا بلا شك تفسير خاطئ. فلا تقوم بذلك الحجة بل هو حجة على عكس ذلك كما بين ذلك الشيخ الألباني رحمه الله، راجع \fullautoref{sec:app_solar_system_albani}. فقد جعل الله جل جلاله الجبال أوتادا لتثبيت سطح الأرض حتى لا تطرب ولا يلزم بهذا ثبات الأرض وعدم حركتها بالكلية. وقد جاء في تفسير ابن كثير في معنى قوله تعالى: \quranayah*[78][6-7]{\footnotesize \surahname*[78]} أي: جعلها لها أوتادا أرساها بها وثبتها وقررها حتى سكنت ولم تضطرب بمن عليها \cite{tafsir_ibnKathir}. وهذا لا يلزم ثباتها بالكلية بل ثبات سطحها. وقد جاء في معجم اللغة أن معنى أوتاد أي وتد: ما رُزَّ وثبت في الحائط أو الأرض من خشب وغيره: «وتد الخيمة». فدل ذلك على أن المراد هو ثبات سطحها وليس ثباتها بالكلية. بل أن ذلك فيه إثبات أصل الحركة للأرض وإلا فما الغاية من تثبيت شئ لا يتحرك. 

كما أن القول بثبات الأرض بالكلية بالجبال ينافي الحجة العقلية، فلا يمكن تثبيت شيئ بالكلية بوتد غرس فيه وللزم بهذا القول أن تغرس الجبال في شئ آخر ثابت غير الأرض مع أرتباطها به لتثبيتها بالكلية وهذا بخلاف الواقع. ولعل مثال ذلك أن وتد الخيمة تربط به الخيمة ويغرس في الأرض الثابتة ولو غرس في الخيمة نفسها لما ثبتت البتة. ولقد جاء هذا المعنى في حديث أنس بن مالك أن النبي ﷺ قال للأعرابيِّ الَّذي تركَ النَّاقةَ سائبةً متوكل علَى اللهِ: فقالَ لهُ: اعقِلها و توَكَّلْ \href{https://shamela.ws/book/21659/1948#p1}{\faExternalLink} \cite{jamaaSagheer}.\footnote{الجامع الصغير: 1948، وقال الألباني حديث حسن.} فلزم بذلك أن تربط الناقة في شئ آخر ثابت كالجدار أو الشجر وهذا معروف. وقد ثبت بالقياس والحساب في علم الجيولوجيا أن سطح الأرض عبارة عن صفائح تكتونية تتحرك بإستمرار وتتصادم مع بعضها البعض مما يسبب العديد من الظواهر الطبيعية الأخرى مثل تكون الجبال وتصاعد الحمم البركانية والزلازل نتيجة لهذه التصادمات والتحركات. وهذا فيه أن الله جل جلاله جعل لهذه الظواهر أسبابها بحكمته وعلمه. وقد ثبت أن الجبال لها جدور عظيمة تمتد في أعماق الأرض وكأنها مسامير تمنع طبقات سطح الأرض من الإنزلاق والتحرك وهذا من فضل الله ورحمته بعباده كما في قوله تعالى: \quranayah*[16][15]{\footnotesize \surahname*[16]}.

ويقول الشيخ الفوزان حفظه الله عندما سأل: هل الشمس تدور على الأرض؟ فقال: بلا شك، هذا هو ما دل عليه القرآن، والشمس تجري. هم يقولون لا، الشمس واقفة والأرض هي التي تجري. هذا عكس ما جاء في القرآن. وإبراهيم عليه السلام قال للنمرود: \quranayah*[2][258][26-38] {\footnotesize (\surahname*[2])}. فكون أننا نترك ما دل عليه القرآن ونأخذ بالنظريات الحديثة، هذا لا يكون من المسلم. ويجب على المسلم أن يتبع القرآن. الله جل وعلى يقول: 
\quranayah*[18][51] {\footnotesize (\surahname*[18])} \href{https://www.youtube.com/watch?v=32PPmN-Tq9g}{\faExternalLink}. وقد تقدم معنا أن الله جل جلاله وصف هذه الآيات على وجهين ولا يصح أن يستدل بظاهرها على حقيقتها. بل إن ذلك مخالف لطرق الإستدلال التي دل الله تعالى عليها في كتابه العظيم. والله جل جلاله أثبت جريان الشمس بالعموم وليس حول الأرض. كما أن دوران الأرض حول نفسها وحول الشمس لا يلزم ثبات الشمس بالكلية وإنما فقط بالنسبة للأرض. وقد تقدم أيضا موافقة الأدلة الشرعية للقول بدوران الأرض حول نفسها وحول الشمس. وكل هذه الأدلة ليست مجرد نظريات بل ثابتة بالقياس والحساب بالعلم الكوني السببي الظاهر والمؤكد والتجريبي. 

وقال الشيخ الفوزان حفظه الله في فتوى أخرى: الله جل وعلى يقول: \quranayah*[18][51][1-8] {\footnotesize (\surahname*[18])}. هذه أمور كونية لا نحيط بها [.] ظاهر القرآن أن الشمس تدور وأن الأرض مستقرة وثابتة والشمس هي التي تدور عليها وجميع الأفلاك: الشمس والقمر والنجوم، كلها تدور على الأرض. هذا هو الذي يدل عليه ظاهر القرآن. أما نظريات الفلكيين، هذه لا يعتمد عليها. والواجب على الإنسان أنه يمسك عن هذه الأمور ولا يدخل فيها. وكونها تدرس في المدارس للإطلاع فقط، ماهو بأجل أن تعتقدها أو تجزم بصحتها، فأنت تدرسها للإطلاع فقط وأن هذا شئ قيل. لكن أنت لا تصدق به ولا تعتقده \href{https://www.youtube.com/watch?v=5j5DDjVbGoQ}{\faExternalLink}. ولا يصح أن يقال: هذه أمور كونية لا نحيط بها. فالعلوم الكونية قسمان: علم ظاهر وهو العلم السببي وعلم غير ظاهر وهو علم الغيب. والعلم الظاهر السببي منه المؤكد الذي أمكن إثباته بالقياس ومنه الغير مؤكد والذي لم يمكن إثباته وهذا هو الذي يقال له نظريات. وأما جعل العلم الظاهر السببي المؤكد كله غير مؤكد ومجرد ظن فهذا لا يصح عقلا ولا شرعا. كما أن الأخذ بالنظر فقط في تقرير حقائق الأشياء لا يكون دائما صحيحا. فالله جل في علاه لم يثبت النظر فقط للإستدلال على هذه الأمور الكونية بل إن الله تعالى أرشد إلى طرق الإستدلال في العلم السببي الظاهر وهي السمع والبصر مع العقل. وذكر سبحانه وتعالى أولى الألباب وأولى الأبصار والحساب مع آيات الليل والنهار والشمس والقمر فدل على أن ذلك من الأمور التي يمكن إدراكها بطرق الإستدلال التي أرشد الله لها. كما أنه لا يوجد في القرآن ما يدل على دوران الشمس حول الأرض بل العكس. وقد تقدم بطلان مركزية الأرض والقول بأن الأرض هي محور الكون بناءا على حركة النجوم في السماء وغير ذلك من طرق الإستدلال ومنها القياس والحساب.

وقال الشيخ الفوزان حفظه الله في فتوى أخرى عندما سأل عن ذلك: والله يثبتون هذا، فإن القرآن على أن الأرض ثابتة وأن الله أرساها بالجبال، وجعل الأرض قرارا. هذا الذي في القرآن. ولأنهم عندهم مخالفة للقرآن فعليهم أن يقيمون البراهين. ولن يقيموا براهين تخالف القرآن أبدا \href{https://www.youtube.com/watch?v=ymkzpVhjJhM}{\faExternalLink}. وقد تقدم أن معنى الجبال أوتادا هو تثبيت الجبال لسطح الأرض وليس للأرض بالكلية. وبهذا تكون الأرض قرارا وسطحها ثابت وهذا لا يتنافى مع دورانها حول نفسها. بل أن هذا دليل على أصل حركتها وإلا فما الغاية من تثبيت شئ ثابت لا يتحرك. فالبراهين الكونية والشرعية متوافقة في هذا المعنى ولا تتعارض في شئ ولله الحمد.

ولقد وافق الشيخ الألباني رحمه الله الحقيقة العلمية الكونية للنظام الشمسي الحديث، فراجع \fullautoref{sec:app_solar_system_albani}. إلا أن الشيخ الألباني رحمه الله أعاز ظاهرة الفصول الأربعة إلى قرب الأرض وبعدها من الشمس وهذا غير صحيح. فمن المعروف  والمؤكد علميا أن زاوية ميلان الأرض بالنسبة لمحور دورانها حول الشمس هي التي تتسبب في ظاهرة الفصول الأربعة بشكل متعاكس في نصف الكرة الأرضية الشمالي والجنوبي. في الحقيقة قد تبين أن الأرض تكون أقرب إلى الشمس في فصل الشتاء في نصفها الشمالي من فصل الصيف. ولكن يتعرض كل نصف لأشعة شمس أقل أو أكثر بحسب اتجاه زاوية ميلان الأرض بالنسبة للشمس. ولهذا يتعرض النصف الشمالي للأرض في الشتاء لأشعة شمس أقل من النصف الجنوبي وهذا يسبب في برودة الجو وتساقط الثلوج. والعكس في الصيف حيث تكون الأرض أبعد من الشمس ولكن زاوية ميلانها تتسبب في تعرض النصف الشمالي لأشعة شمس أكبر من النصف الجنوبي وهذا يسبب في حرارة الجو وجفافه. وتتعاكس الفصول في النصف الجنوبي من الكرة الأرضية بالنسبة للنصف الشمالي بحيث يكون الشتاء في النصف الجنوبي عندما يكون الصيف في النصف الشمالي والعكس. وهكذا مع فصلي الربيع والخريف. وكل هذا سببه زاوية ميلان الأرض بالنسبة للشمس وليس قربها أو بعدها منها. وهذا كله فيه دليل ربوبية الله عز وجل وفضله على خلقه. 

وبعد الإطلاع على أقوال أهل العلم يمكن للباحث رؤية إختلاف منهجية العلماء في تقرير هذه الحقائق الكونية: فمنهم من يأخذ بظاهر الآيات الشرعية فقط دون الرجوع إلى العلوم الكونية المؤكدة ومعاملتها على أنها في مجملها علوم ظنية مثل الشيخ الفوزان حفظه الله. ومنهم من يأخذ بظاهر الآيات الشرعية مع إقرار إمكانية التأويل إذا ثبت الدليل بخلاف ذلك في العلوم الكونية مثل الشيخ العثيمين والشيخ ابن باز رحمهم الله. ومنهم من يأخذ بالحجة العقلية الثابتة في العلوم الكونية في فهم الآيات الشرعية مثل شيخ الإسلام ابن تيمية والشيخ الألباني رحمهم الله. ورغم قلة العلوم الكونية في زمن شيخ الإسلام ابن تيمية مقارنة لما وصلنا إليه اليوم، إلا أنه كان كثير الإطلاع في علوم الهيئة (علم الفلك) وعلم الحساب والعلوم الطبيعية (الفيزياء والكيمياء) ما مكنه من مناقشة هذه الظواهر الطبيعية مناقشة علمية مع محاولة إيجاد وجه التوافق بين الأمر الكوني والأمر الشرعي. ولقد شابهه في ذلك الشيخ الألباني رحمه الله من حيث المنهجية في تقرير الحقيقة العلمية الكونية للنظام الشمسي بعد الإطلاع عليها وعلى أدلتها الثابتة والمؤكدة. ورجوعا لما قرره شيخ الإسلام ابن تيمية في عدم تعارض العلم الكوني المؤكد مع العلم الشرعي الثابت، فإن منهجية شيخ الإسلام التي تبعه فيها الشيخ الألباني رحمهم الله جميعا هي الأصح، والله فوق كل ذي علم عليم.

\section{الإستدلال بالحساب في المواقيت}

إن من أعظم المصالح في شريعة الإسلام هي معرفة المواقيت وإجتماع المسلمين عليها كالصلاة والصوم والحج. فهذه العبادات العظيمة هي من أركان الإسلام والتي جعل الله لها أوقاتها وعلاماتها. ومن ذلك أوقات الصلاة خلال اليوم والليلة والأهلة التي بها تعرف أيام الصوم والإفطار في شهر رمضان وشوال وأيام الحج في شهر ذي الحجة. ولا شك أن إجتماع المسلمين في هذه المواقيت من المصالح العظيمة في دين الإسلام والتي ينبغي الإعتناء بها لما في ذلك من الألفة والتآخي وتوحيد الصف. بينما يبقى الإختلاف في أوقات الصلاة من قطر إلى قطر أمر طبيعي لا إشكال فيه، إلا أن الإختلاف في رؤية الأهلة من قطر إلى قطر وأحيانا داخل القطر نفسه قد ترتب عليه ظهور تفرق المسلمين في أعيادهم كل عام. ولعل من أسباب ذلك إنتشار وسائل التواصل بحميع أنواعها مع سهولة السفر حتى أصبح هذا الإختلاف ظاهرًا. وهذا بلا شك أمر لا بد من البحث والإجتهاد فيه سعيا لتوحيد كلمة المسلمين على الحق الذي يحبه الله ويرضاه.

\comment{
    ولقد تقدم أن شريعة الإسلام جاءت برفع الأمية عن أمة الإسلام ومن ذلك تعلم القراءة والكتابة والحساب حتى ينتفعوا بذلك ويبنوا عليه مصالحهم الدينية والدنيوية. ومن المعلوم أن حساب المواقيت يعتمد على معرفة حركة الأفلاك من الشمس والقمر والأرض وهذا لا يختلف بحسب الأقطار. بينما تختلف الرؤية بحسب مكان النظر من قطر لآخر. ولقد تقدم أن الآيات الكونية يتعلم منها الحساب الصحيح ومن ثم يكون الحساب الصحيح سبيلا لإقامة الميزان الشرعي. فمتى أمكنت الرؤية كانت وسيلة للتحقق من صحة الحساب. فالحساب الصحيح يمكن تقريره وتعلمه من القياس أو النظر ويجب أن يكون موافقا له بحسب الدرجة التي يمكن فيها الرؤية. ولعل من المسائل الدقيقة في هذا: إن توفرت طرق الحساب الصحيح وتوفرت الموانع مثل إنعدام الرؤية، فهل يقدر ذلك بالحساب أم يعلق الأمر دائما بالرؤية وخصوصا أن النبي ﷺ قال: "فإنْ غُمَّ علَيْكُم فَاقْدُرُوا له"؟ وهل يكون أن الرؤية التي دل عليها النبي ﷺ مقيدة بعلة: "إنا أمة أمية لا نكتب ولا نحسب" فيختلف الحكم بالنسبة لحال المكلف؟ وهل يلزم أن يكون الحساب الفلكي موافق للرؤية من حيث الحقيقة العلمية؟ وهل يمكن الإعتماد الحساب أو الرؤية على الإطلاق شرعا دون تقيد؟ وهل المطالع واحدة أم تختلف بإختلاف الأقطار؟ فكل هذه الأسئلة ناقشها أهل العلم من الناحية الشرعية والكونية. وقد يطول الحديث في هذا الموضوع ولكن يمكن تلخيص الأمور فيما يلي.
}

ورغم كثرة إستخدام الحساب في تقدير أوقات الصلاة في الساعات الإلكترونية والهواتف الذكية، إلا أن مسألة ثبوت دخول الشهر من عدمها قد قيدت بالرؤية فقط دون الحساب على الإطلاق. ولإختلاف الرؤية من قطر لآخر، ترتب على هذا إختلاف أيام الصوم، والعيد، وأحيانا حتى أيام الحج. ورغم دقة الحساب في زماننا هذا والذي قد يكون سببا لتوحيد المسلمين في الأهلة،\footnote{لا تختلف الأهلة حسابيا لأن دخول الشهر يكون عندما يتجاوز مركز القمر الخط الإفتراضي الذي يصل بين مركز الأرض ومركز الشمس. ولكن المطالع تختلف بالرؤية بحسب مكانها من بلد لآخر. ولقد قرر جمهور العلماء أنه لا عبرة بإختلاف المطالع وأن المطالع واحدة وإن اختلفت أي يجب على جميع المسلمين الأخذ بالرؤية متى تبثت في أي قطر. وهذا القول موافق للحساب من حيث أن دخول الشهر واحد لا يتعدد إلا من جهة الرؤية ومكانها. عمليا، الأخذ بالرؤية أصبح مقيد بكل قطر بإعتبار أن لكل قطر ولاية خاصة به.} إلا أن أكثر علماء أهل السنة قديما وحديثا اتفقوا على وجوب الأخذ بالرؤية فقط على الإطلاق دون الإعتبار بالحساب لما صح عن عبد الله بن عمر أن النبي ﷺ قال: الشهر تسع وعشرون ليلة، فلا تصوموا حتى تروه، فإن غم عليكم فأكملوا العدة ثلاثين \href{https://shamela.ws/book/1284/1257#p1}{\faExternalLink} \cite{bukhari}.\footnote{صحيح البخاري: 1918.} وفي حديث أخر في صحيح مسلم بعدة روايات عن أبي هريرة أن النبي ﷺ قال: إذا رأيتم الهلال فصوموا. وإذا رأيتموه فأفطروا. فإن غم عليكم فصوموا ثلاثين يوما \href{https://shamela.ws/book/1727/2446#p2}{\faExternalLink}. وفي رواية: صوموا لرؤيته وأفطروا لرؤيته. فإن غمي عليكم فأكملوا العدد \href{https://shamela.ws/book/1727/2447#p2}{\faExternalLink}. وفي رواية: صوموا لرؤيته وأفطروا لرؤيته. فإن غمي عليكم الشهر فعدوا ثلاثين \href{https://shamela.ws/book/1727/2448#p2}{\faExternalLink}. وفي رواية: إذا رأيتموه فصوموا. وإذا رأيتموه فأفطروا. فإن أغمي عليكم. فعدوا ثلاثين \href{https://shamela.ws/book/1727/2449#p2}{\faExternalLink} \cite{muslim}.\footnote{صحيح مسلم: 1081.} فهل هذا الحكم هو على الإطلاق أم مقيد بعلة؟ وهل يوجد ما يدل على وجوب الأخذ بالحساب عند زوال هذه العلة؟ فهذه أسئلة حري بنا أن نتمعن ونبحث فيها لمعرفة الحق في ذلك.

إن حصر المسألة على ما تقدم من الأحاديث يفضي بلا شك إلى وجوب الأخذ بالرؤية فقط على الإطلاق دون الإعتبار بغيرها. إلا أن النصوص الآخرى قد جاءت بألفاظ تبين أن هذا الإطلاق غير صحيح وأن المسألة فيها تفصيل لا يجب أن يغفل عنه. فمن منهج أهل السنة والجماعة الجمع بين النصوص من كتاب الله عز وجل وسنة نبيه ﷺ والتوفيق بينها. ومن منهج أهل البدع والأهواء أنهم يأخذون ما يوافق أهوائهم من النصوص ويتركون مالا يوافقها بل ويردونه. يقول الشيخ فلاح مندكار رحمه الله نقلا عن الشيخ عبد المحسن العباد حفظه الله: ينظرون إلى النصوص بعين عوراء، والحق لا يكتمل، ودين الله جل وعلا لا يتضح، إلا إذا جمعت بين النصوص كلها [.] تخرج بمذهب وصراط (وأشار بيده أي: صراط مستقيم) \href{https://www.youtube.com/watch?v=PEQsmhJX51w}{\faExternalLink}. ويقول الشيخ العثيمين رحمه الله: والذين في قلوبهم زيغ يتبعون ما تشابه منه ويضربون كتاب الله بعضه ببعض ويقولون هذا يناقض هذا وهذا يكذب هذا وما أشبه ذلك أو يلجئون إلى القول بأحد النصين وإلغاء الآخر. ولكن أهل الإيمان والعلم يسلكون مسلك آخر وهو أنه لا تعارض في كلام الله عز وجل ولا تعارض بين كلام الله وبين سنة رسول الله ﷺ التي صحت عنه لأن الكل من عند الله \href{https://www.youtube.com/watch?v=KvRTXkTmeew}{\faExternalLink}.

وهناك العديد من الأدلة الآخرى من كتاب الله عز وجل وسنة النبي ﷺ التي تتعارض مع الحكم بالأخذ بالرؤية على الإطلاق. فقد جاء عن عبدالله بن عمر حديثين مهمين في الصحيحين في هذه المسألة ولا يستبعد أن يكون ذلك في مجلس واحد وهذا لإرتباط الأحاديث من حيث المتن ومن حيث الراوي. فأما الحديث الأول فقد صح عن عبدالله بن عمر أنَّ رَسولَ اللَّهِ ﷺ ذَكَرَ رَمَضَانَ فَقالَ: لا تَصُومُوا حتَّى تَرَوُا الهِلَالَ، ولَا تُفْطِرُوا حتَّى تَرَوْهُ، فإنْ غُمَّ علَيْكُم فَاقْدُرُوا له \href{https://shamela.ws/book/1284/1256#p6}{\faExternalLink} \cite{bukhari}.\footnote{صحيح البخاري: 1917.} وفي رواية أخرى: إذا رأيتموه فصوموا، وإذا رأيتموه فأفطروا، فإن غم عليكم فاقدروا له \href{https://shamela.ws/book/1284/1254#p1}{\faExternalLink} \cite{bukhari} \href{https://shamela.ws/book/1727/2430#p3}{\faExternalLink} \cite{muslim}.\footnote{صحيح البخاري: 1911، صحيح مسلم: 1080.} وأما الحديث الثاني فقد صح عن عبد الله بن عمر أن النبي ﷺ قال: إنَّا أُمَّةٌ أُمِّيَّةٌ، لا نَكْتُبُ ولَا نَحْسُبُ، الشَّهْرُ هَكَذَا وهَكَذَا. يَعْنِي مَرَّةً تِسْعَةً وعِشْرِينَ، ومَرَّةً ثَلَاثِينَ \href{https://shamela.ws/book/1284/1259#p3}{\faExternalLink} \cite{bukhari} \href{https://shamela.ws/book/1727/2443#p2}{\faExternalLink} \cite{muslim}.\footnote{صحيح البخاري: 1924، صحيح مسلم: 1080.} وبالجمع بين هذه الأحاديث مع ما سبق يتبين أن الحكم في هذه المسألة مقيد بعلتين.  فقد جاء في الحديث الأول علة الإغمام ووجوب الأخذ بالتقدير عند الإغمام لإستحالة الرؤية. وأما الحديث الثاني فقد جاء فيه بيان علة الجهل بالحساب. ووجه الجمع بينهما أن النبي ﷺ كان يبين لعبدالله بن عمر رضي الله عنهما دخول شهر رمضان وخروجه وعدد أيام الشهر. وفي هذا الجمع يتقرر أن النبي ﷺ أشار للتقدير بالحساب أي أنه قال: "فاقدروا له" قاصدا الحساب لأنه قال أيضا: "إنَّا أُمَّةٌ أُمِّيَّةٌ، لا نَكْتُبُ ولَا نَحْسُبُ" وهذا لأن الخطاب في قوله "فاقدروا له" ليس في حق من جهل الحساب وإنما هو في حق من علم الحساب فقط.

\vspace{1em}
\fbox{\small
    \begin{minipage}{34em}
    \subsection{هل حكم الأخذ بالرؤية هو على الإطلاق أم مقيد بعلة؟}
    بناء على ما تقدم من الأدلة، فالحكم بالأخذ بالرؤية ليس بالإطلاق وإنما مقيد بعلتين وهما الإغمام والجهل بالحساب. فإن تحققت العلتين أكمل العدد إلى 30. وإن تحققت علة الجهل بالحساب مع ثبوت الرؤية يوم 29 أخذ بالرؤية. وإن تحققت علة الإغمام يوم 29 مع العلم بالحساب وجب التقدير. وإن ثبتت الرؤية يوم 29 مع العلم بالحساب أمكن الأخذ بأيسر الأمور وهو الرؤية أو الأدق في حالة الشك وهو الحساب. ولعدم تأكد علة الإغمام أو الشك قبل ذلك، وجب التقدير تحسبا للأخذ به عند تحقق علة الإغمام أو الشك.

    \subsection{هل يحب الأخذ بالحساب عند زوال علة الجهل به؟}
    أرشد النبي ﷺ إلى وجوب التقدير في حالة الإغمام. والجمع بين الأحاديث فيه أن النبي ﷺ أشار إلى التقدير بالحساب. وقد تقدم معنا في هذا البحث أن التقدير من معاني الحساب ولو كان على وجه التقريب. ومن الأدلة على أن التقدير من معاني الحساب قوله تعالى: \quranayah*[6][96] {\footnotesize (\surahname*[6])}. فجعل الله جل جلاله الحسبان وهو جمع الحساب من التقدير.
\end{minipage}
}
\vspace{1em}

ولهذا فإن من الأهمية بمكان هو الإعتناء بتقرير معنى التقدير الذي دل عليه النبي ﷺ. فإن الإختلاف في تأويل معنى الأحاديث السابقة إنما جاء من هذا الباب. وقد تقدم أن من معاني الحساب العد والتقدير ولكن كثير من العلماء فسر هذا بأن يتم الشهر 30 يوما على الإطلاق ولا يعني ذلك الأخذ بالحساب. إلا أن هذا التفسير ينافي معنى التقدير في اللغة ولا دليل عليه إلا إجتهادا لجعل هذا المعنى موافقا لإكمال العدد ثلاثين ومخالفا لمعنى الحساب وهذا لا يصح. وإنما الذي صح عن النبي ﷺ أنه قال: "فاقدُروا لَهُ" في موضعين. الموضع الأول (شهري): أنَّ رَسولَ اللَّهِ ﷺ ذَكَرَ التقدير في رَمَضَانَ فَقالَ: لا تَصُومُوا حتَّى تَرَوُا الهِلَالَ، ولَا تُفْطِرُوا حتَّى تَرَوْهُ، فإنْ غُمَّ علَيْكُم فَاقْدُرُوا له. وفي رواية لمسلم: وقالَ: فاقْدِرُوا له ولَمْ يَقُلْ ثَلاثِينَ.\footnote{صحيح البخاري: 1906. صحيح مسلم: 1080.} وفي هذه الزيادة الدليل الكافي على علم الراوي بألفاظ الحديث الأخرى التي جاءت بإكمال العدد وأنه أثبت التقدير مع نفي إكمال العدد ثلاثين حتى لا يلتبس الأمر وحتى لا يفسر التقدير بذلك. وأما الموضع الثاني (عام): عندما سأل الصحابة رضوان الله عليهم النبي ﷺ عن فترة مكوث الدجال في الأرض. فقد جاء عن النواس بن سمعان الأنصاري أنه قال: قلنا: يا رسولَ اللَّهِ وما لبثُهُ في الأرضِ؟ قالَ: أربعونَ يومًا، يومٌ كَسنةٍ، ويومٌ كَشَهْرٍ، ويومٌ كَجُمعةٍ، وسائرُ أيَّامِهِ كأيَّامِكُم، فقُلنا: يا رسولَ اللَّهِ هذا اليومُ الَّذي كسَنةٍ أتكفينا فيه صلاةُ يومٍ وليلةٍ؟ قالَ: لا، فاقدُروا لَهُ قدرَهُ.\footnote{صحيح ابن ماجه: 3310، وصححه الألباني.} فهل يعقل أن يفسر التقدير بإكمال العدد الثلاثين وهو ينافي فترة مكوث الدجال إذ لا يوجد 30 يوم وإنما يوم واحد كسنة أو كشهر أو كأسبوع. والله جل جلاله يقول: \quranayah*[4][103][14] {\footnotesize (\surahname*[4])}:  أي: مفروضا في وقته، فدل ذلك على فرضيتها، وأن لها وقتا لا تصح إلا به \cite{tafsir_Saadi}. ويقول تعالى: \quranayah*[2][189][1-8] {\footnotesize (\surahname*[2])} أي: ليعرف الناس بذلك، مواقيت عباداتهم من الصيام، وأوقات الزكاة، والكفارات، وأوقات الحج \cite{tafsir_Saadi}.

ولبيان الحق في هذا وجب النظر في هذا الأمر من عدة وجوه. أولا: أن الله جل جلاله سمى الحساب تقديرا وبالأخص فيما يتعلق بالليل والنهار والشمس والقمر والتي جميعها من أسباب معرفة المواقيت كما في قوله تعالى: \quranayah*[6][96] {\footnotesize (\surahname*[6])}. فدل على أن تقدير الله جل جلاله هو الحساب الذي دلنا عليه وهذا فيه أن من معاني الحساب التقدير. ثانيا: أن النبي ﷺ أرشدنا إلى تقدير أوقات الصلاة والشهر وبالأخص متى إحتاج المسلمين إلى ذلك في قوله ﷺ: "فاقْدِرُوا له" في موضعين مختلفين كما تقدم. ثالثا: موافقة سنة الرسول ﷺ لكتاب الله جل جلاله يلزم بأن يكون معنى "فاقدُروا لَهُ" أي "أكتبوه" لأن الصلاة كانت كتابا موقوتا ولأن لكل موسم ميقات معلوم و "أحسبوه" لأن الله جل جلاله أرشد للحساب في قوله تعالى: (\quranayah*[10][5][10-13]) في موضعين في كتابه الكريم. رابعا: أن الله جل جلاله جعل الصلاة كتابا موقوتا على المسلمين وجعل المواقيت فدل ذلك على وجوب الإعتناء بأوقات الصلاة والمواقيت كتابة وحسابا.

فهذه أربعة حجج ظاهرة بينة تدل على أن معنى قوله ﷺ "فاقدُروا لَهُ" وجوب الأخذ بالحساب  متى أمكن ذلك وأحتاج المسلمين له وتوفرت شروطه كالإغمام أو الشك أو تغير الزمان لأي عارض كفتنة الدجال. وهذا لا يتعارض مع بالأخذ بالرؤية لمن جهل الحساب لأن الله جل جلاله يقول: \quranayah*[22][78][8-14] {\footnotesize (\surahname*[22])}. والأمية كانت متحققة في زمن الرسول ﷺ والصحابة رضوان الله عليهم بنص القرآن وكما بين ذلك النبي ﷺ في قوله: "إنا أمة أمية لا نكتب ولا نحسب". وبناء على كل ما تقدم يكون المعنى الصحيح لهذا الحديث: أنه لولا الأمية التي غلبت على حال الناس لأمر النبي ﷺ الناس بكتابته وحسابه ولكنه بينه لهم بحسب حالهم والذي فيه مراعاةً لأميتهم. ولا نحب أن نخالف الرسول ﷺ والصحابة رضوان الله عليهم والسلف الصالح ولا علماءنا الأفاضل في شئ ولكن النصوص تدل على وجوب مخالفة الصحابة في هذا الأمر على وجه الخصوص لإختلاف حال المكلفين وبالأخص فيما يتعلق بالعلم بالحساب أو الجهل به. وهذا فيه وجوب تعلم الحساب والأخذ والإستدلال به على اللاحقين من أمة الإسلام كما أرشد تعالى في هذه المسألة على وجه الخصوص في قوله: \quranayah*[10][5] {\footnotesize (\surahname*[10])}. وبهذا تجتمع الأدلة من كتاب الله جل جلاله ومن سنة نبيه ﷺ على وجوب الأخذ بالحساب والإستدلال به متى أمكن ذلك وتوفرت شروطه، كما هو الحال مع الكتابة.

وأما الإستدلال بقوله ﷺ: "إنا أمة أمية لا نكتب ولا نحسب" على عدم جواز الأخذ والإستدلال بالحساب فهذا إستدلال باطل يخالف كتاب الله عز وجل وسنة نبيه من عدة وجوه. الوجه الأول: أن الله جل جلاله دل في موضعين في كتابه الكريم على تعلم الحساب من آيات الله الكونية التي بها تعرف المواقيت كالليل والنهار والشمس والقمر. فكيف يدلنا الله جل جلاله على الحساب وبالأخص فيما يتعلق بالمواقيت والنبي ﷺ ينهانا عن الأخذ به؟ ومن قال أن هذا النهي في حق الأخذ بالحساب في الأهلة فقط فلا حجة له في ذلك بل إن هذا مخالف لصريح كتاب الله جل جلاله لأن الله جعل الأهلة مواقيت وجعل لها أسبابها وبين أنها بحساب متقن ليتعلم الناس منها الحساب وليبنوا عليه مصالحهم ومن ذلك حساب المواقيت. الوجه الثاني: أن النبي ﷺ دعى لمعاوية رضي الله عنه تعلم الحساب مع الكتاب. فكيف يدعو النبي ﷺ لمعاوية تعلم الحساب وينهى عن الأخذ به؟ الوجه الثالث: يقوم علم الفرائض والمواريث التي جاءت في كتاب الله وسنة نبيه على الحساب. فكيف ينهى النبي ﷺ عن الحساب وتقام عليه أحكام الدين؟ الوجه الرابع: بدأ الرسول ﷺ هذا الحديث بقوله "أنا أمة أمية". فهل يعقل أنه يفهم من ذلك أن تبقى أمة الإسلام على أميتها لا تكتب ولا تحسب إلى قيام الساعة؟ الوجه الخامس: جاء في هذا الحديث أيضا الكتابة. فكيف يكون هذا النهي في حق الحساب فقط دون الكتابة؟ وهل يعقل أن ينهانا رسولنا الكريم ﷺ عن الكتابة وأول ما أنزل عليه إقرا وفيها قوله تعالى: \quranayah*[96][4-5] {\footnotesize (\surahname*[96])}. الوجه السادس: كان النبي ﷺ حريص على رفع الأمية عن أمة الإسلام ومن ذلك أنه جعل فداء أسرى بدر أن يعلموا أولاد الأنصار الكتابة \href{https://shamela.ws/book/25794/1602#p1}{\faExternalLink} \cite{ahmid}.\footnote{مسند الإمام أحمد: 2216.} فهل يعقل أن يسعى النبي ﷺ  في تعليم أبناء المسلمين الكتابة وهو ينهى عن ذلك؟ 

وعليه فإن المقصود بقوله ﷺ: "إنا أمة أمية لا نكتب ولا نحسب" ليس النهي عن الحساب كما لا يفهم منه النهي عن الكتابة. فأول ما أنزل الله عز وجل إقرا وفيها الحث على تعلم القراءة والكتابة ومن ثم جاء الحث على تعلم الحساب لرفع الأمية بكاملها وحتى يبني عليه المسلمين مصالحهم الدينية والدنيوية. وكان ذلك كله في العهد المكي أي أن هذا الحث كان منذ بداية فجر الإسلام فدل ذلك على أن دين الإسلام هو دين العلم و الحضارة بما يرضي الله تبارك وتعالى. وعليه فإن المقصود في هذا الحديث هو بيان حال الرسول ﷺ وأصحابه رضوان الله عليهم أجمعين أي أنهم في غالبهم أميين لا يكتبون ولا يحسبون كما بين ذلك النبي ﷺ وكما جاء ذلك في كتاب الله عز وجل. فدل على أن هذا الحديث فيه بيان معنى الأمية وما يتعلق بذلك من معرفة أيام الشهر بغير كتابة ولا حساب. بل إن هذا فيه دليل على أن الجهل بالكتابة والحساب من الأمية والتي لولا إنتشارها لأمرهم النبي ﷺ بكتابة ذلك وحسابه. وبهذا المعنى تتوافق الأدلة من كتاب الله عز وجل وسنة نبيه ﷺ ولا عبرة بخلاف ذلك لأنه يعارض الحجة الشرعية والعقلية جملة وتفصيلا.

ولقد نقل شيخ الإسلام ابن تيمية أن من فقهاء البصرة من فسر "أقدروا له" بالحساب فقال: أن فقهاء البصريين ذهبوا إلى أن قوله: "فاقدروا له" تقدير حساب بمنازل القمر وقد روي عن محمد بن سيرين قال: خرجت في اليوم الذي شك فيه فلم أدخل على أحد يؤخذ عنه العلم إلا وجدته يأكل إلا رجلا كان يحسب ويأخذ بالحساب ولو لم يعلمه كان خيرا له. وقد قيل: إن الرجل مطرف بن عبد الله بن الشخير وهو رجل جليل القدر إلا أن هذا إن صح عنه فهي من زلات العلماء. وقد حكي هذا القول عن أبي العباس بن سريج أيضا \href{https://shamela.ws/book/7289/12629#p1}{\faExternalLink} \cite{muslim}.\footnote{مجموع الفتاوى: 25/181.} ويفهم من هذا الأثر أن فقهاء البصرة أفطروا في يوم الشك إكمالا لشعبان ثلاثين يوما إتباعا لقول الرسول ﷺ: " فإن غم عليكم فأكملوا العدة ثلاثين" وهذا الفعل صحيح لمن جهل بالحساب. وأما الرجل الذي أخذ بالحساب في يوم الشك فعمله موافق لقول الرسول ﷺ: "فإن غم عليكم فاقدروا له" في حق من علم الحساب كما تقدم، والله فوق كل ذي علم عليم. ولو كان هذا الرجل هو فعلا مطرف بن عبد الله بن الشخير فهو رجل مبارك مستجاب الدعوة من كبار التابعين ابن صحابي نقل أحاديث الرسول ﷺ عن أبيه وعن أصحاب النبي ﷺ ونقل عنه الحسن البصري وغيره من أتباع التابعين. إلا أن محمد بن سيرين رحمه الله أنكر ذلك عليه بإجتهاده. وشيخ الإسلام ابن تيمية رحمه الله كان لا يرى بالأخذ بالحساب في الأهلة وغلظ القول في ذلك بإجتهاده ولهذا قال عنه: "وهو رجل جليل القدر إلا أن هذا إن صح عنه فهي من زلات العلماء". فهو رجل ثقة عرف بالرسوخ في العلم والثبات في الفتن ولا يضره أنه كان يأخذ بالحساب بل إن هذه منقبة عظيمة له جعل لنا بها سلف، راجع \fullautoref{sec:app_mutrif}.

وأما الإستدلال بقوله ﷺ: "لا تَصُومُوا حتَّى تَرَوُا الهِلَالَ، ولَا تُفْطِرُوا حتَّى تَرَوْهُ" على وجوب الأخذ بالرؤية بالعموم دون تقيد ذلك بشئ مع القول بحرمة الأخذ بالحساب حتى متى أمكن فهذا كله ينافي الكتاب والسنة من عدة وجوه. الوجه الأول: أنه ينافي الغاية من قوله تعالى: \quranayah*[10][5] {\footnotesize (\surahname*[10])}. فالغاية من تعلم العدد والحساب تقتضى أن يكون ذلك للأخذ والإستدلال به حتى يبني عليه مصالحهم كما جاء في التفسير وليس لمجرد التعلم فقط. وأي مصلحة أهم من المواقيت التي أمر الله بها. كما أن هذه الآيات جاءت صريحة  بالحساب ومقيدة بالشمس والقمر والليل والنهار والتي بها تعرف جميع المواقيت سواء مواقيت الصلاة أو الأهلة. الوجه الثاني: أنه ينافي معنى قول الرسول ﷺ "فَاقْدُرُوا له" وهذا لأن التقدير من معاني الحساب كما تقدم بيان ذلك ولا يصح أن يفسر هذا بإكمال الثلاثين بالإطلاق. الوجه الثالث: ينافي الإعتبار بحال المكلفين وبالأخص من ناحية العلم بالحساب أو الجهل به. الوجه الرابع: لا يمكن للإنسان إن يتقن أوقات الصلاة والمواقيت دون حساب صحيح وبالأخص عندما يكون اليوم كسنة أو كشهر أو كأسبوع لإنتفاء أسباب معرفة ذلك من تتابع الليل والنهار كما أخبر النبي ﷺ في قوله: "يومٌ كَسنةٍ، ويومٌ كَشَهْرٍ، ويومٌ كَجُمعةٍ".\footnote{وهذا البطئ في الزمان إنما يكون بتوقف وبطئ دوران الأرض حول نفسها وقد يصحب ذلك ثبات منازل القمر. وقد يكون ذلك مصحوبا بتغير إتجاه دوران الأرض حتى يكون ذلك سببا في ظهور الشمس من مغربها، والله أعلم.} الوجه الخامس: ينافي الإتقان الذي حثنا الرسول ﷺ عليه بقوله: إنَّ اللهَ تعالى يُحِبُّ إذا عمِلَ أحدُكمْ عملًا أنْ يُتقِنَهُ \href{https://shamela.ws/book/21659/2761#p1}{\faExternalLink} \cite{jamaaSagheer}.\footnote{صحيح الجامع: 1880، وقال الألباني حديث حسن.} وهذا بالعموم فكيف بأوقات الصلاة والمواقيت والتي هي أساس الدين ومن أعظم ما يحبه الله ويرضاه. وقد تقدم معنا أن علم الحساب هو علم الضبط والإتقان ومن وسائل إقامة الميزان الشرعي بالقسط. الوجه السادس: ينافي المصلحة العليا في جمع كلمة المسلمين وأن المطالع في حقيقتها واحدة إلا أن الإختلاف كان فقط من جهة الرؤية وأن هذا ترتب عليه إختلاف المسلمين في أعيادهم في عصر سهل فيه التواصل بينهم في جميع الأقطار. الوجه السابع: عدم الإلتفات للحساب والأخذ به فيه الإستهانة بهذا العلم العظيم فهو مفتاح العلوم الكونية السببية النافعة وهذا يترتب عليه تأخر المسلمين -كما هو الحال اليوم- في الأخذ بالأسباب التي بها تقام مصالح المسلمين الدينية والدنيوية ومن ذلك إجتماع الكلمة وإعداد العدة. فتأمل كيف لو كان أهل الإيمان ممن أتقن الحساب مع الأخذ والإستدلال به في أمور الدين والدنيا وفي معرفة الأسباب مع إقامة الحق والميزان، لأجتمعت فيهم من المنافع الدينية والدنيوية ما يستحيل أن يجتمع في غيرهم، ولسبقوا بذلك الأمم الأخرى كلها وكان لهم التمكين الذي وعد الله به.

وعليه فإن الأخذ والإستدلال بالحساب في زماننا هذا الذي توفرت فيه طرق الحساب الصحيحة مع الأجهزة السريعة وأدوات القياس الدقيقة أصبح حقيقة واجبة لا مفر منها لا رغبة في مخالفة المتقدمين ولكن فقط لتغير حال المتأخرين وهذا لإرتباط الحكم بحال المكلفين من حيث العلم أو الجهل بالحساب. وما خلص له في هذا البحث من بدايته إلى نهايته كله يصب في هذا الإتجاه لا لغير ذلك. ورغم قلة السائرين على هذا النهج في تقرير الحقيقة دون تقليد أو تعصب، إلا أن الإنسان ليسعد سعادة الغريب بالأدلة الثابتة والصريحة في ذلك. وإن الباحث عن الحق ليستأنس بأن الله جل جلاله وفق الشيخ القاضي والمحدث أحمد محمد شاكر رحمه الله رحمة واسعة في تقرير ذلك كله رغم قلة الناظرين فيما قرره في رسالته: أوائل الشهور العربية: هل يجوز شرعا إثباتها بالحساب الفلكي؟ \cite{shuhur_ahmidShakir}. فإنا والله مع قلة علمنا وضعف حالنا، إلا أننا  لنجد الحجة في كلامه واضحة جلية موافقة لهذا البحث. فالحمد الله رب العالمين، انظر \fullautoref{sec:app_shuhur_ahmidShakir}.

ومما يأسف عليه الإنسان أن يكون أهل الإيمان في زماننا ممن يتحامل على علم الحساب والأخذ والإستدلال به وجميع الآيات الشرعية جاءت بالحث عليه والآيات الكونية دالة عليه فلا حول ولا قوة إلا بالله العلي العظيم. فهذا والله مصاب عظيم في أمة الإسلام ترتب عليه من الجهل والظلم ما الله به عليم. ويعذر السلف الصالح في هذا الباب لعدم معرفتهم الكافية بعلم الحساب ويلحقهم في ذلك المتقدمين من علماء الإسلام الذين لم يجدوا من يثقون بعلمه في هذا الباب بل وجدوا من يدعى التنجيم وعلم الغيب. ولكن ما حجة أهل الإيمان وعلماء الإسلام اليوم وقد صار علم الحساب إلى درجة عالية من الضبط والإتقان ما يتحقق به اليقين في معرفة حركة الإفلاك والأهلة. ومع ذلك لا زالوا يبنون إجتهاداتهم على كلام شيخ الإسلام ابن تيمية رحمه الله دون مراعاة ما كان عليه علم الحساب قبل مئات السنين ودون الرجوع لأهل هذا العلم من علماء الحساب والفلك. ورغم ذلك فإن شيخ الإسلام إجتهد في معرفة علم الحساب لمعرفة ما يقوم عليه من أسس لتقرير صحة الإستدلال به من عدمها في زمانه. فهل يعقل أن يستدل بذلك في هذا الزمان والعديد من تلك الأسس قد تغيرت وتطورت حتى أصبحت حقائق ثابتة بالحجة والبرهان. وهل هذا يوافق منهجية شيخ الإسلام ابن تيمية حيث أنه كان يجتهد ويطلع على كلام المتخصصين في زمانه من شتى العلوم بحثا عن الحق بالجمع بين الأدلة الكونية والأدلة الشرعية. 

إن الحكم بحرمة الأخذ بعلم الحساب في الأهلة بدون الأخذ بعين الإعتبار ما صار إليه هذا العلم اليوم هو من الإجحاف. فعلم الحساب اليوم ليس هو كعلم الحساب في زمن شيخ الإسلام ابن تيمية ولا هو كعلم الحساب في زمن الصحابة. فعلم الحساب من العلوم التي تطورت كثيرا ولا زالت تتطور حتى وصلت إلى درجة عالية من الدقة والكفاءة في كافة العلوم حتى أصبح ذلك ظاهرا في كافة المجالات ومنها علم الفلك. 

من المعلوم أن حساب المواقيت يعتمد على معرفة حركة الأفلاك من الشمس والقمر والأرض وهذا لا يختلف بحسب الأقطار. بينما تختلف الرؤية بحسب مكان النظر من قطر لآخر. ولقد تقدم أن الآيات الكونية يتعلم منها الحساب الصحيح ومن ثم يكون الحساب الصحيح سبيلا لإقامة الميزان الشرعي. فمتى أمكنت الرؤية كانت وسيلة للتحقق من صحة الحساب. فالحساب الصحيح يمكن تقريره وتعلمه من القياس ويجب أن يكون موافقا له بحسب الدرجة التي يمكن فيها الرؤية. ولعل من المسائل الدقيقة في هذا: هل يلزم أن يكون الحساب الفلكي موافق للرؤية من حيث الحقيقة العلمية؟ وهل المطالع واحدة أم تختلف بإختلاف الأقطار؟ فهذه الأسئلة ناقشها أهل العلم من الناحية الشرعية والكونية. وقد يطول الحديث في هذا الموضوع ولكن يمكن تلخيص الأمور فيما يلي.

لقد قرر جمهور العلماء أنه لا عبرة بإختلاف المطالع وأن المطالغ في حقيقتها إنما تختلف من جهة الرؤية وليس من حيث دخول الشهر لأن دخول الشهر واحد سواء أمكنت رؤية ذلك أم لا. ولكن الأخذ بالرؤية وبالأخص في حق من جهل بالحساب يترتب عليه الإختلاف في المطالع كما كان الحال في زمن الصحابة رضوان الله عليهم. فقد صح عن كريب أن أم الفضل بنت الحارث بعثته إلى معاوية بالشام. قال: فقدمت الشام. فقضيت حاجتها. واستهل علي رمضان وأنا بالشام. فرأيت الهلال ليلة الجمعة. ثم قدمت المدينة في آخر الشهر. فسألني عبد الله بن عباس رضي الله عنهما. ثم ذكر الهلال فقال: متى رأيتم الهلال فقلت: رأيناه ليلة الجمعة. فقال: أنت رأيته؟ فقلت: نعم. ورآه الناس. وصاموا وصام معاوية. فقال: لكنا رأيناه ليلة السبت. فلا نزال نصوم حتى نكمل ثلاثين. أو نراه. فقلت: أو لا تكتفي برؤية معاوية وصيامه؟ فقال: لا. هكذا أمرنا رسول الله صلى الله عليه وسلم \href{https://shamela.ws/book/1727/2460#p3}{\faExternalLink} \cite{muslim}.\footnote{صحيح مسلم: 1087.} 


\section{الحساب بين الإفراط والتفريط}

إن من أعظم أسباب تأخر الإنسان في كل زمان ومكان هو الظلم والجهل وهذا ينافي الأمانة التي كلف الله الإنسان بها كما قال تعالى: \quranayah*[33][72] {\footnotesize (\surahname*[33])}. فالجهل مخالف للعلم الذي يقام به الحق والظلم مخالف للعدل الذي يقام به الميزان. ولا شك أن من أعظم أسباب تخلف المسلمين في زماننا هذا هو مخالفة الحق والميزان وبالأخص في عدم الأخذ والعمل بالحساب الصحيح والإستدلال به في الأمور الكونية والشرعية. وبدلا من ذلك أعتمد الإستدلال بمجرد المشاهدة في تقرير حقائق الكون دون الإعتبار بالحساب وطرق الإستدلال التي أرشد الله تبارك وتعالى إليها من سمع وبصر وعقل. بل ووضع ذلك كله في موضع التعارض مع الدين حتى أصبح المنهج السليم هو الأخذ بظاهر الأشياء دون تعقلها والتبصر فيها في كل الأمور الكونية الظاهرة منها والغير ظاهرة.  

وأصبح الناس في ذلك على نقيضين. منهم الجهال الذين يخوضون في علم الغيب بالباطل ظنا أن العلم السببي يدرك بالعقل الغيبيات التي لا يمكن إدراك أسبابها. فأساؤوا بذلك لأنفسهم ولعلمهم الكوني فردهم أهل العلم الشرعي بالحق في بيان بطلان الإستدلال بالعقل في تقرير الأمور الغيبية. إلا أن العديد غلى في ذلك حتى عطل العقل والإستدلال به في الأمور الظاهرة سواء كانت الكونية منها أو الشرعية. حتى كان لسان حال أغلبهم بطلان الإستدلال بالعقل حتى في العلم السببي الظاهر زعما بأنه مجرد ظن وإلحاقا بالعلم الغيبي الغير ظاهر. وهذا كله باطل ويخالف ما أشرد الله جل جلاله إليه في كتابه العظيم. ولا عجب أن يكون ذلك صدا للأخذ بالأسباب والتطور في الحضارة بما يرضي الله جل جلاله. 

فالعلوم الكونية قسمان: علم ظاهر وهو العلم السببي وعلم غير ظاهر وهو علم الغيب. والعلم الظاهر السببي منه المؤكد الذي أمكن إثباته بالقياس ومنه الغير مؤكد والذي لم يمكن إثباته وهذا هو الذي يقال له نظريات. وأما جعل العلم الظاهر السببي المؤكد كله غير مؤكد ومجرد ظن فهذا لا يصح عقلا ولا شرعا. كما أن الأخذ بالنظر فقط في تقرير حقائق الأشياء لا يكون دائما صحيحا. فالله جل في علاه لم يثبت النظر فقط للإستدلال على هذه الأمور الكونية بل إن الله تعالى أرشد إلى طرق الإستدلال في العلم السببي الظاهر وهي السمع والبصر مع العقل. وذكر سبحانه وتعالى أولى الألباب وأولى الأبصار والحساب مع آيات الليل والنهار والشمس والقمر فدل على أن ذلك من الأمور التي يمكن إدراكها بطرق الإستدلال التي أرشد الله لها.

ينسب لشيخ الإسلام بن تيمية قوله عن الخوارزمي: "وإن كان علمه صحيحا إلا إن العلوم الشرعية مستغنية عنه وعن غيره". ولكن هذا القول لم يثبت عن شيخ الإسلام وهو مخالف لما نقله شيخ الإسلام ومن ذلك حرص علماء السنة على تعلم الحساب والإشتغال به للإستفادة من علم الجبر والمقابلة الذي ألفه الخوارزمي رحمه الله. بل إن شيخ الإسلام قال عن علم الحساب أنه علم صحيح لا يدخل فيه غلط وأن فيه حفظا للشرع كما تقدم. فهذا بلا شك من التدليس والتفريط في هذا العلم العظيم الذي ابتلينا به في زماننا. وإنما كان شيخ الإسلام يرد على من غلى في علم الحساب وأراد أن يجعل الشريعة متوقفة عليه من غلاة المنطق الذين يقدمون العقل على النقل وبالأخص في مسألة وجوب الأخذ بالرؤية\comment{، فقال في ذلك: "قد بينا أن شريعة الإسلام ومعرفتها ليست موقوفة على شيء يتعلم من غير المسلمين أصلا وإن كان طريقا صحيحا. بل طرق الجبر والمقابلة فيها تطويل. يغني الله عنه بغيره كما ذكرنا في المنطق. وهكذا كل ما بعث به النبي صلى الله عليه وسلم مثل العلم بجهة القبلة والعلم بمواقيت الصلاة والعلم بطلوع الفجر والعلم بالهلال؛ فكل هذا يمكن العلم به بالطرق التي كان الصحابة والتابعون لهم بإحسان يسلكونها ولا يحتاجون معها إلى شيء آخر \comment{. وإن كان كثير من الناس قد أحدثوا طرقا أخر؛ وكثير منهم يظن أنه لا يمكن معرفة الشريعة إلا بها. وهذا من جهلهم كما يظن طائفة من الناس أن العلم بالقبلة لا يمكن إلا بمعرفة أطوال البلاد وعروضها. وهو وإن كان علما صحيحا حسابيا يعرف بالعقل لكن معرفة المسلمين بقبلتهم ليست موقوفة على هذا [.] فلهذا كان قدماء علماء " الهيئة " كبطليموس صاحب المجسطي وغيره لم يتكلموا في ذلك بحرف وإنما تكلم فيه بعض المتأخرين مثل كوشيار الديلمي ونحوه لما رأوا الشريعة جاءت باعتبار الرؤية. فأحبوا أن يعرفوا ذلك بالحساب فضلوا وأضلوا.}" \href{https://shamela.ws/book/7289/4480#p1}{\faExternalLink} \cite{ibnTaimia_Majmoo}.\footnote{مجموع الفتاوى 9/215.}}. وعليه فإن شريعة الإسلام غير متوقفة على الحساب كما أنها غير متوقفة على القراءة والكتابة ولهذا كان الرسول ﷺ وأصحابه الكرام في غالبهم أميين. ولكن شريعة الإسلام جاءت بالحث على تعلم القراءة والكتابة أولا لنشر الحق وبيانه، وتعلم الحساب والأخذ به ثانيا لإقامة الميزان بالقسط في المعاملات بين الناس كما بين الله جل جلاله ذلك في كتابه العظيم. فدل هذا على وجوب الأخذ بعلم الحساب مع القراءة والكتابة على اللاحقين من أمة الإسلام وبالأخص متى أحتاج المسلمين لذلك حتى يبنوا عليه مصالحهم الدينية والدنيوية. 

وأما أن يقال على من يستدل بالعقل والحساب في الأمور الكونية الظاهرة السببية بطرق الإستدلال التي دل الله تبارك وتعالى عليها بأنهم علماء كلام وأصحاب نظريات فهذا كله باطل. وهذا أولا لأن أهل الكلام هم من يتكلمون في العلم الشرعي بالفلفسة والتي بها يتم تحكيم العقل وتقديمه على النقل وهذا بلا شك مخالف لصريح الكتاب والسنة ومنهج أهل السنة والجماعة. وليس كل علم هو علم ظني ويبنى على نظريات. فالحساب من أهم طرق الإستدلال التي بها يتم إثبات النظريات أو دحضها. ولهذا فإن من العلم الكوني ما هو مجرد ظن ونظري ومنها ما هو مؤكد بالحجة والبراهين الموافقة للعقل والثابتة بالقياس والتجريب. وأما التعميم على كل ذلك على أنه مجرد ظن فهذا لا يصح لا شرعا ولا كونا. 

ما انتشرت هذه المفاهيم الخاطئة إلا بالأخذ بالمشاهدة فقط دون الإعتبار والنظر والبحث في علم الحساب مع القياس والحجة العقلية مما زاد الفراغ بين العلم الكوني والعلم الشرعي حتى شاع بين الناس تعارض الأمرين وهذا يخالف أمر الله. فالله جل جلاله أرشدنا إلى النظر في آياته الكونية ليس فقط لمجرد التفكر فيها ولكن أيضا لتعقلها وفهمها على الوجه الصحيح ولتعلم الحساب منها. فكان هذا التقصير سببا إلى جمود القلب حتى أصبح الناس أقل نظرا لآيات الله تعالى وأقل إعتبارا بها. وهذا فيه تفويت للعديد من المصالح الدينية والدنيوية ومن أهم ذلك أن النظر في آيات الله الكونية والإعتبار بها والتفكر فيها وفي كفيتها وتعلم تفاصيلها على الوجه الصحيح يورث خشيته سبحانه وتعالى ومعرفته ومعرفة دقة خلقه وفيه دليل ربوبيته وأولوهيته وبيان صفاته. وقد ترتب على هذا الخلط والتضارب العديد من المفاسد حتى شاع بين الناس الفصل والخلاف بين العلم الكوني والعلم الشرعي فأصبح الناس في غالبهم بين نقيضين. فريق يأخذ بالعلم الشرعي ويرد العلم الكوني، وفريق يأخذ بالعلم الكوني ويرد العلم الشرعي. ومن أسباب ذلك تكلم أهل العلم الكوني خلافا للعلم الشرعي فأخطئوا كثيرا وضلوا ضلال بعيدا حتى وصلوا بذلك إلى الإلحاد. وتكلم أهل العلم الشرعي في بعض مسائل الدنيا خلافا للعلم الكوني المؤكد والثابت فأخطئوا في بعضها وأصابوا في بعضه الأخر بحسب ما فهموه وعقلوه.\comment{وكما قيل: من تكلم في غير فنه أتى بالعجائب.} فحبذ لو لم يُتَكَلَّمُ في العلم الكوني أخذا بالعلم الشرعي فقط، وحبذ لو لم يُتَكَلَّمُ في العلم الشرعي أخذا بالعلم الكوني فقط، إلا بعد الجمع بينهما والبحث والإجتهاد في ذلك سعيا إلى معرفة الحق وبيانه. فالله جل جلاله الذي شرع هذا الدين وأنزل القرآن العظيم هو من خلق هذا الكون، فلا يكون أن يتعارض الأمر الكوني المؤكد مع الأمر الشرعي الثابت والصحيح إلا بالفهم الخاطئ. ولهذا فقد ألف شيخ الإسلام بن تيمية في ذلك كتاب درء تعارض العقل والنقل \cite{ibnTaimia_DTAWN}.


ونتيجة لهذه المفاهيم الخاطئة ذهب الناس في ذلك بين مغال أراد أن يجعل الشريعة متوقفة على الحساب، وبين مجاف أراد أن يجعل الشريعة مستغنية عن الحساب. وهذا كله باطل يخالف أمر الله الذي حث عليه في كتابه العظيم. فلو كان الشرع يدرك بالعقل والحساب فقط أو مقدما عليه لما أرسل الله جل جلاله الرسل وأنزل الكتب ولكتفى الناس بعقولهم في إدراك وحساب أمر الله جل جلاله وهذا باطل. ولو كانت الشريعة مستغنية عن الحساب والعقل لما أرشدنا الله جل جلاله في آياته الشرعية بالنظر في آياته الكونية للتفكر فيها ولتعلم الحساب منها في أكثر من موضع في كتابه العظيم. ولكان هذا الإتقان في هذه الآيات الكونية لعبا وقد نزه الله عز وجل نفسه عن ذلك كله فقال جل في علاه: \quranayah*[44][38-39] {\footnotesize (\surahname*[44])}.

فمن أراد أن يجعل الشريعة متوقفة على الحساب والعقل فقط فقد خالف منهج السلف والأخذ بأيسر الأمور التي جاء بها دين الإسلام ومن ذلك غلاة المنطق الذين أرادوا تقديم العقل على النقل فخالفوا صريح الكتاب والسنة كما بين ذلك شيخ الإسلام بن تيمية في مسألة الرؤية. ولقد صح عن النبي ﷺ أنه قال: صومُوا لِرُؤيتِه، وأفْطِرُوا لِرُؤيتِه، وانْسُكُوا لها، فإنْ غُمَّ عليكم فأتِمُّوا ثلاثينَ، فإنْ شَهِدَ شاهِدانِ مُسلِمانِ فصُومُوا وأفْطِرُوا \href{https://shamela.ws/book/25794/15398#p2}{\faExternalLink} \href{https://shamela.ws/book/21659/7259#p1}{\faExternalLink} \cite{ahmid}.\footnote{مسند أحمد: 18895، النسائي: 2116، وصححه الألباني في صحيح الجامع.} ولهذا فقد قرر أهل السنة والجماعة قديما وحديثا بالأخذ بالرؤية للهلال بدلا من الحساب أولا أخذا بأمر النبي ﷺ وثانيا حتى لا يستأثر بذلك أهل المنطق ممن قل فيهم من أتقن الحساب حقا فضلا عن الفهم الصحيح لدين الإسلام. وإلا فإن الله جل جلاله يقول: \quranayah*[55][5]{\footnotesize \surahname*[55]}، وهذا يمكن حسابه حسابا صحيحا لمن أتقن هذا العلم وصدق في ذلك. فلا يمنع أن يحسب ذلك ولو على وجه التقريب مع الأخذ بالرؤية وإجراء دراسات وبحوث والإجتهاد في ذلك لتطوير طرق حسابية دقيقة موافقة للرؤية وإختبار ذلك ومراجعته مع متابعة أهل العلم الشرعي وأهل الفلك لتقريب وجهات النظر بما يوافق العلم الشرعي الصحيح مع العلم الكوني الثابت.

ومن المسائل التي أخطأ فيها بعض أهل العلم الشرعي المعاصرين رغم ثبوت عكس ذلك بالقياس والحساب هي القول بأن الشمس هي التي تدور على الأرض وأن الأرض ثابتة بالكلية وهذا القول خاطئ كما تقدم بيانه. وسبب ذلك هو عدم الإعتبار بما ثبت بالحساب والقياس رغم أن جميع آيات العدد الحساب في القرآن الكريم جائت مقرونة بالليل والنهار والشمس والقمر وفيها الإشارة الواضحة بالإعتبار بالحساب من الله تبارك وتعالى. وهذا فيه مخالفة الصواب وقد يفهم منه عند البعض التزهيد في العلوم الكونية السببية النافعة وكأنها تزييف للواقع وهذا غير صحيح. وقد يكون ذلك سببا في عدم فهم آيات الله الكونية على وجهها الصحيح وأن ذلك قد يؤدي إلى قلة التدبر فيها والإعتبار بها عند أهل الإيمان. وقد يؤدي ذلك أيضا إلى التخلف الحضاري أو تصوير دين الإسلام أنه مخالف للحضارة. وقد يستغل ذلك الملحدين لتضليل الناس عن الدين زعما بأن أهل الإيمان لا يعتبرون بالقياس والعلم المؤكد. وقد يلحد من ثبت له عكس ما أقره أهل الإيمان فهما بأن الإيمان يتعارض مع الواقع، إلى غير ذلك من المصائب. والصواب في هذا أن الأرض تسبح أي تدور حول نفسها وتسير في فلكها حول الشمس مع دورانها وبهذا فالشمس بالنسبة للأرض ثابتة كما أنها ثابتة بالنسبة لسائر الكواكب الأخرى في المجموعة الشمسية بسبب ظاهرة الجاذبية وجميعهم يتحركون معا حول المجرة كما هو معروف. ويحفظ للمخطئ في ذلك من أهل السنة والجماعة فضله ويعرف حقه علينا ولا ينقص ذلك من قدرهم شيئا فقد قدموا لأمة الإسلام الكثير من الخير ما يصغر أمامه هذه الأخطاء التي لا تخالف أساس الدين. ولكن لما كثر إستغلال هذا الأمر من الملحدين وأعداء الدين وضعاف الإيمان وجب البحث في هذا الأمر لبيان الحق في ذلك والتنبيه وعليه. ولقد تعلمنا من علماء السنة قديما وحديثا الرجوع إلى الحق في كل شيء متى إستبان ذلك فرحم الله علماء السنة في كل زمان ومكان. 

ويستفاد من هذا الأمر أن البشر بطبيعتهم يخطئون ويصيبون والله وحده فوق كل ذي علم عليم. قال ابن عباس: الله العليم، وهو فوق كل عالم. وقال أيضا: يكون هذا أعلم من هذا، وهذا أعلم من هذا، والله فوق كل عالم. وقال الحسن البصري: ليس عالم إلا فوقه عالم، حتى ينتهي إلى الله عز وجل \cite{tafsir_ibnKathir}. ويفهم أيضا أن البشر لا يزالون يتقدمون في الحضارة بحسب ما فتح الله عليهم من المعرفة في الأمور الكونية. ودين الإسلام العظيم هو دين الحضارة فحق على أهله أن يكونوا في مقدمة الحضارة في كل العلوم الكونية السببية النافعة بما يرضي الله كما كان الحال في زمن هارون الرشيد رحمه الله. ومن المعلوم أنه يصعب على أهل العلم الشرعي معرفة العلوم الكونية أو التكلم فيها أخذا بظاهر الأدلة الشرعية فقط وبالأخص في الأمور التي تتطلب الخبرة والمعرفة والبحث في المجالات المختلفة كالفلك، والطب، والهندسة، والحساب، وغيرها من علوم الدنيا. كيف لا والنبي ﷺ قد قال بعدما نهى الناس عن تلقيح النخل: أَنْتُمْ أَعْلَمُ بأَمْرِ دُنْيَاكُمْ {\footnotesize (صحيح مسلم)}. وقال أيضا ﷺ: إنْ كانَ يَنْفَعُهُمْ ذلكَ فَلْيَصْنَعُوهُ؛ فإنِّي إنَّما ظَنَنْتُ ظَنًّا، فلا تُؤَاخِذُونِي بالظَّنِّ، وَلَكِنْ إذَا حَدَّثْتُكُمْ عَنِ اللهِ شيئًا، فَخُذُوا به؛ فإنِّي لَنْ أَكْذِبَ علَى اللهِ عَزَّ وَجَلَّ {\footnotesize (صحيح مسلم)}. حتى قال لهم ﷺ: إنَّما أَنَا بَشَرٌ، إذَا أَمَرْتُكُمْ بشَيءٍ مِن دِينِكُمْ، فَخُذُوا به، وإذَا أَمَرْتُكُمْ بشَيءٍ مِن رَأْيِي، فإنَّما أَنَا بَشَرٌ {\footnotesize (صحيح مسلم)}. فجعل كلامه ﷺ في أمور الدنيا ظن ورأي، وقال: (لا تُؤَاخِذُونِي به) ولكن كلامه في أمر الله حق. وبين النبي ﷺ أنه بشر يخطئ ويصيب في أمور الدنيا كغيره من البشر ولهذا قال:  (إنَّما أَنَا بَشَرٌ، إذَا أَمَرْتُكُمْ بشَيءٍ مِن دِينِكُمْ، فَخُذُوا به، وإذَا أَمَرْتُكُمْ بشَيءٍ مِن رَأْيِي، فإنَّما أَنَا بَشَرٌ). وهذا من تواضعه وصدقه ﷺ فهو الصادق الأمين الذي لا يكذب على الله ولا على الناس عليه الصلاة والسلام.

\section{الخلاصة}

وأما الطريق القويم في دين الله عز وجل هو الأخذ بعلم الحساب والبحث فيه وتعلمه من آيات الله الكونية والإنتفاع والإستدلال به كما أرشدنا ربنا سبحانه وتعالى في كتابه العظيم. وهذا فيه تنشيط للعقل والتبصر في الأمور ومعرفة حقائقها بحسب ما أظهر الله لها من أسباب. وفيه قوة الحجة في إثبات صنع الخالق ورد شبه الملحدين. وفيه تفريح للنفس وزيادة في الإيمان حتى يعرف الإنسان أن الله عز وجل لم يخلق هذا الكون عبثا بل إن الله عز وجل قد خلق هذا الكون وأتقنه وفصَّله لغاية عظيمة فكل ذلك فيه دليل ربوبيته وألوهيته سبحانه وتعالى وفيه أيضا بيان صفاته وشأنه في خلقه وحكمته فسبحان الله أسرع الحاسبين. وكل هذا رحمة بنا حتى نخشاه فنتبع أمره رجاء لمغفرته وخوفا من عقابه. فالله جل جلاله كلفنا بالحق والميزان الشرعي معا فكلاهما من الميثاق والأمانة التي بها يقوم الدين وتصلح الدنيا وهذه هي دعوة الأنبياء والرسل عليهم السلام.

