\chapter{الميزان الكوني والميزان الشرعي}

\section{المقدمة}

علم الحساب من العلوم التي تدرك بالعقل والفطرة وقد يكون هو العلم الوحيد الذي يكاد لا يختلف عليه البشر بكافة أجناسهم وألوانهم وبالأخص لمن عرف هذا العلم وتمعن فيه صدقا. وذلك لأن الله جل جلاله خلق كل شئ بقدر معلوم ووضع الميزان الكوني فجعل هذا الكون موزونا ومتناسقا سبحانه. ومن فضله ومنه علي الناس أنه أرسل إليهم الرسل وأنزل الكتب بالحق والميزان الشرعي. ومن حكمته أنه سبحانه فطر الناس على فهم الميزانان وجعل لهم كل ما يحتاجونه من عقل وسمع وبصر. فجعل سبحانه آياته الكونية دليلا على الميزان الكوني وآياته الشرعية دليلا على الميزان الشرعي. وأرشد سبحانه إلى التأمل في آياته الكونية لتعلم العدد والحساب وهذا لحكمته سبحانه فالعدد والحساب يدرك بالعقل والفطرة ولهذا اكتفى سبحانه بالدلالة عليه. أما الميزان الشرعي فهو لا يدرك بالعقل والفطرة فقط وأنما يدرك بالوحي المنزل من عنده سبحانه. والله جل جلاله تكفل بإقامة الميزان الكوني وأرسل الرسل وأنزل الكتب وفرض على الناس إقامة الميزان الشرعي. ولما كان الحساب هو الوسيلة لمعرفة الحقائق وضبطها والطريق لمعرفة الأسباب وربطها وجب النظر والبحث فيه وتعلمه من آيات الله الكونية لفهمها لما في ذلك من مصالح دينية ودنيوية كما أرشد سبحانه في  كتابه العظيم في قوله تعالى: \quranayah*[10][5]{\footnotesize \surahname*[10]}. وفي قوله تعالى:
\quranayah*[17][12]{\footnotesize \surahname*[17]}.

\section{الميزان الكوني}

جعل الله جل جلاله الميزان في آياته الكونية لحكمته وعدله سبحانه. ولهذا فإن الميزان الكوني أمره عظيم عند الله فعَنْ أَبِي هُرَيْرَةَ أَنَّ رَسُولَ اللَّهِ ﷺ قَالَ "يَدُ اللَّهِ مَلأَى لاَ يَغِيضُهَا نَفَقَةٌ، سَحَّاءُ اللَّيْلَ وَالنَّهَارَ، أَرَأَيْتُمْ مَا أَنْفَقَ مُنْذُ خَلَقَ السَّمَوَاتِ وَالأَرْضَ، فَإِنَّهُ لَمْ يَغِضْ مَا فِي يَدِهِ، عَرْشُهُ عَلَى الْمَاءِ وَبِيَدِهِ الأُخْرَى الْمِيزَانُ يَخْفِضُ وَيَرْفَعُ" {\footnotesize (صحيح البخاري)}. وهذا فيه أن الميزان الكوني بيده سبحانه فهو قائم عليه بالقسط والعدل لا تأخذه في ذلك سنة ولا نوم، ولهذا كانت آية الكرسي أعظم آية فقد قال سبحانه:
\quranayah*[2][255][1-12] {\footnotesize (\surahname*[2])}. وَعَن أبي مُوسَى قَالَ قَامَ فِينَا رَسُولُ اللَّهِ ﷺ بِخَمْسِ كَلِمَاتٍ فَقَالَ: إِنَّ اللَّهَ عز وَجل لَا يَنَامُ وَلَا يَنْبَغِي لَهُ أَنْ يَنَامَ يَخْفِضُ الْقِسْطَ وَيَرْفَعُهُ، يُرْفَعُ إِلَيْهِ عَمَلُ اللَّيْلِ قَبْلَ عَمَلِ النَّهَارِ وَعَمَلُ النَّهَارِ قَبْلَ عَمَلِ اللَّيْل حجابه النُّور. {\footnotesize (رَوَاهُ مُسلم وصححه الألباني)}. وهذا فيه أن الله عز وجل كامل في صفاته لا يلحقه نقص وهذا لازم لإقامة الوجود إذ يتعذر على غيره إقامة الميزان الكوني كما دلت على ذلك الآيات القرآنية والأحاديث. وقد جاء في تفسير ابن كثير أن قوله: ( لا تأخذه سنة ولا نوم ) أي: لا يعتريه نقص ولا غفلة ولا ذهول عن خلقه بل هو قائم على كل نفس بما كسبت شهيد على كل شيء لا يغيب عنه شيء ولا يخفى عليه خافية ، ومن تمام القيومية أنه لا يعتريه سنة ولا نوم ، فقوله: ( لا تأخذه ) أي: لا تغلبه سنة وهي الوسن والنعاس ولهذا قال: ( ولا نوم ) لأنه أقوى من السنة, [.] وقوله: ( ولا يئوده حفظهما ) أي: لا يثقله ولا يكرثه حفظ السماوات والأرض ومن فيهما ومن بينهما ، بل ذلك سهل عليه يسير لديه وهو القائم على كل نفس بما كسبت ، الرقيب على جميع الأشياء ، فلا يعزب عنه شيء ولا يغيب عنه شيء والأشياء كلها حقيرة بين يديه متواضعة ذليلة صغيرة بالنسبة إليه ، محتاجة فقيرة وهو الغني الحميد الفعال لما يريد ، الذي لا يسأل عما يفعل وهم يسألون. وهو القاهر لكل شيء الحسيب على كل شيء الرقيب العلي العظيم لا إله غيره ولا رب سواه [هـ].

وبهذا يتبين أن الميزان الكوني بيد الله سبحانه وتعالى وهو قائم عليه بالقسط والعدل يخفضه ويرفعه كما في قوله تعالى: 
\quranayah*[3][18]{\footnotesize \surahname*[3]}. وهو الحي القيوم ولا ينام ولا ينبغي له أن ينام وهذا لازم لوجود الكون وصلاحه فهو مدبره سبحانه وهو مالكه. ولا يعلم على وجه التحديد متى خلق الله جل جلاله هذا الميزان الكوني ولكن يعلم بالضرورة أن هذا الميزان الكوني كان موجودا عندما رفع الله السموات حيث وضعه جل جلاله في يده. وكل ذلك كان بعد خلق القلم واللوح والعرش والماء، فقد جاء عن النبي ﷺ أنه قال: إنَّ أولَ ما خلق اللهُ القلمُ، فقال لهُ: اكتبْ، قال: ربِّ وماذا أكتبُ ؟ قال: اكتُبْ مقاديرَ كلِّ شيءٍ حتى تقومَ الساعةُ. ومن مات على غيرِ هذا فليسَ مِني {\footnotesize (صحيح أبي داود وصححه الألباني)}. وهذا فيه أن الله خلق القلم أولا ثم خلق اللوح المحفوظ وأمر جل جلاله القلم أن يكتب على اللوح المحفوظ المقادير كلها إلى قيام الساعة. وكل هذا كان قبل خلق السموات والأرض بخمسين ألف سنة وكان عرشه جل جلاله على الماء كما صح ذلك عن النبي ﷺ أنه قال:  كَتَبَ اللَّهُ مَقَادِيرَ الخَلَائِقِ قَبْلَ أَنْ يَخْلُقَ السَّمَوَاتِ وَالأرْضَ بِخَمْسِينَ أَلْفَ سَنَةٍ، وَعَرْشُهُ علَى المَاءِ {\footnotesize (صحيح مسلم)}. وسماها الله جل جلاله المقادير لأنه قدرها وعرف قدرها بعلمه وحكمته وعدله ورحمته سبحانه حتى يقوم بنفسه على ذلك في الميزان الكوني الذي وضعه سبحانه في يده بعد رفع السماء كما في قوله تعالى: 
\quranayah*[55][7]{\footnotesize \surahname*[55]}. فيكون المعنى هنا الميزان الذي وضعه الله في يده بعد رفع السموات وهو الميزان الكوني ويحتمل أيضا الميزان الذي وضعه للمكلفين من خلقه من الإنس والجن وهو الميزان الشرعي كما في باقي الآيات الكريمة في قوله تعالى: 
\quranayah*[55][8-9]{\footnotesize \surahname*[55]}.

وكل هذا فيه أن الله جل جلاله قد جعل الميزان الكوني سببا لإستقامة السموات والأرض وصلاحهما كما في قوله تعالى:
\quranayah*[23][71]{\footnotesize \surahname*[23]}. يقول السعدي في تفسيره:
ووجه ذلك أن أهواءهم متعلقة بالظلم والكفر والفساد من الأخلاق والأعمال، فلو اتبع الحق أهواءهم لفسدت السماوات والأرض، لفساد التصرف والتدبير المبني على الظلم وعدم العدل، فالسماوات والأرض ما استقامتا إلا بالحق والعدل[هـ]. وبهذا يعلم أن السموات والأرض تفسد بالظلم وهذا لا يكون لان الله أقام الميزان الكوني بالعدل. ولا يعلم صورة هذا الميزان ولا وصفه إلا بما وصفه الله جل جلاله ورسوله ﷺ ومن ذلك أن الله وضعه في يده ويخفض به ويرفع وهو قائم عليه بنفسه كما قال الرسول ﷺ: 
ما منْ قلبٍ إلا وهوَ معلقٌ بينَ إصبعينَ منْ أصابعِ الرحمنِ، إنْ شاءَ أقامَهُ، و إنْ شاءَ أزاغَهُ، و الميزانُ بيدَ الرحمنَ، يرفعُ أقوامًا، ويخفضُ آخرينَ، إلى يومِ القيامةِ {\footnotesize (صحيح، الجامع الصغير)}. وهذا فيه أن هذا الميزان هو الميزان الكوني التابع لإرادة الله الكونية وأن الله قائم عليه إلى قيام الساعة وهو غير الميزان الشرعي التابع لإرادة الله الشرعية الذي يوضع يوم الحساب كما سيأتي بيانه إن شاء الله تعالى.

فالميزان الكوني قد تكفل به سبحانه وهو دليل على عظمته وقدرته إذ يتعذر على غيره العيش من دونه فضلا عن إقامته، ومن ذلك أن الله عدل في قضاءه ومشيئته وحليم في تدبيره والدليل على هذا قوله تعالى:
\quranayah*[35][41]{\footnotesize \surahname*[35]}. فهذا دليل على وحدانيته سبحانه بالملك والتدبير، فالأدلة العقلية والشرعية دلت على وحدانيته سبحانه ومن ذلك أنه جعل هذا الكون موزونا ومتناسقا بإرادته الكونية لا يشاركه في ذلك أحد كما في قوله تعالى:
\quranayah*[21][22]{\footnotesize \surahname*[21]}. وقد جاء في تفسير السعدي في شرح هذه الآيات أن العالم العلوي والسفلي، على ما يرى، في أكمل ما يكون من الصلاح والانتظام، الذي ما فيه خلل ولا عيب، ولا ممانعة، ولا معارضة، فدل ذلك، على أن مدبره واحد، وربه واحد، وإلهه واحد، فلو كان له مدبران وربان أو أكثر من ذلك، لاختل نظامه، وتقوضت أركانه فإنهما يتمانعان ويتعارضان، وإذا أراد أحدهما تدبير شيء، وأراد الآخر عدمه، فإنه محال وجود مرادهما معا، ووجود مراد أحدهما دون الآخر، يدل على عجز الآخر، وعدم اقتداره واتفاقهما على مراد واحد في جميع الأمور، غير ممكن، فإذًا يتعين أن القاهر الذي يوجد مراده وحده، من غير ممانع ولا مدافع، هو الله الواحد القهار، ولهذا ذكر الله دليل التمانع في قوله:
\quranayah*[23][91]{\footnotesize \surahname*[23]} [هـ].

\section{الميزان الشرعي}

ومن حكمته سبحانه وتعالى أنه تكفل بإقامة الميزان الكوني وفرض على المكلفين من الجن والإنس إقامة الميزان الشرعي كما في قوله:
\quranayah*[55][7-9]{\footnotesize \surahname*[55]}. وقد جاء في تفسير السعدي أن معنى ووضع الله الميزان أي: العدل بين العباد، في الأقوال والأفعال، وليس المراد به الميزان المعروف وحده، بل هو كما ذكرنا، يدخل فيه الميزان المعروف، والمكيال الذي تكال به الأشياء والمقادير، والمساحات التي تضبط بها المجهولات، والحقائق التي يفصل بها بين المخلوقات، ويقام بها العدل بينهم [هـ]. وهذا بالتأكيد يشمل الحساب والذي به تحسب المقادير وتضبط المجهولات فهو أيضا صورة من صور الميزان. وكل هذا فيه أن الله عز وجل فرض على المكلفين من الإنس والجن إقامة الميزان الشرعي في قوله تعالى "ألا تطغوا في الميزان، وأقيموا الوزن بالقسط ولا تخسروا الميزان". وفيه أيضا أن الله عز وجل جعل العدل في آياته الشرعية كما في آياته الكونية وهذا دليل على حكمته وعدله وكماله سبحانه.

إن المقصود والمراد من إقامة الميزان بالمجمل هي إقامة الميزان الشرعي وهو العدل في جميع الأقوال والأفعال وهذا يشمل العبادات التي يحبها الله ويرضاها والتي أمر الله عباده بها عن طريق الرسل والكتب ومن ذلك إقامة الميزان والكيل بالقسط والصدق في القول والعمل ومنه بلا شك الحساب الصحيح. وبهذا يتبين أن الميزان الشرعي، حاله كحال الحق، هو من الأمانة في قوله تعالى:
\quranayah*[33][72]{\footnotesize \surahname*[33]}. وفي تفسير ابن كثير: قال العوفي عن ابن عباس يعني بالأمانة الطاعة [.] وقال قتادة الأمانة الدين والفرائض والحدود [هـ].

الميزان الشرعي هو أيضا من الميثاق الذي ذكره الله في قوله:
\quranayah*[5][7]{\footnotesize \surahname*[5]}. يقول السعدي في تفسيره رحمه الله:
و { مِيثَاقَهُْ} أي: واذكروا ميثاقه { الَّذِي وَاثَقَكُمْ بِهِ ْ} أي: عهده الذي أخذه عليكم. وليس المراد بذلك أنهم لفظوا ونطقوا بالعهد والميثاق، وإنما المراد بذلك أنهم بإيمانهم بالله ورسوله قد التزموا طاعتهما، ولهذا قال: { إِذْ قُلْتُمْ سَمِعْنَا وَأَطَعْنَا ْ} أي: سمعنا ما دعوتنا به من آياتك القرآنية والكونية، سمع فهم وإذعان وانقياد. وأطعنا ما أمرتنا به بالامتثال، وما نهيتنا عنه بالاجتناب. وهذا شامل لجميع شرائع الدين الظاهرة والباطنة. وأن المؤمنين يذكرون في ذلك عهد الله وميثاقه عليهم، وتكون منهم على بال، ويحرصون على أداء ما أُمِرُوا به كاملا غير ناقص. { وَاتَّقُوا اللَّهَ ْ} في جميع أحوالكم { إِنَّ اللَّهَ عَلِيمٌ بِذَاتِ الصُّدُورِ ْ} أي: بما تنطوي عليه من الأفكار والأسرار والخواطر. فاحذروا أن يطلع من قلوبكم على أمر لا يرضاه، أو يصدر منكم ما يكرهه، واعمروا قلوبكم بمعرفته ومحبته والنصح لعباده. فإنكم -إن كنتم كذلك- غفر لكم السيئات، وضاعف لكم الحسنات، لعلمه بصلاح قلوبكم [هـ].

ولهذا فإن الميزان الشرعي يعتبر من الأمانة والميثاق الذي واثق الله به المؤمنين وأمرهم به ليجزى كل نفس بما كسبت. وبهذا يتبين أن الميزان المخلوق الذي يوضع بعد الصراط يوم القيامة إنما هو الميزان الشرعي وهذا من عدل الله إذ جعل الميزان الذي يوزن به الناس يوم القيامة هو الميزان الشرعي الذي كلفهم به. وقد صح عن الرسول ﷺ أن هذا الميزان يوضع في صورة مخلوق بعد الصراط فعن أَنَسِ بْنِ مَالِكٍ قال سألتُ النَّبيَّ صلَّى اللَّهُ عليْهِ وسلَّمَ أن يشفعَ لي يومَ القيامةِ فقالَ: أنا فاعِلٌ، قلتُ: يا رسولَ اللَّهِ فأينَ أطلبُكَ، قالَ: اطلُبني أوَّلَ ما تطلُبُني على الصِّراطِ. قلتُ: فإن لم ألقَكَ على الصِّراطِ، قالَ: فاطلُبني عندَ الميزانِ. قلتُ: فإن لم ألقَكَ عندَ الميزانِ، قالَ: فاطلُبني عندَ الحوضِ فإنِّي لا أخطئُ هذِهِ الثَّلاثَ المواطنَ {\footnotesize (صحيح الترمذي وصححه الألباني)}. وهذا فيه أن الميزان الشرعي يوضع في صورته بعد الصراط مباشرة وقبل الحوض. وقد صح عن أحد أصحاب النبي ﷺ أنهم رأى هذا الميزان الشرعي في منامه كما جاء عن النبي ﷺ أنه ذات يوم قال لأصحابه: من رأَى منكُم رؤْيا؟ فقال رجلٌ: أنا، رأيتُ كأنَّ ميزانًا نزلَ من السَّماءِ فَوُزِنْتَ أنتَ وأبو بكرٍ فَرَجِحْتَ أنتَ بأبيِ بكرٍ، ووُزِنَ عُمَرُ وأبو بكرٍ فَرُجِحَ أبو بكرٍ، ووُزِنَ عُمَرُ وعُثْمانُ فَرُجِحَ عُمَرُ، ثُمَّ رُفِعَ الميزانُ. فَرأَيْنا الكراهيةَ في وجهِ رسولِ اللهِ ﷺ {\footnotesize (صحيح أبي داود، وصححه الألباني)}. وهذا فيه بيان الميزان الشرعي الذي رجح بحسب من فضل الله به النبي ﷺ وأصحابه الكرام رضوان الله عليهم.

والأدلة في وصف الميزان الشرعي ووصف حال المكلفين وأعمالهم وهم يوزنون عليه كثيرة ومنها قوله تعالى:
\quranayah*[7][8-9]{\footnotesize \surahname*[7]}. وقال تعالى:
\quranayah*[18][105]{\footnotesize \surahname*[18]}. وقال تعالى:
\quranayah*[21][47]{\footnotesize \surahname*[21]}. وقال تعالى:
\quranayah*[23][102-103]{\footnotesize \surahname*[23]}. وقال تعالى:
\quranayah*[101][6-9]{\footnotesize \surahname*[101]}. فكل ذلك فيه وصف حال وأحوال الناس وأعمالهم على هذا الميزان الشرعي الذي كلف جل جلاله المكلفين به من الجن والإنس فنسأل الله العفو والعافية. اللهم فاعف عنا وارحمنا وتجاوز عن سيئاتنا. اللهم ارحمنا فأنت ارحم الراحمين.

ولهذا فإن هذا الميزان الشرعي موافق لما شرعه الله تعالى وقد وصف النبي ﷺ وزن الأعمال الصالحة في هذا الميزان الشرعي ومن ذلك قوله: 
ما مِن شيءٍ يوضَعُ في الميزانِ أثقلُ من حُسنِ الخلقِ، وإنَّ صاحبَ حُسنِ الخلقِ ليبلُغُ بِهِ درجةَ صاحبِ الصَّومِ والصَّلاةِ {\footnotesize (صحيح الترمذي، وصححه الألباني)}. وفي رواية أخرى عن أبو الدرداء أن النبي  ﷺ قال:
من أُعطِيَ حظَّه من الرِّفقِ فقد أُعطِيَ حظَّه من الخيرِ ومن حُرِمَ حظُّه من الرِّفقِ ؛ فقد حُرِمَ حظُّه من الخيرِ. أثقلٌ شيءٍ في ميزانِ المؤمنِ يومَ القيامةِ حُسنُ الخُلُقِ، وإنَّ اللهَ لَيبغضُ الفاحشَ البذِيءَ {\footnotesize (صحيح الأدب المفرد، وصححه الألباني)}. وقال ﷺ أيضا: 
كَلِمَتانِ خَفِيفَتانِ علَى اللِّسانِ ، ثَقِيلَتانِ في المِيزانِ، حَبِيبَتانِ إلى الرَّحْمَنِ ، سُبْحانَ اللَّهِ وبِحَمْدِهِ، سُبْحانَ اللَّهِ العَظِيمِ {\footnotesize (صحيح البخاري، وصححه الألباني)}. وقال أيضا:
الطُّهُورُ شَطْرُ الإيمانِ، والْحَمْدُ لِلَّهِ تَمْلأُ المِيزانَ، وسُبْحانَ اللهِ والْحَمْدُ لِلَّهِ تَمْلَآنِ ما بيْنَ السَّمَواتِ والأرْضِ{\footnotesize (صحيح مسلم)}. وقد جاء أيضا أن النبيُّ ﷺ أمرَ عبدَ اللهِ بنَ مسعودٍ أن يصعدَ شجرةً فيأتِيَهُ منها بشيٍء، فنظرَ أصحابُه إلى ساقِ عبدِ اللهِ فضحِكوا من حُمُوشَةِ ساقَيهِ، فقال رسولُ اللهِ ﷺ: ممَّا تضحكون!، لَرِجْلُ عبدِ اللهِ أثقلُ في الميزانِ من أُحُدٍ {\footnotesize (السلسلة الصحيحة للألباني)}. وصح أيضا عن النبي ﷺ أنه قال:
خَصْلتانِ لا يُحافِظُ عليهِما عبدٌ مُسلمٌ إلا دخل الجنةَ، ألا وهُما يَسِيرٌ، ومَن يعملْ بِهِما قَليلٌ، يُسَبِّحُ اللهَ في دُبُرِ كُلِّ صلاةٍ عَشْرًا، ويَحمدُه عشْرًا، ويُكبِّرُه عشْرًا، فذلِكَ خَمسُونَ ومِائَةٌ باللِسانِ، وألفٌ وخَمسُمِائةٌ في المِيزانِ. ويُكبِّرُ أربعًا وثلاثِينَ إذا أخَذَ مَضْجَعَهُ، ويَحمدُه ثلاثًا وثلاثِين، ويُسَبِّحُ ثلاثًا وثلاثِينَ، فتِلكَ مائةٌ باللِسانِ، وألْفٌ في المِيزانِ، فأيُّكمْ يَعْملُ في اليومِ والليلةِ ألْفينِ وخَمسَمائةِ سَيِّئَةٍ {\footnotesize (صحيح الجامع، وصححه الألباني)}. وقد جاء أيضا أن النبي ﷺ قال: بَخٍ بَخٍ - وأشار بيدِه بخَمْسٍ - ما أثقَلَهنَّ في الميزانِ سُبحانَ اللهِ والحمدُ للهِ ولا إلهَ إلَّا اللهُ واللهُ أكبَرُ والولَدُ الصَّالحُ يُتوفَّى للمرءِ المُسلِمِ فيحتسِبُه{\footnotesize (صحيح ابن حبان، والسلسلة الصحيحة للألباني)}. وقال أيضا ﷺ: 
لو أنَّ عِلْمَ عمرَ بنِ الخطابِ رضيَ اللهُ عنهُ وُضِعَ في كفَّةِ الميزانِ، ووُضِعَ عِلْمُ أهلِ الأرضِ في كفَّةٍ، لرجح عِلْمُ عمرَ بنِ الخطابِ رضيَ اللهُ عنهُ {\footnotesize (صححه الألباني)}. والمراد هنا علم عمر الشرعي وهذا فيه فضل عمر رضي الله عنه. فكل ذلك فيه أن المراد في كل هذه الأحاديث هو الميزان الشرعي والذي يوافق أمر الله الشرعي لما يحبه الله ويرضاه من الأعمال والأقوال وهو بخلاف الميزان الكوني الذي جعله الله بيده لتدبير الكون كما تقدم.

\comment{
عن عبدِ اللهِ بنِ عمرٍو حديثًا فيه طُولٌ، وفيه عنه صلَّى اللهُ عليه وسلَّمَ: أنَّ نوحًا لما حضرتْه الوفاةُ دعا بَنيه، فقال: إني قاصٌّ عليكم الوصيةَ، آمُرُكم باثنتينِ، وأنهاكم عن اثنتينِ، أنهاكم عن الشِّركِ والكِبْرِ، وآمُرُكم بلا إلهَ إلَّا اللهُ، فإنَّ السَّمواتِ والأرضَ وما فيها لو وُضِعتْ في كِفَّةِ الميزانِ، ووُضِعتْ لا إلهَ إلَّا اللهُ في الكِفَّةِ الأخرى، كانت أرجَحَ منها، ولو أنَّ السَّمواتِ والأرضَ وما فيها كانت حلْقَةً، فوُضِعتْ [لا إلهَ إلَّا اللهُ] عليها لقَصَمتْهما، وآمُرُكم بسُبحانَ اللهِ وبحَمدِه، فإنَّها صلاةُ كُلِّ شيءٍ، وبها يُرزَقُ كُلُّ شَيءٍ.

قُلتُ لفاطمةَ: لو أتيْتِ النَّبيَّ صلَّى اللهُ عليه وسلَّمَ فسأَلْتِيه خادمًا، فقد أَجهَدكِ الطَّحنُ والعمَلُ؟ -قال حُسَينٌ: إنَّه قد جهَدكِ الطَّحنُ والعمَلُ، وكذلك قال أبو أحمدَ- قالت: فانطلِقْ معي. قال: فانطلَقْتُ معها، فسأَلْناه، فقال النَّبيُّ صلَّى اللهُ عليه وسلَّمَ: ألَا أَدُلُّكما على ما هو خَيرٌ لكما من ذلك؟ إذا أَوَيْتما إلى فِراشِكما فسبِّحا اللهَ ثلاثًا وثلاثينَ، واحمَداه ثلاثًا وثلاثينَ، وكبِّراه أربعًا وثلاثينَ، فتلك مِئَةٌ على اللِّسانِ، وألفٌ في الميزانِ، فقال عليٌّ رضِي اللهُ عنه: ما ترَكْتُها بعدَما سمِعْتُها منَ النَّبيِّ صلَّى اللهُ عليه وسلَّمَ

خَصلتانِ أو خَلَّتانِ لا يحافِظُ عليهما عبدٌ مسلمٌ إلَّا دخلَ الجنَّةَ هما يسيرٌ ومن يعمَلُ بِهِما قليلٌ يسبِّحُ اللَّهَ تعالى دُبُرِ كلِّ صلاةٍ عَشرًا ويحمدُ عشرًا ويُكَبِّرُ عشرًا فذلِكَ خمسونَ ومائةٌ باللِّسانِ وألفٌ وخمسمائةٍ في الميزانِ ويُكَبِّرُ أربعًا وثلاثينَ إذا أخذَ مضجعَهُ ويحمدُ ثلاثًا وثلاثينَ ويسبِّحُ ثلاثًا وثلاثينَ فذلِكَ مائةٌ باللِّسانِ وألفٌ بالميزانِ قال فلَقدْ رأيتُ رسولَ اللَّهِ صلَّى اللَّهُ عليهِ وسلَّمَ يعقدُها بيدِهِ قالوا يا رسولَ اللَّهِ كيفَ هُما يسيرٌ ومن يعمَلُ بِهِما قليلٌ ؟ قالَ : يأتي أحدَكُم يعني الشَّيطانَ في مَنامِهِ فينوِّمُهُ قبلَ أن يقولَهُ

أنَّ رجلًا، قال لرسولِ اللهِ صلَّى اللهُ عليه وسلَّم : رأيتُ كأنَّ ميزانًا دُلِّيَ منَ السماءِ، فوُزِنتَ بأبي بكرٍ فرجَحتَ بأبي بكرٍ، ثم وُزِن أبو بكرٍ بعُمرَ، فرجَح أبو بكرٍ، ثم وُزِن عُمرُ بعثمانَ، فرَجَح عُمرُ، ثم رُفِع الميزانُ، فاستَهَلَّها رسولُ اللهِ صلَّى اللهُ عليه وسلَّم خلافةَ نبوةٍ، ثم يؤتي اللهُ المُلكَ مَن يشاءُ

أنَّ رجلًا قال : يا رسولَ اللهِ رأيتُ كأنَّ مِيزانًا دُلِّي مِنَ السماءِ فَوُزِنْتَ فيه أنت وأبوبكرٍ فَرَجَحْتَ بأبي بكرٍ ثم وُزِنَ فيه أبوبكرٍ وعمرُ فَرَجَحَ أبو بكرٍ بعمرَ ثم وُزِنَ فيه عمرُ وعثمانُ فَرَجَحَ عمرُ بعثمانَ ثم رُفِعَ الميزانُ فاسْتآلهَا يعني تَأَوَّلَها ثم قال : خِلافَةُ نُبُوَّةٍ ثم يُؤتِي اللهُ الملكَ مَنْ يَشاءُ
}



\section{الإرادة الكونية والإرادة الشرعية}

قرر أهل العلم الشرعي من أهل السنة والجماعة في هذا الباب العظيم وبناء على الأدلة والبراهين الواضحة من كتاب الله وسنة نبيه ﷺ وما يوافق العقل والنقل، أن الله جل جلاله له أرادتان وهما الإرادة الكونية والإرادة الشرعية. فالإرادة الكونية هي ما تعلق بمشيئته سبحانه والإرادة الشرعية هي ما تعلق بمحبته ورضاه. فإرادة الله الكونية نافذة كما في قوله تعالى:
\quranayah*[36][82]{\footnotesize \surahname*[36]}. وأما الإرادة الشرعية فهي إرادة بيان لما يحبه الله ويرضاه من الأعمال والأقوال كما في قوله تعالى:
\quranayah*[4][26-28]{\footnotesize \surahname*[4]}.

ولما كان الله عز وجل فعال لما يريد كما في قوله تعالى:
\quranayah*[85][16]{\footnotesize \surahname*[16]}، كان قضاءه تابعا لإرادته أي سبحانه له كذلك قضاء كوني وقضاء شرعي. فالقضاء الكوني هو ما أراده الله كونا فشاء أن يكون فكان بعزته وعلمه وقدرته سبحانه كما في قوله تعالى في هذه الآيات:
\quranayah*[2][117]{\footnotesize \surahname*[2]}. %وقوله:
% \quranayah*[3][47]{\footnotesize \surahname*[3]}. وقوله:
% \quranayah*[19][35]{\footnotesize \surahname*[19]}. وقوله:
% \quranayah*[39][42]{\footnotesize \surahname*[39]}. وقوله:
% \quranayah*[40][68]{\footnotesize \surahname*[40]}. وقوله:
% \quranayah*[41][12]{\footnotesize \surahname*[41]}.
وأما قضاءه الشرعي فهو ما أراده الله شرعا فأمر الله عباده به مثل قوله تعالى:
\quranayah*[17][23]{\footnotesize \surahname*[17]}. فلو كان هذا قضاءا كونيا لكان الناس أمة واحدة على التوحيد ولكن الله نفى ذلك بقضاءه الكوني أي بمشيئته الكونية كما في قوله تعالى: \quranayah*[16][93]{\footnotesize \surahname*[16]}.
ومن ذلك أيضا قوله تعالى:
\quranayah*[33][36]{\footnotesize \surahname*[33]}. فهذا أيضا من القضاء الشرعي ووجه ذلك أنه سبحانه ألزمهم بإتباع أمره الشرعي لأن ذلك من مقتضيات الإيمان ولهذا جاء التحذير في نهاية الآية لمن خالف وعصى أمر الله ورسوله، فلو كان هذا قضاء كونيا لكان ما أراده الله ولم يسع لأحد أن يختار شيئا من ذلك حيث أن أمر الله الكوني نافذ لا محالة.

وكذلك حكم الله تابعا لإرادته وله سبحانه الحكم الكوني وهو تابع لأرادته الكونية والحكم الشرعي وهو تابع لإرادته الشرعية كما في قوله تعالى:
\quranayah*[5][1]{\footnotesize \surahname*[5]}، وهذا في الحكم الشرعي كما دل سياق الآية. وقوله تعالى:
\quranayah*[13][41]{\footnotesize \surahname*[13]}. ويدخل في هذا حكمه الشرعي والقدري (أي الكوني) والجزائي كما جاء في تفسير السعدي رحمه الله.


فحكم الله وقضاءه وأمره الكوني نافذ وماض بإرادته الكونية وبما شاء وهو عدل في ذلك لا يشاركه فيه غيره سبحانه كما قال تعالى:
\quranayah*[40][20]{\footnotesize \surahname*[40]}. وقد جاء في تفسير ابن كثير أن قوله: ( والله يقضي بالحق ) أي: يحكم بالعدل، وقوله: ( والذين يدعون من دونه ) أي: من الأصنام والأوثان والأنداد ، ( لا يقضون بشيء ) أي: لا يملكون شيئا ولا يحكمون بشيء [هـ]. وكما جاء عن النبي ﷺ أنه قال:
«ماضٍ فيَّ حكمُك، عدلٌ فيَّ قضاؤُك» {\footnotesize (أخرجه أحمد وصححه الألباني)}. 

وفي كل ذلك فإن الله هو أحكم الحاكمين كما في قوله تعالى:
\quranayah*[95][8]{\footnotesize \surahname*[95]}.\comment{
وعلى لسان نبيه نوح عليه السلام:
\quranayah*[11][45]{\footnotesize \surahname*[11]}.}
وهو أيضا خير الحاكمين كما في قوله:
\quranayah*[10][109]{\footnotesize \surahname*[10]}.
فالحكم كله لله تعالى ومنه الحكم الجزائي وهو سبحانه خير الفاصلين كما في قوله تعالى:
\quranayah*[6][57][14]{\footnotesize \surahname*[6]}. وذلك لان الله عز وجل عدل في إرادته وحكمه وقضاءه العدل التام المنافي للظلم والدليل قوله تعالى:
\quranayah*[3][108]{\footnotesize \surahname*[3]}. ومن ذلك أن الله عدل في جزاءه وثوابه كما أخبر هو بذلك في قوله تعالى:
\quranayah*[39][69]{\footnotesize \surahname*[69]}. وقوله تعالى:
\quranayah*[10][47]{\footnotesize \surahname*[10]}.

وقد ثبت في السنة أن الله عز وجل حرم الظلم على نفسه في حكمه الكوني وعلى عباده في حكمه الشرعي فعَنِ النَّبيِّ صَلَّى اللَّهُ عليه وسلَّمَ فِيما رَوَى عَنِ اللهِ تَبَارَكَ وَتَعَالَى، أنَّهُ قالَ: يا عِبَادِي، إنِّي حَرَّمْتُ الظُّلْمَ علَى نَفْسِي، وَجَعَلْتُهُ بيْنَكُمْ مُحَرَّمًا، فلا تَظَالَمُوا، يا عِبَادِي، كُلُّكُمْ ضَالٌّ إلَّا مَن هَدَيْتُهُ، فَاسْتَهْدُونِي أَهْدِكُمْ، يا عِبَادِي، كُلُّكُمْ جَائِعٌ إلَّا مَن أَطْعَمْتُهُ، فَاسْتَطْعِمُونِي أُطْعِمْكُمْ، يا عِبَادِي، كُلُّكُمْ عَارٍ إلَّا مَن كَسَوْتُهُ، فَاسْتَكْسُونِي أَكْسُكُمْ، يا عِبَادِي، إنَّكُمْ تُخْطِئُونَ باللَّيْلِ وَالنَّهَارِ، وَأَنَا أَغْفِرُ الذُّنُوبَ جَمِيعًا، فَاسْتَغْفِرُونِي أَغْفِرْ لَكُمْ، يا عِبَادِي، إنَّكُمْ لَنْ تَبْلُغُوا ضَرِّي فَتَضُرُّونِي، وَلَنْ تَبْلُغُوا نَفْعِي فَتَنْفَعُونِي، يا عِبَادِي، لو أنَّ أَوَّلَكُمْ وَآخِرَكُمْ وإنْسَكُمْ وَجِنَّكُمْ، كَانُوا علَى أَتْقَى قَلْبِ رَجُلٍ وَاحِدٍ مِنكُمْ؛ ما زَادَ ذلكَ في مُلْكِي شيئًا، يا عِبَادِي، لوْ أنَّ أَوَّلَكُمْ وَآخِرَكُمْ وإنْسَكُمْ وَجِنَّكُمْ، كَانُوا علَى أَفْجَرِ قَلْبِ رَجُلٍ وَاحِدٍ؛ ما نَقَصَ ذلكَ مِن مُلْكِي شيئًا، يا عِبَادِي، لو أنَّ أَوَّلَكُمْ وَآخِرَكُمْ وإنْسَكُمْ وَجِنَّكُمْ، قَامُوا في صَعِيدٍ وَاحِدٍ فَسَأَلُونِي، فأعْطَيْتُ كُلَّ إنْسَانٍ مَسْأَلَتَهُ؛ ما نَقَصَ ذلكَ ممَّا عِندِي إلَّا كما يَنْقُصُ المِخْيَطُ إذَا أُدْخِلَ البَحْرَ، يا عِبَادِي، إنَّما هي أَعْمَالُكُمْ أُحْصِيهَا لَكُمْ، ثُمَّ أُوَفِّيكُمْ إيَّاهَا، فمَن وَجَدَ خَيْرًا فَلْيَحْمَدِ اللَّهَ، وَمَن وَجَدَ غيرَ ذلكَ فلا يَلُومَنَّ إلَّا نَفْسَهُ. وفي روايةٍ: إنِّي حَرَّمْتُ علَى نَفْسِي الظُّلْمَ وعلَى عِبَادِي، فلا تَظَالَمُوا. {\footnotesize (صحيح مسلم)}.

وهذا فيه أن الله جل جلاله لم يحرم الظلم على عباده كونا بل حرمه شرعا وهذا من حكمته سبحانه فقد بين حكمه الشرعي للمكلفين حتى يغفر لمن يشاء برحمته ويعذب من يشاء بعدله في حكمه الجزائي كما في قوله تعالى: 
\quranayah*[48][14]{\footnotesize \surahname*[48]}.  وقد جاء في تفسير السعدي أن معنى ذلك أن الله تعالى هو المنفرد بملك السماوات والأرض، يتصرف فيهما بما يشاء من الأحكام القدرية، والأحكام الشرعية، والأحكام الجزائية، ولهذا ذكر حكم الجزاء المرتب على الأحكام الشرعية، فقال: { يَغْفِرُ لِمَنْ يَشَاءُ } وهو من قام بما أمره الله به { وَيُعَذِّبُ مَنْ يَشَاءُ } ممن تهاون بأمر الله، { وَكَانَ اللَّهُ غَفُورًا رَحِيمًا } أي: وصفه اللازم الذي لا ينفك عنه المغفرة والرحمة، فلا يزال في جميع الأوقات يغفر للمذنبين، ويتجاوز عن الخطائين، ويتقبل توبة التائبين، وينزل خيره المدرار، آناء الليل والنهار [هـ]. فهذا الحكم الجزائي مرتبط بعدل الله وهدايته فيغفر لمن يشاء بأن يهديه للإسلام والطاعة ويعذب من يشاء بأن يكله إلى نفسه الجاهلة الظالمة المقتضية لعمل الشر فيعمل الشر ويعذب على ذلك كما جاء بيان ذلك في تفسير السعدي رحمه الله.

\section{الهداية الكونية والهداية الشرعية}

وكما أن لله عز وجل له إرادتان الكونية والشرعية فإن له سبحانه أيضا هدايتان وهما الهداية الكونية وهي هداية التوفيق والإنقياد والهداية الشرعية وهي هداية المعرفة والإرشاد. واجتمعت الهدايتان في قوله تعالى:
\quranayah*[42][52]{\footnotesize \surahname*[42]}. ووجه ذلك أن الله عز وجل يهدي من يشاء ومن يريد فهذه الهداية المرتبطة بمشيئته وإرادته سبحانه هي الهداية الكونية وقد وردت في عدة مواضع في القرآن منها قوله تعالى:
\quranayah*[28][56]{\footnotesize \surahname*[28]}. وفي قوله تعالى:
\quranayah*[22][16]{\footnotesize \surahname*[22]}. وأما الهداية الأخرى في قوله تعالى: (وَإِنَّكَ لَتَهْدِي إِلَىٰ صِرَاطٍ مُّسْتَقِيمٍ)
فهي المرتبطة بمعرفة الحق وهي الهداية الشرعية وقد وردت أيضا في عدة مواضع في القرآن منها قوله تعالى:
\quranayah*[4][26]{\footnotesize \surahname*[4]}. وهذه الهداية تكون مرتبطة بالوحي وهي هداية بيان للحق كما في قوله تعالى:
\quranayah*[34][6]{\footnotesize \surahname*[6]}.

وبهذا يتبين أن كل المخلوقات مسييرين بإرادة الله الكونية وأن المكلفين منهم من الجن والإنس مخييرين بإرادة الله الشرعية. والهداية الكونية هي الغالبة، فمن عرف الحق وعمل به فقد هدي شرعا وكونا أي اجتمعت فيه الهدايتان ومثال ذلك أصحاب النبي ﷺ. والناس في كل ذلك درجات برحمة الله وكرمه وبما فضل الله بعضهم على بعض،
ومن عرف الحق ولم يعمل به فقد هدي شرعا ولم يهدى كونا أي لم تجتمع فيه الهدايتان ومثال ذلك ابليس لعنه الله. والناس في ذلك دركات بعدل الله وغضبه وبما أغوى الله بعضهم على بعض. وسيأتي تفصيل ذلك في بينان يوم الحساب الذب فيه يكون الحساب وبعده يأتي الجزاء إما جنة أو نار. 

ولكن الله عز وجل جعل لهدايته الكونية مسببات منها الإنابة إليه كما في قوله تعالى:
\quranayah*[42][13][30]{\footnotesize \surahname*[42]}. وقوله تعالى:
\quranayah*[13][27]{\footnotesize \surahname*[13]}.
ومن ذلك أيضا الإيمان بالله والأعتصام به كما في قوله تعالى:
\quranayah*[4][175]{\footnotesize \surahname*[4]}. ومن ذلك أيضا إتباع أمر الله الشرعي الذي يحبه الله ويرضاه كما في قوله تعالى:
\quranayah*[5][16]{\footnotesize \surahname*[5]}.  وهذا كله من عدل الله ورحمته سبحانه. ومن أسباب الهداية والثبات الدعاء فإن أكثر دعاء نبينا ﷺ كان في ثبات القلب كما جاء عن أم سلمة أن أَكْثرُ دعائِهِ كانَ: يا مُقلِّبَ القلوبِ ثبِّت قلبي على دينِكَ قالَت: فقُلتُ : يا رسولَ اللَّهِ ما أكثرُ دعاءكَ يا مقلِّبَ القلوبِ ثبِّت قلبي على دينِكَ؟ قالَ: يا أمَّ سلمةَ إنَّهُ لَيسَ آدميٌّ إلَّا وقلبُهُ بينَ أصبُعَيْنِ من أصابعِ اللَّهِ، فمَن شاءَ أقامَ، ومن شاءَ أزاغَ . فتلا معاذٌ رَبَّنَا لَا تُزِغْ قُلُوبَنَا بَعْدَ إِذْ هَدَيْتَنَا {\footnotesize (صحيح الترمذي وصححه الألباني)}. وهذا فيه أن الله جل جلاله يقلب القلوب بين أصابعه بإرادته الكونية وأن الدعاء قد يكون سببا للهداية الكونية والتي بها يكون الثبات على الدين والطاعة.  

ومن أعظم أسباب الهداية هي الجهاد في سبيل الله كما في قوله تعالى:  \quranayah*[29][69]{\footnotesize \surahname*[29]}. وقد جاء في تفسير السعدي قوله: دل هذا، على أن أحرى الناس بموافقة الصواب أهل الجهاد، وعلى أن من أحسن فيما أمر به أعانه اللّه ويسر له أسباب الهداية، وعلى أن من جد واجتهد في طلب العلم الشرعي، فإنه يحصل له من الهداية والمعونة على تحصيل مطلوبه أمور إلهية، خارجة عن مدرك اجتهاده، وتيسر له أمر العلم، فإن طلب العلم الشرعي من الجهاد في سبيل اللّه، بل هو أحد نَوْعَي الجهاد، الذي لا يقوم به إلا خواص الخلق، وهو الجهاد بالقول واللسان، للكفار والمنافقين، والجهاد على تعليم أمور الدين، وعلى رد نزاع المخالفين للحق، ولو كانوا من المسلمين [هـ].

\section{الميزان بمعنى العدل والميزان المخلوق}

 العدل مرادف للميزان، والعدل هو صفة من صفات الله جل جلاله والله عدل في إرادته الكونية والشرعية، وبهذا يكون الميزان الكوني تابعا لإرادة الله الكونية والميزان الشرعي تابعا لإرادة الله الشرعية. وكل هذا على وجه الإجمال. وأما على وجه التفصيل، فيكون الميزان الكوني هو العدل في إرادة الله الكونية والميزان الشرعي هو العدل في إرادة الله الشرعية. 

ولكن الله عز وجل جعل الميزانان كل منهما في صورة مخلوق لحكمته ولإظهار عدله سبحانه، فخلق سبحانه الميزان الكوني ووضعه بيده والذي فيه تدبيره للكون وهو قائما عليه بالقسط كما في قوله تعالى:
\quranayah*[3][18]{\footnotesize \surahname*[3]}. ويخلق سبحانه الميزان الشرعي يوم القيامة ويضعه لحساب المكلفين وأعمالهم كما في قوله تعالى: 
\quranayah*[21][47]{\footnotesize \surahname*[21]}.


وبهذا يعلم بالضرورة أن الميزان الكوني المخلوق الذي يضعه الله جل جلاله في يده كما صح ذلك عن النبي ﷺ فيه شأن الله وتدبيره لهذا الكون بما في ذلك السماوات والأرض، وأما الميزان الشرعي المخلوق الذي يوضع بعد الصراط وقبل الحوض كما صح ذلك عن النبي ﷺ فيه شأن حساب المكلفين وأعمالهم يوم القيامة، فلزم بذلك أن يكون الميزان الكوني المخلوق أعظم من الميزان الشرعي المخلوق، وكل منهما من عدل الله ورحمته. وهذا لأن خلق السماوات والأرض أعظم من خلق الناس كما في قوله تعالى: 
\quranayah*[40][57]{\footnotesize \surahname*[40]}. فكل هذا فيه أن شؤون الخلق ومن ذلك البعث والحساب أهون على الله جل جلاله من أمر السموات والأرض كما في قوله تعالى: 
\quranayah*[30][27]{\footnotesize \surahname*[30]}. وقال السعدي في بيان معنى (وَهُوَ أَهْوَنُ عَلَيْهِ): وهذا بالنسبة إلى الأذهان والعقول.

وهذا فيه بيان أن التفاوت هنا في قوله (وهو أهون عليه) ليس في ذات الله وقدرته فهو سبحانه على كل شئ قدير ولا يعجزه شئ ولكن هذا التفاوت إنما هو لبيان الحجة العقلية حيث أن من خلق الإنسان وخلق السموات والأرض وهي أكبر وأعظم، قادر على إحياء الموتى من باب أولى. فتكون الحجة العقلية هنا أن من لم يعجزه الإبتداء لا تعجزه الإعادة وخصوصا لما هو أسهل فهذا أهون. وقد جاء في تفسير ابن كثير وتفسير الطبري: عن ابن عباس قوله: (وَهُوَ أَهْوَنُ عَلَيْهِ) يقول: كلّ شيء عليه هين [هـ]. وقد جاء في تفسير القرطبي: فجعل ما علم من ابتداء خلقه دليلا على ما يخفى من إعادته; استدلالا بالشاهد على الغائب [.]، قال أبو عبيدة: ومن جعل أهون يعبر عن تفضيل شيء على شيء فقوله مردود بقوله تعالى: وكان ذلك على الله يسيرا، وبقوله : ولا يئوده حفظهما. والعرب تحمل أفعل على فاعل، [.] وأنشد أبو عبيدة أيضا: إني لأمنحك الصدود وإنني قسما إليك مع الصدود لأميل (أراد لمائل) [.]، ووجهه أن هذا مثل ضربه الله تعالى لعباده [هـ]. وهذا فيه أن الله جل جلاله يخاطب عباده بما يناسب فهمهم وعقولهم وهذا من عدله ورحمته جل جلاله.

\section{الأصل في هذا الكون استقراره وثباته وتوازنه وبركته}

من المعلوم بالضرورة وما دلت عليه البراهين الشرعية والعقلية أن الأصل في هذا الكون استقراره وثباته وتوازنه وبركته حتى يصلح للحياة ومن ذلك أن الله عزل وجل جعل الأرض مستقرة وثابتة ومبسوطة والجبال أوتادا والسماء مرفوعة والسحاب ممطرة والرياح مسخرة والفلك والأنهار جارية والبحار محسورة والشمس سراجا والقمر نورا والنهار معاشا والليل سكنا والنجوم دليلا والشجار مثمرة والدواب متحركة وسائر المخلوقات المتنوعة وغيرها من الآيات العظيمة الدالة عليه والمرشدة إليه. فكل ذلك من آيات الله الكونية الدالة على عظمته وحكمته سبحانه والتي أراد الله منا بإرادته الكونية أن نراها وبإرادته الشرعية أن نتدبر فيها ونتمعن في تفاصيلها بما أودع فينا من عقل وفطرة. وقد قال تعالى في ذلك: 
\quranayah*[27][93]{\footnotesize \surahname*[27]}. وقوله تعالى:
\quranayah*[41][53]{\footnotesize \surahname*[41]}.

والآيات في ذلك عديدة ومنها قوله تعالى:
\quranayah*[21][30-33]{\footnotesize \surahname*[21]}. وقوله تعالى:
\quranayah*[36][33-40]{\footnotesize \surahname*[36]}. وقوله تعالى:
\quranayah*[2][164]{\footnotesize \surahname*[2]}. وقوله تعالى:
\quranayah*[6][97-99]{\footnotesize \surahname*[6]}.
% \quranayah*[3][190-191]{\footnotesize \surahname*[3]}. وقوله تعالى:
% \quranayah*[13][2-4]{\footnotesize \surahname*[13]}. وقوله تعالى:
% \quranayah*[16][12]{\footnotesize \surahname*[16]}. وقوله تعالى:
% \quranayah*[16][79]{\footnotesize \surahname*[16]}. وقوله تعالى:
% \quranayah*[21][32]{\footnotesize \surahname*[21]}. وقوله تعالى:
% \quranayah*[27][86]{\footnotesize \surahname*[27]}. وقوله تعالى:
% \quranayah*[30][20-27]{\footnotesize \surahname*[30]}. وقوله تعالى:
% \quranayah*[30][37]{\footnotesize \surahname*[30]}. وقوله تعالى:
% \quranayah*[30][46]{\footnotesize \surahname*[30]}. وقوله تعالى:
% \quranayah*[31][31]{\footnotesize \surahname*[31]}. وقوله تعالى:
% \quranayah*[40][13]{\footnotesize \surahname*[40]}. وقوله تعالى:
% \quranayah*[41][37-39]{\footnotesize \surahname*[41]}. وقوله تعالى:
% \quranayah*[42][29]{\footnotesize \surahname*[42]}. وقوله تعالى:
% \quranayah*[42][32-33]{\footnotesize \surahname*[42]}. وقوله تعالى:
% \quranayah*[45][3-5]{\footnotesize \surahname*[45]}. وقوله تعالى:
% \quranayah*[45][13]{\footnotesize \surahname*[45]}. وقوله تعالى:
% \quranayah*[51][20]{\footnotesize \surahname*[51]}. وقوله تعالى:
% \quranayah*[57][17]{\footnotesize \surahname*[57]}. وقوله تعالى:
% \quranayah*[7][58]{\footnotesize \surahname*[7]}.

\comment{

التفكر في السماوات والأرض، والدواب، والليل والنهار، والأرض: 
\quranayah*[45][3-5]{\footnotesize \surahname*[45]}.

التفكر في كتاب الله:
\quranayah*[59][21]{\footnotesize \surahname*[59]}.
\quranayah*[16][44]{\footnotesize \surahname*[16]}.

التفكر في السموات والأرض
\quranayah*[45][13]{\footnotesize \surahname*[45]}.

التفكر في الأرض, والجبال, والأنهار, والثمرات, والأزواج, والليل والنهار:
\quranayah*[13][3]{\footnotesize \surahname*[13]}.

التفكر في النحل:
\quranayah*[16][69]{\footnotesize \surahname*[16]}.

التفكر في الحياة الدنيا:
\quranayah*[10][24]{\footnotesize \surahname*[10]}.

التفكر في الزرع والزيتون والنخيل والأعناب، وكل الثمرات:
\quranayah*[16][11]{\footnotesize \surahname*[16]}.

التفكر في الزواج:
\quranayah*[30][21]{\footnotesize \surahname*[30]}.

التفكر في الموت:
\quranayah*[39][42]{\footnotesize \surahname*[39]}.

النهي عن أغلاق القلوب عن التفكر في كتاب الله:
\quranayah*[22][46]{\footnotesize \surahname*[22]}.

}

وكل هذا فيه الحجة البالغة العقلية والشرعية على استحقاق الله جل جلاله للعبادة وحده لا شريك له بالطريقة التي ارتضاها ومن ذلك وجوب تسبيحه وتقديسه بأسمائه وصفاته بدون تحريف ولا تعطيل ولا تكييف ولاتمثيل إذ قال جل جلاله: 
\quranayah*[87][1-5]{\footnotesize \surahname*[87]}.
وجاء في تفسير السعدي في بيان معنى هذه الآيات أنه تعالى يأمر بتسبيحه المتضمن لذكره وعبادته، والخضوع لجلاله، والاستكانة لعظمته، وأن يكون تسبيحا، يليق بعظمة الله تعالى، بأن تذكر أسماؤه الحسنى العالية على كل اسم بمعناها الحسن العظيم. { الذي خلق فسوى } أي: أتقنها وأحسن خلقها { وَالَّذِي قَدَّرَ } تقديرًا، تتبعه جميع المقدرات { فَهَدَى } إلى ذلك جميع المخلوقات. وهذه الهداية العامة، التي مضمونها أنه هدى كل مخلوق لمصلحته، وتذكر فيها نعمه الدنيوية، ولهذا قال فيها: { وَالَّذِي أَخْرَجَ الْمَرْعَى } أي: أنزل من السماء ماء فأنبت به أنواع النبات والعشب الكثير، فرتع فيها الناس والبهائم وكل حيوان ، ثم بعد أن استكمل ما قدر له من الشباب، ألوى نباته، وصوح عشبه. { فَجَعَلَهُ غُثَاءً أَحْوَى } أي: أسود أي: جعله هشيمًا رميمًا، ويذكر فيها نعمه الدينية [هـ]. فتبارك الله أحسن الخالقين.

فلولا ثبات الكون وإستقراره وبركته لما صلح للحياة ولما كانت الحياة ممكنة ومستقرة ومنتظمة ومنتجة. ولهذا فإن الإنسان يعيش في هذا الكون ويستفيد منه ومن ثماره ونعمه وبركاته وموارده ومعادنه وغيرها من الأشياء التي جعلها الله في هذا الكون ليستفيد منها بفضله ورحمته ومنه علينا.
وفي ثبات الكون وإستقراره غايات عظيمة ومصالح كثيرة ومن أهمها تعلم العدد والحساب كما في قوله تعالى: 
\quranayah*[10][5]{\footnotesize \surahname*[10]}. 

فكل هذه الآيات واضحة في دلالتها على عظمة الخالق وحكمته وعدله ورحمته. ولهذا فقد ذكر الله عزل وجل أن آياته لقوم يعقلون، يعلمون، يفقهون، يؤمنون، يوقنون، أو يتفكرون، وهم أولي الألبات الصادقين حقا مع أنفسهم ومع خالقهم بما أودعه فيهم من هداية وبصيرة بفضله ومنه عليهم. وهم الذين آمنوا بالله حقا على يقين ولم يرتابوا وجاهدوا في الله لنصرة الحق كما في قوله تعالى: 
\quranayah*[49][15]{\footnotesize \surahname*[49]}.
ولهذا ما ينكر هذه الآيات الواضحة إلا المعاندين لها والكافرين بها والمشككين فيها والمعرضين عنها وعن خالقهم كفرا وعدوانا وظلما. ولهذا فقد سماهم الله جل جلاله العمي وحجب عنهم الهداية الكونية ونفى عنهم اليقين بعدله سبحانه فقال لنبيه: 
\quranayah*[27][81-82]{\footnotesize \surahname*[27]}. وهم الذين يجادلون في آيات الله بالباطل فطبع الله على قلوبهم كما في قوله تعالى:
\quranayah*[40][35]{\footnotesize \surahname*[40]}.
وقوله تعالى: 
\quranayah*[40][56]{\footnotesize \surahname*[40]}.
فهم تكبروا عن قبول الحق لكفرهم كما قال تعالى: 
\quranayah*[40][4]{\footnotesize \surahname*[40]}.


\section{الظلم ينافي الميزان الكوني والميزان الشرعي}

ومن عدله وحكمته سبحانه أنه جعل الظلم منافيا ومخالفا للميزان الشرعي كما جعله سبحانه منافيا للميزان الكوني. فهذا فيه أن الكون محفوظ بإمر الله الكوني وبعدله ولكن هذا الحفظ والإستقرار انما جعله الله برهانا واضحا على ربوبيته وألوهيته حتى يقيم المكلفين الحق والميزان الشرعي، ولهذا فإن الله جل جلاله جعل الظلم من أسباب البلاء الذي يقع بإذنه إما لحكمته أو عدله أو رحمته. ويقع هذا البلاء في صور مختلفة منها الجوع والخوف وقلة المطر والزلازل وذهاب البركة وغير ذلك. ومن أعظم الظلم الكفر بالله كالشرك كما في قوله تعالى: 
\quranayah*[31][13]{\footnotesize \surahname*[31]}. أو دعوة الولد له سبحانه والدليل قوله تعالى:
\quranayah*[19][88-91]{\footnotesize \surahname*[19]}. يقول السعدي رحمه الله: { أَنْ دَعَوْا لِلرَّحْمَنِ } أي: من أجل هذه الدعوى القبيحة تكاد هذه المخلوقات، أن يكون منها ما ذكر. والحال أنه: { مَا يَنْبَغِي } أي: لا يليق ولا يكون { لِلرَّحْمَنِ أَنْ يَتَّخِذَ وَلَدًا } وذلك لأن اتخاذه الولد، يدل على نقصه واحتياجه، وهو الغني الحميد. والولد أيضا، من جنس والده، والله تعالى لا شبيه له ولا مثل ولا سمي [هـ]. 

وهذا فيه أن الشرك ونسبة الولد لله سبحانه هو من الظلم الذي لا ينافي فقط الميزان الشرعي الذي أمر الله به، وأنما ينافي أيضا الميزان الكوني الذي به يكاد يحصل الإضراب الذي به يكون خراب هذا الكون. ولهذا فإن دعوة الولد أو الصاحبة لله جل جلاله من شتم الله والإشراك به سبحانه ولهذا فقد نزه سبحانه نفسه عن ذلك كله في قوله:
\quranayah*[6][100-103]{\footnotesize \surahname*[19]}. وقد صح عن عبد الله بن عباس وعن أبي هريرة عن النبي ﷺ أن الله قالَ: كَذَّبَنِي ابنُ آدَمَ، ولَمْ يَكُنْ له ذلكَ، وشَتَمَنِي، ولَمْ يَكُنْ له ذلكَ؛ فأمَّا تَكْذِيبُهُ إيَّايَ فَزَعَمَ أنِّي لا أقْدِرُ أنْ أُعِيدَهُ كما كانَ، وأَمَّا شَتْمُهُ إيَّايَ فَقَوْلُهُ: لي ولَدٌ، فَسُبْحانِي أنْ أتَّخِذَ صاحِبَةً أوْ ولَدًا!. وفي رواية اخرى: أمَّا تَكْذِيبُهُ إيَّايَ أنْ يَقُولَ: إنِّي لَنْ أُعِيدَهُ كما بَدَأْتُهُ، وأَمَّا شَتْمُهُ إيَّايَ أنْ يَقُولَ: اتَّخَذَ اللَّهُ ولَدًا، وأنا الصَّمَدُ الذي لَمْ ألِدْ ولَمْ أُولَدْ، ولَمْ يَكُنْ لي كُفُؤًا أحَدٌ {\footnotesize (صحيح البخاري)}.

ومما ينافي عدل الله الكوني والشرعي أيضا اتباع الهوى بدلا من إقامة الحق ونصرته كما في قوله تعالى:
\quranayah*[23][71]{\footnotesize \surahname*[23]}. يقول السعدي في تفسيره:
ووجه ذلك أن أهواءهم متعلقة بالظلم والكفر والفساد من الأخلاق والأعمال، فلو اتبع الحق أهواءهم لفسدت السماوات والأرض، لفساد التصرف والتدبير المبني على الظلم وعدم العدل، فالسماوات والأرض ما استقامتا إلا بالحق والعدل[هـ].

ومن ذلك أيضا نقصان البركة بسبب المعاصي كما في قوله تعالى:
\quranayah*[30][41]{\footnotesize \surahname*[30]}. فقد ورد في تفسير القرطبي أن ابن عباس قال: هو نقصان البركة بأعمال العباد كي يتوبوا [هـ]. وجاء في تفسير ابن كثير أن زيد بن رفيع قال: ( ظهر الفساد ) يعني انقطاع المطر عن البر يعقبه القحط، وعن البحر تعمى دوابه [هـ]. 

ومن ذلك أيضا الكفر بأنعم الله كما في قوله تعالى:
\quranayah*[16][112]{\footnotesize \surahname*[16]}. وسيأتي توضيح أسباب هذا العذاب في بيان حال الأمم مع الحق والميزان.

\section{المراد بالعلم والميزان}

إن المراد بالعلم عموما هو المعرفة وله أقسام وأنواع ويمكن تقسميه إلى قسمين وهما: العلم الكوني والعلم الشرعي. فالعلم الكوني ينقسم إلى علم ظاهر وهو العلم السببي وعلم غير ظاهر وهو العلم الغيبي. وأما العلم الشرعي فهو ما قام عليه الدليل كما بين ذلك شيخ الإسلام ابن تيمية رحمه الله، وينقسم إلى علم فطري وعلم ديني.

وأما الميزان فالمراد به العدل وهو العمل بعد المعرفة وأقسام الميزان كأقسام العلم، ويمكن تقسيمه إلى ميزان كوني وميزان شرعي. فالميزان الكوني ينقسم إلى الميزان السببي والميزان الغيبي. وأما الميزان الشرعي فهو ينقسم إلى الميزان الفطري والميزان الديني. والميزان الشرعي تابع للعلم الشرعي وهو ما قام عليه الدليل. والميزان الكوني وضعه الله جل جلاله في يده على صورة مخلوق لتدبير الكون بعدله، والميزان الشرعي يضعه جل جلاله بعد الصراط في صورة مخلوق لحساب المكلفين من الجن والإنس بعدله. وهذا لحكمته سبحانه ومن ذلك ليكون عدله ظاهرا في الميزان الكوني والميزان الشرعي.

وعلى ما تقدم، فإن كان المراد المعرفة قيل العلم وإن كان المراد العمل بعد المعرفة قيل الميزان. وفيما يأتي بيان أقسام العلم والميزان واقتصرت التسمية على الميزان فقط لتشمل العمل والمعرفة معا. ولكن نفس التقسيم والشرح ينطبق على العلم أيضا.

\section{أقسام الميزان الكوني}

\subsection{الميزان السببي}

فالقسم الأول هو الميزان السببي أو العلم السببي وهو علم ظاهر يدرك بالعقل والفطرة وما منَّ الله به على خلقه من حواس كالبصر والسمع والإحساس. فهذا الميزان فيه حقيقة الأشياء ومسمياتها وطريق الوصول إليها ومسبباتها. والمخلوقات تتفاوت في المعرفة بهذا الميزان كل بحسب حاله ومقامه ولكن الله جل جلاله أختص الإنسان وفضله على سائر الخلق بأن جعل له عقلا يدرك به من الأسباب ما لا يمكن لغيره من المخلوقات. وهذا لأن الله جل جلاله أراد بحكمته أن يجعله خليفة في الأرض فخلقه على صورته وجعل له من العقل والذكاء ما لم يعطي غيره. وهذا ما فضل به آدم على الملائكة وهو تفضيل في المعرفة السببية فأمرهم بالسجود له سجود التحية والإحترام وليظهر فضله على سائر الخلق وهذا لحكمته سبحانه وعلمه كما في قوله تعالى:
\quranayah*[2][30-34]{\footnotesize \surahname*[2]}.

فأما الملائكة فاعترفوا بهذا الفضل وبأنهم لا علم لهم إلا علمه الله لهم ولكن هذا النقص في العلم السببي لم يمنعهم من الطاعة والإنقياد لأمر الله جل جلاله. وأما إبليس فقد حسد آدم في الصورة التي خلق بها وعلى ما منَّ الله به عليه من العلم والقدرة المعرفية التي استحق بها هذا الثناء والتقدير. ولهذا فما كان للشيطان إلا أن يقول مستكبرا أنه خيرا منه خلق من نار وآدم من طين حسدا منه وكفرا. وهذا فيه جهل ابليس حيث أنه نسب الفضل لمجرد نوع مادة الخلق والقوة الطبيعية لا للعلم والقدرة المعرفية التي منَّ الله بها على آدم عليه السلام. فما كان له إلا أن يسعى لإضلاله وإخراجه من الجنة حسدا منه على هذه الفضائل ولو كان ذلك على حساب هلاكه وسوء مئاله. وهذا أيضا فيه نقص العقل وسوء الفكر نسأل الله السلامة والعافية. ولم يكتفي بذلك بل أخذ العهد على نفسه لإغواء وإضلال كل ذرية بني آدم كما حذرنا سبحانه وتعالى في كتابه الكريم.

وبالمعرفة بهذا الميزان السببي لا زالت تتقدم الحضارات الإنسانية وتتطور في الأخذ بالأسباب لإنجاز ما لم يكن ممكا لما سبق من الأمم من التكنولوجيا وشتى العلوم كالفيزياء والكيمياء والطب وغيرها من العلوم الأخرى التي بها يمكن تحصيل المصالح الدينية والدنيوية. ومفتاح كل هذه العلوم هو علم الحساب حيث به تعرف مقادير الأشياء وتقديرها ولهذا فقد أمر الله تعالى بتعلمه من الآيات الكونية كما تقدم. ولهذا فإن الميزان السببي يحتاج إلى بحث وتمعن في آيات الله الكونية، وما يخفى منه على الإنسان أكثر مما يعلم لهذا قال جل جلاله في ذلك عندما سئل الرسول ﷺ عن حقيفة الروح: 
\quranayah*[17][85]{\footnotesize \surahname*[17]}.

ولهذا فقد ذكر جل جلاله تقدم البشرية من الأقوام السابقة في هذا العلم السببي ولكن هذا التقدم كان سببا في زيادة الغفلة ظنا منهم أنه يغني عن الميزان أو العلم الشرعي ولهذا قال تعالى: 
\quranayah*[40][83]{\footnotesize \surahname*[40]}. وقال تعالى:
\quranayah*[30][7]{\footnotesize \surahname*[30]}. وجاء في تفسيير السعدي بيان ذلك أن هؤلاء الذين لا يعلمون أي: لا يعلمون بواطن الأشياء وعواقبها. وإنما { يَعْلَمُونَ ظَاهِرًا مِنَ الْحَيَاةِ الدُّنْيَا } فينظرون إلى الأسباب ويجزمون بوقوع الأمر الذي في رأيهم انعقدت أسباب وجوده ويتيقنون عدم الأمر الذي لم يشاهدوا له من الأسباب المقتضية لوجوده شيئا، فهم واقفون مع الأسباب غير ناظرين إلى مسببها المتصرف فيها. { وَهُمْ عَنِ الْآخِرَةِ هُمْ غَافِلُونَ } قد توجهت قلوبهم وأهواؤهم وإراداتهم إلى الدنيا وشهواتها وحطامها فعملت لها وسعت وأقبلت بها وأدبرت وغفلت عن الآخرة، فلا الجنة تشتاق إليها ولا النار تخافها وتخشاها ولا المقام بين يدي اللّه ولقائه يروعها ويزعجها وهذا علامة الشقاء وعنوان الغفلة عن الآخرة. ومن العجب أن هذا القسم من الناس قد بلغت بكثير منهم الفطنة والذكاء في ظاهر الدنيا إلى أمر يحير العقول ويدهش الألباب. وأظهروا من العجائب الذرية والكهربائية والمراكب البرية والبحرية والهوائية ما فاقوا به وبرزوا وأعجبوا بعقولهم ورأوا غيرهم عاجزا عما أقدرهم اللّه عليه، فنظروا إليهم بعين الاحتقار والازدراء وهم مع ذلك أبلد الناس في أمر دينهم وأشدهم غفلة عن آخرتهم وأقلهم معرفة بالعواقب، قد رآهم أهل البصائر النافذة في جهلهم يتخبطون وفي ضلالهم يعمهون وفي باطلهم يترددون نسوا اللّه فأنساهم أنفسهم أولئك هم الفاسقون. ثم نظروا إلى ما أعطاهم اللّه وأقدرهم عليه من الأفكار الدقيقة في الدنيا وظاهرها و[ما] حرموا من العقل العالي فعرفوا أن الأمر للّه والحكم له في عباده وإن هو إلا توفيقه وخذلانه فخافوا ربهم وسألوه أن يتم لهم ما وهبهم من نور العقول والإيمان حتى يصلوا إليه، ويحلوا بساحته [وهذه الأمور لو قارنها الإيمان وبنيت عليه لأثمرت الرُّقِيَّ العالي والحياة الطيبة، ولكنها لما بني كثير منها على الإلحاد لم تثمر إلا هبوط الأخلاق وأسباب الفناء والتدمير] [هـ]. وهذا والله فيه البيان الكافي في أن الطريق الواضح والسليم للرقي بالحضارة بما يرضي الله لا يكون إلا بالأخذ بالعلم الشرعي مع العلم السببي والتوجه إلى الله بالتوحيد والإخلاص والتقوى والعمل الصالح. وهذا ما يجب على الإنسان أن يعمل به ويعمل على تحقيقه ويتوكل على الله في ذلك كله والله المستعان.

ومن الأمثلة على تقدم الأمم السابقة في العلم السببي (ولو بالنسبة لقريش والعرب) والذي كان سببا لتكبرها وتجبرها على أمر الله الشرعي هم عاد قوم هود عليه السلام حيث قال تعالى فيهم:
\quranayah*[46][26]{\footnotesize \surahname*[46]}. وهذا فيه أن الله جل جلاله يسر لعاد أسباب التمكين في الدنيا على نحو لم يمكن به غيرهم من العرب وكفار قريش كما جاء في عدة تفاسير. إلا أن هذا التمكين لم يكن سببا لهدايتهم رغم ما كان لهم من سمع وأبصار وقلوب ولكنهم كفروا واستهزؤا بأمر الله فغضب الله عليهم وأنزل عليهم عذابه. وهذا الأمر قد تكرر مع العديد من الأقوام السابقة كما في قوله تعالى: \quranayah*[6][6]{\footnotesize \surahname*[6]}. فهذه هي سنة الله ودأبه في الأمم السابقين واللاحقين، فاعتبروا بمن قص الله عليكم نبأهم. فالحمد الله الذي بين لنا الحق البيان الكامل.

\comment{
    ومن الأمم السابقة التي تقدمت في العلم السببي وبالأخص علم الهندسة والبناء وكان سببا في تكبرها وكفرها هم الفراعنة قوم فرعون حيث قال تعالى: 
    \quranayah*[40][36-37]{\footnotesize \surahname*[40]}. فلا زلنا نرى بقايا معمار الأهرامات والتي صنفها المؤرخون أنها أحد عجائب الدنيا الباقية والتي لا زال الباحثون يسعون لكشف ألغازها إلى يومنا هذا. وهذا فيه أن الفراعنة كان لهم من العلم والقوة في نقل وصقل وتصميم وتنفيذ البناء المعقد والذي لم يستطع الإنسان الحديث صنع مثله إلا بعد فترة طويلة من الزمن. يذكر المؤرخون أن الهرم الأكبر في مصر كان أطول هيكل في العالم  بإرتفاع يصل إلى 146 متر لأكثر من 3800 عام حتى بناء كاتدرائية لينكولن في إنجلترا في عام 1311م.
}

وأما الجن فلم يكن لهم التقدم في الحضارة وهذا لنقص عقولهم في إدراك الميزان أو العلم السببي. ولهذا كان كل الأنبياء والرسل من البشر فهم أكمل عقلأ وأعلى فكرا وأعظم علما. وهذا لأن الجن لا يدركون من الأسباب ما يمكن للبشر إدراكه ومن ذلك ما بينه جل جلاله في قوله عن الجن: 
\quranayah*[34][14]{\footnotesize \surahname*[34]}. حيث أنهم خدموا سليمان عليه السلام وهو ميت ظنا منهم أنه حيا. فلم يدركوا بعقولهم الناقصة أنه إذا لم يتحرك لفترة طويلة من الزمن فقد يكون قد مات، وهذا ما يستطيع إدراكه العاقل من البشر بكل سهولة. وقد جاء في تفسير السعدي بيان ذلك أن الجن كانوا قد موهوا على الإنس, وأخبروهم أنهم يعلمون الغيب, ويطلعون على المكنونات، فأراد اللّه تعالى أن يُرِيَ العباد كذبهم في هذه الدعوى, فمكثوا يعملون على عملهم، وقضى اللّه الموت على سليمان عليه السلام, واتَّكأ على عصاه, وهي المنسأة، فصاروا إذا مروا به وهو متكئ عليها, ظنوه حيا, وهابوه. فغدوا على عملهم كذلك سنة كاملة على ما قيل, حتى سلطت دابة الأرض على عصاه, فلم تزل ترعاها, حتى باد وسقط فسقط سليمان عليه السلام وتفرقت الشياطين وتبينت الإنس أن الجن { لَوْ كَانُوا يَعْلَمُونَ الْغَيْبَ مَا لَبِثُوا فِي الْعَذَابِ الْمُهِينِ } وهو العمل الشاق عليهم، فلو علموا الغيب, لعلموا موت سليمان, الذي هم أحرص شيء عليه, ليسلموا مما هم فيه [هـ].

وكل هذا فيه البيان من الله جل جلاله للناس أن الجن ليس فقط لا يعلمون الغيب بل هم أنقص عقلا وفكرا وإن كانوا أكثر قوة بطبيعة مادة خلقهم وهي النار. إلا أن الإنسان قادر على إدراك الأسباب التي تمكنه من التفوق على الجن في القوة بالعلم السببي، ومن ذلك قصة عرش بالقيس حيث قال تعالى مخبرا عن سليمان عليه السلام:
\quranayah*[27][38-40]{\footnotesize \surahname*[27]}. وقد جاء في تفسير السعدي أن هذا الذي عنده علم من الكتاب هو رجل عالم صالح عند سليمان يقال له: "آصف بن برخيا" كان يعرف اسم الله الأعظم الذي إذا دعا الله به أجاب وإذا سأل به أعطى، بأن يدعو الله بذلك الاسم فيحضر حالا وأنه دعا الله فحضر. فالله أعلم هل هذا المراد، أم أن عنده علما من الكتاب يقتدر به على جلب البعيد وتحصيل الشديد [هـ].
والأقرب والله أعلى وأعلم أن آصف كان لديه هذا العلم السببي الذي به يقرب البعيد ولهذا فقد نسب جل جلاله فعله للعلم وهو العلم السببي ولم ينسب فعله لإيمانه أو دعائه كما هو الحال مع سائر الأنبياء مثل ينوس عليه السلام في قوله تعالى:
\quranayah*[21][88]{\footnotesize \surahname*[21]}. وهذا العلم الذي كان لدى آصف يسمى علم التنقل الآني للمحسوسات وإلى زماننا يبدو هذا العلم مستحيلا إدراكه إلا أنه يبقى علم سببي قد يدركه الإنسان إن علم أسبابه الموصلة إليه. ومن المعلوم أنه الإنسان في زماننا قد تمكن من تحقيق التنقل الآني لغير المحسوسات كالصوت والصور وكافة البيانات الرقمية وغيرها من الأشياء المستخدمة في وسائل الاتصال التي كانت تبدوا مستحيلة في الماضي القريب. فلك أن تتأمل في الأسباب الموصلة لهذا العلم العظيم الذي إن أدركه المسلمون لسبقوا كل الأمم والحضارات وكتب لهم التمكين إن أقاموا الحق والميزان مع الأخذ بهذا السبب العظيم.

ومن فضل الله ومنه على عباده أنه يفتح على من يشاء من هذا العلم السببي لعباده الصالحين القائمين بالميزان الشرعي فجعل لهم من أسباب التمكين ما يعجز عليه غيرهم وهذا لحكمته سبحانه، كما فتح على ذي القرنين  وعلى داوود وسليمان عليهما السلام وعلى آصف رحمه الله وعلى نبينا محمد ﷺ وعلى الصالحين من بعده من أصحابه رضي الله عنهم إلى زمان عمر بن عبدالعزيز رحمه الله ومن بعده هارون الرشيد رحمه الله وما سيأتي في آخر الزمان في عهد المهدي المنتظر وعيسى عليه السلام، وسيأتي بيان وتفصيل كل ذلك في فصل الحكم الرشيد. 

\subsection{الميزان الغيبي}

أما القسم الثاني فهو الميزان الغيبي أو العلم الغيبي وهو علم غير ظاهر لا يعلمه بالكلية ولا يملك مفاتحه إلا الله جل جلاله كما في قوله تعالى: 
\quranayah*[6][59]{\footnotesize \surahname*[6]}. فهو سبحانه  عالم الغيب والشهادة كما في قوله تعالى: 
\quranayah*[59][22]{\footnotesize \surahname*[59]}. ومنه ما أذن الله بمعرفته ومنه ما لم يأذن سبحانه بمعرفته والدليل على هذا قوله تعالى: 
\quranayah*[72][26-27]{\footnotesize \surahname*[72]}. وقد صح عن النبي ﷺ أنه قال: اللهمّ إني أسألُك بكلِّ اسمٍ هو لك سميتَ به نفسَك أو أنزلتَه في كتابِك أو علمته أحدًا من خلقِك أو استأثرت به في علمِ الغيبِ عندك ، أن تجعلَ القرآنَ العظيمَ ربيعَ قلبي ونورَ صدري وجلاءَ حزني وذهابَ همّي وغمّي {\footnotesize (صحيحه الألباني في السلسلة الصحيحة)}. وهذا فيه أن الله جل جلاله عنده علم لم يأذن بمعرفته لأحد من خلقه وهو العلم الذي استأثر به في علم الغيب عنده.

ومن الأمثلة على العلم الغيبي الذي لم يأذن الله بمعرفته كعلم موعد وقوع الساعة كما في قوله تعالى: 
\quranayah*[33][63]{\footnotesize \surahname*[33]}.
ولقد صح عن النبي وهو أشرف الناس أنه قال لجربيل هو أشرف الملائكة عن الساعة عندما سأله: فمتَى السَّاعةُ؟ قالَ ما المسئولُ عنها بأعلمَ منَ السَّائلِ {\footnotesize (صحيح البخاري)}. ولهذا فقد أمر جل جلاله نبيه ببيان هذا الأمر وهو العلم الغيبي الذي لم يأذن الله بمعرفته في قوله تعالى:
\quranayah*[6][50]{\footnotesize \surahname*[6]}. وقوله تعالى:
\quranayah*[7][188]{\footnotesize \surahname*[7]}.

وأما العلم الغيبي الذي أذن الله بمعرفته فمنه ما علمه الله الأنبياء والرسل في الكتب المنزلة من القصص الفائتة كما في قوله تعالى:
\quranayah*[11][49]{\footnotesize \surahname*[11]}. وغير ذلك من الأخبار الفائتة كفترة مكوث أهل الكهف وعددهم، وقصص الأنبياء عليهم السلام. ومن ذلك أيضا الأحداث القادمة كما في قوله تعالى:
\quranayah*[34][3]{\footnotesize \surahname*[34]}. وغير ذلك من الآيات والأحاديث التي تخبر بالأمور التي تحدث في المستقبل وفي آخر الزمان كنزول عيسى عليه السلام، وخروج الدابة، وإنهيار سد ذي القرنين، وخروج يأجوج ومأجوج. فكل ما سبق فيه الدليل الواضح على أن الله سبحانه وتعالى أذن بمعرفة بعض العلم الغيبي لمن شاء من خلقه ولم يأذن بمعرفة بعضه الآخر. والعلم الغيبي الذي لم يأذن الله بمعرفته أكثر مما علم الخلق  كما في قوله تعالى: 
\quranayah*[17][85]{\footnotesize \surahname*[17]}.

ومن المعلوم أن العلم الغيبي لا يمكن إدراكه بالكلية بالعلم السببي وإنما يدرك منه فقط ما أذن الله بمعرفته فجعل الله أسبابه وعلامته واضحة ومنتظمة لكل عاقل لما في ذلك من مصالح دينية ودنيوية. فمثلا يتنبأ بالمطر من الغيم الأسود, وبالإنجاب من علامات الحمل، وبنهاية الشهر بمنازل القمر، وبطلوع الزرع بعد نزول المطر، وبطريق السير من سير النجم وغير ذلك من الأسباب التي جعلها الله دلالات واضحة على ما ستئول إليه الأشياء والأحوال في المستقبل ولو على وجه التقريب. وهنا يأتي علم الحساب حيث به تحسب المقادير والأوزان والأحجام والأشكال والألوان والأصوات والحركات والأمكنة والأزمنة وغير ذلك من الأشياء التي تعرف بها الأسباب والقوانين التي جعلها سبحانه في هذا الكون. وتتفاوت هذه الأمور في إمكانية حسابها فمنها ما يستحيل حسابه كالساعة سواء الكبرى أو الصغيرى، ومنها ما يصعب حسابه كالمطر، ومنها ما يسهل ويعرف حسابه كأيام الشهر وساعات اليوم. 

فكل هذه الأسباب مرجعها للعلم السببي ويدرك بها فقط جزء من العلم الغيبي الذي أذن الله به ومثال ذلك أن ذي القرنين بخبرته بأسباب الحديد علم أن مئال سده إلى الإنهيار لما علمه من تآكل الحديد كما في قوله تعالى: 
\quranayah*[18][98]{\footnotesize \surahname*[18]}. ومن ذلك أيضا علم التسيير الذي به يتنبأ بالمكان والزمان لمعرفة الطريق بإستخدام مواضع النجوم كما في قوله تعالى: 
\quranayah*[6][97]{\footnotesize \surahname*[6]}. وقد فسر السعدي ذلك أن الله جعل النجوم هداية للخلق إلى السبل، التي يحتاجون إلى سلوكها لمصالحهم، وتجاراتهم، وأسفارهم. منها: نجوم لا تزال ترى، ولا تسير عن محلها، ومنها: ما هو مستمر السير، يعرف سيرَه أهل المعرفة بذلك، ويعرفون به الجهات والأوقات. ودلت هذه الآية ونحوها، على مشروعية تعلم سير الكواكب ومحالّها الذي يسمى علم التسيير، فإنه لا تتم الهداية ولا تمكن إلا بذلك [هـ].

ولهذا فإن التنبأ بالمستقبل بالحساب لا يكون دائما صحيحا 
أو دقيقا وبالأخص في الأمور التي يصعب حسابها، فأمر الله الواقع قد يحجب عن خلقه لحكمته ومثال ذلك قوم هود إذ ظنوا أن الغيم الأسود علامة للمطر كما هو معتاد ولكنه كان عذاب الله كما في قوله تعالى: 
\quranayah*[46][24-25]{\footnotesize \surahname*[46]}. لهذا فإن الوقائع المستقبلية يختلف حسابها بحسب ما أذن الله بعرفته ومعرفة أسبابه ومقاديره، فالتي لا تخضع لنمط معين ومعروف لا يمكن التنبأ بها إلا على وجه التقريب. وتزداد دقة الحساب مع الزيادة في معرفة هذه الأسباب والمقادير والتي تتأتى بالبحث والتجربة والقياس والملاحظة وكل ذلك ممكن بالرجوع لآيات الله الكونية التي منها يتعلم الحساب. 

ومن فضل الله ومنه أنه اختص وفتح من هذا العلم الغيبي على ما شاء من أنبياءه ورسله كل حسب حاله ومقامه. ومن ذلك ما فتح الله به على الخضر عليه السلام حيث علم من الأمور الغيبية ما لم يعلمه موسى عليه السلام. فكان الخضر عليه السلام أعلم من موسى في العلم الغيبي وكان موسى عليه السلام أعلم من الخضر في العلم الديني، وكل منهما كان نبيا ويأتيه الوحي من الله جل جلاله ولكن تفاوتوا في نوع العلم والفضل فموسى عليه السلام من أولى العزم من الرسل وكلم الله موسى تكليما، فكل بحسب مقامه وما فضله الله به كما في قوله تعالى:
\quranayah*[2][253][1-21]{\footnotesize \surahname*[2]}.

\comment{
وقد فسر النبي ﷺ قصة موسى مع الخضر عليهم السلام كما صح ذلك عن أبي بن كعب عندما سأل بن عباس رضي الله عنهم كما جاء ذلك في صحيح البخاري. وتبدأ هذه القصة أنه ذات يوم قَامَ مُوسَى خَطِيبًا في بَنِي إسْرَائِيلَ، فَسُئِلَ: أيُّ النَّاسِ أعْلَمُ؟ فَقالَ: أنَا، فَعَتَبَ اللَّهُ عليه؛ إذْ لَمْ يَرُدَّ العِلْمَ إلَيْهِ، فأوْحَى اللَّهُ إلَيْهِ: إنَّ لي عَبْدًا بمَجْمَعِ البَحْرَيْنِ هو أعْلَمُ مِنْكَ. فلما إلتقى موسى بالخضر 
فَسَلَّمَ عليه مُوسَى، فَقالَ الخَضِرُ: وأنَّى بأَرْضِكَ السَّلَامُ؟ قالَ: أنَا مُوسَى، قالَ: مُوسَى بَنِي إسْرَائِيلَ؟ قالَ: نَعَمْ، أتَيْتُكَ لِتُعَلِّمَنِي ممَّا عُلِّمْتَ رَشَدًا، قالَ: {إِنَّكَ لَنْ تَسْتَطِيعَ مَعِيَ صَبْرًا} [الكهف: 67]، يا مُوسَى، إنِّي علَى عِلْمٍ مِن عِلْمِ اللَّهِ عَلَّمَنِيهِ، لا تَعْلَمُهُ أنْتَ، وأَنْتَ علَى عِلْمٍ مِن عِلْمِ اللَّهِ عَلَّمَكَهُ اللَّهُ، لا أعْلَمُهُ. \footnote{وهذا فيه أن الخضر لديه من العلم الغيبي الذي لا يعلمه موسى وأن موسى لديه من العلم الديني الذي لا يعلمه الخضر.} 

فأصر موسى على اتباعه فَقالَ: {سَتَجِدُنِي إِنْ شَاءَ اللَّهُ صَابِرًا وَلَا أَعْصِي لَكَ أَمْرًا} [الكهف: 69]، فَقالَ له الخَضِرُ: {فَإِنِ اتَّبَعْتَنِي فَلَا تَسْأَلْنِي عَنْ شَيْءٍ حَتَّى أُحْدِثَ لَكَ مِنْهُ ذِكْرًا} [الكهف: 70]، فَانْطَلَقَا يَمْشِيَانِ علَى سَاحِلِ البَحْرِ، فَمَرَّتْ سَفِينَةٌ، فَكَلَّمُوهُمْ أنْ يَحْمِلُوهُمْ، فَعَرَفُوا الخَضِرَ، فَحَمَلُوهُمْ بغيرِ نَوْلٍ، فَلَمَّا رَكِبَا في السَّفِينَةِ لَمْ يَفْجَأْ إلَّا والخَضِرُ قدْ قَلَعَ لَوْحًا مِن ألْوَاحِ السَّفِينَةِ بالقَدُومِ، فَقالَ له مُوسَى: قَوْمٌ قدْ حَمَلُونَا بغيرِ نَوْلٍ عَمَدْتَ إلى سَفِينَتِهِمْ فَخَرَقْتَهَا {لِتُغْرِقَ أَهْلَهَا لَقَدْ جِئْتَ شَيْئًا إِمْرًا * قَالَ أَلَمْ أَقُلْ إِنَّكَ لَنْ تَسْتَطِيعَ مَعِيَ صَبْرًا * قَالَ لَا تُؤَاخِذْنِي بِمَا نَسِيتُ وَلَا تُرْهِقْنِي مِنْ أَمْرِي عُسْرًا} [الكهف: 71 - 73]، قالَ: وقالَ رَسولُ اللَّهِ صَلَّى اللهُ عليه وسلَّمَ: وكَانَتِ الأُولَى مِن مُوسَى نِسْيَانًا، قالَ: وجَاءَ عُصْفُورٌ، فَوَقَعَ علَى حَرْفِ السَّفِينَةِ، فَنَقَرَ في البَحْرِ نَقْرَةً، فَقالَ له الخَضِرُ: ما عِلْمِي وعِلْمُكَ مِن عِلْمِ اللَّهِ إلَّا مِثْلُ ما نَقَصَ هذا العُصْفُورُ مِن هذا البَحْرِ. \footnote{وهذا فيه أن الإنسان لم يؤتى من العلم إلا قليلا كما تقدم في معنى قوله تعالى:
\quranayah*[17][85]{\footnotesize \surahname*[17]}.}

ثُمَّ خَرَجَا مِنَ السَّفِينَةِ، فَبيْنَا هُما يَمْشِيَانِ علَى السَّاحِلِ إذْ أبْصَرَ الخَضِرُ غُلَامًا يَلْعَبُ مع الغِلْمَانِ، فأخَذَ الخَضِرُ رَأْسَهُ بيَدِهِ، فَاقْتَلَعَهُ بيَدِهِ فَقَتَلَهُ، فَقالَ له مُوسَى: (أَقَتَلْتَ نَفْسًا زَاكِيَةً) {بِغَيْرِ نَفْسٍ لَقَدْ جِئْتَ شَيْئًا نُكْرًا * قَالَ أَلَمْ أَقُلْ لَكَ إِنَّكَ لَنْ تَسْتَطِيعَ مَعِيَ صَبْرًا} [الكهف: 74 - 75]، قالَ: وهذِه أشَدُّ مِنَ الأُولَى، قالَ: {قَالَ إِنْ سَأَلْتُكَ عَنْ شَيْءٍ بَعْدَهَا فَلَا تُصَاحِبْنِي قَدْ بَلَغْتَ مِنْ لَدُنِّي عُذْرًا * فَانْطَلَقَا حَتَّى إِذَا أَتَيَا أَهْلَ قَرْيَةٍ اسْتَطْعَمَا أَهْلَهَا فَأَبَوْا أَنْ يُضَيِّفُوهُمَا فَوَجَدَا فِيهَا جِدَارًا يُرِيدُ أَنْ يَنْقَضَّ} [الكهف: 76، 77]، قالَ: مَائِلٌ، فَقَامَ الخَضِرُ فأقَامَهُ بيَدِهِ، فَقالَ مُوسَى: قَوْمٌ أتَيْنَاهُمْ فَلَمْ يُطْعِمُونَا ولَمْ يُضَيِّفُونَا، {لَوْ شِئْتَ لَاتَّخَذْتَ عَلَيْهِ أَجْرًا} [الكهف: 77]، قالَ: {قَالَ هَذَا فِرَاقُ بَيْنِي وَبَيْنِكَ} إلى قَوْلِهِ: {ذَلِكَ تَأْوِيلُ مَا لَمْ تَسْطِعْ عَلَيْهِ صَبْرًا} [الكهف: 78 - 82]، فَقالَ رَسولُ اللَّهِ صَلَّى اللهُ عليه وسلَّمَ: وَدِدْنَا أنَّ مُوسَى صَبَرَ حتَّى يَقُصَّ اللَّهُ عَلَيْنَا مِن خَبَرِهِما. 
}

وقد جاء في تفسيير بن كثير عن بن عباس أن النبي ﷺ قال أن موسى قال للخضر: جئتك لتعلمني مما علمت رشدا. قال: يكفيك التوراة بيدك، وأن الوحي يأتيك. يا موسى، إن لي علما لا ينبغي لك أن تعلمه، وإن لك علما لا ينبغي لي أن أعلمه.\footnote{وهذا فيه أن الخضر لديه من العلم الغيبي الذي لا يعلمه موسى وأن موسى لديه من العلم الديني الذي لا يعلمه الخضر.} وقال بن عباس رضي الله عنه عن الخضر عليه السلام أنه كان رجلا يعلم علم الغيب قد علم ذلك - فقال موسى: بلى. قال: ( وكيف تصبر على ما لم تحط به خبرا ) ؟ أي: إنما تعرف ظاهر ما ترى من العدل، ولم تحط من علم الغيب بما أعلم [هـ]. وجاء أيضا ما يأكد هذا المعنى في تفسير القرطبي: وعلمناه من لدنا علما أي علم الغيب، وقال ابن عطية: كان علم الخضر علم معرفة بواطن قد أوحيت إليه، لا تعطي ظواهر الأحكام أفعاله بحسبها; وكان علم موسى علم الأحكام والفتيا بظاهر أقوال الناس وأفعالهم [هـ].
وبينما الخضر وموسى عليهم السلام على السفينة، جَاءَ عُصْفُورٌ، فَوَقَعَ علَى حَرْفِ السَّفِينَةِ، فَنَقَرَ في البَحْرِ نَقْرَةً، فَقالَ له الخَضِرُ: ما عِلْمِي وعِلْمُكَ مِن عِلْمِ اللَّهِ إلَّا مِثْلُ ما نَقَصَ هذا العُصْفُورُ مِن هذا البَحْرِ {\footnotesize (صحيح البخاري)}.\footnote{وهذا فيه أن الإنسان لم يؤتى من العلم إلا قليلا كما تقدم في معنى قوله تعالى:
\quranayah*[17][85]{\footnotesize \surahname*[17]}.} 

وكل هذا فيه أن الخضر علم ما ستؤول إليه الأمور بما علَّمه الله له من علم الغيب فعلم بذلك أن السفينة لو لم تعاب لأخذها الملك غصبا من المساكين، وأن الغلام لو لم يقتل فسوف يرهق أبواه المؤمنين طغيانا وكفرا، وأن الجدار لو لم يقام لسقط ولسرق كنز الغلامين اليتيمين أبناء الرجل الصالح. وما تصرف الخضر في هذه الأمور إلا لعلمه أن جميع هذه الأمور سيقع في علم الغيب كما أوحى الله إليه ذلك. ولكن من رحمة الله ومنه فقد أذن للخضر أن يصلح ذلك ولهذا فقد قال: 
\quranayah*[18][82][25]{\footnotesize \surahname*[18]}. وبهذا يتبين أن الخضر إنما أوتي هذا العلم العظيم لمصلحة الناس وليس للإضرار بهم أي للإصلاح الدنيوي وهذا فيه بيان الرشد والذي به تجلب المصالح الدينية والدنيوية معا. وهذا ما أراد موسى تعلمه لهذا: 
\quranayah*[18][66]{\footnotesize \surahname*[18]}. وهذا فيه أن المصالح الدنيوية لا تجلب فقط بالعلم الديني الذي به يكون الإصلاح الديني بل إن ذلك يتطلب العلم الذي به يكون الإصلاح الدنيوي ولهذا فقد قال الخضر عليه السلام: 
 \quranayah*[18][67-68]{\footnotesize \surahname*[18]}. وهذا فيه أن الخضر عليه السلام اختص بالإصلاح الدنيوي مع ما كان لديه من الإصلاح الديني بينما موسى عليه السلام اختص بالإصلاح الديني فقط. وفيه أيضا أنه بالعلم والخبرة تجلب المصالح الدنيوية وبالعلم الديني تجلب المصالح الدينية وسيأتي تفصييل ذلك في فصل الحكم الرشيد بإذن الله. وفي هذه القصة بيان تدبير الله جل جلاله فإنه يعلم سبحانه ما كان وما سيكون وما لم يكن لو كان كيف يكون فسبحان الله الذي وسع كل شئ وأحاط به علما.

\section{أقسام الميزان الشرعي}

\subsection{الميزان الفطري}

فالميزان الفطري أو العلم الفطري فهو الذي يدرك بالفطرة السليمة الموافقة للعقل والتي فطر الله الناس عليها فيعرف به الخير من الشر والعدل من الظلم والإسلام من الكفر وغير ذلك من الأمور التي فطر الله الناس عليها والدليل على هذا أن النبيُّ صَلَّى اللهُ عليه وسلَّمَ قال: ما مِن مَوْلُودٍ إلَّا يُولَدُ علَى الفِطْرَةِ، فأبَوَاهُ يُهَوِّدَانِهِ أوْ يُنَصِّرَانِهِ، أوْ يُمَجِّسَانِهِ، كما تُنْتَجُ البَهِيمَةُ بَهِيمَةً جَمْعَاءَ، هلْ تُحِسُّونَ فِيهَا مِن جَدْعَاءَ، ثُمَّ يقولُ أبو هُرَيْرَةَ رَضِيَ اللَّهُ عنْه: {فِطْرَتَ اللَّهِ الَّتي فَطَرَ النَّاسَ عَلَيْهَا} [الروم: 30] الآيَةَ {\footnotesize (صحيح البخاري)}. ومن الأمور التي تخالف الميزان الفطري مثل الشرك (وبالأخص شرك الربوبية) والمجاهرة بالمعاصي والقتل واللواط والسرقة والغش في الكيل فهي أمور تدرك بالفطرة السليمة الموافقة للعقل ويمكن إثباتها لكل ذي عقل حتى بدون وحي. 
ومن ذلك أن أغلب الأمم مسلمة كانت أو كافرة اتفقت على فرض عقوبات على السرقة والغش على سبيل المثال لموافقة ذلك للفطرة السليمة. 

ولهذا فإن مخالفة الميزان الفطري هي أشد جرما من مخالفة الميزان الديني لأنها تعارض فطرة الله التي فطر الناس عليها والتي يمكن إدراكها حتى بدون وحي. ويعتبر الميزان الفطري أدنى مرتبة من الميزان الديني وفي معارضة هذا الميزان تعجيل سخط الله وعقوبته في الدنيا قبل الآخرة. ويعتبر الميزان الفطري ناقص حيث لا يمكن به إدراك العديد من الأمور الشرعية التي يحتاج إلى الوحي لإدراكها ومنها العقيدة والعبادات والأحكام الشرعية وغيرها من الأمور التي لا يمكن إدراكها بالفطرة السليمة فقط. ولهذا فقد أرسل الله جل جلاله الرسل وأنزل الكتب لبيان الميزان الديني والذي به يكتمل بيان الميزان الشرعي الذي أمر الله عباده به.

والميزان الفطري فيه الحجة لإدراك دين الإسلام لموافقته الفطرة كما سيأتي في بيان الميزان الديني. فقد جاء في الحديث القدسي عنِ اللهِ تعالى: إني خلقتُ عبادي حنفاءَ فاجتالتْهم الشياطينُ فحرَّمتْ عليهم ما أحللتُ لهم وأمرتهم أن يشركوا بي ما لم أُنزِّل به سلطانًا {\footnotesize (صحيح، مجموع الفتاوى لابن تيمية)}.
ولهذا فإن الشياطين لا تسعى لإفساد الميزان الديني فقط وأنما تسعى لإفساد الميزان الفطري والديني معا كما في قوله تعالى عن ابليس: 
\quranayah*[4][119]{\footnotesize \surahname*[4]}. وقد جاء بيان ذلك في تفسير السعدي أن الله تعالى خلق عباده حنفاء مفطورين على قبول الحق وإيثاره، فجاءتهم الشياطين فاجتالتهم عن هذا الخلق الجميل، وزينت لهم الشر والشرك والكفر والفسوق والعصيان. فإن كل مولود يولد على الفطرة ولكن أبواه يهوِّدانه أو ينصِّرانه أو يمجِّسانه، ونحو ذلك مما يغيرون به ما فطر الله عليه العباد من توحيده وحبه ومعرفته. فافترستهم الشياطين في هذا الموضع افتراس السبع والذئاب للغنم المنفردة. لولا لطف الله وكرمه بعباده المخلصين لجرى عليهم ما جرى على هؤلاء المفتونين، وهذا الذي جرى عليهم من توليهم عن ربهم وفاطرهم وتوليهم لعدوهم المريد لهم الشر من كل وجه، فخسروا الدنيا والآخرة، ورجعوا بالخيبة والصفقة الخاسرة [هـ].

وقد جاء في تفسير ابن كثير أن ابن عباس قال: 
أتى علي زمان وأنا أقول: أولاد المسلمين مع أولاد المسلمين، وأولاد المشركين مع المشركين. حتى حدثني فلان عن فلان: أن رسول الله ﷺ سئل عنهم فقال: "الله أعلم بما كانوا عاملين". فأمسكت عن قولي [هـ]. وهذا فيه أن ابن عباس امسك عن قوله بفصل أولاد المسلمين عن أولاد المشركين في اللعب عندما علم قول الرسول ﷺ أن أولاد المشركين أيضا على الفطرة السمحة التي فطر الله الناس عليها. وهذا ما يوافق باقي الأحاديث والآيات كما تقدم. 

وجاء أيضا في تفسير ابن كثير عن الفطرة أنه لا يولد أحد إلا على ذلك ، ولا تفاوت بين الناس في ذلك [هـ]. وبهذا يعلم أن المكلفين قد تساوا في الميزان الفطري عند نشأتهم وهذا من عدل الله إذ أعطاهم سبحانه الفطرة السليمة الموافقة للعقل حتى يدركوا بذلك الميزان الديني. ولكن هذه الفطرة قد تفسد فيضل صاحبها عند البلوغ فإن شاء الله هداه وإن شاء أزاغه وكل ذلك بهداية الله الكونية. ونقل هذا المعنى القرطبي في تفسيره عن شيخه أبو العباس قوله: قال شيخنا في عبارته: إن الله تعالى خلق قلوب بني آدم مؤهلة لقبول الحق، كما خلق أعينهم وأسماعهم قابلة للمرئيات والمسموعات، فما دامت باقية على ذلك القبول وعلى تلك الأهلية أدركت الحق ودين الإسلام وهو الدين الحق. وقد دل على صحة هذا المعنى قوله: كما تنتج البهيمة بهيمة جمعاء هل تحسون فيها من جدعاء يعني أن البهيمة تلد ولدها كامل الخلقة سليما من الآفات، فلو ترك على أصل تلك الخلقة لبقي كاملا بريئا من العيوب، لكن يتصرف فيه فيجدع أذنه ويوسم وجهه، فتطرأ عليه الآفات والنقائص فيخرج عن الأصل; وكذلك الإنسان، وهو تشبيه واقع ووجهه واضح [هـ]. فرحم الله علماء قرطبة من الأندلس الأسبانية الذين بينوا هذا المعنى العظيم.

\subsection{الميزان الديني}

وأما القسم الثاني من الميزان الشرعي فهو الميزان الديني أو العلم الديني وهو موافق للميزان الفطري ومكمل له. ويدرك بالوحي بالمنزل من عند الله تبارك وتعالى على الأنبياء والمرسلين عليهم الصلاة والسلام. ولهذا فقد أثبت سبحانه موافقة دينه للفطرة التي فطر الناس عليها في قوله تعالى: 
\quranayah*[30][30]{\footnotesize \surahname*[30]}. وهذه فيه أن الميزان الديني الذي أنزله الله كان ولا يزال موافقا للفطرة ومكملا لها وهو دين الإسلام الذي أرسلت به كل الرسل والأنبياء عليهم الصلاة والسلام من آدم عليه السلام إلى محمد ﷺ.

ومن الأمور التي تخالف الوحيي مثل الشرك (وبالأخص شرك الألوهية)، ومنع الزكاة، والحكم بغيير ما أنزل الله كتحريم ما أحل الله أو تحيل ما حرم الله وغير ذلك من الأمور التي تخالف أمر الله ورسوله والتي يمكن إدراكها بالوحي المنزل وبالحجة الواضحة والبينة إستنادا إلى جاء في كتاب الله عز وجل، أو صح في سنه نبيه الكريم، أو ثبت عن سبيل المؤمنين من السلف الصالحين. 

ولهذا فقد أمر الله عز وجل جميع الأنبياء لدعوة المشركين لعبادة الله وحده لا شريك له (أي إلى توحيد الألوهية) وإقامة الحجة عليهم بإيمانهم بأن الله هو من خلقهم وخلق السموات والأرض وهو مدبر الكون (أي بإيمانهم بتوحيد الربوبية) كما في قوله تعالى:
\quranayah*[39][38]{\footnotesize \surahname*[39]}.
وفي قوله تعالى: 
\quranayah*[43][87]{\footnotesize \surahname*[43]}. فوصف الله جل جلاله هؤلاء بأن أكثرهم لا يعقلون في قوله:
\quranayah*[29][63]{\footnotesize \surahname*[29]}.
ووصفهم جل جلاله أيضا بأنهم لا يعلمون في قوله:
\quranayah*[31][25]{\footnotesize \surahname*[31]}.
؛ أولا لا يعقلون لمخالفتهم الفطرة الموافقة للعقل وثانيا لا يعلمون لمخالفتهم العلم الشرعي وهو أمر الله المنزل من عنده. 

ولهذا فقد بين ذلك إبراهيم عليه السلام لأبيه كما في قوله تعالى:
\quranayah*[19][42-45]{\footnotesize \surahname*[19]}. فحاجه أولا بالميزان الفطري الذي يقام بالحجة العقلية في بطلان عبادة ما لا يسمع ولا يبصر ولا ينفع ولا يضر، وحاجه ثانيا بالميزان الديني الذي يقام بالعلم الصحيح المنزل من الله تبارك وتعالى وهو الوحي الموافق للفطرة والمكمل لها. وبين له الحكم الجزائي للشرك الموجب لعذاب الله فأقام عليه بذلك الحجة الكاملة والواضحة. 


\section{الغاية من إرسال الرسل وإنزال الكتب}

أرسل الله عز وجل رسله بالكتاب أولا لبيان الحق وهو العلم الصحيح ومن ثم لإقامة الميزان الشرعي بالقسط والعدل بين الناس حيث يقول جل جلاله:
\quranayah*[57][25]{\footnotesize \surahname*[57]}. والكتاب هو الحق كما في قوله تعالى:
\quranayah*[2][144][21]{\footnotesize \surahname*[2]}. والقسط في الكيل والوزن والعدل بين الناس من إقامة الميزان الشرعي وهو من الأمور التي أوصى الله تعالى بها. وقد دلنا سبحانه في هذه الآية على الحديد والأخذ به لنصرة الله جل جلاله ورسله ونصرة الحق الذي جاءوا به. وفي هذا دليل على أن نصرة الله ورسله تكون بثلاثة أمور وهي: (1) إقامة الحق بالعلم الصحيح، (2) إقامة الميزان الشرعي بالعدل والقسط، (3) الأخذ بأسباب القوى كالحديد وما يلزم ذلك من علوم كالحساب والطب وغيرها من العلوم السببية التي تكمن المسلمين من دحر الأعداء ونشر الحق ونصرته. وهذه الأمور الثلاثة هي أركان الحكم الرشيد التي بها يكون التمكين كما سيأتي بيان ذلك في فصل الحكم الرشيد.

فالله جل جلاله أنزل كتابه لتحقيق هذه الغاية العظيمة وهي إقامة الحق والميزان الشرعي كما بين ذلك في قوله تعالى:
\quranayah*[42][17]{\footnotesize \surahname*[42]}.
فذكر الله الميزان إلحاقا بالحق لان الحق لا يكون إلا بالعلم الصحيح وهو يقتضي الميزان الشرعي الذي لا يكون إلا بالعمل الصحيح الموافق للحق ومنه العدل والقسط كما أمر تعالى ومن ذلك بلا شك الحساب الصحيح كما سيأتي. فإقامة الميزان الشرعي من الوصايا العشر من سورة الأنعام في قوله تعالى:
\quranayah*[6][152][12]{\footnotesize \surahname*[6]}. فقرن الله عز وجل في هذه الآيات بين الكيل والميزان والعدل في القول والوفاء بالعهد. وفيه أن الميزان والكيل لا يكون إلا بالقسط وهو العدل. وفيه أن العدل ذكر مع ذا القربى ولذلك يكون العدل بين الناس.
وأيضا من الوصايا التي ذكرها الله في سورة الإسراء في قوله تعالى:
\quranayah*[17][35]{\footnotesize \surahname*[17]}. يقول السعدي في تفسيره:
وهذا أمر بالعدل وإيفاء المكاييل والموازين بالقسط من غير بخس ولا نقص. ويؤخذ من عموم المعنى النهي عن كل غش في ثمن أو مثمن أو معقود عليه والأمر بالنصح والصدق في المعاملة [هـ]. ومن ذلك بلا الشك الحساب ولذلك وجب الوفاء والصدق فيه من غير غش ولا تضليل وإقامة الميزان فيه بالقسط كما في الكيل.


\section{تفاوت الرسل في العلم والفضل ودعوتهم واحدة}

لا شك أن الأنبياء والرسل قد تفاوتوا في العلم والفضل والدليل قوله تعالى: 
\quranayah*[2][253]{\footnotesize \surahname*[2]}. وقوله تعالى:
\quranayah*[17][55]{\footnotesize \surahname*[17]}. ومن أمثلة ذلك ما تقدم في التفاوت في العلم والفضل بين موسى عليه السلام والخضر عليه السلام. 

ولكن جميع الرسل دعوتهم  واحدة وهي الدعوة إلى عبادة الله وحده لا شريك له كما في قوله تعالى: 
\quranayah*[16][36]{\footnotesize \surahname*[16]}. وقوله تعالى:
\quranayah*[21][25]{\footnotesize \surahname*[21]}. وهذه الدعوة إنما هي دين الإسلام الذي أرسل الله به جميع الرسل والأنبياء عليهم الصلاة والسلام. وهذا فيه أن الدعوة إلى الإسلام واحدة وهي دعوة إلى توحيد الألوهية والربوبية والأسماء والصفات والعبادات والأحكام الشرعية وغيرها من الأمور التي تدعو إليها الرسل والأنبياء عليهم الصلاة والسلام كما في قوله تعالى: 
\quranayah*[42][13]{\footnotesize \surahname*[42]}.

وبهذا يتبين أنه لا يمكن التفريق أو التفضيل بين دعوة الرسل فدعوتهم واحدة كما قال جل جلاله: \quranayah*[2][136]{\footnotesize \surahname*[2]}. وقال تعالى:
\quranayah*[2][285]{\footnotesize \surahname*[2]}.

ولهذا فقد نهى النبي ﷺ عن التفضيل بينه وبين الأنبياء على سبيل الإستنقاص أو على وجه التعصب أو التفريق أو التفاخر لما في ذلك بطر للحق ومن أعظم ذلك عدم إرجاع هذا الفضل لله تبارك وتعالى والذي بيده هذا التفضيل كما تقدم دليل ذلك، ومن ذلك حديث أبي هريرة رضي الله عنه أنه قال: بيْنَما يَهُودِيٌّ يَعْرِضُ سِلْعَةً له أُعْطِيَ بهَا شيئًا، كَرِهَهُ -أَوْ لَمْ يَرْضَهُ فقالَ: لَا، وَالَّذِي اصْطَفَى مُوسَى عليه السَّلَامُ علَى البَشَرِ، قالَ: فَسَمِعَهُ رَجُلٌ مِنَ الأنْصَارِ فَلَطَمَ وَجْهَهُ، قالَ: تَقُولُ: وَالَّذِي اصْطَفَى مُوسَى عليه السَّلَامُ علَى البَشَرِ وَرَسولُ اللهِ صَلَّى اللَّهُ عليه وَسَلَّمَ بيْنَ أَظْهُرِنَا؟! قالَ: فَذَهَبَ اليَهُودِيُّ إلى رَسولِ اللهِ صَلَّى اللَّهُ عليه وَسَلَّمَ، فَقالَ: يا أَبَا القَاسِمِ، إنَّ لي ذِمَّةً وَعَهْدًا، وَقالَ: فُلَانٌ لَطَمَ وَجْهِي، فَقالَ رَسولُ اللهِ صَلَّى اللَّهُ عليه وَسَلَّمَ: لِمَ لَطَمْتَ وَجْهَهُ؟ قالَ: قالَ يا رَسولَ اللهِ: وَالَّذِي اصْطَفَى مُوسَى عليه السَّلَامُ علَى البَشَرِ، وَأَنْتَ بيْنَ أَظْهُرِنَا! قالَ: فَغَضِبَ رَسولُ اللهِ صَلَّى اللَّهُ عليه وَسَلَّمَ حتَّى عُرِفَ الغَضَبُ في وَجْهِهِ، ثُمَّ قالَ: لا تُفَضِّلُوا بيْنَ أَنْبِيَاءِ اللهِ؛ فإنَّه يُنْفَخُ في الصُّورِ ، فَيَصْعَقُ مَن في السَّمَوَاتِ وَمَن في الأرْضِ إلَّا مَن شَاءَ اللَّهُ، قالَ: ثُمَّ يُنْفَخُ فيه أُخْرَى، فأكُونُ أَوَّلَ مَن بُعِثَ فَإِذَا مُوسَى عليه السَّلَامُ آخِذٌ بالعَرْشِ، فلا أَدْرِي أَحُوسِبَ بصَعْقَتِهِ يَومَ الطُّورِ، أَوْ بُعِثَ قَبْلِي، وَلَا أَقُولُ: إنَّ أَحَدًا أَفْضَلُ مِن يُونُسَ بنِ مَتَّى عليه السَّلَامُ {\footnotesize (صحيح مسلم، وصححه الألباني)}. وفي رواية اخرى: لا تُخَيِّروني على موسى؛ فإن الناسَ يُصْعَقون، فأكونُ أولَ مَن يَفِيقُ، فإذا موسى باطشٌ في جانبِ العرشِ، فلا أدري أكانَ ممَن صُعِقَ فأفاق قبلي، أو كان ممَن استَثْنَى اللهُ عزَّ وجلَّ {\footnotesize (صحيح أبي داود، وصححه الألباني)}. 

فهذا النهي جاء لأنه لم يكن لبيان فضل الله على أنبياءه ورسله بما فضلهم الله به وإنما كان على سبيل الإستنقاص أو التعصب أو التفريق أو التفاخر أو غير ذلك من الأمور التي تعارض ما جائوا به كأنما دعوتهم ليست بواحدة وهذا بخلاف التفضيل الذي فاضل الله به بين أنبياءه ورسله والذي فيه عدل الله تبارك وتعالى حيث اثبت سبحانه التفاوت بينهم في العلم والفضل ونفى التفريق بينهم في الدعوة. وبهذا يكون المعني الصحيح لقوله ﷺ (لا تفضلوا بين أنبياء الله) أي لا تفضلوا بين أنبياء الله تفضيلا يخالف التفضيل الذي فاضلهم الله به. وكذلك يقال بين نبينا ﷺ وبين موسى عليه السلام أو بين نبينا ﷺ وبين يونس عليه السلام. والله أعلى وأعلم.

صحيح البخاري أنس بن مالك
1. السماء الأولى السماء الدنيا
آدم، نهران النيل والفرات، قصر من لؤلؤ وزبرجد وهو الكوثر
2. السماء الثانية
ادريس
3. السماء الثالثة

4. السماء الرابعة
هارون
5. السماء الخامسة
؟
6. السماء السادسة
ابراهيم
7. السماء السابعة
موسى
8. فوق السماء السابعة
سدرة المنتهى

صحيح البخاري أبو ذر الغفاري
1. السماء الأولى السماء الدنيا
آدم
2-5 السماء الثانية
ادريس، وموسى، وعيسى ويحي، 
6. السماء السادسة
ابراهيم
7. السماء السابعة
سدرة المنتهى والجنة 
8. فوق السماء السابعة
العرش، صريف الأقلام, فرضت الصلاة

صحيح البخاري مالك بن صعصة الأنصاري
1. السماء الأولى السماء الدنيا
آدم
2. السماء الثانية
عيسى ويحي
3. السماء الثالثة
يوسف
4. السماء الرابعة
إدريس
5. السماء الخامسة
هارون
6. السماء السادسة
موسى
7. السماء السابعة
ابراهيم, البيت المعمور، سدرة المنتهى، 
8. فوق السماء السابعة
العرش

https://dorar.net/aqeeda/1463/%D8%A7%D9%84%D9%85%D8%B7%D9%84%D8%A8-%D8%A7%D9%84%D8%A3%D9%88%D9%84-%D8%AA%D8%B9%D9%8A%D9%8A%D9%86-%D8%A3%D9%88%D9%84%D9%8A-%D8%A7%D9%84%D8%B9%D8%B2%D9%85

\section{الإصلاح وأنواعه}
قد تقدم معنا تفاوت الأنبياء في العلم والفضل ولكن دعوتهم واحدة وهي دعوة إلى عبادة الله وحده لا شريك له. وهذه الدعوة تحتاج إلى إصلاح الناس وهذا الإصلاح يكون بالعلم الصحيح والعمل الصالح والدعوة إلى الحق والمعروف والنهي عن المنكر. وهذا الإصلاح يكون إما إصلاح ديني أو إصلاح دنيوي أو كلاهما ومثال ذلك قصة موسى عليه السلام مع الخضر عليه السلام. فالخضر عليه السلام اختصه الله بعلم الغيب للإصلاح الدنيوي وأما موسى عليه السلام فاختصه الله بالعلم الديني للإصلاح الديني. فكان للخضر الرشاد والذي فيه الإصلاح الديني والدنيوي معا وكان لموسى الحكمة التي فيها الإصلاح الديني فقط. بينما فضل الله جل جلاله موسى عليه السلام في الفضل والعلم لما كان معه من العلم والحكمة في أمر الله الشرعي. 

فالإنبياء جاءوا بالحق والميزان لحكمة الله عز وجل ومن ذلك الإصلاح الديني والدنيوي. ولكن أغلب الرسل اختصهم الله جل جلاله للإصلاح الديني لأن الإصلاح الديني فيه صلاح العباد والذي به تدرك مصالح الدنيا والآخرة ومن ذلك الصدق والأمانة وغيرها من الخصال التي توافق الفطرة والدين وتعود بالنفع الديني والدنيوي. إلا أن العديد من المنافع الدنيوية الآخرى لا تدرك بالعلم الديني فقط وإنما تدرك بالعلم السببي كعلم الطب والهندسة وغيرها من العلوم النافعة التي ينتفع بها الناس في أمور دنياهم وآخرتهم. وقد من الله سبحانه وتعالى على كل البشر فجعل لهم كل ما يحتاجونه من عقل وفطرة لإدراك هذا العلم السببيي كما تقدم بيانه. 

ومما يأكد هذا المعنى ما رواه العديد من الصحابة رضوان الله عليهم في قصة تلقيح النخل، ومنها أن أنس بن مالك قال: أنَّ النبيَّ صَلَّى اللَّهُ عليه وَسَلَّمَ مَرَّ بقَوْمٍ يُلَقِّحُونَ، فَقالَ: لو لَمْ تَفْعَلُوا لَصَلُحَ قالَ: فَخَرَجَ شِيصًا، فَمَرَّ بهِمْ فَقالَ: ما لِنَخْلِكُمْ؟ قالوا: قُلْتَ كَذَا وَكَذَا، قالَ: أَنْتُمْ أَعْلَمُ بأَمْرِ دُنْيَاكُمْ {\footnotesize (صحيح مسلم)}. 
وحديث طلحة بن عبيدالله حيث قال: مَرَرْتُ مع رَسولِ اللهِ ﷺ بقَوْمٍ علَى رُؤُوسِ النَّخْلِ، فَقالَ: ما يَصْنَعُ هَؤُلَاءِ؟ فَقالوا: يُلَقِّحُونَهُ؛ يَجْعَلُونَ الذَّكَرَ في الأُنْثَى فيَلْقَحُ، فَقالَ رَسولُ اللهِ صَلَّى اللَّهُ عليه وَسَلَّمَ: ما أَظُنُّ يُغْنِي ذلكَ شيئًا، قالَ: فَأُخْبِرُوا بذلكَ فَتَرَكُوهُ، فَأُخْبِرَ رَسولُ اللهِ صَلَّى اللَّهُ عليه وَسَلَّمَ بذلكَ، فَقالَ: إنْ كانَ يَنْفَعُهُمْ ذلكَ فَلْيَصْنَعُوهُ؛ فإنِّي إنَّما ظَنَنْتُ ظَنًّا، فلا تُؤَاخِذُونِي بالظَّنِّ، وَلَكِنْ إذَا حَدَّثْتُكُمْ عَنِ اللهِ شيئًا، فَخُذُوا به؛ فإنِّي لَنْ أَكْذِبَ علَى اللهِ عَزَّ وَجَلَّ {\footnotesize (صحيح مسلم)}. وحديث رافع بن خديج حيث قال: قَدِمَ نَبِيُّ اللهِ صَلَّى اللَّهُ عليه وَسَلَّمَ المَدِينَةَ وَهُمْ يَأْبُرُونَ النَّخْلَ، يقولونَ: يُلَقِّحُونَ النَّخْلَ ، فَقالَ: ما تَصْنَعُونَ؟ قالوا: كُنَّا نَصْنَعُهُ، قالَ: لَعَلَّكُمْ لو لَمْ تَفْعَلُوا كانَ خَيْرًا، فَتَرَكُوهُ، فَنَفَضَتْ -أَوْ فَنَقَصَتْ- قالَ: فَذَكَرُوا ذلكَ له، فَقالَ: إنَّما أَنَا بَشَرٌ، إذَا أَمَرْتُكُمْ بشَيءٍ مِن دِينِكُمْ، فَخُذُوا به، وإذَا أَمَرْتُكُمْ بشَيءٍ مِن رَأْيِي، فإنَّما أَنَا بَشَرٌ {\footnotesize (صحيح مسلم)}. 

وكل ما تقدم فيه أن نبينا ﷺ أقر لهم بأنهم أعلم بأمور دنياهم أي بالعلم السببي لهذا فقد قال: (إن كان ينفعهم ذلك فَلْيَصْنَعُوهُ)، وفرق عليه ﷺ بين أمر الله جل جلاله الذي به يكون الإصلاح الديني وبين أمور الدنيا التي بها يكون الإصلاح الدنيوي. فجعل كلامه ﷺ في أمور الدنيا ظن وقال: (لا تُؤَاخِذُونِي به) ولكن كلامه في أمر الله حق. وبين النبي ﷺ أنه بشر يخطئ ويصيب في أمور الدنيا كغيره من البشر ولهذا قال:  (إنَّما أَنَا بَشَرٌ، إذَا أَمَرْتُكُمْ بشَيءٍ مِن دِينِكُمْ، فَخُذُوا به، وإذَا أَمَرْتُكُمْ بشَيءٍ مِن رَأْيِي، فإنَّما أَنَا بَشَرٌ). وهذا من تواضعه وصدقه ﷺ فهو الصادق الأمين الذي لا يكذب على الله ولا على الناس عليه الصلاة والسلام. 

ولقد قال جل جلاله في كتابه: 
\quranayah*[18][110]{\footnotesize \surahname*[18]}. وقد جاء في تفسير الطبري أن معنى ذلك: أي قل لهؤلاء المشركين يا محمد: إنما أنا بشر مثلكم من بني آدم لا علم لي إلا ما علمني الله وإن الله يوحي إليّ أن معبودكم الذي يجب عليكم أن تعبدوه ولا تشركوا به شيئا، معبود واحد لا ثاني له، ولا شريك ( فَمَنْ كَانَ يَرْجُوا لِقَاءَ رَبِّهِ ) يقول: فمن يخاف ربه يوم لقائه، ويراقبه على معاصيه، ويرجو ثوابه على طاعته ( فَلْيَعْمَلْ عَمَلا صَالِحًا ) يقول: فليخلص له العبادة، وليفرد له الربوبية [هـ].


\section{الحكمة والرشاد}

ذكر جل جلاله الحكمة والرشاد في مواضع مختلفة في كتابه الكريم. فالحكمة فيها الإصلاح الديني بالعلم الشرعي الصحيح الموافق للحق وفيها العمل الموافق للميزان الشرعي. وأما الرشاد فهو أعم ومنه الرشاد الديني وهي الهداية ويشمل الحكمة مع العلم بالأسباب والأخذ بها، أي الإصلاح الديني والدنيوي معا، من إقامة الحق بالعلم الصحيح وإقامة الميزان بالعدل والعلم بالأسباب والأخذ بها. ولا يلزم أن الرشاد أفضل من الحكمة على وجه الإطلاق.

ولقد اختص جل جلاله من خلقه من يشاء بالحكمة فقال سبحانه وتعالى: 
\quranayah*[2][269]{\footnotesize \surahname*[2]}. وقد ذكر السعدي في تفسيره: 
لما أمر تعالى بهذه الأوامر العظيمة المشتملة على الأسرار والحكم وكان ذلك لا يحصل لكل أحد، بل لمن منَّ عليه وآتاه الله الحكمة، وهي العلم النافع والعمل الصالح ومعرفة أسرار الشرائع وحكمها، وإن من آتاه الله الحكمة فقد آتاه خيرا كثيرا وأي خير أعظم من خير فيه سعادة الدارين والنجاة من شقاوتهما! وفيه التخصيص بهذا الفضل وكونه من ورثة الأنبياء، فكمال العبد متوقف على الحكمة، إذ كماله بتكميل قوتيه العلمية والعملية فتكميل قوته العلمية بمعرفة الحق ومعرفة المقصود به، وتكميل قوته العملية بالعمل بالخير وترك الشر، وبذلك يتمكن من الإصابة بالقول والعمل وتنزيل الأمور منازلها في نفسه وفي غيره، وبدون ذلك لا يمكنه ذلك، ولما كان الله تعالى قد فطر عباده على عبادته ومحبة الخير والقصد للحق، فبعث الله الرسل مذكرين لهم بما ركز في فطرهم وعقولهم، ومفصلين لهم ما لم يعرفوه، انقسم الناس قسمين قسم أجابوا دعوتهم فتذكروا ما ينفعهم ففعلوه، وما يضرهم فتركوه، وهؤلاء هم أولو الألباب الكاملة، والعقول التامة، وقسم لم يستجيبوا لدعوتهم، بل أجابوا ما عرض لفطرهم من الفساد، وتركوا طاعة رب العباد، فهؤلاء ليسوا من أولي الألباب، فلهذا قال تعالى: ( وما يذكر إلا أولو الألباب ) [هـ].

ومن رحمة الله جل جلاله أنه جعل نبينا الكريم ﷺ معلما لأمته لهذه الحكمة فقال جل جلاله: 
\quranayah*[2][151]{\footnotesize \surahname*[2]}. وقد جاء في تفسير ابن كثير أنه تعالى يذكر عباده المؤمنين ما أنعم به عليهم من بعثة الرسول محمد صلى الله عليه وسلم إليهم ، يتلو عليهم آيات الله مبينات ويزكيهم ، أي : يطهرهم من رذائل الأخلاق ودنس النفوس وأفعال الجاهلية ، ويخرجهم من الظلمات إلى النور ، ويعلمهم الكتاب وهو القرآن والحكمة وهي السنة [هـ]. وقال السعدي أن معنى { وَيُعَلِّمُكُمُ الْكِتَابَ } أي: القرآن, ألفاظه ومعانيه، { وَالْحِكْمَةَ } قيل: هي السنة, وقيل: الحكمة, معرفة أسرار الشريعة والفقه فيها, وتنزيل الأمور منازلها. فيكون - على هذا - تعليم السنة داخلا في تعليم الكتاب, لأن السنة, تبين القرآن وتفسره, وتعبر عنه [هـ].

وأما الرشاد قد يكون إصلاح ديني فقط بمعنى الهداية لإتباع الحق كقوله تعالى: 
\quranayah*[40][38]{\footnotesize \surahname*[40]}، وهذا لأن قوم فرعون كان لهم العلم بالأسباب الذي به يكون الإصلاح الدنيوي. أو إصلاح دنيوي فقط كقوله تعالى:
\quranayah*[18][66]{\footnotesize \surahname*[18]}، وهذا لأن موسى عليه السلام كان لديه الإصلاح الديني وهو أعلم من الخضر عليه السلام في ذلك. أو كلاهما معا كقوله تعالى: \quranayah*[18][10]{\footnotesize \surahname*[10]}، وهذا فيه أولا صلاح الدين حيث أن الله جل جلاله هداهم وزادهم هدى، وثانيا  صلاح الدنيا حيث جعل سبحانه وتعالى لهم حفظ البدن مع طول الفترة المكوث في الكهف فرارا من قومهم. والرشاد لا يدركه الإنسان إلا بتوفيق الله ولهذا فقد أمر جل جلاله نبيه الكريم ﷺ فقال: \quranayah*[18][24][5]{\footnotesize \surahname*[18]}. وقال السعدي في تفسيره: فأمره أن يدعو الله ويرجوه، ويثق به أن يهديه لأقرب الطرق الموصلة إلى الرشد. وحري بعبد، تكون هذه حاله، ثم يبذل جهده، ويستفرغ وسعه في طلب الهدى والرشد، أن يوفق لذلك، وأن تأتيه المعونة من ربه، وأن يسدده في جميع أموره [هـ].

والرشاد إن جمع مع الهداية يكون بمعنى الإصلاح الدنيوي، وتكون الهداية بمعنى الحكمة والتي بها يكون الإصلاح الديني. ولهذا فإن نبينا الكريم ﷺ بدأ بالإصلاح الديني وسمى الخلفاء من بعده بالخلفاء الراشدين حتى يقوموا بالإصلاح الديني والدنيوي معا حيث قال ﷺ: عَليكم بِسنَّتي وسنَّةِ الخلفاءِ الرَّاشدينَ المَهْديِّينَ مِن بَعدي {\footnotesize (صحيح الألباني)}.

\section{مكانة أهل العلم الشرعي}

إن أهل العلم الشرعي هم القائمين بالإصلاح الديني وهم الذين يأمرون الناس بالقسط فيأمرون بالمعروف وينهون عن المنكر فهم أعلم الناس بالحق كما في قوله تعالى:
\quranayah*[34][6]{\footnotesize \surahname*[34]}. وهذا لأن الحق يهدي إلى الطريق المستقيم كما ذكر تعالى في هذه الآية وفي قوله تعالى:
\quranayah*[22][54]{\footnotesize \surahname*[22]}. فأهل العلم الشرعي يبينون للناس الحق الذي جاءت به الرسل والأنبياء ساعين إلى الهداية الشرعية للناس وراجين لهم الهداية الكونية من الله جل جلاله. وقد جعل الله جل جلاله أهل العلم حجة على الناس لما معهم من الحق فكانوا بذلك هم ورثة الأنبياء في الأرض فقد قال تعالى:
\quranayah*[17][107]{\footnotesize \surahname*[17]}.
وقوله تعالى:
\quranayah*[29][49]{\footnotesize \surahname*[29]}.


ولهذا فقد رفع الله مكانة أهل الإيمان وأهل العلم في الدنيا والأخرة لما عرفوا من الحق كما في قوله تعالى:
\quranayah*[58][11][19]{\footnotesize \surahname*[58]}.
ومن أعظم ذلك قوله تعالى:
\quranayah*[3][18]{\footnotesize \surahname*[3]}.
يقول السعدي في تفسيره:
هذا تقرير من الله تعالى للتوحيد بأعظم الطرق الموجبة له، وهي شهادته تعالى وشهادة خواص الخلق وهم الملائكة وأهل العلم [.] وأما شهادة أهل العلم فلأنهم هم المرجع في جميع الأمور الدينية خصوصا في أعظم الأمور وأجلها وأشرفها وهو التوحيد، فكلهم من أولهم إلى آخرهم قد اتفقوا على ذلك ودعوا إليه وبينوا للناس الطرق الموصلة إليه، فوجب على الخلق التزام هذا الأمر المشهود عليه والعمل به، وفي هذا دليل على أن أشرف الأمور علم التوحيد لأن الله شهد به بنفسه وأشهد عليه خواص خلقه، والشهادة لا تكون إلا عن علم ويقين، بمنزلة المشاهدة للبصر، ففيه دليل على أن من لم يصل في علم التوحيد إلى هذه الحالة فليس من أولي العلم. وفي هذه الآية دليل على شرف العلم من وجوه كثيرة، منها: أن الله خصهم بالشهادة على أعظم مشهود عليه دون الناس، ومنها: أن الله قرن شهادتهم بشهادته وشهادة ملائكته، وكفى بذلك فضلا، ومنها: أنه جعلهم أولي العلم، فأضافهم إلى العلم، إذ هم القائمون به المتصفون بصفته، ومنها: أنه تعالى جعلهم شهداء وحجة على الناس، وألزم الناس العمل بالأمر المشهود به، فيكونون هم السبب في ذلك، فيكون كل من عمل بذلك نالهم من أجره، وذلك فضل الله يؤتيه من يشاء، ومنها: أن إشهاده تعالى أهل العلم يتضمن ذلك تزكيتهم وتعديلهم وأنهم أمناء على ما استرعاهم عليه [هـ].


ومن أعظم المنكر قتل الأنبياء أو الذين يأمرون الناس بالقسط القائمين بالإصلاح الديني أو الإصلاح الدنيوي أو كلاهما. ولهذا فقد قال جل جلاله ذمه لأهل الكتاب لبيان هذا الجرم العظيم في قوله تعالى:
\quranayah*[3][21]{\footnotesize \surahname*[3]}. وقد جاء في تفسير السعدي أن هؤلاء الذين أخبر الله عنهم في هذه الآية، أشد الناس جرما وأي: جرم أعظم من الكفر بآيات الله التي تدل دلالة قاطعة على الحق الذي من كفر بها فهو في غاية الكفر والعناد ويقتلون أنبياء الله الذين حقهم أوجب الحقوق على العباد بعد حق الله، الذين أوجب الله طاعتهم والإيمان بهم، وتعزيرهم، وتوقيرهم، ونصرهم وهؤلاء قابلوهم بضد ذلك، ويقتلون أيضا الذين يأمرون الناس بالقسط الذي هو العدل، وهو الأمر بالمعروف والنهي عن المنكر الذي حقيقته إحسان إلى المأمور ونصح له، فقابلوهم شر مقابلة، فاستحقوا بهذه الجنايات المنكرات أشد العقوبات، وهو العذاب المؤلم البالغ في الشدة إلى غاية لا يمكن وصفها، ولا يقدر قدرها المؤلم للأبدان والقلوب والأرواح [هـ]. 

وقد صح عن النبي ﷺ أنه قال: إن الإسلام بدأ غريبًا، وسيعودُ غريبًا كما بدأَ، فطُوبَى للغُرباءِ قيل : من هم يا رسولَ اللهِ ؟ قال : الذينَ يصلحونَ إذا فسدَ الناسُ {\footnotesize (صحيحه الألباني في السلسلة الصحيحة)}. وفي زيادة: وليأرَزنَّ الإسلامُ إلى ما بين المسجدَيْن كما تأرَزُ الحيَّةُ إلى جُحرِها {\footnotesize (غريب أورده ابن حجر العسقلاني في موافقة الخبر الخبر)}. وفي رواية: الذين يُصْلِحُون ما أفسَدَ الناسُ مِن بعدِي مِن سُنتي {\footnotesize (قال الألباني ضعيف جدا)}.

عن أنسٍ رضي اللَّه عنه قالَ: قيلَ يا رسولَ اللَّهِ متى نتركُ الأمرَ بالمعروفِ والنَّهيَ عنِ المنكرِ قالَ إذا ظهرَ فيكم ما ظهرَ في الأممِ السابقة وفي روايةٍ في بني إسرائيلَ قالوا يا رسولَ اللَّهِ وما ظهرَ في الأممِ قبلنا قالَ المُلكُ في صغارِكم والفاحِشةُ في كبارِكم والعلمُ في رُذالتِكم {\footnotesize (حسنه السخاوي والوادعي وضعفه الألباني)}.


فالأمر بالمعروف واجب على كل مسلم إلى أن يشاء الله. 



\section{حال الأنبياء وأتباعهم مع الميزان الشرعي}

ولما كان الأنبياء أعلم الناس بأمر الله وأحرصهم, فقد أقاموا الميزان الشرعي حق إقامته في حكمهم الرشيد بين الخلق. ومثال ذلك يوسف عليه السلام في قوله تعالى:
\quranayah*[12][55]{\footnotesize \surahname*[12]}.
وهذا فيه حرصه عليه السلام على إقامة الكيل والوزن بما يرضي الله وهذا من الإصلاح الذي أمر الله به حيث قال لإخوته عن الكيل:
\quranayah*[12][59-60]{\footnotesize \surahname*[12]}.

فلا شك أن التفريط في الكيل والوزن من أعظم البلايا التي حذرنا الله عز وجل منها في كتابه الكريم فيقول تعالى:
\quranayah*[83][1-6]{\footnotesize \surahname*[83]}. فالظلم في الكيل والوزن من الإفساد العظيم ومن أسباب تعجيل العذاب في الدنيا قبل الأخرة, وفي قصة مدين مع نبيهم شعيبا العبرة الواضحة في ذلك. يقول تعالى على لسان نبيه شعيب محذرا قومه:
\quranayah*[7][85]{\footnotesize \surahname*[7]}.
وفي موضع أخر من سورة الشعراء:
\quranayah*[26][181-183]{\footnotesize \surahname*[26]}.
وفي سورة هود:
\quranayah*[11][84-85]{\footnotesize \surahname*[11]}. وهذا فيه أن شعيبا عليه السلام دعا قومه لإقامة الحق أولا وهو التوحيد بإفراد الله بالعبادة وثانيا لإقامة الميزان الشرعي وهو الكيل والوزن بالقسط. وفيه أن بخس الناس أشيائهم والخسران والنقصان في الكيل والوزن من الظلم والفساد الموجب لسخط الله وعذابه العاجل.

ونبينا ﷺ كان من أحرص الناس في إقامة الكيل والميزان وحذر من الفساد في ذلك في العديد من المواضع منها ما ورد عن ابن عباس رضي الله عنه أنه قَالَ: قَالَ رَسُولُ اللَّهِ صَلَّى اللَّهُ عَلَيْهِ وَسَلَّمَ لِأَصْحَابِ الْكَيْلِ وَالْمِيزَانِ: «إِنَّكُمْ قَدْ وُلِّيتُمْ أَمْرَيْنِ هَلَكَتْ فِيهِمَا الْأُمَمُ السَّابِقَة قبلكُمْ».
{\footnotesize رَوَاهُ التِّرْمِذِيّ}.
وعَنْ عَبْدِ اللَّهِ بْنِ عُمَرَ، قَالَ أَقْبَلَ عَلَيْنَا رَسُولُ اللَّهِ ـ ﷺ ـ فَقَالَ «يَا مَعْشَرَ الْمُهَاجِرِينَ خَمْسٌ إِذَا ابْتُلِيتُمْ بِهِنَّ وَأَعُوذُ بِاللَّهِ أَنْ تُدْرِكُوهُنَّ: لَمْ تَظْهَرِ الْفَاحِشَةُ فِي قَوْمٍ قَطُّ حَتَّى يُعْلِنُوا بِهَا إِلاَّ فَشَا فِيهِمُ الطَّاعُونُ وَالأَوْجَاعُ الَّتِي لَمْ تَكُنْ مَضَتْ فِي أَسْلاَفِهِمُ الَّذِينَ مَضَوْا, وَلَمْ يَنْقُصُوا الْمِكْيَالَ وَالْمِيزَانَ إِلاَّ أُخِذُوا بِالسِّنِينَ وَشِدَّةِ الْمَؤُنَةِ وَجَوْرِ السُّلْطَانِ عَلَيْهِمْ, وَلَمْ يَمْنَعُوا زَكَاةَ أَمْوَالِهِمْ إِلاَّ مُنِعُوا الْقَطْرَ مِنَ السَّمَاءِ وَلَوْلاَ الْبَهَائِمُ لَمْ يُمْطَرُوا, وَلَمْ يَنْقُضُوا عَهْدَ اللَّهِ وَعَهْدَ رَسُولِهِ إِلاَّ سَلَّطَ اللَّهُ عَلَيْهِمْ عَدُوًّا مِنْ غَيْرِهِمْ فَأَخَذُوا بَعْضَ مَا فِي أَيْدِيهِمْ, وَمَا لَمْ تَحْكُمْ أَئِمَّتُهُمْ بِكِتَابِ اللَّهِ وَيَتَخَيَّرُوا مِمَّا أَنْزَلَ اللَّهُ إِلاَّ جَعَلَ اللَّهُ بَأْسَهُمْ بَيْنَهُمْ».
{\footnotesize أخرجه ابن ماجه وصححه الألباني}.
فكل ما ذكره النبي ﷺ في هذه الحديث هي من المعاصي التي تخالف الميزان الشرعي. فالمجاهرة بالمعاصي وإنقاص الكيل لا يعارض فقط الميزان الشرعي بل أيضا الميزان الفطري والذي هو أساس الميزان الشرعي والذي يمكن إدراكه بالفطرة السليمة. وفي تقديم هذا النوع بيان المبالغة في المعصية. وأما منع الزكاة، ومخالفة أمر الله ورسوله، والحكم بغير ما أنزل الله فهي تخالف الميزان الشرعي الذي يدرك بالوحي. وكل هذه الأمور من أسباب البلاء العظيم ومنها الفقر والجوع وجور السلطان. وفيه الدليل على نبوته ﷺ فقد وقع ذلك كما أخبر بعد أن تهاون الكثير من المسلمين في أمر الميزان والمكيال إلا من رحم الله. وقد علم الصحابة والتابعين بأهمية إقامة الميزان والمكيال وأن الفساد فيهما من أسباب سخط الله ومنه ما ورد عَنْ يَحْيَى بْنِ سَعِيدٍ، أَنَّهُ سَمِعَ سَعِيدَ بْنَ الْمُسَيَّبِ، يَقُولُ إِذَا جِئْتَ أَرْضًا يُوفُونَ الْمِكْيَالَ وَالْمِيزَانَ فَأَطِلِ الْمُقَامَ بِهَا وَإِذَا جِئْتَ أَرْضًا يُنَقِّصُونَ الْمِكْيَالَ وَالْمِيزَانَ فَأَقْلِلِ الْمُقَامَ بِهَا.

ومن ذلك أيضا الربا وبيع العينة لما فيه من التلاعب بالميزان الذي أمر الله بإقامته فعَنِ ابْنِ عُمَرَ، قَالَ سَمِعْتُ رَسُولَ اللَّهِ ﷺ يَقُولُ "إِذَا تَبَايَعْتُمْ بِالْعِينَةِ وَأَخَذْتُمْ أَذْنَابَ الْبَقَرِ وَرَضِيتُمْ بِالزَّرْعِ وَتَرَكْتُمُ الْجِهَادَ سَلَّطَ اللَّهُ عَلَيْكُمْ ذُلاًّ لاَ يَنْزِعُهُ حَتَّى تَرْجِعُوا إِلَى دِينِكُمْ" {\footnotesize (صححه الألباني)}. يقول الشيخ العثيمين في بيان العينة:
أن يبيع شيئا بثمن مؤجل ثم يشتريه ممن باعه عليه بأقل منه نقدًا [.] وسُمي بذلك لأن المشتري لم يُرد السلعة وإنما أراد العين أي: النقد، النقد لينتفع به، ودليل ذلك: أنه اشتراها بثمن زائد مؤجل، ثم باعها على من اشتراها منه بنقد، فكأنه لم يقصد هذه السلعة وإنما قصد الثمن الدراهم، فلهذا سمي بيع عينة [.] والغالب أن هذا ملازم لهذا، يعني أن الذي ينهمك في طلب الدنيا ويتحيل على الحصول عليها حتى بما حرم الله، الغالب أنه يترك الجهاد، لأن قلبه انشغل بالدنيا عنه.
[هـ]. وهذا فيه أن التفريط في الميزان حبا في الدنيا من أسباب عقاب الله وتسلط الأعداء, فعَنْ ثَوْبَانَ، قَالَ قَالَ رَسُولُ اللَّهِ ﷺ "يُوشِكُ الأُمَمُ أَنْ تَدَاعَى عَلَيْكُمْ كَمَا تَدَاعَى الأَكَلَةُ إِلَى قَصْعَتِهَا ". فَقَالَ قَائِلٌ وَمِنْ قِلَّةٍ نَحْنُ يَوْمَئِذٍ قَالَ "بَلْ أَنْتُمْ يَوْمَئِذٍ كَثِيرٌ وَلَكِنَّكُمْ غُثَاءٌ كَغُثَاءِ السَّيْلِ وَلَيَنْزِعَنَّ اللَّهُ مِنْ صُدُورِ عَدُوِّكُمُ الْمَهَابَةَ مِنْكُمْ وَلَيَقْذِفَنَّ اللَّهُ فِي قُلُوبِكُمُ الْوَهَنَ". فَقَالَ قَائِلٌ يَا رَسُولَ اللَّهِ وَمَا الْوَهَنُ قَالَ "حُبُّ الدُّنْيَا وَكَرَاهِيَةُ الْمَوْتِ" {\footnotesize (صححه الألباني)}. وفيه إن التفريط في الميزان الشرعي الذي أمر الله به حبا في الدنيا هو من أسباب الذل والهوان في الدنيا قبل الأخرة.

\section{حال الأمم مع الميزان الشرعي}

الحساب الصحيح يبنى على الميزان. فإن وافق هذا الحساب أيات الله الشرعية أو الفطرة السليمة فهو من العدل والإصلاح الذي أمر الله به. وإن خالف أمر الله أو الفطرة التي فطر الله الناس عليها فهو من الظلم والفساد الذي لا يرضى الله به ومن أسباب تعجيل سخط الله في الدنيا قبل الأخرة. وهذا بالعموم على المسلمين وغيرهم. فقد قال تعالى:
\quranayah*[11][117]{\footnotesize \surahname*[11]}. وقد جاء في تفسير الطبري أن معنى ذلك أن الله جل جلاله لم يكن ليهلكهم بشركهم بالله. وذلك قوله " بظلم " يعني: بشرك ، (وأهلها مصلحون)، فيما بينهم لا يتظالمون، ولكنهم يتعاطَون الحقّ بينهم ، وإن كانوا مشركين، إنما يهلكهم إذا تظالموا [هـ]. والحق هنا هو العدل الموافق للفطرة. 
وجاء أيضا تفصيل ذلك في تفسير القرطبي أن معنى وأهلها مصلحون أي فيما بينهم في تعاطي الحقوق; أي لم يكن ليهلكهم بالكفر وحده حتى ينضاف إليه الفساد، كما أهلك قوم شعيب ببخس المكيال والميزان، وقوم لوط باللواط; ودل هذا على أن المعاصي أقرب إلى عذاب الاستئصال في الدنيا من الشرك، وإن كان عذاب الشرك في الآخرة أصعب [هـ]. وفي صحيح الترمذي من حديث أبي بكر الصديق - رضي الله عنه - قال: سمعت رسول الله - ﷺ - يقول: إن الناس إذا رأوا الظالم فلم يأخذوا على يديه أوشك أن يعمهم الله بعقاب من عنده.
قال ابن القيم رحمه الله:
«أصل كل خير في الدنيا والآخرة هو العلم والعدل،
وأصل كل شرٍ في الدنيا والآخرة الجهل والظلم،      
والعدل مرجعه إلى العلم لإنّ من لم يعلم لا يستطيع أن يعدل.[هـ] يحتاج تحقيق من المصدر

وقوله تعالى:
\quranayah*[10][13-14]{\footnotesize \surahname*[10]}.
وفيه أن الله جل جلاله ناظر على أعمالنا وأعمال الأمم ومجازيها بعدله سبحانه وتعالى. فحال الأمم يدور مع الحق والميزان في أربعة أحوال من الأقل تمكينا إلى الأكثر تمكينا:

\begin{compactitem}
    \item الدولة الكافرة الظالمة
    \item الدولة المسلمة الظالمة
    \item الدولة الكافرة العادلة
    \item الدولة المؤمنة العادلة
\end{compactitem}

يقول شيخ الإسلام ابن تمية رحمه الله:
ولهذا يروى ان الله ينصر الدولة العادلة وان كانت كافرة ولا ينصر الدولة الظالمة وان كانت مؤمنة
[هـ] {\footnotesize (مجموع الفتاوى 28/63)}.
والأصح أن يقال: "ولا ينصر الدولة الظالمة وإن كانت مسلمة" وهذا لان الظلم ينافي الغاية من الإيمان وكماله وهو بلا شك معصية لله ورسوله والدليل قوله تعالى:
\quranayah*[49][14]{\footnotesize \surahname*[49]}. فالدولة المؤمنة هي التي تطيع الله ورسوله ومن ذلك الحكم بالعدل، فإن خالفت أمر الله فقد خالفت الغاية من الإيمان وتكون أدنى مرتبة في الدين وهي مرتبة الإسلام.
وقد قال الشيخ الألباني رحمه الله تعالى في توضيح هذا المعنى:
ذلك لأنَّ الظلم هو سبب خراب البلاد وهلاك العباد، فإذا كانت الأمة أو الدولة كافرة ولكنها تحكم بالعدل فيما بينها، هذا العدل الذي يعرفه الناس بفِطَرهم، فإذا كانوا يحكمون بذلك فستقوم دولتهم وتستمرُّ مدَّة طويلة، والتاريخ يحفل بهذا
[هـ].
والحساب يكون سبيل إلى الحكم بالعدل إن كان صحيحا إن يوافق الفطرة السليمة ونجاة من سخط الله وعذابه العاجل في الدنيا. فإن وافق الحساب الحكم الشرعي مع الفطرة كان ذلك نجاة في الدنيا والأخرة وكان حكما راشدا. ولهذا كان الحساب لإقامة الحق والميزان من بنيان الحكم الرشيد. وعليه يكون علم الحساب من الدين بالضرورة وليس بخلاف ذلك إذ يتعذر إقامة الميزان حق إقامته من دون حساب صحيح.

فالدولة الكافرة عدلها في موافقة الفطرة والدولة المسلمة عدلها في موافقة الشرع بما عرفوا من الحق عن طريق الرسل التي جائتهم بالكتب والتي تبين حكم الله الشرعي كما في قوله تعالى:
\quranayah*[3][23]{\footnotesize \surahname*[3]}.

بالتتبع والإستقراء يتبين أن الله يقيم الأمم التي تحكم بالعدل الذي يوافق الفطرة وإن كانت كافرة. فإن في ذلك سلامة من عذاب الله في الدنيا كما تقدم. فإن كانت مؤمنة وتحكم بالعدل كانت حكما راشدا ووعدها الله بالتمكين في الدنيا والفوز في الأخرة, يقول تعالى:
\quranayah*[24][55]{\footnotesize \surahname*[24]}.
وإن كانت مسلمة ولا تحكم بالعدل فقد خالفت حكمة الله والغاية من إيمانها الذي يقتضي إقامة العدل والميزان ويصدق فيها قوله تعالى:
\quranayah*[49][14]{\footnotesize \surahname*[49]}. وهذا هو حال أغلب أمة الإسلام في يومنا هذا كما هو معروف. فتكون بذلك الأمة الكافرة التي تحكم بالعدل قائمة فوق الأمة المسلمة التي لا تحكم بالعدل. وأما الأمة المؤمنة التي تقيم الحق وأجله التوحيد وما يقتضيه ذلك من إقامة الميزان ومنه العدل بين الناس تكون هي فوقهم جميعا كما دلت على ذلك الآيات والأحاديث. وهذا فيه الحكمة البالغة من الله عز وجل ومنه أن الله لا يرضى لعباده الظلم ولا يزال ذلك حال الأمة المسلمة حتى تقيم الميزان والكيل والعدل الذي أمر الله به. قال تعالى:
\quranayah*[13][11][12]{\footnotesize \surahname*[13]}. وهذا يشمل الراعي والرعية.

وبهذا يعلم أن الأمم إنما تقام بإقامة الميزان ومنه العدل بين الناس فإن تحقق ذلك سلمت سخط الله وعذابه في الدنيا وإن كانت كافرة. فإن لم تقم الميزان والعدل بين الناس فتكون بذلك قد جنت على نفسها عقاب الله العاجل في الدنيا من فقر وجوع وذل وجور السلطان  وإن كانت مسلمة. وأما إن كانت مؤمنة وأقامت الحق مع إقامة الميزان كما أمر الله كانت حكما راشدا وتحقق لها التمكين في الدنيا والفوز في الأخرة. وأما إقامة التوحيد دون إقامة الميزان وما يقتضيه من العدل بين الناس فهذا ينافي حكمة الله وأمره الذي بينه في كتابه وعلى لسان نبيه ﷺ. ولهذا وجب على المسلمين ودعاتهم الرجوع إلى أمر الله وعدم التهاون في ذلك ومنه العناية بإقامة الحق ومنه التوحيد وإقامة الميزان ومنه العدل بين الناس على حد السواء حتى يكون لهم التمكين الذي أمر الله به. ولذلك وجب علينا العناية بالحساب الصحيح بحثا وتطبيقا سعيا لتحقيق هذه الغاية العظيمة التي أمرنا الله بها وهي إقامة الحق والميزان الذي يبنى عليه الحكم الرشيد.

فقد روى الإمام أحمد في "المسند" (30 / 355) عَنِ حُذَيْفَةُ، قال: قَالَ رَسُولُ اللهِ صَلَّى اللهُ عَلَيْهِ وَسَلَّمَ: (تَكُونُ النُّبُوَّةُ فِيكُمْ مَا شَاءَ اللهُ أَنْ تَكُونَ، ثُمَّ يَرْفَعُهَا إِذَا شَاءَ أَنْ يَرْفَعَهَا، ثُمَّ تَكُونُ خِلَافَةٌ عَلَى مِنْهَاجِ النُّبُوَّةِ، فَتَكُونُ مَا شَاءَ اللهُ أَنْ تَكُونَ، ثُمَّ يَرْفَعُهَا إِذَا شَاءَ اللهُ أَنْ يَرْفَعَهَا، ثُمَّ تَكُونُ مُلْكًا عَاضًّا، فَيَكُونُ مَا شَاءَ اللهُ أَنْ يَكُونَ، ثُمَّ يَرْفَعُهَا إِذَا شَاءَ أَنْ يَرْفَعَهَا، ثُمَّ تَكُونُ مُلْكًا جَبْرِيَّةً، فَتَكُونُ مَا شَاءَ اللهُ أَنْ تَكُونَ، ثُمَّ يَرْفَعُهَا إِذَا شَاءَ أَنْ يَرْفَعَهَا، ثُمَّ تَكُونُ خِلَافَةٌ عَلَى مِنْهَاجِ نُبُوَّةٍ).


\section{التجارة مع الله ليوم الحساب}

\subsection{البعث ليوم الحساب}


\subsection{الحساب يبدأ عند الميزان وقبل الجزاء}

نقل القرطبي أن الحساب يكون قبل الميزان وفيه تعرض الأعمال فقط قبل أن توزن على الميزان وهذا لا يصح. لأن الحساب لا يكتمل إلا بعد وزن الأعمال كلها كبيرها وصغيرها وعدها وجمعها. ولهذا فإن الحساب يبدأ مع بداية وزن الأعمال وينتهي عند الإنتهاء من وزنها وعدها وجمعها. يبدأ الحساب مع وزن الأعمال وهنا تعرض الأعمال وتوزن على الميزان كلها صغيرها وكبيرها حتى يحسب حسابها الحساب الكامل والوافي الذي لا ظلم فيه، وبناءا على الحساب الكلي يعطى الجزاء إما جنة وإما نار. 


العجائب: 

إنَّ اللَّهَ سيُخَلِّصُ رجلًا من أمَّتي على رؤوسِ الخلائقِ يومَ القيامةِ فينشُرُ علَيهِ تسعةً وتسعينَ سجلًّا ، كلُّ سجلٍّ مثلُ مدِّ البصرِ ثمَّ يقولُ : أتنكرُ من هذا شيئًا ؟ أظلمَكَ كتبتي الحافِظونَ ؟يقولُ : لا يا ربِّ ، فيقولُ : أفلَكَ عذرٌ ؟ فيقولُ : لا يا ربِّ ، فيقولُ : بلَى ، إنَّ لَكَ عِندَنا حسنةً ، وإنَّهُ لا ظُلمَ عليكَ اليومَ ، فيخرجُ بطاقةً فيها أشهدُ أن لا إلَهَ إلَّا اللَّهُ ، وأشهدُ أنَّ محمَّدًا عبدُهُ ورسولُهُ ، فيقولُ : احضُر وزنَكَ فيقولُ يا ربِّ ، ما هذِهِ البطاقةُ مع هذِهِ السِّجلَّاتِ ؟ فقالَ : فإنَّكَ لا تُظلَمُ ، قالَ : فتوضَعُ السِّجلَّاتُ في كفَّةٍ ، والبطاقةُ في كفَّةٍ فطاشتِ السِّجلَّاتُ وثقُلتِ البطاقةُ ، ولا يثقلُ معَ اسمِ اللَّهِ شيءٌ {\footnotesize (صحيح الترمذي، صححه الألباني)}


\subsection{الجزاء إما جنة أو نار}

الجنة 

مائة درجة 

وهي مائة درجة كما جاء ذلك عن النبي ﷺ حيث قال: مَن آمَنَ باللَّهِ وبِرَسولِهِ، وأَقامَ الصَّلاةَ، وصامَ رَمَضانَ؛ كانَ حَقًّا علَى اللَّهِ أنْ يُدْخِلَهُ الجَنَّةَ، جاهَدَ في سَبيلِ اللَّهِ أوْ جَلَسَ في أرْضِهِ الَّتي وُلِدَ فيها، فقالوا: يا رَسولَ اللَّهِ، أفَلا نُبَشِّرُ النَّاسَ؟ قالَ: إنَّ في الجَنَّةِ مِئَةَ دَرَجَةٍ، أعَدَّها اللَّهُ لِلْمُجاهِدِينَ في سَبيلِ اللَّهِ، ما بيْنَ الدَّرَجَتَيْنِ كما بيْنَ السَّماءِ والأرْضِ، فإذا سَأَلْتُمُ اللَّهَ، فاسْأَلُوهُ الفِرْدَوْسَ؛ فإنَّه أوْسَطُ الجَنَّةِ وأَعْلَى الجَنَّةِ -أُراهُ- فَوْقَهُ عَرْشُ الرَّحْمَنِ ، ومِنْهُ تَفَجَّرُ أنْهارُ الجَنَّةِ. {\footnotesize (صحيح البخاري)}.

أعلى مراتب الجنة هي الفردوس الأعلى كما جاء عن أم المؤمنين عائشة رضي الله عنها أن النبيُّ صَلَّى اللهُ عليه وسلَّمَ كانَ يقولُ وهو صَحِيحٌ: إنَّه لَمْ يُقْبَضْ نَبِيٌّ حتَّى يَرَى مَقْعَدَهُ مِنَ الجَنَّةِ، ثُمَّ يُخَيَّرَ فَلَمَّا نَزَلَ به، ورَأْسُهُ علَى فَخِذِي غُشِيَ عليه، ثُمَّ أفَاقَ فأشْخَصَ بَصَرَهُ إلى سَقْفِ البَيْتِ، ثُمَّ قالَ: اللَّهُمَّ الرَّفِيقَ الأعْلَى . فَقُلتُ: إذًا لا يَخْتَارُنَا، وعَرَفْتُ أنَّه الحَديثُ الذي كانَ يُحَدِّثُنَا وهو صَحِيحٌ، قالَتْ: فَكَانَتْ آخِرَ كَلِمَةٍ تَكَلَّمَ بهَا: اللَّهُمَّ الرَّفِيقَ الأعْلَى {\footnotesize (صحيح البخاري)}. فعرفت عائشة رضي الله عنه أن الرسول ﷺ أوري مقعده في الجنة فاختار الرفيق الأعلى  قبل أن تقض روحه ﷺ.


وأدناهم مرتبة هو آخر رجل يدخل الجنة وهو آخر رجل يخرج من النار كما صح ذلك عن النبي ﷺ حيث قال: إنِّي لأعلَمُ آخِرَ أهلِ النَّارِ خُرُوجًا مِنها، وآخِرَ أهلِ الجَنَّةِ دُخُولًا الجَنَّةَ: رَجُلٌ يَخرُجُ مِنَ النَّارِ حَبْوًا، فيَقُولُ اللهُ تباركَ وتعالى لَه: اذهَب فادخُلِ الجَنَّةَ، فيَأتيها فيُخَيَّلُ إليه أنَّها مَلأى، فيَرجِعُ فيَقُولُ: يا رَبِّ، وجَدْتُها مَلأى، فيَقُولُ اللهُ تباركَ وتعالى لَه: اذهَبْ فادخُلِ الجَنَّةَ، قال: فيَأتيها فيُخَيَّلُ إليه أنَّها مَلأى، فيَرجِعُ فيَقُولُ: يا رَبِّ، وجَدْتُها مَلأى، فيَقُولُ اللهُ لَه: اذهَبْ فادخُلِ الجَنَّةَ؛ فإنَّ لَكَ مِثلَ الدُّنيا وعَشَرةَ أمثالِها -أو إنَّ لَكَ عَشَرةَ أمثالِ الدُّنيا- قال: فيَقُولُ: أتَسخَرُ بي -أو أتضحَكُ بي- وأنتَ المَلِكُ؟ قال: لَقَد رَأيتُ رَسولَ اللهِ صلَّى اللهُ عليه وسلَّم ضَحِكَ حَتَّى بَدَت نَواجِذُه، قال: فكانَ يُقالُ: ذاكَ أدنى أهلِ الجَنَّةِ مَنزِلةً {\footnotesize (صحيح البخاري، صحيح مسلم واللفظ له)}.




