\chapter{الميزان الكوني والميزان الشرعي}

\section{المقدمة}

علم الحساب من العلوم التي تدرك بالعقل والفطرة وقد يكون هو العلم الوحيد الذي يكاد لا يختلف عليه البشر بكافة أجناسهم وألوانهم وبالأخص لمن عرف هذا العلم وتمعن فيه صدقا. وذلك لأن الله جل جلاله خلق كل شئ بقدر معلوم ووضع الميزان الكوني فجعل هذا الكون موزونا ومتناسقا سبحانه. ومن فضله ومنه علي الناس أنه أرسل إليهم الرسل وأنزل الكتاب وما فرط فيه من شئ وأنزل معه الميزان الشرعي. ومن حكمته أنه سبحانه فطر الناس على فهم الميزانان وجعل لهم كل ما يحتاجونه من عقل وسمع وبصر. فجعل سبحانه آياته الكونية دليلا على الميزان الكوني وآياته الشرعية دليلا على الميزان الشرعي. وأرشد سبحانه إلى التأمل في آياته الكونية لتعلم العدد والحساب وهذا لحكمته سبحانه فالعدد والحساب يدرك بالعقل والفطرة ولهذا اكتفى سبحانه بالدلالة عليه. أما الميزان الشرعي فهو لا يدرك بالعقل والفطرة فقط وأنما يدرك بالوحي المنزل من عنده سبحانه. والله جل جلاله تكفل بإقامة الميزان الكوني وأنزل الكتب وأرسل الرسل وفرض على الناس إقامة الميزان الشرعي. ولما كان الحساب هو الوسيلة لإقامة الميزان الشرعي وجب النظر فيه وتعلمه من آيات الله الكونية لما في ذلك من مصالح دينية ودنيوية كما أرشد سبحانه في  كتابه العظيم في قوله تعالى: \quranayah*[10][5]{\footnotesize \surahname*[10]}. وفي قوله تعالى:
\quranayah*[17][12]{\footnotesize \surahname*[17]}.

جعل الله جل جلاله الميزان في آياته الكونية والشرعية لحكمته وعدله سبحانه. ولهذا فإن الميزان أمره عظيم عند الله فعَنْ أَبِي هُرَيْرَةَ، أَنَّ رَسُولَ اللَّهِ ﷺ قَالَ "يَدُ اللَّهِ مَلأَى لاَ يَغِيضُهَا نَفَقَةٌ، سَحَّاءُ اللَّيْلَ وَالنَّهَارَ ـ وَقَالَ ـ أَرَأَيْتُمْ مَا أَنْفَقَ مُنْذُ خَلَقَ السَّمَوَاتِ وَالأَرْضَ، فَإِنَّهُ لَمْ يَغِضْ مَا فِي يَدِهِ ـ وَقَالَ ـ عَرْشُهُ عَلَى الْمَاءِ وَبِيَدِهِ الأُخْرَى الْمِيزَانُ يَخْفِضُ وَيَرْفَعُ" {\footnotesize (صحيح البخاري)}. وهذا فيه أن الميزان الكوني بيده سبحانه فهو قائم عليه بالقسط والعدل لا تأخذه في ذلك سنة ولا نوم، ولهذا كانت آية الكرسي أعظم آية فقد قال سبحانه:
\quranayah*[2][255][1-12]{\footnotesize \surahname*[2]}. وَعَن أبي مُوسَى قَالَ قَامَ فِينَا رَسُولُ اللَّهِ ﷺ بِخَمْسِ كَلِمَاتٍ فَقَالَ: إِنَّ اللَّهَ عز وَجل لَا يَنَامُ وَلَا يَنْبَغِي لَهُ أَنْ يَنَامَ يَخْفِضُ الْقِسْطَ وَيَرْفَعُهُ يُرْفَعُ إِلَيْهِ عَمَلُ اللَّيْلِ قَبْلَ عَمَلِ النَّهَارِ وَعَمَلُ النَّهَارِ قَبْلَ عَمَلِ اللَّيْل حجابه النُّور. {\footnotesize (رَوَاهُ مُسلم وصححه الألباني)}. وهذا فيه أن الله عز وجل كامل في صفاته لا يلحقه نقص وهذا لازم لإقامة الوجود إذ يتعذر على غيره إقامة الميزان الكوني كما دلت على ذلك الآيات القرآنية والأحاديث. وقد جاء في تفسير ابن كثير أن قوله : ( لا تأخذه سنة ولا نوم ) أي : لا يعتريه نقص ولا غفلة ولا ذهول عن خلقه بل هو قائم على كل نفس بما كسبت شهيد على كل شيء لا يغيب عنه شيء ولا يخفى عليه خافية ، ومن تمام القيومية أنه لا يعتريه سنة ولا نوم ، فقوله : ( لا تأخذه ) أي : لا تغلبه سنة وهي الوسن والنعاس ولهذا قال : ( ولا نوم ) لأنه أقوى من السنة, [.] وقوله : ( ولا يئوده حفظهما ) أي : لا يثقله ولا يكرثه حفظ السماوات والأرض ومن فيهما ومن بينهما ، بل ذلك سهل عليه يسير لديه وهو القائم على كل نفس بما كسبت ، الرقيب على جميع الأشياء ، فلا يعزب عنه شيء ولا يغيب عنه شيء والأشياء كلها حقيرة بين يديه متواضعة ذليلة صغيرة بالنسبة إليه ، محتاجة فقيرة وهو الغني الحميد الفعال لما يريد ، الذي لا يسأل عما يفعل وهم يسألون . وهو القاهر لكل شيء الحسيب على كل شيء الرقيب العلي العظيم لا إله غيره ولا رب سواه [هـ].

ومن حكمته سبحانه وتعالى أنه وضع لنا الميزان الشرعي وهو العدل كما في قوله:
\quranayah*[55][7-9]{\footnotesize \surahname*[55]}. وجاء في تفسير السعدي أن معنى ووضع الله الميزان أي: العدل بين العباد، في الأقوال والأفعال، وليس المراد به الميزان المعروف وحده، بل هو كما ذكرنا، يدخل فيه الميزان المعروف، والمكيال الذي تكال به الأشياء والمقادير، والمساحات التي تضبط بها المجهولات، والحقائق التي يفصل بها بين المخلوقات، ويقام بها العدل بينهم [هـ]. وهذا بالتأكيد يشمل الحساب والذي هو أيضا صورة من صور الميزان. وكل هذا فيه أن الله عز وجل فرض على الناس إقامة الميزان الشرعي في قوله تعالى "ألا تطغوا في الميزان, وأقيموا الوزن بالقسط ولا تخسروا الميزان". وفيه أيضا أن الله عز وجل جعل العدل في آياته الشرعية كما في آياته الكونية وهذا دليل على حكمته وعدله وكماله سبحانه. وبهذا يتبين أن الميزان الشرعي هو من الأمانة في قوله تعالى:
\quranayah*[33][72]{\footnotesize \surahname*[33]}. وفي تفسير ابن كثير: قال العوفي عن ابن عباس يعني بالأمانة الطاعة [.] وقال قتادة الأمانة الدين والفرائض والحدود [هـ]. وهو الميثاق الذي ذكره الله في قوله:
\quranayah*[5][7]{\footnotesize \surahname*[5]}. يقول السعدي في تفسيره رحمه الله:
و { مِيثَاقَهُْ} أي: واذكروا ميثاقه { الَّذِي وَاثَقَكُمْ بِهِ ْ} أي: عهده الذي أخذه عليكم. وليس المراد بذلك أنهم لفظوا ونطقوا بالعهد والميثاق، وإنما المراد بذلك أنهم بإيمانهم بالله ورسوله قد التزموا طاعتهما، ولهذا قال: { إِذْ قُلْتُمْ سَمِعْنَا وَأَطَعْنَا ْ} أي: سمعنا ما دعوتنا به من آياتك القرآنية والكونية، سمع فهم وإذعان وانقياد. وأطعنا ما أمرتنا به بالامتثال، وما نهيتنا عنه بالاجتناب. وهذا شامل لجميع شرائع الدين الظاهرة والباطنة. وأن المؤمنين يذكرون في ذلك عهد الله وميثاقه عليهم، وتكون منهم على بال، ويحرصون على أداء ما أُمِرُوا به كاملا غير ناقص. { وَاتَّقُوا اللَّهَ ْ} في جميع أحوالكم { إِنَّ اللَّهَ عَلِيمٌ بِذَاتِ الصُّدُورِ ْ} أي: بما تنطوي عليه من الأفكار والأسرار والخواطر. فاحذروا أن يطلع من قلوبكم على أمر لا يرضاه، أو يصدر منكم ما يكرهه، واعمروا قلوبكم بمعرفته ومحبته والنصح لعباده. فإنكم -إن كنتم كذلك- غفر لكم السيئات، وضاعف لكم الحسنات، لعلمه بصلاح قلوبكم [هـ].

ولقد قرر أهل السنة والجماعة في هذا الباب العظيم وبناء على الأدلة والبراهين الواضحة من كتاب الله وسنة نبيه ﷺ وما يوافق العقل والنقل، أن الله جل جلاله له أرادتان وهما الإرادة الكونية والإرادة الشرعية. فالإرادة الكونية هي ما تعلق بمشيئته سبحانه والإرادة الشرعية هي ما تعلق بمحبته ورضاه. ومن صفات الله أنه عدل في إرادته الكونية والشرعية، وبهذا يكون الميزان الكوني تابعا لإرادة الله الكونية والميزان الشرعي تابعا لإرادة الله الشرعية. إن المقصود والمراد من إقامة الميزان بالمجمل هي إقامة الميزان الشرعي وهو يشمل جميع العبادات والأعمال التي يحبها الله ويرضاها والتي أمر الله عباده بها عن طريق الرسل والكتب ومن ذلك إقامة العدل والكيل بالقسط والحساب الصحيح. وأما الميزان الكوني فقد تكفل به سبحانه وهو دليل على عظمته وقدرته إذ يتعذر على الناس العيش من دونه فضلا عن إقامته، ومن ذلك أن الله عدل في قضاءه ومشيئته وحليم في تدبيره والدليل على هذا قوله تعالى:
\quranayah*[35][41]{\footnotesize \surahname*[35]}. فهذا دليل على وحدانيته سبحانه بالملك والتدبير، فالأدلية العقلية والشرعية دلت على وحدانيته سبحانه ومن ذلك أنه جعل هذا الكون موزونا ومتناسقا بإرادته الكونية لا يشاركه في ذلك أحد كما في قوله تعالى:
\quranayah*[21][22]{\footnotesize \surahname*[21]}.

ومن عدله وحكمته سبحانه أنه جعل الظلم منافيا ومخالفا للميزان الشرعي وسببا للإخلال بالميزان الكوني لولا أن مشيئته سبحانه نافذة على غيره، ومن أعظم ذلك الشرك بالله كما في قوله تعالى:
\quranayah*[31][13]{\footnotesize \surahname*[31]}، أو دعوة الولد له سبحانه والدليل قوله تعالى:
\quranayah*[19][88-91]{\footnotesize \surahname*[19]}. يقول السعدي رحمه الله: { أَنْ دَعَوْا لِلرَّحْمَنِ } أي: من أجل هذه الدعوى القبيحة تكاد هذه المخلوقات، أن يكون منها ما ذكر. والحال أنه: { مَا يَنْبَغِي } أي: لا يليق ولا يكون { لِلرَّحْمَنِ أَنْ يَتَّخِذَ وَلَدًا } وذلك لأن اتخاذه الولد، يدل على نقصه واحتياجه، وهو الغني الحميد. والولد أيضا، من جنس والده، والله تعالى لا شبيه له ولا مثل ولا سمي [هـ]. ومن ذلك أيضا اتباع الهوى بدلا من إقامة الحق ونصرته كما في قوله تعالى:
\quranayah*[23][71]{\footnotesize \surahname*[23]}. يقول السعدي في تفسيره:
ووجه ذلك أن أهواءهم متعلقة بالظلم والكفر والفساد من الأخلاق والأعمال، فلو اتبع الحق أهواءهم لفسدت السماوات والأرض، لفساد التصرف والتدبير المبني على الظلم وعدم العدل، فالسماوات والأرض ما استقامتا إلا بالحق والعدل[هـ]. ومن ذلك أيضا نقصان البركة بسبب المعاصي كما في قوله تعالى:
\quranayah*[30][41]{\footnotesize \surahname*[30]}. فقد ورد في تفسير القرطبي أن ابن عباس قال: هو نقصان البركة بأعمال العباد كي يتوبوا [هـ]. وجاء في تفسيير ابن كثير أن زيد بن رفيع قال: ( ظهر الفساد ) يعني انقطاع المطر عن البر يعقبه القحط، وعن البحر تعمى دوابه [هـ].

وهذا فيه أن الله جل جلاله عدل في أحكامه الشرعية والكونية. وبهذا يكون العدل مرادفا للميزان، فيكون الميزان الشرعي هو العدل في إرادة الله الشرعية والميزان الكوني هو العدل في إرادة الله الكونية. فالله عز وجل عدل في إرادته ومشيئته وقضاءه والدليل قوله تعالى:

\quranayah*[3][108]{\footnotesize \surahname*[3]}.


ومن ذلك أن الله عدل في جزاءه وثوابه كما أخبر هو بذلك في قوله تعالى:
\quranayah*[39][69]{\footnotesize \surahname*[69]}.

فحكم الله نافذ وماض بإراته الكونية وبما شاء وقضاءه عدل سبحانه كما قال تعالى:
\quranayah*[40][20]{\footnotesize \surahname*[40]}. وقد جاء في تقسير ابن كثير أن قوله : ( والله يقضي بالحق ) أي : يحكم بالعدل، وقوله : ( والذين يدعون من دونه ) أي : من الأصنام والأوثان والأنداد ، ( لا يقضون بشيء ) أي : لا يملكون شيئا ولا يحكمون بشيء [هـ].

وكما جاء عن النبي ﷺ أنه قال:
«ماضٍ فيَّ حكمُك، عدلٌ فيَّ قضاؤُك» {\footnotesize (أخرجه أحمد وصححه الألباني)}


\section{الغاية من إنزال الكتب}

أنزل الله جل جلاله كتابه لغاية عظيمة وهي إقامة الحق والميزان فقال تعالى:
\quranayah*[42][17]{\footnotesize \surahname*[42]}.
فذكر الله الميزان إلحاقا بالحق لان الحق لا يكون إلا بالعلم الصحيح وهو يقتضي الميزان الشرعي الذي لا يكون إلا بالعمل الصحيح ومنه العدل والقسط كما أمر تعالى ومن ذلك بلا شك الحساب الصحيح كما سيأتي. فإقامة الميزان الشرعي من الوصايا العشر من سورة الأنعام في قوله تعالى:
\quranayah*[6][152][12]{\footnotesize \surahname*[6]}. فقرن الله عز وجل في هذه الآيات بين الكيل والميزان والعدل في القول والوفاء بالعهد. وفيه أن الميزان والكيل لا يكون إلا بالقسط وهو العدل. وفيه أن العدل ذكر مع ذا القربى ولذلك يكون العدل بين الناس.
وأيضا من الوصايا التي ذكرها الله في سورة الإسراء في قوله تعالى:
\quranayah*[17][35]{\footnotesize \surahname*[17]}. يقول السعدي في تفسيره:
وهذا أمر بالعدل وإيفاء المكاييل والموازين بالقسط من غير بخس ولا نقص. ويؤخذ من عموم المعنى النهي عن كل غش في ثمن أو مثمن أو معقود عليه والأمر بالنصح والصدق في المعاملة [هـ]. ومن ذلك بلا الشك الحساب ولذلك وجب الوفاء والصدق فيه من غير غش ولا تضليل وإقامة الميزان فيه بالقسط كما في الكيل كما سيأتي.

\section{الغاية من إرسال الرسل}

وقد أرسل الله عز وجل رسله بالكتاب أولا لبيان الحق وهو العلم الصحيح ومن ثم لإقامة الميزان الشرعي بالقسط والعدل بين الناس حيث يقول جل جلاله:
\quranayah*[57][25]{\footnotesize \surahname*[57]}. فالقسط في الكيل والوزن والعدل بين الناس من إقامة الميزان الشرعي وهو من الأمور التي أوصى الله تعالى بها وهي من أسباب القوى ولهذا فقد دلنا سبحانه على الحديد والأخذ به لنصرة الله جل جلاله ورسله ونصرة الحق الذي جاءوا به. وفي هذا دليل على أن نصرة الله ورسله تكون بثلاثة أمور وهي إقامة الحق بالعلم الصحيح وإقامة الميزان الشرعي بالعدل والقسط والأخذ بأسباب القوى كالحديد وما يلزم ذلك من علوم كالحساب والطب وغيرها من العلوم التي تكمن المسلمين من دحر الأعداء ونشر الحق ونصرته. فهذا كله من نصر الله ورسله وقد وعد سبحانه بنصر من ينصره كما في قوله تعالى:
\quranayah*[47][7]{\footnotesize \surahname*[47]}.


\section{مكانة أهل العلم الشرعي}

ولما كان الحق هو العلم الصحيح, كان أهل العلم الشرعي هم أعلم الناس بالحق كما في قوله تعالى:
\quranayah*[34][6]{\footnotesize \surahname*[34]}. وفيه أن الحق يهدي إلى الطريق المستقيم كما ذكر تعالى في هذه الآية وفي قوله تعالى:
\quranayah*[22][54]{\footnotesize \surahname*[22]}. فقد جعل الله جل جلاله أهل العلم حجة على الناس لما معهم من الحق فكانوا بذلك هم ورثة الأنبياء في الأرض فقد قال تعالى:
\quranayah*[17][107]{\footnotesize \surahname*[17]}.
وقوله تعالى:
\quranayah*[29][49]{\footnotesize \surahname*[29]}.


ولهذا فقد رفع الله مكانة أهل العلم في الدنيا والأخرة لما عرفوا من الحق كما في قوله تعالى:
\quranayah*[58][11][19]{\footnotesize \surahname*[58]}.
ومن أعظم ذلك قوله تعالى:
\quranayah*[3][18]{\footnotesize \surahname*[3]}.
يقول السعدي في تفسيره:
هذا تقرير من الله تعالى للتوحيد بأعظم الطرق الموجبة له، وهي شهادته تعالى وشهادة خواص الخلق وهم الملائكة وأهل العلم [.] وأما شهادة أهل العلم فلأنهم هم المرجع في جميع الأمور الدينية خصوصا في أعظم الأمور وأجلها وأشرفها وهو التوحيد، فكلهم من أولهم إلى آخرهم قد اتفقوا على ذلك ودعوا إليه وبينوا للناس الطرق الموصلة إليه، فوجب على الخلق التزام هذا الأمر المشهود عليه والعمل به، وفي هذا دليل على أن أشرف الأمور علم التوحيد لأن الله شهد به بنفسه وأشهد عليه خواص خلقه، والشهادة لا تكون إلا عن علم ويقين، بمنزلة المشاهدة للبصر، ففيه دليل على أن من لم يصل في علم التوحيد إلى هذه الحالة فليس من أولي العلم. وفي هذه الآية دليل على شرف العلم من وجوه كثيرة، منها: أن الله خصهم بالشهادة على أعظم مشهود عليه دون الناس، ومنها: أن الله قرن شهادتهم بشهادته وشهادة ملائكته، وكفى بذلك فضلا، ومنها: أنه جعلهم أولي العلم، فأضافهم إلى العلم، إذ هم القائمون به المتصفون بصفته، ومنها: أنه تعالى جعلهم شهداء وحجة على الناس، وألزم الناس العمل بالأمر المشهود به، فيكونون هم السبب في ذلك، فيكون كل من عمل بذلك نالهم من أجره، وذلك فضل الله يؤتيه من يشاء، ومنها: أن إشهاده تعالى أهل العلم يتضمن ذلك تزكيتهم وتعديلهم وأنهم أمناء على ما استرعاهم عليه [هـ].

\section{حال الأنبياء مع الميزان الشرعي}

ولما كان الأنبياء أعلم الناس بأمر الله وأحرصهم, فقد أقاموا الميزان الشرعي حق إقامته في حكمهم الرشيد بين الخلق. ومثال ذلك يوسف عليه السلام في قوله تعالى:
\quranayah*[12][55]{\footnotesize \surahname*[12]}.
وهذا فيه حرصه عليه السلام على إقامة الكيل والوزن بما يرضي الله وهذا من الإصلاح الذي أمر الله به حيث قال لإخوته عن الكيل:
\quranayah*[12][59-60]{\footnotesize \surahname*[12]}.

فلا شك أن التفريط في الكيل والوزن من أعظم البلايا التي حذرنا الله عز وجل منها في كتابه الكريم فيقول تعالى:
\quranayah*[83][1-6]{\footnotesize \surahname*[83]}. فالظلم في الكيل والوزن من الإفساد العظيم ومن أسباب تعجيل العذاب في الدنيا قبل الأخرة, وفي قصة مدين مع نبيهم شعيبا العبرة الواضحة في ذلك. يقول تعالى على لسان نبيه شعيب محذرا قومه:
\quranayah*[7][85]{\footnotesize \surahname*[7]}.
وفي موضع أخر من سورة الشعراء:
\quranayah*[26][181-183]{\footnotesize \surahname*[26]}.
وفي سورة هود:
\quranayah*[11][84-85]{\footnotesize \surahname*[11]}. وهذا فيه أن شعيبا عليه السلام دعا قومه لإقامة الحق أولا وهو التوحيد بإفراد الله بالعبادة وثانيا لإقامة الميزان الشرعي وهو الكيل الوزن بالقسط. وفيه أن بخس الناس أشيائهم والخسران والنقصان في الكيل والوزن من الظلم الموجب لسخط الله وعذابه العاجل.

ونبينا ﷺ كان من أحرص الناس في إقامة الكيل والميزان وحذر من الفساد في ذلك في العديد من المواضع منها ما ورد عن ابن عباس رضي الله عنه أنه قَالَ: قَالَ رَسُولُ اللَّهِ صَلَّى اللَّهُ عَلَيْهِ وَسَلَّمَ لِأَصْحَابِ الْكَيْلِ وَالْمِيزَانِ: «إِنَّكُمْ قَدْ وُلِّيتُمْ أَمْرَيْنِ هَلَكَتْ فِيهِمَا الْأُمَمُ السَّابِقَة قبلكُمْ».
{\footnotesize رَوَاهُ التِّرْمِذِيّ}.
وعَنْ عَبْدِ اللَّهِ بْنِ عُمَرَ، قَالَ أَقْبَلَ عَلَيْنَا رَسُولُ اللَّهِ ـ ﷺ ـ فَقَالَ «يَا مَعْشَرَ الْمُهَاجِرِينَ خَمْسٌ إِذَا ابْتُلِيتُمْ بِهِنَّ وَأَعُوذُ بِاللَّهِ أَنْ تُدْرِكُوهُنَّ: لَمْ تَظْهَرِ الْفَاحِشَةُ فِي قَوْمٍ قَطُّ حَتَّى يُعْلِنُوا بِهَا إِلاَّ فَشَا فِيهِمُ الطَّاعُونُ وَالأَوْجَاعُ الَّتِي لَمْ تَكُنْ مَضَتْ فِي أَسْلاَفِهِمُ الَّذِينَ مَضَوْا, وَلَمْ يَنْقُصُوا الْمِكْيَالَ وَالْمِيزَانَ إِلاَّ أُخِذُوا بِالسِّنِينَ وَشِدَّةِ الْمَؤُنَةِ وَجَوْرِ السُّلْطَانِ عَلَيْهِمْ, وَلَمْ يَمْنَعُوا زَكَاةَ أَمْوَالِهِمْ إِلاَّ مُنِعُوا الْقَطْرَ مِنَ السَّمَاءِ وَلَوْلاَ الْبَهَائِمُ لَمْ يُمْطَرُوا, وَلَمْ يَنْقُضُوا عَهْدَ اللَّهِ وَعَهْدَ رَسُولِهِ إِلاَّ سَلَّطَ اللَّهُ عَلَيْهِمْ عَدُوًّا مِنْ غَيْرِهِمْ فَأَخَذُوا بَعْضَ مَا فِي أَيْدِيهِمْ, وَمَا لَمْ تَحْكُمْ أَئِمَّتُهُمْ بِكِتَابِ اللَّهِ وَيَتَخَيَّرُوا مِمَّا أَنْزَلَ اللَّهُ إِلاَّ جَعَلَ اللَّهُ بَأْسَهُمْ بَيْنَهُمْ».
{\footnotesize أخرجه ابن ماجه وصححه الألباني}.
فذكر ﷺ أن النقص في الكيل والوزن من أسباب البلاء العظيم ومنها الفقر والجوع وجور السلطان. وفيه الدليل على نبوته ﷺ فقد وقع ذلك كما أخبر بعد أن تهاون الكثير من المسلمين في أمر الميزان والمكيال إلا من رحم الله. وقد علم الصحابة والتابعين بأهمية إقامة الميزان والمكيال وأن الفساد فيهما من أسباب سخط الله ومنه ما ورد عَنْ يَحْيَى بْنِ سَعِيدٍ، أَنَّهُ سَمِعَ سَعِيدَ بْنَ الْمُسَيَّبِ، يَقُولُ إِذَا جِئْتَ أَرْضًا يُوفُونَ الْمِكْيَالَ وَالْمِيزَانَ فَأَطِلِ الْمُقَامَ بِهَا وَإِذَا جِئْتَ أَرْضًا يُنَقِّصُونَ الْمِكْيَالَ وَالْمِيزَانَ فَأَقْلِلِ الْمُقَامَ بِهَا.

ومن ذلك أيضا الربا وبيع العينة لما فيه من التلاعب بالميزان الذي أمر الله بإقامته فعَنِ ابْنِ عُمَرَ، قَالَ سَمِعْتُ رَسُولَ اللَّهِ صلى الله عليه وسلم يَقُولُ "إِذَا تَبَايَعْتُمْ بِالْعِينَةِ وَأَخَذْتُمْ أَذْنَابَ الْبَقَرِ وَرَضِيتُمْ بِالزَّرْعِ وَتَرَكْتُمُ الْجِهَادَ سَلَّطَ اللَّهُ عَلَيْكُمْ ذُلاًّ لاَ يَنْزِعُهُ حَتَّى تَرْجِعُوا إِلَى دِينِكُمْ" {\footnotesize (صححه الألباني)}. يقول الشيخ العثيمين في بيان العينة:
أن يبيع شيئا بثمن مؤجل ثم يشتريه ممن باعه عليه بأقل منه نقدًا [.] وسُمي بذلك لأن المشتري لم يُرد السلعة وإنما أراد العين أي: النقد، النقد لينتفع به، ودليل ذلك: أنه اشتراها بثمن زائد مؤجل، ثم باعها على من اشتراها منه بنقد، فكأنه لم يقصد هذه السلعة وإنما قصد الثمن الدراهم، فلهذا سمي بيع عينة [.] والغالب أن هذا ملازم لهذا، يعني أن الذي ينهمك في طلب الدنيا ويتحيل على الحصول عليها حتى بما حرم الله، الغالب أنه يترك الجهاد، لأن قلبه انشغل بالدنيا عنه.
[هـ]. وهذا فيه أن التفريط في الميزان حبا في الدنيا من أسباب عقاب الله وتسلط الأعداء, فعَنْ ثَوْبَانَ، قَالَ قَالَ رَسُولُ اللَّهِ صلى الله عليه وسلم "يُوشِكُ الأُمَمُ أَنْ تَدَاعَى عَلَيْكُمْ كَمَا تَدَاعَى الأَكَلَةُ إِلَى قَصْعَتِهَا ". فَقَالَ قَائِلٌ وَمِنْ قِلَّةٍ نَحْنُ يَوْمَئِذٍ قَالَ "بَلْ أَنْتُمْ يَوْمَئِذٍ كَثِيرٌ وَلَكِنَّكُمْ غُثَاءٌ كَغُثَاءِ السَّيْلِ وَلَيَنْزِعَنَّ اللَّهُ مِنْ صُدُورِ عَدُوِّكُمُ الْمَهَابَةَ مِنْكُمْ وَلَيَقْذِفَنَّ اللَّهُ فِي قُلُوبِكُمُ الْوَهَنَ". فَقَالَ قَائِلٌ يَا رَسُولَ اللَّهِ وَمَا الْوَهَنُ قَالَ "حُبُّ الدُّنْيَا وَكَرَاهِيَةُ الْمَوْتِ" {\footnotesize (صححه الألباني)}. وفيه إن التفريط في الميزان الشرعي الذي أمر الله به حبا في الدنيا هو من أسباب الذل والهوان في الدنيا قبل الأخرة.

\section{حال الأمم مع الميزان}

الحساب الصحيح يبنى على الميزان. فإن وافق هذا الحساب أيات الله الشرعية أو الفطرة السليمة فهو من العدل والإصلاح الذي أمر الله به. وإن خالف أمر الله أو الفطرة التي فطر الله الناس عليها فهو من الظلم والفساد الذي لا يرضى الله به ومن أسباب تعجيل سخط الله في الدنيا قبل الأخرة. وهذا بالعموم على المسلمين وغيرهم. فقد قال تعالى:
\quranayah*[11][117]{\footnotesize \surahname*[11]}.
وقوله تعالى:
\quranayah*[10][13-14]{\footnotesize \surahname*[10]}.
وفيه أن الله جل جلاله ناظر على أعمالنا وأعمال الأهم ومجازيها بعدله سبحانه وتعالى.

فحال الأمم يدور مع الحق والميزان في أربعة أحوال من الأقل تمكينا إلى الأكثر تمكينا:

\begin{compactitem}
    \item الدولة الكافرة الظالمة
    \item الدولة المسلمة الظالمة
    \item الدولة الكافرة العادلة
    \item الدولة المؤمنة العادلة
\end{compactitem}

يقول شيخ الإسلام ابن تمية رحمه الله:
ولهذا يروى ان الله ينصر الدولة العادلة وان كانت كافرة ولا ينصر الدولة الظالمة وان كانت مؤمنة
[هـ] {\footnotesize (مجموع الفتاوى 28/63)}.
والأصح أن يقال: "ولا ينصر الدولة الظالمة وإن كانت مسلمة" وهذا لان الظلم ينافي الغاية من الإيمان وكماله وهو بلا شك معصية لله ورسوله والدليل قوله تعالى:
\quranayah*[49][14]{\footnotesize \surahname*[49]}.
وقد قال الشيخ الألباني رحمه الله تعالى في توضيح هذا المعنى:
ذلك لأنَّ الظلم هو سبب خراب البلاد وهلاك العباد، فإذا كانت الأمة أو الدولة كافرة ولكنها تحكم بالعدل فيما بينها، هذا العدل الذي يعرفه الناس بفِطَرهم، فإذا كانوا يحكمون بذلك فستقوم دولتهم وتستمرُّ مدَّة طويلة، والتاريخ يحفل بهذا
[هـ].
والحساب يكون سبيل إلى الحكم بالعدل إن كان صحيحا إن يوافق الفطرة السليمة ونجاة من سخط الله وعذابه العاجل في الدنيا. فإن وافق الحساب الحكم الشرعي مع الفطرة كان ذلك نجاة في الدنيا والأخرة وكان حكما راشدا. ولهذا كان الحساب لإقامة الحق والميزان من بنيان الحكم الرشيد. وعليه يكون علم الحساب من الدين بالضرورة وليس بخلاف ذلك إذ يتعذر إقامة الميزان حق إقامته من دون حساب صحيح.

بالتتبع والإستقراء يتبين أن الله يقيم الأمم التي تحكم بالعدل الذي يوافق الفطرة وإن كانت كافرة. فإن في ذلك سلامة من عذاب الله في الدنيا كما تقدم. فإن كانت مؤمنة وتحكم بالعدل كانت حكما راشدا ووعدها الله بالتمكين في الدنيا والفوز في الأخرة, يقول تعالى:
\quranayah*[24][55]{\footnotesize \surahname*[24]}.
وإن كانت مسلمة ولا تحكم بالعدل فقد خالفت حكمة الله والغاية من إيمانها الذي يقتضي إقامة العدل والميزان ويصدق فيها قوله تعالى:
\quranayah*[49][14]{\footnotesize \surahname*[49]}. وهذا هو حال أغلب أمة الإسلام في يومنا هذا كما هو معروف. فتكون بذلك الأمة الكافرة التي تحكم بالعدل قائمة فوق الأمة المسلمة التي لا تحكم بالعدل. وأما الأمة المؤمنة التي تقيم الحق وأجله التوحيد وما يقتضيه ذلك من إقامة الميزان ومنه العدل بين الناس تكون هي فوقهم جميعا كما دلت على ذلك الآيات والأحاديث. وهذا فيه الحكمة البالغة من الله عز وجل ومنه أن الله لا يرضى لعباده الظلم ولا يزال ذلك حال الأمة المسلمة حتى تقيم الميزان والكيل والعدل الذي أمر الله به. قال تعالى:
\quranayah*[13][11][12]{\footnotesize \surahname*[13]}. وهذا يشمل الراعي والرعية.

وبهذا يعلم أن الأمم إنما تقام بإقامة الميزان ومنه العدل بين الناس فإن تحقق ذلك سلمت سخط الله وعذابه في الدنيا وإن كانت كافرة. فإن لم تقم الميزان والعدل بين الناس فتكون بذلك قد جنت على نفسها عقاب الله العاجل في الدنيا من فقر وجوع وذل وجور السلطان  وإن كانت مسلمة. وأما إن كانت مؤمنة وأقامت الحق مع إقامة الميزان كما أمر الله كانت حكما راشدا وتحقق لها التمكين في الدنيا والفوز في الأخرة. وأما إقامة التوحيد دون إقامة الميزان وما يقتضيه من العدل بين الناس فهذا ينافي حكمة الله وأمره الذي بينه في كتابه وعلى لسان نبيه ﷺ. ولهذا وجب على المسلمين ودعاتهم الرجوع إلى أمر الله وعدم التهاون في ذلك ومنه العناية بإقامة الحق ومنه التوحيد وإقامة الميزان ومنه العدل بين الناس على حد السواء حتى يكون لهم التمكين الذي أمر الله به. ولذلك وجب علينا العناية بالحساب الصحيح بحثا وتطبيقا سعيا لتحقيق هذه الغاية العظيمة التي أمرنا الله بها وهي إقامة الحق والميزان الذي يبنى عليه الحكم الرشيد.

فقد روى الإمام أحمد في "المسند" (30 / 355) عَنِ حُذَيْفَةُ، قال: قَالَ رَسُولُ اللهِ صَلَّى اللهُ عَلَيْهِ وَسَلَّمَ: (تَكُونُ النُّبُوَّةُ فِيكُمْ مَا شَاءَ اللهُ أَنْ تَكُونَ، ثُمَّ يَرْفَعُهَا إِذَا شَاءَ أَنْ يَرْفَعَهَا، ثُمَّ تَكُونُ خِلَافَةٌ عَلَى مِنْهَاجِ النُّبُوَّةِ، فَتَكُونُ مَا شَاءَ اللهُ أَنْ تَكُونَ، ثُمَّ يَرْفَعُهَا إِذَا شَاءَ اللهُ أَنْ يَرْفَعَهَا، ثُمَّ تَكُونُ مُلْكًا عَاضًّا، فَيَكُونُ مَا شَاءَ اللهُ أَنْ يَكُونَ، ثُمَّ يَرْفَعُهَا إِذَا شَاءَ أَنْ يَرْفَعَهَا، ثُمَّ تَكُونُ مُلْكًا جَبْرِيَّةً، فَتَكُونُ مَا شَاءَ اللهُ أَنْ تَكُونَ، ثُمَّ يَرْفَعُهَا إِذَا شَاءَ أَنْ يَرْفَعَهَا، ثُمَّ تَكُونُ خِلَافَةٌ عَلَى مِنْهَاجِ نُبُوَّةٍ).


\section{الغاية من علم الحساب}

رغم أن علم الحساب قد يدرس لغايات كونية محضة إلا أن العلم به من الضروريات التي يحتاج إليها الناس في أمور دينهم ودنياهم. فعلم الحساب هو الوسيلة لتحقيق الغاية العظيمة التي أمر الله بها وهي إقامة الميزان والعدل. ولهذا كان البحث في علم الحساب من الأمور التي حث الله تعالى عليها في موضعين في كتابه. قال تعالى: \quranayah*[17][12]{\footnotesize \surahname*[17]}.
يقول السعدي رحمه الله في تفسيره:
{ وَلِتَعْلَمُوا } بتوالي الليل والنهار واختلاف القمر { عَدَدَ السِّنِينَ وَالْحِسَابَ } فتبنون عليها ما تشاءون من مصالحكم. { وَكُلَّ شَيْءٍ فَصَّلْنَاهُ تَفْصِيلًا } أي: بينا الآيات وصرفناه لتتميز الأشياء ويستبين الحق من الباطل كما قال تعالى: { مَا فَرَّطْنَا فِي الْكِتَابِ مِنْ شَيْءٍ }
[هـ]. وقال تعالى: \quranayah*[10][5]{\footnotesize \surahname*[10]}.
يقول السعدي رحمه الله في تفسير هذه الأيات:
وفي هذه الآيات الحث والترغيب على التفكر في مخلوقات الله، والنظر فيها بعين الاعتبار، فإن بذلك تنفتح البصيرة، ويزداد الإيمان والعقل، وتقوى القريحة، وفي إهمال ذلك، تهاون بما أمر الله به، وإغلاق لزيادة الإيمان، وجمود للذهن والقريحة
[هـ].

وفي هذا الحث والترغيب في علم الحساب الحكمة البالغة من الله جل جلاله. ومن ذلك أن علم الحساب هو مفتاح جميع العلوم التي يمكن فيها القياس والعد ولا يمكن فهمها فهما صحيحا من دون الحساب الصحيح. فعلم الحساب يحتاج لفهم الميزان الكوني من آيات الله الكونية وإقامة الميزان الشرعي باتباع آيات الله الشرعية. من الأمثلة لتطبيقات علم الحساب في اتباع آيات الله الشرعية كعلم المواريث والبيع والشراء وغيرها من المعاملات التي يحتاج إليها الناس.  ومن الأمثلة على تطبيقات علم الحساب في فهم آيات الله الكونية كحركة الشمس والقمر وغيرها من الظواهر الطبيعية التي خلقها الله لما في ذلك من تفكر في عظمة الله وزيادة في الإيمان. وإن أفضل طريقة لفهم علم الحساب هي التأمل والتفكر في آيات الله الكونية وهي الظواهر الطبيعية التي خلقها الله وجعل لها الميزان الكوني لفهمها وحسابها. فهي المرجع لنا حتى نتحقق من صحة وسلامة الحساب. وهذا النهج هو نهج القرآن وهو أفضل الطرق وأحسنها. ويمكن أيضا دراسة علم الحساب مجردا من أي تطبيقات وهذا نهج معروف. ولكن الجمع بين العلوم الطبيعية كعلم الفيزياء والحساب لمحاولة محاكاة الظواهر الطبيعية هي الطريق الأمثل لتطوير علم الحساب وهذا معروف لأهل هذا العلم. وبهذا يكون الميزان الكوني طريقا لتعلم الحساب الصحيح ومن ثم يكون الحساب الصحيح وسيلة لإقامة الميزان الشرعي الذي أمر الله به.

والحساب الصحيح لا يقام إلا بالميزان وهو صورة من صور الكيل إلا أن الكيل يكون بالميزان الحسي وأما الحساب يكون بالميزان العقلي. وعليه فإن الحساب لا يكون صحيحا إلا بالقسط والعدل كما في الكيل الوزن تماما. ومن رحمة الله أنه فطر الناس على هذا وجعل لهم كل ما يحتاجونه من عقل وسمع وبصر لفهم الحساب والعدد وليبنوا عليه مصالحهم الدينية والدنيوية. ولهذا فإن الآيات الشرعية التي تأمر بالعدل في الكيل وإقامة الميزان بالقسط فهي بلا شك تشمل الحساب بالميزان العقلي كما هو الحال مع الكيل بالميزان الحسي.


\section{الحساب الصحيح من القسط}
ولما كان الحساب هو مثل الكيل تماما ولكن بالميزان العقلي بدلا من الميزان الحسي, وجب إقامته بالقسط وهو العدل الظاهر.

\section{الجهل بالحساب من الأمية}

ومن أعظم البلايا في زماننا هذا أن المسلمين في غالبهم قد أضاعوا هذا العلم العظيم ظنا منهم أنه ليس من الدين في شئ بعد أن كانوا روادا فيه ووضعوا أسسه وقواعده في زمن هارون الرشيد كما سيأتي. فاعتنت وتسابقت وتهالفت عليه الأمم الأخرى وكان سببا في نهوضها وإزدهارها بل وأيضا تسلطها على أمة الإسلام. فتضييع علم الحساب من الأمية التي جاء الإسلام بالحث على خلافها من طلب العلم ونشره وإقامة الحق والميزان. فالأمية لا تكون فقط بعدم القدرة على القراءة والكتابة كما هو شائع, وإنما ايضا بعدم القدرة على الحساب.
ومما يأكد هذا الطرح قوله ﷺ عندما سأل عن عدد الأيام في الشهر فقال ﷺ:
إنَّا أُمَّةٌ أُمِّيَّةٌ، لا نَكْتُبُ ولَا نَحْسُبُ، الشَّهْرُ هَكَذَا وهَكَذَا. يَعْنِي مَرَّةً تِسْعَةً وعِشْرِينَ، ومَرَّةً ثَلَاثِينَ"
{\footnotesize (صحيح البخاري)}.
فجعل ﷺ الجهل بعلم الحساب من الأمية.

ولهذا جاءت الشريعة بالحث أولا على القراءة ومن ثم الحساب. فكان أول ما أنزل الله "إقرا" وفيه الحث لأمة الإسلام على تعلم القراءة والكتابة وطلب العلم ونشره.
يقول تعالى:
\quranayah*[96][1-5]{\footnotesize \surahname*[96]}.
ومن ثم جاء الحث على التأمل في آيات الله الكونية في مواضع كثيرة ليس فقط لمجرد التفكر في خلق الله ولكن أيضا لتعلم العدد والحساب كما جاء في قوله تعالى:
\quranayah*[10][5]{\footnotesize \surahname*[10]}.

\comment{
ومن المعلوم أن الرسول ﷺ كان أميا كما قال تعالى:
\quranayah*[7][158][21]{\footnotesize \surahname*[7]}.
وكذلك الصحابة رضى الله عنهم أجمعين وصفهم الله بالأمية في قوله تعالى:
\quranayah*[62][2]{\footnotesize \surahname*[62]}. فالرسول ﷺ والصحابة رضى الله عنهم في غالبهم كانوا أمة أمية لا يقرأون ولا يكتبون ولا بحسبون كما ورد هذا في كتاب الله وسنة نبيه ﷺ.
يقول الشيخ اين باز رحمه الله تعالى:
فكان النبي ﷺ مثلما قال الله: وَوَجَدَكَ ضَالًّا فَهَدَى [الضحى:7]، جاهلاً بالعلوم التي جاء بها الوحي، لم يكن عنده علم بما شرع الله له في كتابه العظيم، ولم يكن عنده علم بعلوم الأولين المرسلين، ولم يكن يكتب ويخط حتى جاءه هذا الخير العظيم والوحي العظيم عليه الصلاة والسلام، فكل إنسان لم يتعلم ولم يكتب يقال له: أمي، والأمة العربية هكذا كان الغالب عليها أنها أمية لا تكتب ولا تقرأ، هذا الغالب على أمة محمد ﷺ
[هـ].
}

ولهذا فإن كل إنسان لم يتعلم الحساب مع القراءة والكتابة يكون أميا كما بين ذلك النبي ﷺ. والله حث هذه الأمة الأمية في كتابه العظيم على العلم الذي يتأتى بالقراءة والكتابة حتى تقيم الحق والميزان الشرعي الذي يتأتى بالعدل والقسط في الكيل والحساب.


\section{الحساب الصحيح يبنى على الوزن}

والحساب الصحيح لا يقام إلا بالميزان فعَنْ أَبِي سَعِيدٍ وَأَبِي هُرَيْرَةَ: أَنَّ رَسُولَ اللَّهِ صَلَّى اللَّهُ عَلَيْهِ وَسلم اسْتَعْمَلَ رَجُلًا عَلَى خَيْبَرَ فَجَاءَهُ بِتَمْرٍ جَنِيبٍ فَقَالَ: «أَكُلُّ تَمْرِ خَيْبَرَ هَكَذَا؟» قَالَ: لَا وَاللَّهِ يَا رَسُولَ اللَّهِ, إِنَّا لَنَأْخُذُ الصَّاعَ مِنْ هَذَا بِالصَّاعَيْنِ وَالصَّاعَيْنِ بِالثَّلَاثِ فَقَالَ: «لَا تَفْعَلْ بِعِ الْجَمْعَ بِالدَّرَاهِمِ ثُمَّ ابْتَعْ بِالدَّرَاهِمِ جَنِيبًا». وَقَالَ: «فِي الْمِيزَانِ مِثْلَ ذَلِكَ»
{\footnotesize مُتَّفَقٌ عَلَيْهِ}.
وهذا فيه حرص النبي ﷺ حيث انه من المعلوم أن من اخذ صاع اضافيا لا يثبت قيمة البيع. فيكون من أخذ صاعين بدل صاع فقد اشترى بنصف قيمة ما باع بينما من أخذ ثلاثة بدل اثنين فقد اشترى بثلثي قيمة ما باع وهذا من الظلم الذي لا يقع إلا خطأ أو جهلا أو غشا. فأخبر النبي ﷺ أن هذا بخلاف الميزان وهو الحساب الصحيح في البيع والشراء, بل ونهى عن ذلك وأمر بأخذ القيمة عند البيع ومن ثم الشراء حتى تثبت القيمة. وفيه أيضا أن الرسول ﷺ سمى الحساب الصحيح ميزانا في قوله "في الميزان مثل ذلك". وهذا فيه دليل على نبوته ﷺ فهو أمي لا يحسب ولكن لا ينطق إلا بالحق كما أخبر ذلك الله عز وجل في كتابه العظيم:
\quranayah*[53][3-4]{\footnotesize \surahname*[53]}.

ومما يؤكد ما سبق أيضا قوله ﷺ:
الذهب بالذهب وزنًا بوزن، مثلًا بمثل، سواءً بسواء، يدًا بيد.
يقول الشيخ ابن باز رحمه الله تعالى في بيان معنى هذا الحديث: ولا فرق بين كونه جديدًا أو قديمًا، أو كون هذا أطيب وهذا أطيب، ما دام جنس الذهب لابد أن يكونا متساويين في الوزن يدًا بيد، يقبض في الحال [هـ]. وعن أبي سعيد الخدري رضي الله عنه أن رسول الله ﷺ قال: لا تبيعوا الذهب بالذهب إلا مثلا بمثل ولا تشفوا -أي تفاضلوا- بعضها على بعض، ولا تبيعوا منها غائبا بناجز
{\footnotesize (متفق عليه)}.

والحساب يبنى على التقدير العددي والتقدير الفكري. أما التقدير العددي فهو يضبط بالوزن والتقدير فكري يضبط بالحق. ولهذا يكون تقدير المخلوق محدود وناقص بما توفر لديه من علم وإدراك. وأما تقدير الخالق فهو تقدير كامل لا نقص فيه لأن الله هو العليم بكل شئ والقادر على كل شئ. ومن تقدير الله التقدير الكوني والتقدير الشرعي.ولما كان تقدير الله هو الحق وهو الميزان الذي لا نقص فيه, وافق تقديره الكوني سبحانه الميزان الكومي وتقديره الشرعي الميزان الشرعي.

ولهذا فإن الحساب الصحيح يبنى على التقدير العددي والوزن وهو ما نعرفه اليوم بالتساوي. فقد سماه الخوارزمي رحمه الله تعالى بالجبر والمقابلة بحيث يبنى الحساب على التساوي بين المتغيرات لجبر ما اختل من الميزان, عليه يمكن حساب ما جهل منها. فقد قال ابن تيمية رحمه الله عن هذا:
وأما حساب الفرائض ومعرفة أصول المسائل وتصحيحها والمناسخات وقسمة التركات، وهذا الثاني كله علم معقول يُعلم بالعقل، كسائر حساب المعاملات وغير ذلك من الأنواع التي يحتاج إليها الناس [.] ثم قد ذكروا حساب المجهول الملقب بحساب الجبر والمقابلة في ذلك، وهو علم قديم [.] أول من عرف أنه أدخله فيها محمد بن موسى الخوارزمي. وبعض الناس يذكر عن علي بن أبي طالب أنه تكلم فيه، وأنه تعلم ذلك من يهودي، وهذا كذب على علي
[هـ].
{\footnotesize (مجموع الفتاوى 9/214)}.
ويقول اين تيمية فيه أيضا:
وكذلك كثير من متأخري أصحابنا يشتغلون وقت بطالتهم بعلم الفرائض والحساب والجبر والمقابلة والهندسة ونحو ذلك، لأن فيه تفريحاً للنفس، وهو علم صحيح لا يدخل فيه غلط
[هـ].
{\footnotesize (مجموع الفتاوى ج9/ص129)}.
ويقول الشيخ الفوزان حفظه الله:
أما ما كان من علم الحساب الذي ينتفع به في معرفة المواقيت ومعرفة القبلة فهذا مباح وهو ما يسمى علم التسيير [.] وهو معرفة الحساب الذي به ينتفع الناس في مواقيت عباداتهم ومعاملاتهم ومواقيت زروعهم وغرس أشجارهم ويستدلون به على القبلة فهذا مباح وقد يجب تعلمه إذا كان يعين على أداء العبادات في مواقيتها [هـ].
ويقول الشيخ ابن باز رحمه الله:
ولكن على الأمة أن تتعلم أيضًا ما ينفعها في دنياها: من الصناعات النافعة، ومن الاستعانة بها على قتال الأعداء وجهاد الأعداء، فيتعلم شؤون الزراعة، ويتعلم شؤون استخراج خزائن الأرض: من البترول والمعادن وغير ذلك؛ حتى تستغني عن أعداء الله، وتستخرج من بطون الأرض ومن خزائن الأرض ما ينفعها.
[هـ].

\section{أمثلة حسابية من القرآن}
\subsection{مكوث أهل الكهف}
يقول جل جلاله عن مدة مكوث أهل الكهف في سورة الكهف:
\quranayah*[18][25]{\footnotesize \surahname*[18]}. فقد يسأل السائل لماذا جاء النص مع "وازدادوا تسعا" وهذا فيه الحكمة البالغة منه سبحانه. فمن المعلوم أن الرسول ﷺ كان مخاطبا لأهل الكتاب وأن أهل الكتاب يستعملون السنوات الشمسية وأما المسلمين فهم يستعملون السنوات القمرية. فالمطلوب هنا حساب المدة بعدد السنوات الشمسية ليوافق ذلك حساب أهل الكتاب وتحويل ذلك إلى عدد السنوات القمرية الذي يستعمله المسلمين في تاريخهم. وبالحساب الصحيح يتين الأتي:

عدد الأيام في السنة القمرية = 354 يوم

عدد الأيام في السنة الشمسية = 365 يوم

وبهذا يكون الفارق في عدد الأيام بينهما = 11 يوم

وعليه يكون في المئة سنة شمسية 36500 يوم وفي المئة سنة قمرية 35400 يوم.
الفارق هو  1100 يوم. بتقسيم هذا الفارق على 354 نجد أن الفارق هو 3 سنوات قمرية. وبهذا يعلم أن لكل 100 سنة شمسية توجد 103 سنة قمرية تقريبا. وعليه يكون في كل 300 سنة شمسية هناك 309 سنة قمرية. ويكون الفارق هو فقط تسعة سنوات ولهذا جاء لفظ "وازدادوا تسعا" للبيان فهي 300 سنة شمسية بالنسبة لأهل الكتاب وزيادة عليها 9 سنوات لتوافق بذلك 309 سنة قمرية بالنسبة للمسلمين. وهذا ما يعرف في علم الحساب بوحدة قياس الأعداد. فبالحساب يمكن تحويل وحدة قياس من نوع إلى أخر. وفي هذا المثال كانت وحدة القياس هي الزمن وبه علم التحويل من القياس الشمسي إلى القياس القمري والخلاف بينهما لا يعني التعارض بل لكل وحدة قياس حسابها الخاص. وهذا فيه بيان حكمة الله وعلمه سبحانه وأن كلامه هو الحق لهذا جاء بعد هذه الأية قوله تعالى:
\quranayah*[18][26]{\footnotesize \surahname*[18]}.
ويقول ابن كثير في تفسير الآية التي ذكر فيها عدد السنوات:
هذا خبر من الله تعالى لرسوله ﷺ بمقدار ما لبث أصحاب الكهف في كهفهم، منذ أرقدهم الله إلى أن بعثهم وأعثر عليهم أهل ذلك الزمان، وأنه كان مقداره ثلاثمائة سنة وتسع سنين بالهلالية، وهي ثلاثمائة سنة بالشمسية، فإن تفاوت ما بين كل مائة سنة بالقمرية إلى الشمسية ثلاث سنين؛ فلهذا قال بعد الثلاثمائة : (وازدادوا تسعا) [هـ].

\subsection{مكوث الوحي من عيسى عليه السلام إلى محمد ﷺ}
وفي مثال أخر يشه المثال السابق في تفسير قوله تعالى:
\quranayah*[5][19]{\footnotesize \surahname*[5]}. يقول ابن كثير في تفسيره عن هذا: (على فترة من الرسل) أي : بعد مدة متطاولة ما بين إرساله وعيسى ابن مريم.
وهو أنه ستمائة سنة. ومنهم من يقول : ستمائة وعشرون سنة. ولا منافاة بينهما، فإن القائل الأول أراد ستمائة سنة شمسية، والآخر أراد قمرية، وبين كل مائة سنة شمسية وبين القمرية نحو من ثلاث سنين; ولهذا قال تعالى في قصة أصحاب الكهف:
\quranayah*[18][25]{\footnotesize \surahname*[18]}. أي : قمرية ، لتكميل الثلاثمائة الشمسية التي كانت معلومة لأهل الكتاب
[هـ]. ولهذا فقد فهم المفسرين رحمهم الله بالقران الفرق بين عدد السنوات الشمسية والقمرية كما تقدم. وفيه أيضا عناية السلف رحمهم الله بالحساب وحساب الزمن وتحويله من وحدة قياس إلى أخرى.

\subsection{عدد ساعات اليوم والليلة}

عَنْ جَابِرِ بْنِ عَبْدِ اللَّهِ عَنْ رَسُولِ اللَّهِ ﷺ قَالَ: يَوْمُ الْجُمُعَةِ اثْنَتَا عَشْرَةَ سَاعَةً، لَا يُوجَدُ فِيهَا عَبْدٌ مُسْلِمٌ يَسْأَلُ اللَّهَ شَيْئًا إِلَّا آتَاهُ إِيَّاهُ فَالْتَمِسُوهَا آخِرَ سَاعَةٍ بَعْدَ الْعَصْرِ.{\footnotesize (أبو داود(1048), والنسائي(1389) وصححه الألباني)}. وفيه أن النبي ﷺ علم أن عدد ساعات اليوم بالمتوسط هو 12 ساعة. فإن كان لعدد ساعات الليلة مثل ذلك كانت عدد ساعات اليوم والليلة معا 24 ساعة. وهذا ما أعتاد عليه الناس في زماننا من حساب عدد ساعات اليوم واليلة. فإن ساعات اليوم والليلة تطول وتقصر خلال العام وتتغير بتغير المكان ولكن بالإجمال فهي 24 ساعة. وهذا فيه دليل نبوته ﷺ ففي زمانه لم يكن هناك الساعات الدقيقة التي نعرفها اليوم.

وعدد ساعات اليوم والليل تضبط بالتاريخ الشمسي كما موضخ في %\ref{fig:Hours}.


وكما موضح أن الأماكن القريبة من القطبين يغيب فيها النهار أو الليل خلال 24 ساعة وبهذا يمضي اليوم كاملا ويتعذر ضبط أوقات الصلاة بالطريقة المعتادة. وقد رخص أهل العلم على جواز ضبط وقت الصلاة فيها كما اعتاد الناس في سائر أوقات السنة أو قياسا على غيرها من الأماكن. فعَنِ النَّوَّاسِ بْنِ سَمْعَانَ الْكِلاَبِيِّ، قَالَ ذَكَرَ رَسُولُ اللَّهِ صلى الله عليه وسلم الدَّجَّالَ فَقَالَ "إِنْ يَخْرُجْ وَأَنَا فِيكُمْ فَأَنَا حَجِيجُهُ دُونَكُمْ وَإِنْ يَخْرُجْ وَلَسْتُ فِيكُمْ فَامْرُؤٌ حَجِيجُ نَفْسِهِ وَاللَّهُ خَلِيفَتِي عَلَى كُلِّ مُسْلِمٍ فَمَنْ أَدْرَكَهُ مِنْكُمْ فَلْيَقْرَأْ عَلَيْهِ فَوَاتِحَ سُورَةِ الْكَهْفِ فَإِنَّهَا جِوَارُكُمْ مِنْ فِتْنَتِهِ" . قُلْنَا وَمَا لُبْثُهُ فِي الأَرْضِ قَالَ " أَرْبَعُونَ يَوْمًا يَوْمٌ كَسَنَةٍ وَيَوْمٌ كَشَهْرٍ وَيَوْمٌ كَجُمُعَةٍ وَسَائِرُ أَيَّامِهِ كَأَيَّامِكُمْ" . فَقُلْنَا يَا رَسُولَ اللَّهِ هَذَا الْيَوْمُ الَّذِي كَسَنَةٍ أَتَكْفِينَا فِيهِ صَلاَةُ يَوْمٍ وَلَيْلَةٍ قَالَ "لاَ اقْدُرُوا لَهُ قَدْرَهُ ثُمَّ يَنْزِلُ عِيسَى ابْنُ مَرْيَمَ عِنْدَ الْمَنَارَةِ الْبَيْضَاءِ شَرْقِيَّ دِمَشْقَ فَيُدْرِكُهُ عِنْدَ بَابِ لُدٍّ فَيَقْتُلُهُ {\footnotesize (صححه الألباني)}.
وفي هذا يقول الشيخ ابن باز رحمه الله:
الواجب على سكان هذه المناطق التي يطول فيها النهار أو الليل أن يصلوا الصلوات الخمس بالتقدير إذا لم يكن لديهم زوال ولا غروب لمدة أربع وعشرين ساعة، كما صح ذلك عن النبي ﷺ في حديث النواس بن سمعان، المخرج في صحيح مسلم في يوم الدجال الذي كسنة، سأل الصحابة رسول الله ﷺ عن ذلك، فقال: اقدروا له قدره. وهكذا حكم اليوم الثاني من أيام الدجال، وهو اليوم الذي كشهر، وهكذا اليوم الذي كأسبوع. أما المكان الذي يقصر فيه الليل ويطول فيه النهار أو العكس في أربع وعشرين ساعة فحكمه واضح: يصلون فيه كسائر الأيام، ولو قصر الليل جدا أو النهار؛ لعموم الأدلة، والله ولي التوفيق [هـ]. وهذا فيه أهمية الحساب وتسجيل بيانات أوقات الصلاة حتى يمكن تقديرها تقديرا صحيحا بقدر المستطاع إن تعذر معرفة ذلك من بغياب الليل أو النهار خلال 24 ساعة كما في فتنة المسيح الدجال.

\begin{figure}
    \centering
    \includegraphics[width=\textwidth] %,natwidth=512,natheight=288
    {Hours_of_daylight_vs_latitude_vs_day_of_year_with_tropical_and_polar_circles.png}
    \caption{عدد ساعات النهار بحسب خطوط العرض}
    \label{fig:Hours}
\end{figure}

\subsection{نسبية الوقت في القرآن}

قال تعالى في كتابه:
\quranayah*[22][47]{\footnotesize \surahname*[22]}.
ويقول ابن كثير في تفسيره: هو تعالى لا يعجل، فإن مقدار ألف سنة عند خلقه كيوم واحد عنده بالنسبة إلى حكمه. وهذا فيه الدليل على أن حساب الوقت نسبي ويختلف كما أخبر الله سبحانه وتعالى في أكثر من موضع. ومن ذلك قوله تعالى:
\quranayah*[32][5]{\footnotesize \surahname*[32]}.
وأيضا قوله تعالى:
\quranayah*[70][4]{\footnotesize \surahname*[70]}. ففي هذه الأيات الدليل الواضح على أن الوقت لا يجري بنفس السرعة فبين الله تعالى ذلك بالنسبة لوقتنا في قوله "مما تعدون".

ويمكن توضيح ذلك المعنى عن طريق حساب فارق الوقت بين الجنة والأرض. فإن حملنا معنى "عند ربك" أي في الجنة كما في قوله:
\quranayah*[3][169]{\footnotesize \surahname*[3]}, فبهذا يكون لكل يوم واحد في الجنة ألف سنة مما نعد في الأرض كما يلي:

\begin{center}
    1 يوم في الجنة = 1000 سنة في الأرض
\end{center}

نبدأ أولا بتحويل اليوم الواحد إلى ثواني. فإن علمنا أن اليوم به 12 ساعة بالمتوسط (اليوم والليلة معا 24 ساعة) كما في المثال السابق, وأن الساعة بها 60 دقيقة وأن الدقيقة بها 60 ثانية, عليه يكون اليوم الواحد به 43200 ثانية. ثانيا نحسب عدد الأيام في كل 1000 سنة. فإن إستخدمنا عدد الأيام في السنة القمرية يكون عدد الأيام الكلي في كل 1000 سنة قمرية 354000 يوم. وإبقاء الميزان وإستبدال وحدات القياس يكون الناتج:

\begin{center}
    43200 ثانية في الجنة = 354000 يوما في الأرض
\end{center}

فإن قسمنا عدد الثواني في الجنة على عدد الأيام يكون:

\begin{center}
    1 ثانية في الجنة = 1944.8 يوما في الأرض
\end{center}

وعليه بمرور ثانية واحدة في الجنة تمر علينا 8 أيام و4 ساعات و39 دقيقة و22 ثانية في الأرض مما نعد ونحسب. وهذا لا يجب أن يفهم أن الوقت في الجنة بطئ جدا بالنسبة لمن هو في الجنة ولكن فقط بالنسبة للأرض. فالوقت هو نفسه كوحدة قياس ولكن قياس نفس القيمة في نطاق مختلف لا يلزم التساوي وأنما كل قياس يرجع لنطاقه الذي قيس فيه.

وهذه الظاهرة تم إكتشافها حديثا في بداية القرن العشرين على يدي العالم الفيزيائي ألبيرت آنشتاين وتعرف بظاهرة التمدد الزمني وهي جزء من النظرية النسبية. وهي ظاهرة مثبتة ويمكن حسابها بدقة بإستخدام مفاهيم معروفة وأهمها ثبات سرعة الضوء في الفراغ. وهي ظاهرة مهمة جدا وتستخدم لموائمة أنظمة الإتصال وأنظمة تحديد الموقع مع الأقمار الصناعية. بدون الأخذ بعين الإعتبار ظاهرة التمدد الزمني كل هذه الأنظمة تتعطل.

ومن المثبت أيضا في النظرية النسبية أن مرور الوقت نسبي وهو يبطأ مع زيادة السرعة أو الجاذبية. ومن المعلوم أن الجاذبية تزداد مع زيادة كتلة الكواكب. وبهذا يعلم أن الوقت على القمر يمر أسرع بقليل بالنسبة للأرض بينما الوقت على الأرض يمر أسرع بالنسبة للشمس. وهذا الفارق يزيد من زيادة الفرق في الحجم. وعليه فإن مرور الوقت في الجنة أبطأ بكثير من الأرض دل على عظم الجنة بالنسبة للأرض. وهذا يتوافق مع قوله تعالى:
\quranayah*[3][133]{\footnotesize \surahname*[3]}.


\subsection{ظاهرة الغلاف الجوي}

\quranayah*[36][36-40]{\footnotesize \surahname*[36]}.


\subsection{ظاهرة تعاقب الليل والنهار}

\quranayah*[3][27]{\footnotesize \surahname*[3]}.

\quranayah*[24][44]{\footnotesize \surahname*[24]}.


\subsection{ظاهرة توسع الكون}

من أعظم بلايا هذا الزمان هو تصوير ووضع العلم في إيطار منفصل عن الايمان بل والاسوأ في ايطار انكار وجود الخالق وهذا والله من اعظم الضلال والجهل.

بينما في الحقيقة العلم هو الطريق للتأمل في آيات الله الكونية والتي جميعها تنادي بوجود الخالق وقدرته وعظمته. فكل ما نراه من تناسق في هذا الكون من ليل ونهار وشمس وقمر ومطر وشجر وحجر ودواب كلها من آيات الله الكونية.

ومن اعظم آيات الله الكونية أن الله عز وجل لم يجعل السماء ثابتة بل جعلها تتوسع فلو كانت ثابتة لقال الكثير ان هذا الكون ليس له بداية وهي ثابتة ازلا وبهذا ينكرون وجود الخالق. ولكن الله جل جلاله جعل السماء تتمدد ليكون هذا التمدد دليلا على ان اطراف السماء كلها جاءت من نقطة واحدة وهذا هو الدليل القاطع على بداية الكون.

الإقرار بأن هذا الكون له بداية كما تشير كل الأدلة والمفاهيم التي توصل لها البشر في القرن العشرين ومنها ما جاء في نظرية الإنفجار العظيم يبطل كل ما تم طرحه من أصحاب النظريات الإلحادية إذا يتعذر على شي له بداية أن يبدأ من لا شئ.

قال تعالى:
أَوَلَمْ يَرَ الَّذِينَ كَفَرُوا أَنَّ السَّمَاوَاتِ وَالْأَرْضَ كَانَتَا رَتْقًا فَفَتَقْنَاهُمَا ۖ وَجَعَلْنَا مِنَ الْمَاءِ كُلَّ شَيْءٍ حَيٍّ ۖ أَفَلَا يُؤْمِنُونَ [الأنبياء : 30]

فقد اتفق المفسرون على ان معنى هذه الآية ان السماوات والأرض كانتا ملتصقتين وهذا ما يتوافق مع فهمنا اليوم بناء على المشاهدة أن الكون بدأ من نقطة واحدة.

فقد جاء في تفسير القرطبي ان ابن عباس والحسن وعطاء والضحاك وقتادة قالوا في تفسير هذه الآية: يعني أنها كانت شيئا واحدا ملتزقتين ففصل الله بينهما بالهواء. وكذلك قال كعب : خلق الله السماوات والأرض بعضها على بعض ثم خلق ريحا بوسطها ففتحها بها ، وجعل السماوات سبعا والأرضين سبعا. وقول ثان قاله مجاهد والسدي وأبو صالح : كانت السماوات مؤتلفة طبقة واحدة ففتقها فجعلها سبع سماوات، وكذلك الأرضين كانت مرتتقة طبقة واحدة ففتقها فجعلها سبعا

وقال الطبري في تفسير هذه الآيات: أو لم ينظر هؤلاء الذي كفروا بالله بأبصار قلوبهم، فيروا بها، ويعلموا أن السماوات والأرض كانتا رَتْقا: يقول: ليس فيهما ثقب، بل كانتا ملتصقتين،

وقد جاء في تفسير ابن كثير:

ألم يروا ( أن السماوات والأرض كانتا رتقا ) أي: كان الجميع متصلا بعضه ببعض متلاصق متراكم، بعضه فوق بعض في ابتداء الأمر ، ففتق هذه من هذه . فجعل السماوات سبعا، والأرض سبعا، وفصل بين سماء الدنيا والأرض بالهواء، فأمطرت السماء وأنبتت الأرض; ولهذا قال: ( وجعلنا من الماء كل شيء حي أفلا يؤمنون ) أي : وهم يشاهدون المخلوقات تحدث شيئا فشيئا عيانا ، وذلك دليل على وجود الصانع الفاعل المختار القادر على ما يشاء

ويقول جل جلاله:
وَالسَّمَاءَ بَنَيْنَاهَا بِأَيْدٍ وَإِنَّا لَمُوسِعُونَ [الذاريات : 47]

وقد جاء في تفسير السعدي رحمه الله:
{ بِأَيْدٍ } أي: بقوة وقدرة عظيمة { وَإِنَّا لَمُوسِعُونَ } لأرجائها وأنحائها،


\subsection{ظاهرة المجال المغنطيسي}

\quranayah*[21][32]{\footnotesize \surahname*[21]}.


