\chapter{الميزان الكوني والميزان الشرعي}

\section{المقدمة}

علم الحساب من العلوم التي تدرك بالعقل والفطرة وقد يكون هو العلم الوحيد الذي يكاد لا يختلف عليه البشر بكافة أجناسهم وألوانهم وبالأخص لمن عرف هذا العلم وتمعن فيه صدقا. وذلك لأن الله جل جلاله خلق كل شئ بقدر معلوم ووضع الميزان الكوني فجعل هذا الكون موزونا ومتناسقا سبحانه. ومن فضله ومنه علي الناس أنه أرسل إليهم الرسل وأنزل الكتب بالحق والميزان الشرعي. ومن حكمته أنه سبحانه فطر الناس على فهم الميزانان وجعل لهم كل ما يحتاجونه من عقل وسمع وبصر. فجعل سبحانه آياته الكونية دليلا على الميزان الكوني وآياته الشرعية دليلا على الميزان الشرعي. وأرشد جل جلاله إلى التأمل في آياته الكونية لتعلم العدد والحساب وهذا لحكمته فالعدد والحساب يدرك بالعقل والفطرة ولهذا اكتفى سبحانه بالدلالة عليه. أما الميزان الشرعي فهو لا يدرك بالعقل والفطرة فقط وأنما يدرك بالوحي المنزل من عند الله تبارك وتعالى. والله جل جلاله تكفل بإقامة الميزان الكوني وأرسل الرسل وأنزل الكتب وفرض على الناس إقامة الميزان الشرعي. ولما كان الحساب هو الوسيلة لمعرفة الحقائق وضبطها والطريق لمعرفة الأسباب وربطها، وجب النظر والبحث فيه وتعلمه من آيات الله الكونية لفهمها لما في ذلك من مصالح دينية ودنيوية كما أرشد سبحانه في  كتابه العظيم.

\comment{
في قوله تعالى: \quranayah*[10][5]{\footnotesize \surahname*[10]}. وفي قوله تعالى:
\quranayah*[17][12]{\footnotesize \surahname*[17]}.
}

\section{الغاية من علم الحساب}

علم الحساب من الضروريات التي يحتاج إليها الناس في أمور دينهم ودنياهم. فعلم الحساب هو الوسيلة لتحقيق الغاية العظيمة التي أمر الله بها وهي إقامة الميزان والعدل. ولهذا كان البحث في علم الحساب من الأمور التي حث الله تعالى عليها في موضعين في كتابه. قال تعالى: \quranayah*[17][12]{\footnotesize \surahname*[17]}.
يقول السعدي رحمه الله في تفسيره:
(وَلِتَعْلَمُوا) بتوالي الليل والنهار واختلاف القمر (عَدَدَ السِّنِينَ وَالْحِسَابَ) فتبنون عليها ما تشاءون من مصالحكم. (وَكُلَّ شَيْءٍ فَصَّلْنَاهُ تَفْصِيلًا) أي: بينا الآيات وصرفناه لتتميز الأشياء ويستبين الحق من الباطل كما قال تعالى: (مَا فَرَّطْنَا فِي الْكِتَابِ مِنْ شَيْءٍ) \href{https://shamela.ws/book/42/1001#p1}{\faExternalLink} \cite{tafsir_Saadi}.
وقال تعالى: \quranayah*[10][5]{\footnotesize \surahname*[10]}.
يقول السعدي رحمه الله في تفسير هذه الأيات:
وفي هذه الآيات الحث والترغيب على التفكر في مخلوقات الله، والنظر فيها بعين الاعتبار، فإن بذلك تنفتح البصيرة، ويزداد الإيمان والعقل، وتقوى القريحة، وفي إهمال ذلك، تهاون بما أمر الله به، وإغلاق لزيادة الإيمان، وجمود للذهن والقريحة
\href{https://shamela.ws/book/42/729#p6}{\faExternalLink} \cite{tafsir_Saadi}.

وفي هذا الحث والترغيب في علم الحساب الحكمة البالغة من الله جل جلاله. ومن ذلك أن علم الحساب هو مفتاح جميع العلوم الكونية التي يمكن فيها العد والقياس والتي لا يمكن فهمها فهما صحيحا من دون الحساب الصحيح. فعلم الحساب به يفهم الميزان الكوني من آيات الله الكونية وبه يقام الميزان الشرعي باتباع آيات الله الشرعية. وهذا لأن علم الحساب ما هو إلا صورة من صور الميزان ولكن صورة معنوية وليست حسية، وبه تضبط المقادير وتعرف المجاهيل بعدة طرق منها ما هو سهل وبسيط ويحسب ذهنيا ومنها ما هو صعب ومعقد ويحسب كتابيا أو بإستخدام طرق حديثة. ومن الأمثلة لتطبيقات علم الحساب في اتباع آيات الله الشرعية كحساب أوقات الصلاة وعلم المواريث والزكاة والبيع والشراء وغيرها من المعاملات التي يحتاج إليها الناس.  ومن الأمثلة على تطبيقات علم الحساب في فهم آيات الله الكونية كمعرفة حركة الشمس والقمر وغيرها من الظواهر الطبيعية التي خلقها الله لما في ذلك من مصالح دينية مثل التفكر في عظمة الله والزيادة في الإيمان ومصالح دنيوية مثل حساب الوقت والمواسم.

وإن أفضل طريقة لفهم علم الحساب هي التأمل والتفكر في آيات الله الكونية وهي الظواهر الطبيعية التي خلقها الله وجعل لها الميزان الكوني لفهمها وحسابها. فهي المرجع لنا لتعلم الحساب وللتحقق من صحته. وهذا النهج هو نهج القرآن وهو أفضل الطرق وأحسنها. ويمكن أيضا دراسة علم الحساب مجردا من أي تطبيقات وهذا نهج معروف. ولكن الجمع بين العلوم الطبيعية كعلم الفيزياء والحساب لمحاولة محاكاة الظواهر الطبيعية هي الطريق الأمثل لتطوير علم الحساب وهذا معروف لأهل هذا العلم. وبهذا يكون الميزان الكوني طريقا لتعلم الحساب الصحيح ومن ثم يكون الحساب الصحيح وسيلة لإقامة الميزان الشرعي الذي أمر الله به.

\section{الميزان الكوني}

جعل الله جل جلاله الميزان في آياته الكونية لحكمته وعدله سبحانه ومن ذلك أنه جعل قيام الكون كله بالقسط أي بالعدل الظاهر \href{https://shamela.ws/book/1736/428#p5}{\faExternalLink} \cite{askaari_Furuq}.\footnote{جاء في معجم الفروق اللغوية لأبي هلال العسكري \comment{في الفرق بين القسط والعدل} أن القسط هو العدل البين الظاهر ومنه سمي المكيال قسطا والميزان قسطا لأنه يصور لك العدل في الوزن حتى تراه ظاهرا وقد يكون من العدل ما يخفى\comment{ ولهذا قلنا إن القسط هو النصيب الذي بينت وجوهه وتقسط القوم الشيء تقاسموا بالقسط}.} فالميزان الكوني أمره عظيم لأن الله جل جلاله شهد به لنفسه ووصف نفسه به وفيه شأن الله وتدبيره لهذا الكون، فعَنْ أَبِي هُرَيْرَةَ أَنَّ رَسُولَ اللَّهِ ﷺ قَالَ: "يَدُ اللَّهِ مَلْأَى لَا يَغِيضُهَا نَفَقَةٌ، سَحَّاءُ اللَّيْلَ وَالنَّهَارَ"، وَقَالَ: "أَرَأَيْتُمْ مَا أَنْفَقَ مُنْذُ خَلَقَ السَّمَاوَاتِ وَالْأَرْضَ؛ فَإِنَّهُ لَمْ يَغِضْ مَا فِي يَدِهِ"، وَقَالَ: "عَرْشُهُ عَلَى الْمَاءِ، وَبِيَدِهِ الْأُخْرَى الْمِيزَانُ يَخْفِضُ وَيَرْفَعُ" \href{https://shamela.ws/book/1284/4620#p2}{\faExternalLink} \cite{bukhari}.\footnote{صحيح البخاري: 7406.} وهذا فيه أن الميزان الكوني بيده سبحانه فهو قائم عليه بالقسط لا تأخذه في ذلك سنة ولا نوم، ولهذا كانت آية الكرسي أعظم آية في كتاب الله فقد قال سبحانه:
\quranayah*[2][255][1-12] {\footnotesize (\surahname*[2])}. وَعَن أبي مُوسَى قَالَ: قَامَ فِينَا رَسُولُ اللَّهِ ﷺ بِخَمْسِ كَلِمَاتٍ فَقَالَ:  "إِنَّ اللَّهَ عَزَّ وَجَلَّ لَا يَنَامُ وَلَا يَنْبَغِي لَهُ أَنْ يَنَامَ. يَخْفِضُ الْقِسْطَ وَيَرْفَعُهُ. يُرْفَعُ إِلَيْهِ عَمَلُ اللَّيْلِ قَبْلَ عَمَلِ النَّهَارِ. وَعَمَلُ النَّهَارِ قَبْلَ عَمَلِ اللَّيْلِ. حِجَابُهُ النُّورُ.  لَوْ كَشَفَهُ لَأَحْرَقَتْ سُبُحَاتُ وَجْهِهِ مَا انْتَهَى إِلَيْهِ بَصَرُهُ مِنْ خَلْقِهِ" \href{https://shamela.ws/book/1727/393#p5}{\faExternalLink} \cite{muslim}.\footnote{صحيح مسلم: 179، وصححه الألباني.} وهذا فيه أن الله قائم على الميزان الكوني بنفسه بالعدل الظاهر ولذلك وصفه الله جل جلاله ورسوله ﷺ بالقسط.

ومن أعظم ذلك أن الله جعل قيامه على الميزان الكوني بالقسط في أعظم شهادة في كتابه الكريم فقال جل في علاه: \quranayah*[3][18]{\footnotesize \surahname*[3]}. وقال شيخ الإسلام ابن تيمية: (قائما بالقسط) أي: متكلما بالعدل مخبرا به آمرا به: كان هذا تحقيقا لكون الشهادة شهادة عدل وقسط وهي أعدل من كل شهادة كما أن الشرك أظلم من كل ظلم وهذه الشهادة أعظم الشهادات [.] وقيامه بالقسط يتضمن أنه يقول الصدق ويعمل بالعدل كما قال: (وَتَمَّتْ كَلِمَةُ رَبِّكَ صِدْقًا وَعَدْلًا) وقال هود: (إنَّ رَبِّي عَلَى صِرَاطٍ مُسْتَقِيمٍ) فأخبر أن الله على صراط مستقيم وهو العدل الذي لا عوج فيه [.] وقال: (هَلْ يَسْتَوِي هُوَ وَمَنْ يَأْمُرُ بِالْعَدْلِ وَهُوَ عَلَى صِرَاطٍ مُسْتَقِيمٍ) وهو مثل ضربه الله لنفسه ولما يشرك به من الأوثان كما ذكر ذلك في قوله: (قُلْ هَلْ مِنْ شُرَكَائِكُمْ مَنْ يَهْدِي إلَى الْحَقِّ قُلِ اللَّهُ يَهْدِي لِلْحَقِّ) الآية. وقال: (أَفَمَنْ يَخْلُقُ كَمَنْ لَا يَخْلُقُ) الآيات  [.] ولهذا أمرنا الله سبحانه أن نسأله أن يهدينا الصراط المستقيم; صراط الذين أنعم عليهم: من النبيين والصديقين والشهداء والصالحين وصراطهم هو العدل والميزان; ليقوم الناس بالقسط والصراط المستقيم هو العمل بطاعته وترك معاصيه فالمعاصي كلها ظلم مناقض للعدل مخالف للقيام بالقسط والعدل. والله سبحانه أعلم \href{https://shamela.ws/book/7289/7239#p1}{\faExternalLink} \cite{ibnTaimia_Majmoo}.\footnote{مجموع الفتاوى 14/176.}

وجاء في تفسير القرطبي عن الكلبي وأورده أيضا شيخ الإسلام ابن تيمية في سبب نزول هذه الآية أنه لما ظهر رسول الله ﷺ بالمدينة قدم عليه حبران من أحبار أهل الشام; فلما أبصرا المدينة، قال أحدهما لصاحبه: ما أشبه هذه المدينة بصفة مدينة النبي الذي يخرج في آخر الزمان. فلما دخلا على النبي ﷺ عرفاه بالصفة والنعت، فقالا له: أنت محمد؟ قال: نعم. قالا: وأنت أحمد؟ قال: نعم. قالا: نسألك عن شهادة، فإن أنت أخبرتنا بها آمنا بك وصدقناك. فقال لهما رسول الله ﷺ: سلاني. فقالا: أخبرنا عن أعظم شهادة في كتاب الله. فأنزل الله تعالى على نبيه ﷺ: \quranayah*[3][18]{\footnotesize \surahname*[3]}، فأسلم الرجلان وصدقا برسول الله ﷺ \href{https://shamela.ws/book/20855/1383#p4}{\faExternalLink} \cite{tafsir_Qurtubi}.

\comment{
وبهذا يتبين أن الميزان الكوني بيد الله سبحانه وتعالى وهو قائم عليه بالقسط يخفضه ويرفعه وهو الحي القيوم ولا ينام ولا ينبغي له أن ينام وهذا لازم لوجود الكون وصلاحه فهو مدبره سبحانه وهو مالكه. وفيه أيضا أن الله عز وجل كامل في صفاته لا يلحقه نقص وهذا لازم لإقامة الوجود إذ يتعذر على غيره إقامة الميزان الكوني كما دلت على ذلك الآيات القرآنية والأحاديث. وقد جاء في تفسير ابن كثير أن قوله: (لا تأخذه سنة ولا نوم) أي: لا يعتريه نقص ولا غفلة ولا ذهول عن خلقه بل هو قائم على كل نفس بما كسبت شهيد على كل شيء لا يغيب عنه شيء ولا يخفى عليه خافية، ومن تمام القيومية أنه لا يعتريه سنة ولا نوم، فقوله: (لا تأخذه) أي: لا تغلبه سنة وهي الوسن والنعاس ولهذا قال: (ولا نوم) لأنه أقوى من السنة، [.] وقوله: (ولا يئوده حفظهما) أي: لا يثقله ولا يكرثه حفظ السماوات والأرض ومن فيهما ومن بينهما، بل ذلك سهل عليه يسير لديه وهو القائم على كل نفس بما كسبت، الرقيب على جميع الأشياء، فلا يعزب عنه شيء ولا يغيب عنه شيء والأشياء كلها حقيرة بين يديه متواضعة ذليلة صغيرة بالنسبة إليه، محتاجة فقيرة وهو الغني الحميد الفعال لما يريد، الذي لا يسأل عما يفعل وهم يسألون. وهو القاهر لكل شيء الحسيب على كل شيء الرقيب العلي العظيم لا إله غيره ولا رب سواه [هـ].
}

ولا يعلم على وجه التحديد متى خلق الله جل جلاله هذا الميزان الكوني ولكن يعلم بالضرورة أن هذا الميزان الكوني كان موجودا عندما رفع الله السموات حيث وضعه جل جلاله في يده. وكل ذلك كان بعد خلق القلم واللوح والعرش والماء، فقد جاء عن النبي ﷺ أنه قال: إنَّ أولَ ما خلق اللهُ القلمُ، فقال لهُ: اكتبْ، قال: ربِّ وماذا أكتبُ؟ قال: اكتُبْ مقاديرَ كلِّ شيءٍ حتى تقومَ الساعةُ. وفي رواية: اكتُبْ القدَرَ، ما كان و ما هو كائِنٌ إلى الأبَدِ. ومن مات على غيرِ هذا فليسَ مِني \href{https://shamela.ws/book/117359/3975#p2}{\faExternalLink} \cite{SunanAbiDawood}.\footnote{أبي داود: 4700 واللفظ له، أحمد: 22705، الترمذي: 3319، وصححه الألباني في صحيح الجامع.} وهذا فيه أن الله خلق القلم أولا ثم خلق اللوح المحفوظ وأمر جل جلاله القلم أن يكتب على اللوح المحفوظ المقادير كلها إلى قيام الساعة. وكل هذا كان قبل خلق السموات والأرض بخمسين ألف سنة وكان عرشه جل جلاله على الماء كما صح ذلك عن النبي ﷺ أنه قال:  كَتَبَ اللَّهُ مَقَادِيرَ الخَلَائِقِ قَبْلَ أَنْ يَخْلُقَ السَّمَوَاتِ وَالأرْضَ بِخَمْسِينَ أَلْفَ سَنَةٍ، وَعَرْشُهُ علَى المَاءِ \href{https://shamela.ws/book/1727/6683#p2}{\faExternalLink} \cite{muslim}.\footnote{صحيح مسلم: 2653، وصححه الألباني في شرح الطحاوية.} وسماها الله جل جلاله المقادير لأنه قدرها وعرف قدرها بعلمه وحكمته وعدله ورحمته سبحانه حتى يقوم بنفسه على ذلك في الميزان الكوني الذي وضعه سبحانه في يده بعد رفع السماء كما في قوله تعالى: 
\quranayah*[55][7]{\footnotesize \surahname*[55]}. فيكون المعنى هنا الميزان الذي وضعه الله في يده بعد رفع السموات وهو الميزان الكوني ويحتمل أيضا الميزان الذي وضعه للمكلفين من خلقه من الإنس والجن وهو الميزان الشرعي كما في باقي الآيات الكريمة في قوله تعالى: 
\quranayah*[55][8-9]{\footnotesize \surahname*[55]}.

وكل هذا فيه أن الله جل جلاله قد جعل الميزان الكوني سببا لإستقامة السموات والأرض وصلاحهما كما في قوله تعالى:
\quranayah*[23][71]{\footnotesize \surahname*[23]}. يقول السعدي في تفسيره:
ووجه ذلك أن أهواءهم متعلقة بالظلم والكفر والفساد من الأخلاق والأعمال، فلو اتبع الحق أهواءهم لفسدت السماوات والأرض، لفساد التصرف والتدبير المبني على الظلم وعدم العدل، فالسماوات والأرض ما استقامتا إلا بالحق والعدل \href{https://shamela.ws/book/42/1255#p12}{\faExternalLink} \cite{tafsir_Saadi}. وبهذا يعلم أن السموات والأرض تفسد بالظلم وهذا لا يكون لأن الله أقامهما بالميزان الكوني وهو قائم بنفسه سبحانه على ذلك بالقسط. ولا يعلم صورة هذا الميزان ولا وصفه إلا بما وصفه الله جل جلاله ورسوله ﷺ به، ومن ذلك أن الله وضعه في يده ويخفض به ويرفع وهو قائم عليه بنفسه كما قال الرسول ﷺ: ما منْ قلبٍ إلا وهوَ معلقٌ بينَ إصبعينَ منْ أصابعِ الرحمنِ، إنْ شاءَ أقامَهُ، و إنْ شاءَ أزاغَهُ، والميزانُ بيدَ الرحمنَ، يرفعُ أقوامًا، ويخفضُ آخرينَ، إلى يومِ القيامةِ \href{https://shamela.ws/book/21659/10685#p1}{\faExternalLink} \cite{jamaaSagheer}.\footnote{الجامع الصغير وزيادته: 10685، وصححه الألباني في صحيح الجامع.} وهذا فيه أن هذا الميزان هو الميزان الكوني التابع لإرادة الله الكونية وأن الله قائم عليه بالقسط إلى قيام الساعة وهو غير الميزان الشرعي التابع لإرادة الله الشرعية الذي يوضع يوم القيامة لحساب المكلفين كما سيأتي بيانه. فالميزان الكوني فيه شأن الله وتدبيره لهذا الكون الذي قال فيه جل في علاه: \quranayah*[55][29]{\footnotesize \surahname*[55]}. وبهذا يكون المراد بالميزان الذي وضع بعد رفع السماء هو الميزان الكوني الذي فيه شأن الله وتدبيره للكون كما قال تعالى: 
\quranayah*[55][7]{\footnotesize \surahname*[55]}. والمراد بالميزان الذي أمر الله به هو الميزان الشرعي الذي فيه شأن العباد وتدبيرهم لأنفسهم بما كلفهم الله به كما قال تعالى:
\quranayah*[55][8-9]{\footnotesize \surahname*[55]}.

والميزان الكوني قد تكفل به سبحانه عدلا وتدبيرا وهو دليل على عظمته وقدرته إذ يتعذر على غيره العيش من دونه فضلا عن إقامته، ومن ذلك أن الله عدل في قضاءه ومشيئته وحليم في تدبيره والدليل على هذا قوله تعالى:
\quranayah*[35][41]{\footnotesize \surahname*[35]}. فهذا دليل على وحدانيته سبحانه بالملك والتدبير، فالأدلة العقلية والشرعية دلت على وحدانيته سبحانه ومن ذلك أنه جعل هذا الكون موزونا ومتناسقا بإرادته الكونية لا يشاركه في ذلك أحد كما في قوله تعالى:
\quranayah*[21][22]{\footnotesize \surahname*[21]}. وقد جاء في تفسير السعدي في شرح هذه الآيات أن العالم العلوي والسفلي، على ما يرى، في أكمل ما يكون من الصلاح والانتظام، الذي ما فيه خلل ولا عيب، ولا ممانعة، ولا معارضة، فدل ذلك، على أن مدبره واحد، وربه واحد، وإلهه واحد، فلو كان له مدبران وربان أو أكثر من ذلك، لاختل نظامه، وتقوضت أركانه فإنهما يتمانعان ويتعارضان، وإذا أراد أحدهما تدبير شيء، وأراد الآخر عدمه، فإنه محال وجود مرادهما معا، ووجود مراد أحدهما دون الآخر، يدل على عجز الآخر، وعدم اقتداره واتفاقهما على مراد واحد في جميع الأمور، غير ممكن، فإذًا يتعين أن القاهر الذي يوجد مراده وحده، من غير ممانع ولا مدافع، هو الله الواحد القهار، ولهذا ذكر الله دليل التمانع في قوله:
\quranayah*[23][91]{\footnotesize \surahname*[23]} \cite{tafsir_Saadi}.

\section{الميزان الشرعي}

ومن حكمته سبحانه وتعالى أنه تكفل بإقامة الميزان الكوني وفرض على المكلفين من الجن والإنس إقامة الميزان الشرعي كما في قوله:
\quranayah*[55][7-9]{\footnotesize \surahname*[55]}. وقد جاء في تفسير السعدي أن معنى ووضع الله الميزان أي: العدل بين العباد، في الأقوال والأفعال، وليس المراد به الميزان المعروف وحده، بل هو كما ذكرنا، يدخل فيه الميزان المعروف، والمكيال الذي تكال به الأشياء والمقادير، والمساحات التي تضبط بها المجهولات، والحقائق التي يفصل بها بين المخلوقات، ويقام بها العدل بينهم \cite{tafsir_Saadi}. وهذا بالتأكيد يشمل علم الحساب والذي به تضبط المقادير وتحسب المجهولات فهو أيضا صورة من صور الميزان. وكل هذا فيه أن الله عز وجل فرض على المكلفين من الإنس والجن إقامة الميزان الشرعي. وفيه أيضا أن الله عز وجل جعل العدل في آياته الشرعية كما في آياته الكونية وهذا دليل على حكمته وكمال عدله سبحانه.

وبهذا يتبين أن المقصود والمراد من إقامة الميزان بالمجمل هي إقامة الميزان الشرعي الذي كلفنا الله جل جلاله به، وهو العدل في جميع الأقوال والأفعال، وهو الإستقامة على دين الله جل جلاله، وهذا يشمل جميع العبادات التي يحبها الله ويرضاها والتي أمر الله عباده بها عن طريق الرسل والكتب ومن ذلك إقامة الميزان والكيل بالقسط والصدق في القول والعمل ومنه بلا شك الحساب الصحيح. وبهذا يتبين أن الميزان الشرعي، حاله كحال الحق، هو من الأمانة في قوله تعالى:
\quranayah*[33][72]{\footnotesize \surahname*[33]}. وفي تفسير ابن كثير: قال العوفي عن ابن عباس يعني بالأمانة الطاعة [.] وقال قتادة الأمانة الدين والفرائض والحدود [هـ].

والميزان الشرعي هو أيضا من الميثاق الذي ذكره الله في قوله:
\quranayah*[5][7]{\footnotesize \surahname*[5]}. يقول السعدي في تفسيره رحمه الله:
و (مِيثَاقَهُْ) أي: واذكروا ميثاقه (الَّذِي وَاثَقَكُمْ بِهِ ْ) أي: عهده الذي أخذه عليكم. وليس المراد بذلك أنهم لفظوا ونطقوا بالعهد والميثاق، وإنما المراد بذلك أنهم بإيمانهم بالله ورسوله قد التزموا طاعتهما، ولهذا قال: (إِذْ قُلْتُمْ سَمِعْنَا وَأَطَعْنَا ْ) أي: سمعنا ما دعوتنا به من آياتك القرآنية والكونية، سمع فهم وإذعان وانقياد. وأطعنا ما أمرتنا به بالامتثال، وما نهيتنا عنه بالاجتناب. وهذا شامل لجميع شرائع الدين الظاهرة والباطنة. وأن المؤمنين يذكرون في ذلك عهد الله وميثاقه عليهم، وتكون منهم على بال، ويحرصون على أداء ما أُمِرُوا به كاملا غير ناقص. (وَاتَّقُوا اللَّهَ ْ) في جميع أحوالكم (إِنَّ اللَّهَ عَلِيمٌ بِذَاتِ الصُّدُورِ ْ) أي: بما تنطوي عليه من الأفكار والأسرار والخواطر. فاحذروا أن يطلع من قلوبكم على أمر لا يرضاه، أو يصدر منكم ما يكرهه، واعمروا قلوبكم بمعرفته ومحبته والنصح لعباده. فإنكم -إن كنتم كذلك- غفر لكم السيئات، وضاعف لكم الحسنات، لعلمه بصلاح قلوبكم \cite{tafsir_Saadi}.

ولهذا فإن الميزان الشرعي يعتبر من الأمانة والميثاق الذي واثق الله به المؤمنين وأمرهم به ليجزى كل نفس بما كسبت. وبهذا يتبين أن الميزان المخلوق الذي يوضع بعد الصراط يوم القيامة إنما هو الميزان الشرعي وهذا من عدل الله إذ جعل الميزان الذي يوزن به الناس يوم القيامة هو الميزان الشرعي الذي كلفهم به. وقد صح عن الرسول ﷺ أن هذا الميزان يوضع في صورة مخلوق بعد الصراط فعن أَنَسِ بْنِ مَالِكٍ قال سألتُ النَّبيَّ صلَّى اللَّهُ عليْهِ وسلَّمَ أن يشفعَ لي يومَ القيامةِ فقالَ: أنا فاعِلٌ، قلتُ: يا رسولَ اللَّهِ فأينَ أطلبُكَ، قالَ: اطلُبني أوَّلَ ما تطلُبُني على الصِّراطِ. قلتُ: فإن لم ألقَكَ على الصِّراطِ، قالَ: فاطلُبني عندَ الميزانِ. قلتُ: فإن لم ألقَكَ عندَ الميزانِ، قالَ: فاطلُبني عندَ الحوضِ فإنِّي لا أخطئُ هذِهِ الثَّلاثَ المواطنَ {\footnotesize (صحيح الترمذي وصححه الألباني)}. وهذا فيه أن الميزان الشرعي يوضع في صورته بعد الصراط مباشرة وقبل الحوض. وقد صح عن أحد أصحاب النبي ﷺ أنهم رأى هذا الميزان الشرعي في منامه كما جاء عن النبي ﷺ أنه ذات يوم قال لأصحابه: من رأَى منكُم رؤْيا؟ فقال رجلٌ: أنا، رأيتُ كأنَّ ميزانًا نزلَ من السَّماءِ فَوُزِنْتَ أنتَ وأبو بكرٍ فَرَجِحْتَ أنتَ بأبيِ بكرٍ، ووُزِنَ عُمَرُ وأبو بكرٍ فَرُجِحَ أبو بكرٍ، ووُزِنَ عُمَرُ وعُثْمانُ فَرُجِحَ عُمَرُ، ثُمَّ رُفِعَ الميزانُ. فَرأَيْنا الكراهيةَ في وجهِ رسولِ اللهِ ﷺ {\footnotesize (صحيح أبي داود، وصححه الألباني)}. وهذا فيه بيان الميزان الشرعي الذي رجح بحسب من فضل الله به النبي ﷺ وأصحابه الكرام رضوان الله عليهم.

والأدلة في وصف الميزان الشرعي ووصف حال المكلفين وأعمالهم وهم يوزنون عليه كثيرة ومنها قوله تعالى:
\quranayah*[7][8-9]{\footnotesize \surahname*[7]}. وقال تعالى:
\quranayah*[21][47]{\footnotesize \surahname*[21]}. وقال تعالى:
\quranayah*[23][102-103]{\footnotesize \surahname*[23]}. وقال تعالى:
\quranayah*[101][6-9]{\footnotesize \surahname*[101]}. ومن سوء حال الكافرين بآيات الله ولقائه أنهم يوم الحساب تحبط أعمالهم فلا يتعدون الصراط فيلقون في جهنم قبل أن يدركوا الميزان الشرعي كما قال تعالى: \quranayah*[18][105]{\footnotesize \surahname*[18]}. فكل ذلك فيه وصف حال وأحوال الناس وأعمالهم قبل وعند وبعد الميزان الشرعي الذي كلف الله جل جلاله المكلفين به من الجن والإنس فنسأل الله العفو والعافية. اللهم فاعف عنا وارحمنا وتجاوز عن سيئاتنا. اللهم ارحمنا فأنت ارحم الراحمين.

ولهذا فإن هذا الميزان الشرعي موافق لما شرعه الله تعالى وقد وصف النبي ﷺ وزن الأعمال الصالحة في هذا الميزان الشرعي ومن ذلك قوله: 
ما مِن شيءٍ يوضَعُ في الميزانِ أثقلُ من حُسنِ الخلقِ، وإنَّ صاحبَ حُسنِ الخلقِ ليبلُغُ بِهِ درجةَ صاحبِ الصَّومِ والصَّلاةِ {\footnotesize (صحيح الترمذي، وصححه الألباني)}. وفي رواية أخرى عن أبو الدرداء أن النبي  ﷺ قال:
من أُعطِيَ حظَّه من الرِّفقِ فقد أُعطِيَ حظَّه من الخيرِ ومن حُرِمَ حظُّه من الرِّفقِ ؛ فقد حُرِمَ حظُّه من الخيرِ. أثقلٌ شيءٍ في ميزانِ المؤمنِ يومَ القيامةِ حُسنُ الخُلُقِ، وإنَّ اللهَ لَيبغضُ الفاحشَ البذِيءَ {\footnotesize (صحيح الأدب المفرد، وصححه الألباني)}. وقال ﷺ أيضا: 
كَلِمَتانِ خَفِيفَتانِ علَى اللِّسانِ، ثَقِيلَتانِ في المِيزانِ، حَبِيبَتانِ إلى الرَّحْمَنِ، سُبْحانَ اللَّهِ وبِحَمْدِهِ، سُبْحانَ اللَّهِ العَظِيمِ {\footnotesize (صحيح البخاري، وصححه الألباني)}. وقال أيضا:
الطُّهُورُ شَطْرُ الإيمانِ، والْحَمْدُ لِلَّهِ تَمْلأُ المِيزانَ، وسُبْحانَ اللهِ والْحَمْدُ لِلَّهِ تَمْلَآنِ ما بيْنَ السَّمَواتِ والأرْضِ{\footnotesize (صحيح مسلم)}.  وقد جاء أيضا أن النبي ﷺ قال: بَخٍ بَخٍ - وأشار بيدِه بخَمْسٍ - ما أثقَلَهنَّ في الميزانِ سُبحانَ اللهِ والحمدُ للهِ ولا إلهَ إلَّا اللهُ واللهُ أكبَرُ والولَدُ الصَّالحُ يُتوفَّى للمرءِ المُسلِمِ فيحتسِبُه {\footnotesize (صحيح ابن حبان، والسلسلة الصحيحة للألباني)}. وقد جاء أيضا أن النبيُّ ﷺ أمرَ عبدَ اللهِ بنَ مسعودٍ أن يصعدَ شجرةً فيأتِيَهُ منها بشيٍء، فنظرَ أصحابُه إلى ساقِ عبدِ اللهِ فضحِكوا من حُمُوشَةِ ساقَيهِ، فقال رسولُ اللهِ ﷺ: ممَّا تضحكون!، لَرِجْلُ عبدِ اللهِ أثقلُ في الميزانِ من أُحُدٍ {\footnotesize (السلسلة الصحيحة للألباني)}. وقال أيضا ﷺ: لو أنَّ عِلْمَ عمرَ بنِ الخطابِ رضيَ اللهُ عنهُ وُضِعَ في كفَّةِ الميزانِ، ووُضِعَ عِلْمُ أهلِ الأرضِ في كفَّةٍ، لرجح عِلْمُ عمرَ بنِ الخطابِ رضيَ اللهُ عنهُ {\footnotesize (صححه الألباني)}. والمراد هنا علم عمر الشرعي وهذا فيه فضل عمر رضي الله عنه فقد بشره بذلك النبي ﷺ فقال: بَيْنا أنا نائِمٌ أُتِيتُ بقَدَحِ لَبَنٍ، فَشَرِبْتُ منه، ثُمَّ أعْطَيْتُ فَضْلِي عُمَرَ بنَ الخَطَّابِ قالوا: فَما أوَّلْتَهُ يا رَسولَ اللَّهِ؟ قالَ: العِلْمَ {\footnotesize (صحيح البخاري)}. أي العلم الشرعي ورؤييا الأنبياء حق. فكل ذلك فيه أن المراد في كل هذه الأحاديث هو الميزان الشرعي والذي يوافق أمر الله الشرعي لما يحبه الله ويرضاه من الأعمال والأقوال وهو بخلاف الميزان الكوني الذي جعله الله بيده لتدبير الكون كما تقدم.

ومن رحمة الله جل جلاله ومنه على المكلفين أنه جعل وزن الأعمال الصالحة في الميزان الشرعي تتضاعف وأقل ذلك عشرة أضعاف كما في قوله تعالى: 
\quranayah*[6][160]{\footnotesize \surahname*[6]}. ويزيد سبحانه وتعالى في فضله على عباده كيف يشاء فيضاعف الحسنات أضعافا كثيرة كما في قوله: 
\quranayah*[2][245]{\footnotesize \surahname*[2]}.
وعن عبد الله بن عباس عَنْ رَسولِ اللهِ ﷺ فِيما يَرْوِي عن رَبِّهِ تَبارَكَ وتَعالَى قالَ: إنَّ اللَّهَ كَتَبَ الحَسَناتِ والسَّيِّئاتِ، ثُمَّ بَيَّنَ ذلكَ، فمَن هَمَّ بحَسَنَةٍ فَلَمْ يَعْمَلْها، كَتَبَها اللَّهُ عِنْدَهُ حَسَنَةً كامِلَةً، وإنْ هَمَّ بها فَعَمِلَها، كَتَبَها اللَّهُ عزَّ وجلَّ عِنْدَهُ عَشْرَ حَسَناتٍ إلى سَبْعِ مِئَةِ ضِعْفٍ إلى أضْعافٍ كَثِيرَةٍ، وإنْ هَمَّ بسَيِّئَةٍ فَلَمْ يَعْمَلْها، كَتَبَها اللَّهُ عِنْدَهُ حَسَنَةً كامِلَةً، وإنْ هَمَّ بها فَعَمِلَها، كَتَبَها اللَّهُ سَيِّئَةً واحِدَةً. وفي رواية: وزادَ: ومَحاها اللَّهُ ولا يَهْلِكُ علَى اللهِ إلَّا هالِكٌ {\footnotesize (صحيح مسلم)}. ولكن الله جل جلاله وضع شرط لهذا الفضل العظيم وهو أن يلقى العبد ربه وهو لا يشرك به شيئا ولهذا فقد جاء عن أبو ذر الغفاري أن النبي ﷺ قال: يقولُ اللهُ تعالَى: مَنْ عمِلَ حسنةً، فلهُ عشرُ أمثالِها. وأَزيدُ، ومَنْ عمِلَ سيِّئةً فجزاؤُها مِثلُها، أوْ أغفِرُ، ومَنْ عمِلَ قُرابَ الأرضِ خطيئةً، ثمَّ لَقِيَنِي لا يُشرِكُ بي شيئًا جعلتُ لهُ مِثلَها مَغفرِةً، ومَنِ اقترَبَ إلىَّ شِبرًا، اقْتربتُ إليه ذِراعًا، ومَنِ اقترَبَ إليَّ ذِراعًا، اقتربتُ إليه باعًا، ومَنْ أتانِي يمشِي، أتيتُهُ هرْولَةً {\footnotesize (صحيح الجامع، وصححه الألباني)}.

ومن واسع فضل الله جل جلاله أنه جعل الحسنات في الميزان الشرعي تذهب السيئات وتبدلها فعن عبد الله بن مسعود رضي الله عنه قال: أنَّ رَجُلًا أصابَ مِنَ امْرَأَةٍ قُبْلَةً، فأتَى النبيَّ صَلَّى اللهُ عليه وسلَّمَ، فأخْبَرَهُ فأنْزَلَ اللَّهُ عزَّ وجلَّ: \quranayah*[11][114]{\footnotesize \surahname*[11]} فقالَ الرَّجُلُ: يا رَسولَ اللَّهِ ألِي هذا؟ قالَ: لِجَمِيعِ أُمَّتي كُلِّهِمْ {\footnotesize (صحيح البخاري)}. ولقد بين النبي ﷺ حساب العشرة الأضعاف من الحسنات خلال اليوم والليلة في الذكر وأنها تغلب السيئات فقال: خَصْلتانِ لا يُحافِظُ عليهِما عبدٌ مُسلمٌ إلا دخل الجنةَ، ألا وهُما يَسِيرٌ، ومَن يعملْ بِهِما قَليلٌ، يُسَبِّحُ اللهَ في دُبُرِ كُلِّ صلاةٍ عَشْرًا (10)، ويَحمدُه عشْرًا (10)، ويُكبِّرُه عشْرًا (10)، فذلِكَ خَمسُونَ ومِائَةٌ باللِسانِ (150 = 30 × 5 أي في الصلوات الخمس)، وألفٌ وخَمسُمِائةٌ في المِيزانِ (1500 = 150 × 10 أي عشرة أضعافها). ويُكبِّرُ أربعًا وثلاثِينَ إذا أخَذَ مَضْجَعَهُ (34)، ويَحمدُه ثلاثًا وثلاثِين (33)، ويُسَبِّحُ ثلاثًا وثلاثِينَ (33)، فتِلكَ مائةٌ باللِسانِ (100 = 34 + 33 + 33)، وألْفٌ في المِيزانِ (1000 = 100 × 10 أي عشرة أضعافها)، فأيُّكمْ يَعْملُ في اليومِ والليلةِ ألْفينِ وخَمسَمائةِ سَيِّئَةٍ (2500 = 1500 + 1000 أي عدد الحسنات الكلي) {\footnotesize (صحيح الجامع، وصححه الألباني)}. 

وكل هذا فيه أن الله تبارك وتعالى يبارك في الأعمال الصالحة على الميزان الشرعي وينميها كما في قوله تعالى: 
\quranayah*[2][261]{\footnotesize \surahname*[2]}. ولهذا فقد وصف الله جل جلاله من جاء بالأعمال الصالحة التي ترضي الله بالفائزين فقال: 
\quranayah*[24][52]{\footnotesize \surahname*[24]}.
 وقد ميز سبحانه بذلك أصحاب الجنة عن أصحاب النار فقال جل جلاله: 
\quranayah*[59][20]{\footnotesize \surahname*[59]}. وأما أصحاب النار فهم الخاسرون بعدل الله وذلك لأنه لم يكن لهم من الأعمال الصالحة ما فيه نجاتهم بما يرضي الله جل جلاله كما قال تعالى: 
\quranayah*[2][27]{\footnotesize \surahname*[2]}. ومن أعظم موجبات الخسران هو الكفر بالله وآياته والشرك به فهذا هو الخسران المبيين كما قال تعالى: 
\quranayah*[39][15]{\footnotesize \surahname*[39]}. وبذلك فإن الخاسرين يوم القيامة يخسرون الإنتفاع بفضل الله العظيم ما فيه نجاتهم من النار، من مضاعفة الأعمال الصالحة وذهاب السيئات بالحسنات فيكونون من أصحاب السعير والعياذ بالله ولهذا فقد قال النبي ﷺ: ولا يَهْلِكُ علَى اللهِ إلَّا هالِكٌ {\footnotesize (صحيح مسلم)}.

\comment{
عن عبدِ اللهِ بنِ عمرٍو حديثًا فيه طُولٌ، وفيه عنه صلَّى اللهُ عليه وسلَّمَ: أنَّ نوحًا لما حضرتْه الوفاةُ دعا بَنيه، فقال: إني قاصٌّ عليكم الوصيةَ، آمُرُكم باثنتينِ، وأنهاكم عن اثنتينِ، أنهاكم عن الشِّركِ والكِبْرِ، وآمُرُكم بلا إلهَ إلَّا اللهُ، فإنَّ السَّمواتِ والأرضَ وما فيها لو وُضِعتْ في كِفَّةِ الميزانِ، ووُضِعتْ لا إلهَ إلَّا اللهُ في الكِفَّةِ الأخرى، كانت أرجَحَ منها، ولو أنَّ السَّمواتِ والأرضَ وما فيها كانت حلْقَةً، فوُضِعتْ [لا إلهَ إلَّا اللهُ] عليها لقَصَمتْهما، وآمُرُكم بسُبحانَ اللهِ وبحَمدِه، فإنَّها صلاةُ كُلِّ شيءٍ، وبها يُرزَقُ كُلُّ شَيءٍ.

قُلتُ لفاطمةَ: لو أتيْتِ النَّبيَّ صلَّى اللهُ عليه وسلَّمَ فسأَلْتِيه خادمًا، فقد أَجهَدكِ الطَّحنُ والعمَلُ؟-قال حُسَينٌ: إنَّه قد جهَدكِ الطَّحنُ والعمَلُ، وكذلك قال أبو أحمدَ- قالت: فانطلِقْ معي. قال: فانطلَقْتُ معها، فسأَلْناه، فقال النَّبيُّ صلَّى اللهُ عليه وسلَّمَ: ألَا أَدُلُّكما على ما هو خَيرٌ لكما من ذلك؟إذا أَوَيْتما إلى فِراشِكما فسبِّحا اللهَ ثلاثًا وثلاثينَ، واحمَداه ثلاثًا وثلاثينَ، وكبِّراه أربعًا وثلاثينَ، فتلك مِئَةٌ على اللِّسانِ، وألفٌ في الميزانِ، فقال عليٌّ رضِي اللهُ عنه: ما ترَكْتُها بعدَما سمِعْتُها منَ النَّبيِّ صلَّى اللهُ عليه وسلَّمَ

خَصلتانِ أو خَلَّتانِ لا يحافِظُ عليهما عبدٌ مسلمٌ إلَّا دخلَ الجنَّةَ هما يسيرٌ ومن يعمَلُ بِهِما قليلٌ يسبِّحُ اللَّهَ تعالى دُبُرِ كلِّ صلاةٍ عَشرًا ويحمدُ عشرًا ويُكَبِّرُ عشرًا فذلِكَ خمسونَ ومائةٌ باللِّسانِ وألفٌ وخمسمائةٍ في الميزانِ ويُكَبِّرُ أربعًا وثلاثينَ إذا أخذَ مضجعَهُ ويحمدُ ثلاثًا وثلاثينَ ويسبِّحُ ثلاثًا وثلاثينَ فذلِكَ مائةٌ باللِّسانِ وألفٌ بالميزانِ قال فلَقدْ رأيتُ رسولَ اللَّهِ صلَّى اللَّهُ عليهِ وسلَّمَ يعقدُها بيدِهِ قالوا يا رسولَ اللَّهِ كيفَ هُما يسيرٌ ومن يعمَلُ بِهِما قليلٌ؟قالَ: يأتي أحدَكُم يعني الشَّيطانَ في مَنامِهِ فينوِّمُهُ قبلَ أن يقولَهُ

أنَّ رجلًا، قال لرسولِ اللهِ صلَّى اللهُ عليه وسلَّم: رأيتُ كأنَّ ميزانًا دُلِّيَ منَ السماءِ، فوُزِنتَ بأبي بكرٍ فرجَحتَ بأبي بكرٍ، ثم وُزِن أبو بكرٍ بعُمرَ، فرجَح أبو بكرٍ، ثم وُزِن عُمرُ بعثمانَ، فرَجَح عُمرُ، ثم رُفِع الميزانُ، فاستَهَلَّها رسولُ اللهِ صلَّى اللهُ عليه وسلَّم خلافةَ نبوةٍ، ثم يؤتي اللهُ المُلكَ مَن يشاءُ

أنَّ رجلًا قال: يا رسولَ اللهِ رأيتُ كأنَّ مِيزانًا دُلِّي مِنَ السماءِ فَوُزِنْتَ فيه أنت وأبوبكرٍ فَرَجَحْتَ بأبي بكرٍ ثم وُزِنَ فيه أبوبكرٍ وعمرُ فَرَجَحَ أبو بكرٍ بعمرَ ثم وُزِنَ فيه عمرُ وعثمانُ فَرَجَحَ عمرُ بعثمانَ ثم رُفِعَ الميزانُ فاسْتآلهَا يعني تَأَوَّلَها ثم قال: خِلافَةُ نُبُوَّةٍ ثم يُؤتِي اللهُ الملكَ مَنْ يَشاءُ
}

\section{الإرادة الكونية والإرادة الشرعية}

قرر أهل العلم الشرعي من أهل السنة والجماعة في هذا الباب العظيم وبناء على الأدلة والبراهين الواضحة من كتاب الله وسنة نبيه ﷺ وما يوافق النقل والعقل، أن الله جل جلاله له أرادتان وهما الإرادة الكونية والإرادة الشرعية. فالإرادة الكونية هي ما تعلق بمشيئته سبحانه والإرادة الشرعية هي ما تعلق بمحبته ورضاه. فإرادة الله الكونية نافذة كما في قوله تعالى:
\quranayah*[36][82]{\footnotesize \surahname*[36]}. وأما الإرادة الشرعية فهي إرادة بيان لما يحبه الله ويرضاه من الأعمال والأقوال كما في قوله تعالى:
\quranayah*[4][26-28]{\footnotesize \surahname*[4]}.

ولما كان الله عز وجل فعال لما يريد كما في قوله تعالى:
\quranayah*[85][16]{\footnotesize \surahname*[16]}، كان قضاءه تابعا لإرادته أي سبحانه له كذلك قضاء كوني وقضاء شرعي. فالقضاء الكوني هو ما أراده الله كونا فشاء أن يكون فكان بعزته وعلمه وقدرته سبحانه كما في قوله تعالى في هذه الآيات:
\quranayah*[2][117]{\footnotesize \surahname*[2]}. %وقوله:
% \quranayah*[3][47]{\footnotesize \surahname*[3]}. وقوله:
% \quranayah*[19][35]{\footnotesize \surahname*[19]}. وقوله:
% \quranayah*[39][42]{\footnotesize \surahname*[39]}. وقوله:
% \quranayah*[40][68]{\footnotesize \surahname*[40]}. وقوله:
% \quranayah*[41][12]{\footnotesize \surahname*[41]}.
وأما قضاءه الشرعي فهو ما أراده الله شرعا فأمر الله عباده به مثل قوله تعالى:
\quranayah*[17][23]{\footnotesize \surahname*[17]}. فلو كان هذا قضاءا كونيا لكان الناس أمة واحدة على التوحيد ولكن الله نفى ذلك بقضاءه الكوني أي بمشيئته الكونية كما في قوله تعالى: \quranayah*[16][93]{\footnotesize \surahname*[16]}.
ومن ذلك أيضا قوله تعالى:
\quranayah*[33][36]{\footnotesize \surahname*[33]}. فهذا أيضا من القضاء الشرعي ووجه ذلك أنه سبحانه ألزمهم بإتباع أمره الشرعي لأن ذلك من مقتضيات الإيمان ولهذا جاء التحذير في نهاية الآية لمن خالف وعصى أمر الله ورسوله، فلو كان هذا قضاء كونيا لكان ما أراده الله ولم يسع لأحد أن يختار شيئا من ذلك حيث أن أمر الله الكوني نافذ لا محالة.

وكذلك حكم الله تابعا لإرادته وله سبحانه الحكم الكوني وهو تابع لأرادته الكونية والحكم الشرعي وهو تابع لإرادته الشرعية كما في قوله تعالى:
\quranayah*[5][1]{\footnotesize \surahname*[5]}، وهذا في الحكم الشرعي كما دل سياق الآية. وقوله تعالى:
\quranayah*[13][41]{\footnotesize \surahname*[13]}. ويدخل في هذا حكمه الشرعي والقدري (أي الكوني) والجزائي كما جاء في تفسير السعدي رحمه الله.


فحكم الله وقضاءه وأمره الكوني نافذ وماض بإرادته الكونية وبما شاء وهو عدل في ذلك لا يشاركه فيه غيره سبحانه كما قال تعالى:
\quranayah*[40][20]{\footnotesize \surahname*[40]}. وقد جاء في تفسير ابن كثير أن قوله: (والله يقضي بالحق) أي: يحكم بالعدل، وقوله: (والذين يدعون من دونه) أي: من الأصنام والأوثان والأنداد، (لا يقضون بشيء) أي: لا يملكون شيئا ولا يحكمون بشيء [هـ]. وكما جاء عن النبي ﷺ أنه قال:
«ماضٍ فيَّ حكمُك، عدلٌ فيَّ قضاؤُك» {\footnotesize (أخرجه أحمد وصححه الألباني)}. 

وفي كل ذلك فإن الله هو أحكم الحاكمين كما في قوله تعالى:
\quranayah*[95][8]{\footnotesize \surahname*[95]}.\comment{
وعلى لسان نبيه نوح عليه السلام:
\quranayah*[11][45]{\footnotesize \surahname*[11]}.}
وهو أيضا خير الحاكمين كما في قوله:
\quranayah*[10][109]{\footnotesize \surahname*[10]}.
فالحكم كله لله تعالى ومنه الحكم الجزائي وهو سبحانه خير الفاصلين كما في قوله تعالى:
\quranayah*[6][57][14]{\footnotesize \surahname*[6]}. وذلك لان الله عز وجل عدل في إرادته وحكمه وقضاءه العدل التام المنافي للظلم والدليل قوله تعالى:
\quranayah*[3][108]{\footnotesize \surahname*[3]}. ومن ذلك أن الله عدل في جزاءه وثوابه كما أخبر هو بذلك في قوله تعالى:
\quranayah*[39][69]{\footnotesize \surahname*[69]}. وقوله تعالى:
\quranayah*[10][47]{\footnotesize \surahname*[10]}.

وقد ثبت في السنة أن الله عز وجل حرم الظلم على نفسه في حكمه الكوني وعلى عباده في حكمه الشرعي فعَنِ النَّبيِّ صَلَّى اللَّهُ عليه وسلَّمَ فِيما رَوَى عَنِ اللهِ تَبَارَكَ وَتَعَالَى، أنَّهُ قالَ: يا عِبَادِي، إنِّي حَرَّمْتُ الظُّلْمَ علَى نَفْسِي، وَجَعَلْتُهُ بيْنَكُمْ مُحَرَّمًا، فلا تَظَالَمُوا، يا عِبَادِي، كُلُّكُمْ ضَالٌّ إلَّا مَن هَدَيْتُهُ، فَاسْتَهْدُونِي أَهْدِكُمْ، يا عِبَادِي، كُلُّكُمْ جَائِعٌ إلَّا مَن أَطْعَمْتُهُ، فَاسْتَطْعِمُونِي أُطْعِمْكُمْ، يا عِبَادِي، كُلُّكُمْ عَارٍ إلَّا مَن كَسَوْتُهُ، فَاسْتَكْسُونِي أَكْسُكُمْ، يا عِبَادِي، إنَّكُمْ تُخْطِئُونَ باللَّيْلِ وَالنَّهَارِ، وَأَنَا أَغْفِرُ الذُّنُوبَ جَمِيعًا، فَاسْتَغْفِرُونِي أَغْفِرْ لَكُمْ، يا عِبَادِي، إنَّكُمْ لَنْ تَبْلُغُوا ضَرِّي فَتَضُرُّونِي، وَلَنْ تَبْلُغُوا نَفْعِي فَتَنْفَعُونِي، يا عِبَادِي، لو أنَّ أَوَّلَكُمْ وَآخِرَكُمْ وإنْسَكُمْ وَجِنَّكُمْ، كَانُوا علَى أَتْقَى قَلْبِ رَجُلٍ وَاحِدٍ مِنكُمْ؛ ما زَادَ ذلكَ في مُلْكِي شيئًا، يا عِبَادِي، لوْ أنَّ أَوَّلَكُمْ وَآخِرَكُمْ وإنْسَكُمْ وَجِنَّكُمْ، كَانُوا علَى أَفْجَرِ قَلْبِ رَجُلٍ وَاحِدٍ؛ ما نَقَصَ ذلكَ مِن مُلْكِي شيئًا، يا عِبَادِي، لو أنَّ أَوَّلَكُمْ وَآخِرَكُمْ وإنْسَكُمْ وَجِنَّكُمْ، قَامُوا في صَعِيدٍ وَاحِدٍ فَسَأَلُونِي، فأعْطَيْتُ كُلَّ إنْسَانٍ مَسْأَلَتَهُ؛ ما نَقَصَ ذلكَ ممَّا عِندِي إلَّا كما يَنْقُصُ المِخْيَطُ إذَا أُدْخِلَ البَحْرَ، يا عِبَادِي، إنَّما هي أَعْمَالُكُمْ أُحْصِيهَا لَكُمْ، ثُمَّ أُوَفِّيكُمْ إيَّاهَا، فمَن وَجَدَ خَيْرًا فَلْيَحْمَدِ اللَّهَ، وَمَن وَجَدَ غيرَ ذلكَ فلا يَلُومَنَّ إلَّا نَفْسَهُ. وفي روايةٍ: إنِّي حَرَّمْتُ علَى نَفْسِي الظُّلْمَ وعلَى عِبَادِي، فلا تَظَالَمُوا. {\footnotesize (صحيح مسلم)}.

وهذا فيه أن الله جل جلاله لم يحرم الظلم على عباده كونا بل حرمه شرعا وهذا من حكمته سبحانه فقد بين حكمه الشرعي للمكلفين حتى يغفر لمن يشاء برحمته ويعذب من يشاء بعدله في حكمه الجزائي كما في قوله تعالى: 
\quranayah*[48][14]{\footnotesize \surahname*[48]}.  وقد جاء في تفسير السعدي أن معنى ذلك أن الله تعالى هو المنفرد بملك السماوات والأرض، يتصرف فيهما بما يشاء من الأحكام القدرية، والأحكام الشرعية، والأحكام الجزائية، ولهذا ذكر حكم الجزاء المرتب على الأحكام الشرعية، فقال: (يَغْفِرُ لِمَنْ يَشَاءُ) وهو من قام بما أمره الله به (وَيُعَذِّبُ مَنْ يَشَاءُ) ممن تهاون بأمر الله، (وَكَانَ اللَّهُ غَفُورًا رَحِيمًا) أي: وصفه اللازم الذي لا ينفك عنه المغفرة والرحمة، فلا يزال في جميع الأوقات يغفر للمذنبين، ويتجاوز عن الخطائين، ويتقبل توبة التائبين، وينزل خيره المدرار، آناء الليل والنهار \cite{tafsir_Saadi}. فهذا الحكم الجزائي مرتبط بعدل الله وهدايته فيغفر لمن يشاء بأن يهديه للإسلام والطاعة ويعذب من يشاء بأن يكله إلى نفسه الجاهلة الظالمة المقتضية لعمل الشر فيعمل الشر ويعذب على ذلك كما جاء بيان ذلك في تفسير السعدي رحمه الله. وهذا المعنى هو المعنى الصحيح ولقد بينه جل في علاه في قوله: \quranayah*[6][125]{\footnotesize \surahname*[6]}.

\section{الهداية الكونية والهداية الشرعية}

وكما أن لله عز وجل له إرادتان الكونية والشرعية فإن له سبحانه أيضا هدايتان وهما الهداية الكونية وهي هداية التوفيق والإنقياد والهداية الشرعية وهي هداية المعرفة والإرشاد. واجتمعت الهدايتان في قوله تعالى:
\quranayah*[42][52]{\footnotesize \surahname*[42]}. ووجه ذلك أن الله عز وجل يهدي من يشاء ومن يريد فهذه الهداية المرتبطة بمشيئته وإرادته سبحانه هي الهداية الكونية وقد وردت في عدة مواضع في القرآن منها قوله تعالى:
\quranayah*[28][56]{\footnotesize \surahname*[28]}. وفي قوله تعالى:
\quranayah*[22][16]{\footnotesize \surahname*[22]}. وأما الهداية الأخرى في قوله تعالى: (وَإِنَّكَ لَتَهْدِي إِلَىٰ صِرَاطٍ مُّسْتَقِيمٍ)
فهي المرتبطة بمعرفة الحق وهي الهداية الشرعية وقد وردت أيضا في عدة مواضع في القرآن منها قوله تعالى:
\quranayah*[4][26]{\footnotesize \surahname*[4]}. وهذه الهداية تكون مرتبطة بالوحي وهي هداية بيان للحق كما في قوله تعالى:
\quranayah*[34][6]{\footnotesize \surahname*[6]}.

وبهذا يتبين أن كل المخلوقات مسييرين بإرادة الله الكونية وأن المكلفين منهم من الجن والإنس مخييرين بإرادة الله الشرعية. والهداية الكونية الراجعة إلى مشيئة الله هي الغالبة، فمن عرف الحق وعمل به فقد هدي شرعا وكونا أي اجتمعت فيه الهدايتان ومثال ذلك أصحاب النبي ﷺ. والناس في كل ذلك درجات برحمة الله وكرمه وبما فضل الله بعضهم على بعض،
ومن عرف الحق ولم يعمل به فقد هدي شرعا ولم يهدى كونا أي لم تجتمع فيه الهدايتان ومثال ذلك ابليس لعنه الله. والناس في ذلك دركات بعدل الله وغضبه وبما أغوى الله بعضهم على بعض. وسيأتي تفصيل ذلك في بيان يوم الحساب الذي فيه يكون الحساب بوزن الأعمال وبعده يأتي الجزاء إما جنة أو نار. 

ولقد الله عز وجل جعل لهدايته الكونية مسببات منها الإنابة إليه كما في قوله تعالى:
\quranayah*[42][13][30]{\footnotesize \surahname*[42]}. وقوله تعالى:
\quranayah*[13][27]{\footnotesize \surahname*[13]}.
ومن ذلك أيضا الإيمان بالله والأعتصام به كما في قوله تعالى:
\quranayah*[4][175]{\footnotesize \surahname*[4]}. ومن ذلك أيضا إتباع أمر الله الشرعي الذي يحبه الله ويرضاه كما في قوله تعالى:
\quranayah*[5][16]{\footnotesize \surahname*[5]}.  وهذا كله من عدل الله ورحمته سبحانه. ومن أسباب الهداية والثبات الدعاء فإن أكثر دعاء نبينا ﷺ كان في ثبات القلب كما جاء عن أم سلمة أن أَكْثرُ دعائِهِ كانَ: يا مُقلِّبَ القلوبِ ثبِّت قلبي على دينِكَ قالَت: فقُلتُ: يا رسولَ اللَّهِ ما أكثرُ دعاءكَ يا مقلِّبَ القلوبِ ثبِّت قلبي على دينِكَ؟ قالَ: يا أمَّ سلمةَ إنَّهُ لَيسَ آدميٌّ إلَّا وقلبُهُ بينَ أصبُعَيْنِ من أصابعِ اللَّهِ، فمَن شاءَ أقامَ، ومن شاءَ أزاغَ \.فتلا معاذٌ رَبَّنَا لَا تُزِغْ قُلُوبَنَا بَعْدَ إِذْ هَدَيْتَنَا {\footnotesize (صحيح الترمذي وصححه الألباني)}. وهذا فيه أن الله جل جلاله يقلب القلوب بين أصابعه بإرادته الكونية وأن الدعاء قد يكون سببا للهداية الكونية والتي بها يكون الثبات على الدين والطاعة.  

ومن أعظم أسباب الهداية هي الجهاد في سبيل الله كما في قوله تعالى:  \quranayah*[29][69]{\footnotesize \surahname*[29]}. وقد جاء في تفسير السعدي قوله: دل هذا، على أن أحرى الناس بموافقة الصواب أهل الجهاد، وعلى أن من أحسن فيما أمر به أعانه اللّه ويسر له أسباب الهداية، وعلى أن من جد واجتهد في طلب العلم الشرعي، فإنه يحصل له من الهداية والمعونة على تحصيل مطلوبه أمور إلهية، خارجة عن مدرك اجتهاده، وتيسر له أمر العلم، فإن طلب العلم الشرعي من الجهاد في سبيل اللّه، بل هو أحد نَوْعَي الجهاد، الذي لا يقوم به إلا خواص الخلق، وهو الجهاد بالقول واللسان، للكفار والمنافقين، والجهاد على تعليم أمور الدين، وعلى رد نزاع المخالفين للحق، ولو كانوا من المسلمين \cite{tafsir_Saadi}.

\section{الميزان بمعنى العدل والميزان المخلوق}

 العدل مرادف للميزان، والعدل هو صفة من صفات الله جل جلاله والله عدل في إرادته الكونية والشرعية، فالميزان الكوني تابعا لإرادة الله الكونية والميزان الشرعي تابعا لإرادة الله الشرعية. وكل هذا على وجه الإجمال. وأما على وجه التفصيل، فالميزان الكوني هو العدل في إرادة الله الكونية والميزان الشرعي هو العدل في إرادة الله الشرعية. 

ولكن الله عز وجل جعل الميزانان كل منهما في صورة مخلوق لحكمته ولإظهار عدله سبحانه، فخلق سبحانه الميزان الكوني ووضعه بيده والذي فيه تدبيره للكون وهو قائما عليه بالقسط كما في قوله تعالى:
\quranayah*[3][18]{\footnotesize \surahname*[3]}. ويخلق سبحانه الميزان الشرعي يوم القيامة ويضعه لحساب المكلفين وأعمالهم كما في قوله تعالى: 
\quranayah*[21][47]{\footnotesize \surahname*[21]}.

وبهذا يعلم بالضرورة أن الميزان الكوني المخلوق الذي يضعه الله جل جلاله في يده كما صح ذلك عن النبي ﷺ فيه شأن الله وتدبيره لهذا الكون بما في ذلك السماوات والأرض. وأما الميزان الشرعي المخلوق الذي يوضع بعد الصراط وقبل الحوض كما صح ذلك عن النبي ﷺ فيه شأن حساب المكلفين وأعمالهم يوم القيامة. فدل ذلك على أن الميزان الكوني المخلوق أعظم من الميزان الشرعي المخلوق، وكل منهما من عدل الله ورحمته. وهذا لأن خلق السماوات والأرض أعظم من خلق الناس كما في قوله تعالى: 
\quranayah*[40][57]{\footnotesize \surahname*[40]}. فكل هذا فيه أن شؤون الخلق ومن ذلك البعث والحساب أهون على الله جل جلاله من أمر السموات والأرض كما في قوله تعالى: 
\quranayah*[30][27]{\footnotesize \surahname*[30]}. وقال السعدي في بيان معنى (وَهُوَ أَهْوَنُ عَلَيْهِ): وهذا بالنسبة إلى الأذهان والعقول.

وهذا فيه بيان أن التفاوت هنا في قوله (وهو أهون عليه) ليس في ذات الله وقدرته فهو سبحانه على كل شئ قدير ولا يعجزه شئ ولكن هذا التفاوت إنما هو لبيان الحجة العقلية حيث أن من خلق الإنسان وخلق السموات والأرض وهي أكبر وأعظم، قادر على إحياء الموتى من باب أولى. فتكون الحجة العقلية هنا أن من لم يعجزه الإبتداء لا تعجزه الإعادة وخصوصا لما هو أسهل فهذا أهون. وقد جاء في تفسير ابن كثير وتفسير الطبري: عن ابن عباس قوله: (وَهُوَ أَهْوَنُ عَلَيْهِ) يقول: كلّ شيء عليه هين [هـ]. وقد جاء في تفسير القرطبي: فجعل ما علم من ابتداء خلقه دليلا على ما يخفى من إعادته; استدلالا بالشاهد على الغائب  [.]، قال أبو عبيدة: ومن جعل أهون يعبر عن تفضيل شيء على شيء فقوله مردود بقوله تعالى: (وكان ذلك على الله يسيرا)، وبقوله: (ولا يئوده حفظهما). والعرب تحمل أفعل على فاعل [.] وأنشد أبو عبيدة أيضا: إني لأمنحك الصدود وإنني قسما إليك مع الصدود لأميل (أراد لمائل) [.] ووجهه أن هذا مثل ضربه الله تعالى لعباده [هـ]. وهذا فيه أن الله جل جلاله يخاطب عباده بما يناسب فهمهم وعقولهم وهذا من عدله ورحمته جل جلاله.

\section{الأصل في هذا الكون استقراره وثباته وتوازنه وبركته}

من المعلوم بالضرورة وما دلت عليه البراهين الشرعية والعقلية أن الأصل في هذا الكون استقراره وثباته وتوازنه وبركته حتى يصلح للحياة ومن ذلك أن الله عز وجل جعل الأرض مستقرة وثابتة ومبسوطة والجبال أوتادا والسماء مرفوعة والسحاب والرياح مسخرة والفلك والأنهار جارية والبحار محسورة والشمس سراجا والقمر نورا والنهار معاشا والليل سكنا والنجوم دليلا والشجار مثمرة والدواب متحركة وسائر المخلوقات المتنوعة وغيرها من الآيات العظيمة الدالة عليه والمرشدة إليه. فكل ذلك من آيات الله الكونية الدالة على عظمته وحكمته سبحانه والتي أراد الله منا بإرادته الكونية أن نراها وبإرادته الشرعية أن نتدبر فيها ونتمعن في تفاصيلها بما أودع فينا من عقل وفطرة. وقد قال تعالى في ذلك: 
\quranayah*[27][93]{\footnotesize \surahname*[27]}. وقوله تعالى:
\quranayah*[41][53]{\footnotesize \surahname*[41]}.

والآيات في ذلك عديدة ومنها قوله تعالى:
\quranayah*[21][30-33]{\footnotesize \surahname*[21]}. وقوله تعالى:
\quranayah*[36][33-40]{\footnotesize \surahname*[36]}. وقوله تعالى:
\quranayah*[2][164]{\footnotesize \surahname*[2]}. وقوله تعالى:
\quranayah*[6][97-99]{\footnotesize \surahname*[6]}.
% \quranayah*[3][190-191]{\footnotesize \surahname*[3]}. وقوله تعالى:
% \quranayah*[13][2-4]{\footnotesize \surahname*[13]}. وقوله تعالى:
% \quranayah*[16][12]{\footnotesize \surahname*[16]}. وقوله تعالى:
% \quranayah*[16][79]{\footnotesize \surahname*[16]}. وقوله تعالى:
% \quranayah*[21][32]{\footnotesize \surahname*[21]}. وقوله تعالى:
% \quranayah*[27][86]{\footnotesize \surahname*[27]}. وقوله تعالى:
% \quranayah*[30][20-27]{\footnotesize \surahname*[30]}. وقوله تعالى:
% \quranayah*[30][37]{\footnotesize \surahname*[30]}. وقوله تعالى:
% \quranayah*[30][46]{\footnotesize \surahname*[30]}. وقوله تعالى:
% \quranayah*[31][31]{\footnotesize \surahname*[31]}. وقوله تعالى:
% \quranayah*[40][13]{\footnotesize \surahname*[40]}. وقوله تعالى:
% \quranayah*[41][37-39]{\footnotesize \surahname*[41]}. وقوله تعالى:
% \quranayah*[42][29]{\footnotesize \surahname*[42]}. وقوله تعالى:
% \quranayah*[42][32-33]{\footnotesize \surahname*[42]}. وقوله تعالى:
% \quranayah*[45][3-5]{\footnotesize \surahname*[45]}. وقوله تعالى:
% \quranayah*[45][13]{\footnotesize \surahname*[45]}. وقوله تعالى:
% \quranayah*[51][20]{\footnotesize \surahname*[51]}. وقوله تعالى:
% \quranayah*[57][17]{\footnotesize \surahname*[57]}. وقوله تعالى:
% \quranayah*[7][58]{\footnotesize \surahname*[7]}.

\comment{

التفكر في السماوات والأرض، والدواب، والليل والنهار، والأرض: 
\quranayah*[45][3-5]{\footnotesize \surahname*[45]}.

التفكر في كتاب الله:
\quranayah*[59][21]{\footnotesize \surahname*[59]}.
\quranayah*[16][44]{\footnotesize \surahname*[16]}.

التفكر في السموات والأرض
\quranayah*[45][13]{\footnotesize \surahname*[45]}.

التفكر في الأرض، والجبال، والأنهار، والثمرات، والأزواج، والليل والنهار:
\quranayah*[13][3]{\footnotesize \surahname*[13]}.

التفكر في النحل:
\quranayah*[16][69]{\footnotesize \surahname*[16]}.

التفكر في الحياة الدنيا:
\quranayah*[10][24]{\footnotesize \surahname*[10]}.

التفكر في الزرع والزيتون والنخيل والأعناب، وكل الثمرات:
\quranayah*[16][11]{\footnotesize \surahname*[16]}.

التفكر في الزواج:
\quranayah*[30][21]{\footnotesize \surahname*[30]}.

التفكر في الموت:
\quranayah*[39][42]{\footnotesize \surahname*[39]}.

النهي عن أغلاق القلوب عن التفكر في كتاب الله:
\quranayah*[22][46]{\footnotesize \surahname*[22]}.

}

وكل هذا فيه الحجة البالغة العقلية والشرعية على استحقاق الله جل جلاله للعبادة وحده لا شريك له بالطريقة التي ارتضاها ومن ذلك وجوب تسبيحه وتقديسه بأسمائه وصفاته بدون تحريف ولا تعطيل ولا تكييف ولاتمثيل إذ قال جل جلاله: 
\quranayah*[87][1-5]{\footnotesize \surahname*[87]}.
وجاء في تفسير السعدي في بيان معنى هذه الآيات أنه تعالى يأمر بتسبيحه المتضمن لذكره وعبادته، والخضوع لجلاله، والاستكانة لعظمته، وأن يكون تسبيحا، يليق بعظمة الله تعالى، بأن تذكر أسماؤه الحسنى العالية على كل اسم بمعناها الحسن العظيم. (الذي خلق فسوى) أي: أتقنها وأحسن خلقها (وَالَّذِي قَدَّرَ) تقديرًا، تتبعه جميع المقدرات (فَهَدَى) إلى ذلك جميع المخلوقات. وهذه الهداية العامة، التي مضمونها أنه هدى كل مخلوق لمصلحته، وتذكر فيها نعمه الدنيوية، ولهذا قال فيها: (وَالَّذِي أَخْرَجَ الْمَرْعَى) أي: أنزل من السماء ماء فأنبت به أنواع النبات والعشب الكثير، فرتع فيها الناس والبهائم وكل حيوان، ثم بعد أن استكمل ما قدر له من الشباب، ألوى نباته، وصوح عشبه. (فَجَعَلَهُ غُثَاءً أَحْوَى) أي: أسود أي: جعله هشيمًا رميمًا، ويذكر فيها نعمه الدينية \cite{tafsir_Saadi}. فتبارك الله أحسن الخالقين.

فلولا ثبات الكون وإستقراره وبركته لما صلح للحياة ولما كانت الحياة ممكنة ومستقرة ومنتظمة ومنتجة. ولهذا فإن الإنسان يعيش في هذا الكون ويستفيد منه ومن ثماره ونعمه وبركاته وموارده ومعادنه وغيرها من الأشياء التي جعلها الله في هذا الكون ليستفيد منها بفضله ورحمته ومنه علينا.
وفي ثبات الكون وإستقراره غايات عظيمة ومصالح كثيرة ومنها تعلم العدد والحساب كما في قوله تعالى: 
\quranayah*[10][5]{\footnotesize \surahname*[10]}. 

فكل هذه الآيات واضحة في دلالتها على عظمة الخالق وحكمته وعدله ورحمته. ولهذا فقد ذكر الله عزل وجل أن آياته لقوم يعقلون، يعلمون، يفقهون، يؤمنون، يوقنون، أو يتفكرون، وهم أولي الألبات الصادقين حقا مع أنفسهم ومع خالقهم بما أودعه فيهم من هداية وبصيرة بفضله ومنه عليهم. وهم الذين آمنوا بالله حقا على يقين ولم يرتابوا وجاهدوا في الله لنصرة الحق كما في قوله تعالى: 
\quranayah*[49][15]{\footnotesize \surahname*[49]}.
ولهذا ما ينكر هذه الآيات الواضحة إلا المعاندين لها والكافرين بها والمشككين فيها والمعرضين عنها وعن خالقهم كفرا وعدوانا وظلما. ولهذا فقد سماهم الله جل جلاله العمي وحجب عنهم الهداية الكونية ونفى عنهم اليقين بعدله سبحانه فقال لنبيه: 
\quranayah*[27][81-82]{\footnotesize \surahname*[27]}. وهم الذين يجادلون في آيات الله بالباطل فطبع الله على قلوبهم كما في قوله تعالى:
\quranayah*[40][35]{\footnotesize \surahname*[40]}.
وقوله تعالى: 
\quranayah*[40][56]{\footnotesize \surahname*[40]}.
فهم تكبروا عن قبول الحق لكفرهم كما قال تعالى: 
\quranayah*[40][4]{\footnotesize \surahname*[40]}.


\section{الظلم ينافي الميزان الكوني والميزان الشرعي}

ومن عدله وحكمته سبحانه أنه جعل الظلم منافيا ومخالفا للميزان الشرعي كما جعله سبحانه منافيا للميزان الكوني. فهذا فيه أن الكون محفوظ بإمر الله الكوني وبعدله ولكن هذا الحفظ والإستقرار إنما جعله الله برهانا واضحا على ربوبيته وألوهيته حتى يقيم المكلفين الحق والميزان الشرعي، ولهذا فإن الله جل جلاله جعل الظلم من أسباب البلاء الذي يقع بإذنه إما لحكمته أو عدله أو رحمته. ويقع هذا البلاء في صور مختلفة منها الجوع والخوف وقلة المطر والزلازل وذهاب البركة وغير ذلك. ومن أعظم الظلم الكفر بالله كالشرك كما في قوله تعالى: 
\quranayah*[31][13]{\footnotesize \surahname*[31]}. أو دعوة الولد له سبحانه والدليل قوله تعالى:
\quranayah*[19][88-91]{\footnotesize \surahname*[19]}. يقول السعدي رحمه الله: أي من أجل هذه الدعوى القبيحة تكاد هذه المخلوقات، أن يكون منها ما ذكر. والحال أنه: (مَا يَنْبَغِي) أي: لا يليق ولا يكون (لِلرَّحْمَنِ أَنْ يَتَّخِذَ وَلَدًا) وذلك لأن اتخاذه الولد، يدل على نقصه واحتياجه، وهو الغني الحميد. والولد أيضا، من جنس والده، والله تعالى لا شبيه له ولا مثل ولا سمي \cite{tafsir_Saadi}. 

وهذا فيه أن الشرك ونسبة الولد لله سبحانه هو من الظلم الذي لا ينافي فقط الميزان الشرعي الذي أمر الله به، وأنما ينافي أيضا الميزان الكوني فيكاد يحصل الإضراب الذي به يكون خراب هذا الكون. ولهذا فإن دعوة الولد أو الصاحبة لله جل جلاله من شتم الله والإشراك به سبحانه ولهذا فقد نزه سبحانه نفسه عن ذلك كله في قوله:
\quranayah*[6][100-103]{\footnotesize \surahname*[19]}. وقد صح عن عبد الله بن عباس وعن أبي هريرة عن النبي ﷺ أن الله قالَ: كَذَّبَنِي ابنُ آدَمَ، ولَمْ يَكُنْ له ذلكَ، وشَتَمَنِي، ولَمْ يَكُنْ له ذلكَ؛ فأمَّا تَكْذِيبُهُ إيَّايَ فَزَعَمَ أنِّي لا أقْدِرُ أنْ أُعِيدَهُ كما كانَ، وأَمَّا شَتْمُهُ إيَّايَ فَقَوْلُهُ: لي ولَدٌ، فَسُبْحانِي أنْ أتَّخِذَ صاحِبَةً أوْ ولَدًا!. وفي رواية اخرى: أمَّا تَكْذِيبُهُ إيَّايَ أنْ يَقُولَ: إنِّي لَنْ أُعِيدَهُ كما بَدَأْتُهُ، وأَمَّا شَتْمُهُ إيَّايَ أنْ يَقُولَ: اتَّخَذَ اللَّهُ ولَدًا، وأنا الصَّمَدُ الذي لَمْ ألِدْ ولَمْ أُولَدْ، ولَمْ يَكُنْ لي كُفُؤًا أحَدٌ {\footnotesize (صحيح البخاري)}.

ومما ينافي عدل الله الكوني والشرعي أيضا اتباع الهوى بدلا من إقامة الحق ونصرته كما في قوله تعالى:
\quranayah*[23][71]{\footnotesize \surahname*[23]}. يقول السعدي في تفسيره:
ووجه ذلك أن أهواءهم متعلقة بالظلم والكفر والفساد من الأخلاق والأعمال، فلو اتبع الحق أهواءهم لفسدت السماوات والأرض، لفساد التصرف والتدبير المبني على الظلم وعدم العدل، فالسماوات والأرض ما استقامتا إلا بالحق والعدل\cite{tafsir_Saadi}.

ومن ذلك أيضا نقصان البركة بسبب المعاصي كما في قوله تعالى:
\quranayah*[30][41]{\footnotesize \surahname*[30]}. فقد ورد في تفسير القرطبي أن ابن عباس قال: هو نقصان البركة بأعمال العباد كي يتوبوا [هـ]. وجاء في تفسير ابن كثير أن زيد بن رفيع قال: (ظهر الفساد) يعني انقطاع المطر عن البر يعقبه القحط، وعن البحر تعمى دوابه [هـ]. 

ومن ذلك أيضا الكفر بأنعم الله كما في قوله تعالى:
\quranayah*[16][112]{\footnotesize \surahname*[16]}. وقوله تعالى:
\quranayah*[28][58]{\footnotesize \surahname*[28]}. وجاء في تفسير ابن كثير معنى ذلك أي: طغت وأشرت وكفرت نعمة الله، فيما أنعم به عليهم من الأرزاق [هـ]. وسيأتي توضيح أسباب هذا العذاب في بيان حال الأمم مع الحق والميزان.

\section{المراد بالعلم والميزان}

إن المراد بالعلم عموما هو المعرفة وله أقسام وأنواع ويمكن تقسميه إلى قسمين وهما: العلم الكوني والعلم الشرعي. فالعلم الكوني ينقسم إلى علم ظاهر وهو العلم السببي وعلم غير ظاهر وهو العلم الغيبي. وأما العلم الشرعي فهو ما قام عليه الدليل كما بين ذلك شيخ الإسلام ابن تيمية رحمه الله، وبهذا يكون العلم الشرعي الصحيح هو ما قام عليه الدليل من كتاب الله عز وجل أو سنة نبيه ﷺ أو الإجماع، والعلم الشرعي الغير صحيح هو ما لم يقم عليه الدليل أو خالفه أو خالف الإجماع، ومن ذلك كل البدع والمحدثات كما بين ذلك النبي ﷺ في حديثه: إِيَّاكُمْ ومُحدَثاتِ الأمورِ، فإنَّ كُلَّ مُحدَثَةٍ بِدعَةٌ، وكلَّ بدعةٍ ضلالةٌ، وكلَّ ضلالَةٍ في النَّارِ {\footnotesize (صحيحه الألباني في وجوب الأخذ بحديث الآحاد)}. وينقسم العلم الشرعي إلى علم فطري وعلم ديني. 

والعلم بكل أقسامه إما علم نافع وإما علم غير نافع بحسب حال صاحبه (أي حامله). فالعلم النافع هو العلم الذي يورث صاحبه خشية الله والعمل الصالح وإن لم يكن علما شرعيا، والعلم الغير نافع هو العلم الذي لا يورث صاحبه خشية الله ولا العمل الصالح وإن كان علما شرعيا. ولكن العلم الشرعي الصحيح هو من أعظم أسباب الهداية إلى معرفة الله وحقه وحق عباده علينا المعرفة التي تورث خشية الله والعمل الصالح ولهذا يأتي لفظ العلم بالإطلاق على العلم الشرعي الصحيح. وقد قال تعالى: \quranayah*[35][28][8]{\footnotesize \surahname*[35]}. وقد جاء سياق الآية في التدبر في آيات الله الكونية كالسماء ونزول المطر وتنوع الجبال والناس والدواب والثمرات فدل ذلك على أن العبرة ليست بنوع العلم فقط ولكن العبرة بالخشية والتقوى والعمل الصالح مع الإقرار بأن هذا الكون هو من صنع العزيز العليم الذي هو على كل شئ قدير. ولهذا فقد جاء في تفسير ابن كثير عن معنى ذلك عن ابن عباس قال: الذين يعلمون أن الله على كل شيء قدير، وعن ابن مسعود أنه قال: ليس العلم عن كثرة الحديث، ولكن العلم عن كثرة الخشية. وقد كان النبي ﷺ يرغب في العلوم النافعة بالعموم وأمر أصحابه بالدعاء فقال: سَلوا اللهَ علمًا نافعًا، و تَعوَّذوا باللهِ من علمٍ لا ينفعُ {\footnotesize (صححه الألباني في السلسلة)}. وكان النبي ﷺ يدعوا بذلك فقال: اللَّهمَّ إنِّي أسأَلُكَ عِلمًا نافعًا وأعوذُ بكَ مِن عِلْمٍ لا ينفَعُ {\footnotesize (صحيح ابن حبان)}. وقال النبي ﷺ: مَنْ سَلَكَ طَرِيقًا يَلْتَمِسُ فِيهِ عِلْمًا سَهَّلَ اللَّهُ لَهُ طَرِيقًا إِلَى الجَنَّةِ {\footnotesize (صحيح مسلم)}. وهذا لا يكون إلا للعلم الذي يورث خشية الله والعمل الصالح.

ومن الأمثلة على العلم النافع هو العلم الشرعي الذي يورث صاحبه خشية الله فيعمل به ليكون حجة له يوم القيامة، ومن العلم النافع أيضا علم التسيير للإستدلال بالنجوم الثابتة في السماء الذي يورث صاحبه خشية الله والتدبر في آياته الكونية ومعرفة عظمة الخالق سبحانه وتعالى. ومن الأمثلة على العلم الغير النافع هو العلم الشرعي الذي لا يورث صاحبه خشية الله فلا يعمل به ويكون حجة عليه يوم القيامة، ومن العلم الغير نافع أيضا علم السحر الذي لا يورث إلا الكفر بالله والشرك به. 

وأما الميزان فالمراد به العدل، وفي اللغة هو الإنصاف، والإنتصاف، والإعتدال، والتوسط، والإستقامة، والتسوية في الحقوق، ومنه الحساب ولهذا سماه الخوارزمي الجبر والمقابلة. ولهذا يكون المراد بالميزان بالنسبة للمعرفة هو العمل بعد المعرفة. ويقام الميزان بأداء الحقوق وبموافقة العمل للمعرفة، ويبخس الميزان ويطغى عليه بإضاعة الحقوق وبمخالفة العمل للمعرفة. وأقسام الميزان كأقسام العلم، ويمكن تقسيمه إلى ميزان كوني وميزان شرعي. فالميزان الكوني ينقسم إلى الميزان السببي والميزان الغيبي. وأما الميزان الشرعي فهو ينقسم إلى الميزان الفطري والميزان الديني. 

والميزان الشرعي تابع للعلم الشرعي وهو ما قام عليه الدليل. ويقام الميزان الشرعي بالحكم بالحق وإتباعه والعمل به، والحق هو العلم الشرعي الصحيح وهذا يكون بأداء حق الله بإخلاص مع أداء حق الناس بالقسط وهذا ما كلفنا الله به بحسب القدرة. وأما الميزان الكوني فقد تكفل به الله ووضعه جل جلاله في يده على صورة مخلوق لتدبير الكون بالقسط أي العدل الظاهر. والله جل في علاه يضع الميزان الشرعي بعد الصراط في صورة مخلوق لحساب المكلفين من الجن والإنس. والله قائم على الميزان الكوني بالقسط وسيقوم على الميزان الشرعي بالقسط يوم الحساب، وهذا لحكمته وتمام عدله سبحانه ومن ذلك ليكون عدله نافذا في الميزان الشرعي كما في الميزان الكوني.

وعلى ما تقدم، فإن كان المراد المعرفة قيل العلم وإن كان المراد العمل بعد المعرفة قيل الميزان. وفيما يأتي بيان أقسام العلم والميزان واقتصرت التسمية على الميزان فقط لتشمل المعرفة والعمل معا. ولكن نفس التقسيم والشرح ينطبق على العلم أيضا.

\section{أقسام الميزان الكوني}

\subsection{الميزان السببي}

فالقسم الأول هو الميزان السببي أو العلم السببي وهو علم ظاهر يدرك بالعقل والفطرة وما منَّ الله به على خلقه من حواس كالبصر والسمع والإحساس. فهذا الميزان فيه حقيقة الأشياء ومسمياتها وطريق الوصول إليها ومسبباتها ولهذا كان مفتاح هذا العلم هو علم الحساب. والمخلوقات تتفاوت في المعرفة بهذا الميزان كل بحسب حاله ومقامه ولكن الله جل جلاله أختص الإنسان وفضله على سائر الخلق بأن جعل له عقلا يدرك به من الأسباب وحقيقتها ومسمياتها ما لا يمكن لغيره من المخلوقات. وهذا لأن الله جل جلاله أراد بحكمته أن يجعله خليفة في الأرض فخلقه على صورته وجعل له من العقل والذكاء ما لم يعطي غيره. وهذا ما فضل الله جل جلاله به آدم على الملائكة وهو تفضيل في المعرفة السببية فأمرهم بالسجود له سجود التحية والإحترام وليظهر فضله على سائر الخلق وهذا لحكمته سبحانه وعلمه كما في قوله تعالى:
\quranayah*[2][30-34]{\footnotesize \surahname*[2]}.

فأما الملائكة فاعترفوا بهذا الفضل وبأنهم لا علم لهم إلا ما علمه الله لهم ولكن هذا النقص في العلم السببي لم يمنعهم من الطاعة والإنقياد لأمر الله جل جلاله. وأما إبليس فقد حسد آدم في الصورة التي خلق بها وعلى ما منَّ الله به عليه من العلم والقدرة المعرفية التي استحق بها هذا الثناء والتقدير. ولهذا فما كان للشيطان إلا أن يقول مستكبرا أنه خيرا منه خلق من نار وآدم من طين حسدا منه وكفرا. وهذا فيه جهل ابليس حيث أنه نسب الفضل لمجرد نوع مادة الخلق والقوة الطبيعية لا للعلم والقدرة المعرفية التي منَّ الله بها على آدم عليه السلام. فما كان له إلا أن يسعى لإضلاله وإخراجه من الجنة حسدا منه على هذه الفضائل ولو كان ذلك على حساب هلاكه وسوء مئاله. وهذا أيضا فيه نقص العقل وسوء الفكر نسأل الله السلامة والعافية. ولم يكتفي بذلك بل أخذ العهد على نفسه لإغواء وإضلال كل ذرية بني آدم كما حذرنا سبحانه وتعالى في كتابه الكريم.

وبالمعرفة بهذا الميزان السببي لا زالت تتقدم الحضارات الإنسانية وتتطور في الأخذ بالأسباب لإنجاز ما لم يكن ممكا لما سبق من الأمم من التكنولوجيا وشتى العلوم كالفيزياء والكيمياء والطب، والذكاء الإصطناعي وغيرها من العلوم الأخرى التي بها يمكن تحصيل المصالح الدينية والدنيوية. ومفتاح كل هذه العلوم هو علم الحساب حيث به تعرف مقادير الأشياء وتقديرها ولهذا فقد أمر الله تعالى بتعلمه من الآيات الكونية كما تقدم. ولهذا فإن الميزان السببي يحتاج إلى بحث وتمعن في آيات الله الكونية، وما يخفى منه على الإنسان أكثر مما يعلم لهذا قال جل جلاله في ذلك عندما سئل الرسول ﷺ عن حقيقة الروح: 
\quranayah*[17][85]{\footnotesize \surahname*[17]}. وهذا فيه أن الروح لا يمكن قياسها ولا عدها وهي ليست من الميزان السببي وإنما هي من الميزان الغيبي. 

ولهذا فقد ذكر جل جلاله تقدم البشرية من الأقوام السابقة في هذا العلم السببي ولكن هذا التقدم كان سببا في زيادة الغفلة ظنا منهم أن العلم السببي يغني عن العلم الشرعي ولهذا قال تعالى: 
\quranayah*[40][83]{\footnotesize \surahname*[40]}. وقال تعالى:
\quranayah*[30][7]{\footnotesize \surahname*[30]}. وجاء في تفسيير السعدي بيان ذلك أن هؤلاء الذين لا يعلمون أي: لا يعلمون بواطن الأشياء وعواقبها. وإنما {يَعْلَمُونَ ظَاهِرًا مِنَ الْحَيَاةِ الدُّنْيَا} فينظرون إلى الأسباب ويجزمون بوقوع الأمر الذي في رأيهم انعقدت أسباب وجوده ويتيقنون عدم الأمر الذي لم يشاهدوا له من الأسباب المقتضية لوجوده شيئا، فهم واقفون مع الأسباب غير ناظرين إلى مسببها المتصرف فيها. (وَهُمْ عَنِ الْآخِرَةِ هُمْ غَافِلُونَ) قد توجهت قلوبهم وأهواؤهم وإراداتهم إلى الدنيا وشهواتها وحطامها فعملت لها وسعت وأقبلت بها وأدبرت وغفلت عن الآخرة، فلا الجنة تشتاق إليها ولا النار تخافها وتخشاها ولا المقام بين يدي اللّه ولقائه يروعها ويزعجها وهذا علامة الشقاء وعنوان الغفلة عن الآخرة. ومن العجب أن هذا القسم من الناس قد بلغت بكثير منهم الفطنة والذكاء في ظاهر الدنيا إلى أمر يحير العقول ويدهش الألباب. وأظهروا من العجائب الذرية والكهربائية والمراكب البرية والبحرية والهوائية ما فاقوا به وبرزوا وأعجبوا بعقولهم ورأوا غيرهم عاجزا عما أقدرهم اللّه عليه، فنظروا إليهم بعين الاحتقار والازدراء وهم مع ذلك أبلد الناس في أمر دينهم وأشدهم غفلة عن آخرتهم وأقلهم معرفة بالعواقب، قد رآهم أهل البصائر النافذة في جهلهم يتخبطون وفي ضلالهم يعمهون وفي باطلهم يترددون نسوا اللّه فأنساهم أنفسهم أولئك هم الفاسقون. ثم نظروا إلى ما أعطاهم اللّه وأقدرهم عليه من الأفكار الدقيقة في الدنيا وظاهرها و[ما] حرموا من العقل العالي فعرفوا أن الأمر للّه والحكم له في عباده وإن هو إلا توفيقه وخذلانه فخافوا ربهم وسألوه أن يتم لهم ما وهبهم من نور العقول والإيمان حتى يصلوا إليه، ويحلوا بساحته [وهذه الأمور لو قارنها الإيمان وبنيت عليه لأثمرت الرُّقِيَّ العالي والحياة الطيبة، ولكنها لما بني كثير منها على الإلحاد لم تثمر إلا هبوط الأخلاق وأسباب الفناء والتدمير] \cite{tafsir_Saadi}.

وهذا فيه البيان الكافي في أن الطريق الواضح والسليم للرقي بالحضارة بما يرضي الله لا يكون إلا بالأخذ بالعلم الشرعي مع العلم السببي والتوجه إلى الله بالتوحيد والإخلاص والتقوى والعمل الصالح. وهذا ما يجب على الإنسان أن يعمل به ويعمل على تحقيقه ويتوكل على الله في ذلك كله.\comment{
ويقول الشيخ ابن باز رحمه الله: ولكن على الأمة أن تتعلم أيضًا ما ينفعها في دنياها: من الصناعات النافعة، ومن الاستعانة بها على قتال الأعداء وجهاد الأعداء، فيتعلم شؤون الزراعة، ويتعلم شؤون استخراج خزائن الأرض: من البترول والمعادن وغير ذلك؛ حتى تستغني عن أعداء الله، وتستخرج من بطون الأرض ومن خزائن الأرض ما ينفعها {\footnotesize (فتاوى الدروس، ما حكم من ينكر تعلم العلوم الدنيوية؟)}.} ويقول الشيخ ابن باز رحمه الله: لا شكَّ أن الدراسة للعلوم الدنيوية أمرٌ مطلوبٌ، والله يقول سبحانه: وَأَعِدُّوا لَهُمْ مَا اسْتَطَعْتُمْ مِنْ قُوَّةٍ [الأنفال:60]، فالمسلمون بحاجةٍ إلى العلوم الدنيوية حتى يستعينوا بها على طاعة الله، وعلى الاستغناء عمَّا في أيدي الناس، وعلى جهاد أعداء الله، مع كون المسلمين يتعلَّمون الجيولوجيا والهندسة والطب وغير ذلك مما يُعينهم، وكذلك ما يخترعون في القوة التي يُجاهد بها الأعداء؛ كل ما يُعينهم على جهاد الأعداء واتِّقاء شرِّ الأعداء ويغنيهم عن الأعداء فهو أمرٌ مطلوبٌ: يَا أَيُّهَا الَّذِينَ آمَنُوا خُذُوا حِذْرَكُمْ [النساء:71]، وَأَعِدُّوا لَهُمْ مَا اسْتَطَعْتُمْ مِنْ قُوَّةٍ [الأنفال:60]. فالاشتغال بالعلوم الدنيوية التي تنفع المسلمين إن كان لله؛ أُجِرَ عليها، مع فائدتها العظيمة، وإن كان يتعلَّمها للدنيا ليستفيد في دنياه فهذا مباحٌ ولا يضرُّه ذلك.
لكن العلوم الدينية أهم، فليأخذ منها بنصيبٍ، ويجتهد في تعلم دينه، والتَّفقه في دينه، ثم مع ذلك يتعلم ما ينفعه في دنياه إذا استطاع ذلك، وإذا جمع بين الأمرين فهو خيرٌ إلى خيرٍ، يقول النبيُّ ﷺ: مَن يُرد الله به خيرًا يُفقهه في الدين، فإذا تفقَّه في دينه واستفاد مع ذلك في دنياه طبًّا أو صنعةً أخرى تنفعه أو أشياء مما ينفع العبدَ في هذه الدنيا؛ فذلك خيرٌ إلى خيرٍ [.] فتعلم العلوم الدنيوية أمرٌ مفيدٌ ونافعٌ، بشرط ألا يشغل عن علم الآخرة وعمَّا ينفعه في الآخرة، فإن جمع بينهما فقد جمع خيرًا إلى خيرٍ، وإن صلحت نيته في علوم الدنيا كانت عبادةً، وإذا تعلَّمها للدنيا فليس له ولا عليه، تعلم شيئًا مباحًا، لا حرج عليه، لكن متى صلحت نيته وأراد بهذا نفع المسلمين وتقويتهم ضدّ عدوهم؛ جمع الله له الخيرين: الأجر، ومع ذلك النَّفع بهذا المُتعلَّم {\footnotesize (فتاوى الدروس، ما حكم وأهمية دراسة العلوم الدنيوية؟)}.


ومن الأمثلة على تقدم الأمم السابقة في العلم السببي (ولو بالنسبة لقريش والعرب) والذي كان سببا لتكبرها وتجبرها على أمر الله الشرعي وكفرها بالأنبياء والرسل هم عاد قوم هود عليه السلام حيث قال تعالى فيهم:
\quranayah*[46][26]{\footnotesize \surahname*[46]}. وهذا فيه أن الله جل جلاله يسر لعاد أسباب التمكين في الدنيا على نحو لم يمكن به غيرهم من العرب وكفار قريش كما جاء في عدة تفاسير. إلا أن هذا التمكين لم يكن سببا لهدايتهم رغم ما كان لهم من سمع وأبصار وقلوب ولكنهم كفروا واستهزؤا بأمر الله فغضب الله عليهم وأنزل عليهم عذابه. وهذا الأمر قد تكرر مع العديد من الأقوام السابقة كما في قوله تعالى: \quranayah*[6][6]{\footnotesize \surahname*[6]}. وقوله تعالى: \quranayah*[30][9]{\footnotesize \surahname*[30]}. فهذه هي سنة الله ودأبه في الأمم السابقين التي كذبت رسلها كما سيأتي بيان ذلك في حال الأمم مع الحق والميزان.

\comment{
    ومن الأمم السابقة التي تقدمت في العلم السببي وبالأخص علم الهندسة والبناء وكان سببا في تكبرها وكفرها هم الفراعنة قوم فرعون حيث قال تعالى: 
    \quranayah*[40][36-37]{\footnotesize \surahname*[40]}. فلا زلنا نرى بقايا معمار الأهرامات والتي صنفها المؤرخون أنها أحد عجائب الدنيا الباقية والتي لا زال الباحثون يسعون لكشف ألغازها إلى يومنا هذا. وهذا فيه أن الفراعنة كان لهم من العلم والقوة في نقل وصقل وتصميم وتنفيذ البناء المعقد والذي لم يستطع الإنسان الحديث صنع مثله إلا بعد فترة طويلة من الزمن. يذكر المؤرخون أن الهرم الأكبر في مصر كان أطول هيكل في العالم  بإرتفاع يصل إلى 146 متر لأكثر من 3800 عام حتى بناء كاتدرائية لينكولن في إنجلترا في عام 1311م.
}

وأما الجن فلم يكن لهم التقدم في الحضارة وهذا لنقص عقولهم في إدراك العلم السببي. ولهذا كان كل الأنبياء والرسل من البشر فهم أكمل عقلأ وأعلى فكرا وأعظم علما. وهذا لأن الجن لا يدركون من الأسباب ما يمكن للبشر إدراكه ومن ذلك ما بينه جل جلاله في قوله عن الجن: 
\quranayah*[34][14]{\footnotesize \surahname*[34]}. حيث أنهم خدموا سليمان عليه السلام وهو ميت ظنا منهم أنه حيا. فلم يدركوا بعقولهم الناقصة أنه إذا لم يتحرك لفترة طويلة من الزمن فقد مات، وهذا ما يستطيع إدراكه العاقل بل وحتى الطفل من البشر بسهولة. وقد جاء في تفسير السعدي بيان ذلك أن الجن كانوا قد موهوا على الإنس، وأخبروهم أنهم يعلمون الغيب، ويطلعون على المكنونات، فأراد اللّه تعالى أن يُرِيَ العباد كذبهم في هذه الدعوى، فمكثوا يعملون على عملهم، وقضى اللّه الموت على سليمان عليه السلام، واتَّكأ على عصاه، وهي المنسأة، فصاروا إذا مروا به وهو متكئ عليها، ظنوه حيا، وهابوه. فغدوا على عملهم كذلك سنة كاملة على ما قيل، حتى سلطت دابة الأرض على عصاه، فلم تزل ترعاها، حتى باد وسقط فسقط سليمان عليه السلام وتفرقت الشياطين وتبينت الإنس أن الجن (لَوْ كَانُوا يَعْلَمُونَ الْغَيْبَ مَا لَبِثُوا فِي الْعَذَابِ الْمُهِينِ) وهو العمل الشاق عليهم، فلو علموا الغيب، لعلموا موت سليمان، الذي هم أحرص شيء عليه، ليسلموا مما هم فيه \cite{tafsir_Saadi}. 

وكل هذا فيه البيان من الله جل جلاله للناس أن الجن ليس فقط لا يعلمون الغيب بل هم أنقص عقلا وفكرا وإن كانوا أكثر قوة بطبيعة مادة خلقهم وهي النار. إلا أن الإنسان قادر على إدراك الأسباب التي تمكنه من التفوق على الجن في القوة بالعلم السببي، ومن ذلك قصة عرش بالقيس حيث قال تعالى مخبرا عن سليمان عليه السلام:
\quranayah*[27][38-40]{\footnotesize \surahname*[27]}. وقد جاء في تفسير السعدي أن هذا الذي عنده علم من الكتاب هو رجل عالم صالح عند سليمان يقال له: "آصف بن برخيا" كان يعرف اسم الله الأعظم الذي إذا دعا الله به أجاب وإذا سأل به أعطى، بأن يدعو الله بذلك الاسم فيحضر حالا وأنه دعا الله فحضر. فالله أعلم هل هذا المراد، أم أن عنده علما من الكتاب يقتدر به على جلب البعيد وتحصيل الشديد \cite{tafsir_Saadi}. والأقرب والله أعلى وأعلم أن آصف رحمه الله كان لديه هذا العلم السببي الذي به يقرب البعيد ولهذا فقد نسب جل جلاله فعله للعلم وهو العلم السببي ولم ينسب فعله لإيمانه أو دعائه كما هو الحال مع سائر الأنبياء مثل ينوس عليه السلام في قوله تعالى:
\quranayah*[21][88]{\footnotesize \surahname*[21]}. وهذا العلم الذي كان لدى آصف يسمى علم التنقل الآني للمحسوسات وإلى زماننا يبدو هذا العلم مستحيلا إدراكه إلا أنه يبقى علم سببي قد يدركه الإنسان إن علم أسبابه الموصلة إليه. ومن المعلوم أنه الإنسان في زماننا قد تمكن من تحقيق التنقل الآني لغير المحسوسات كالصوت والصور وكافة البيانات الرقمية وغيرها من الأشياء المستخدمة في وسائل الاتصال التي كانت تبدوا مستحيلة في الماضي القريب. فلك أن تتأمل في الأسباب الموصلة لهذا العلم العظيم الذي إن أدركه المسلمون لسبقوا كل الأمم والحضارات وكتب لهم التمكين إن أقاموا الحق والميزان مع الأخذ بهذا السبب العظيم.

 ولقد رغب النبي ﷺ أصحابه في العلم السببي الذي به يقاتل الأعداء ومن ذلك علم الرمي لحاجة المسلمين له فقال: مَن عَلِمَ الرَّمْيَ، ثُمَّ تَرَكَهُ، فليسَ مِنَّا، أوْ قدْ عَصَى {\footnotesize (صحيح مسلم، وصححه الألباني)}. ومن العلوم النافعة أيضا علم الفيزياء والكيمياء التي بها يعرف سلوك المواد وأسبابها للإستفادة منها والإنتفاع بها كالحديد فقد قال تعالى عنه: \quranayah*[57][25][12-18]{\footnotesize \surahname*[57]}. ولقد بين ذلك السعدي رحمه الله في تفسيره فقال: وهو ما يشاهد من نفعه في أنواع الصناعات والحرف، والأواني وآلات الحرث، حتى إنه قل أن يوجد شيء إلا وهو يحتاج إلى الحديد \cite{tafsir_Saadi}.ومن فضل الله ومنه على عباده أنه يفتح على من يشاء من هذا العلم السببي لعباده الصالحين القائمين بالميزان الشرعي فجعل لهم من أسباب التمكين ما يعجز عليه غيرهم وهذا لحكمته سبحانه، كما فتح على ذي القرنين  وعلى داوود وسليمان عليهما السلام وعلى آصف رحمه الله وعلى نبينا محمد ﷺ وعلى الصالحين من بعده من أصحابه رضي الله عنهم إلى زمان عمر بن عبدالعزيز رحمه الله ومن بعده هارون الرشيد رحمه الله وما سيأتي في آخر الزمان في عهد المهدي المنتظر وعيسى عليه السلام، وسيأتي بيان وتفصيل كل ذلك في فصل الحكم الرشيد. 


\comment{
وحث النبي ﷺ على الأخلاص في الأقوال والأعمال فقال: 
إنَّ أوَّلَ النَّاسِ يُقْضَى يَومَ القِيامَةِ عليه رَجُلٌ اسْتُشْهِدَ، فَأُتِيَ به فَعَرَّفَهُ نِعَمَهُ فَعَرَفَها، قالَ: فَما عَمِلْتَ فيها؟قالَ: قاتَلْتُ فِيكَ حتَّى اسْتُشْهِدْتُ، قالَ: كَذَبْتَ، ولَكِنَّكَ قاتَلْتَ لأَنْ يُقالَ: جَرِيءٌ، فقَدْ قيلَ، ثُمَّ أُمِرَ به فَسُحِبَ علَى وجْهِهِ حتَّى أُلْقِيَ في النَّارِ، ورَجُلٌ تَعَلَّمَ العِلْمَ، وعَلَّمَهُ وقَرَأَ القُرْآنَ، فَأُتِيَ به فَعَرَّفَهُ نِعَمَهُ فَعَرَفَها، قالَ: فَما عَمِلْتَ فيها؟قالَ: تَعَلَّمْتُ العِلْمَ، وعَلَّمْتُهُ وقَرَأْتُ فِيكَ القُرْآنَ، قالَ: كَذَبْتَ، ولَكِنَّكَ تَعَلَّمْتَ العِلْمَ لِيُقالَ: عالِمٌ، وقَرَأْتَ القُرْآنَ لِيُقالَ: هو قارِئٌ، فقَدْ قيلَ، ثُمَّ أُمِرَ به فَسُحِبَ علَى وجْهِهِ حتَّى أُلْقِيَ في النَّارِ، ورَجُلٌ وسَّعَ اللَّهُ عليه، وأَعْطاهُ مِن أصْنافِ المالِ كُلِّهِ، فَأُتِيَ به فَعَرَّفَهُ نِعَمَهُ فَعَرَفَها، قالَ: فَما عَمِلْتَ فيها؟قالَ: ما تَرَكْتُ مِن سَبِيلٍ تُحِبُّ أنْ يُنْفَقَ فيها إلَّا أنْفَقْتُ فيها لَكَ، قالَ: كَذَبْتَ، ولَكِنَّكَ فَعَلْتَ لِيُقالَ: هو جَوادٌ، فقَدْ قيلَ، ثُمَّ أُمِرَ به فَسُحِبَ علَى وجْهِهِ، ثُمَّ أُلْقِيَ في النَّارِ. {\footnotesize (صحيح مسلم)}.
}

\subsection{الميزان الغيبي}

أما القسم الثاني فهو الميزان الغيبي أو العلم الغيبي وهو علم غير ظاهر لا يعلمه بالكلية ولا يملك مفاتحه إلا الله جل جلاله كما في قوله تعالى: 
\quranayah*[6][59]{\footnotesize \surahname*[6]}. وقد أخبر النبي ﷺ عن عدد هذه المفاتيح فقال: مَفاتِيحُ الغَيْبِ خَمْسٌ، ثُمَّ قَرَأَ: (إنَّ اللَّهَ عِنْدَهُ عِلْمُ السَّاعَةِ) {\footnotesize (صحيح البخاري)}. فهو سبحانه  عالم الغيب والشهادة كما في قوله تعالى: 
\quranayah*[59][22]{\footnotesize \surahname*[59]}. ومن علم الغيب ما أذن الله بمعرفته ومنه ما لم يأذن سبحانه بمعرفته والدليل على هذا قوله تعالى: 
\quranayah*[72][26-27]{\footnotesize \surahname*[72]}. وقد صح عن النبي ﷺ أنه قال: اللهمّ إني أسألُك بكلِّ اسمٍ هو لك سميتَ به نفسَك أو أنزلتَه في كتابِك أو علمته أحدًا من خلقِك أو استأثرت به في علمِ الغيبِ عندك، أن تجعلَ القرآنَ العظيمَ ربيعَ قلبي ونورَ صدري وجلاءَ حزني وذهابَ همّي وغمّي {\footnotesize (صحيحه الألباني في السلسلة الصحيحة)}. وهذا فيه أن الله جل جلاله عنده علم لم يأذن بمعرفته لأحد من خلقه وهو العلم الذي استأثر به في علم الغيب عنده.

ومن الأمثلة على العلم الغيبي الذي لم يأذن الله بمعرفته كعلم موعد وقوع الساعة كما في قوله تعالى: 
\quranayah*[33][63]{\footnotesize \surahname*[33]}.
ولقد صح عن النبي وهو أشرف الناس أنه قال لجربيل هو أشرف الملائكة عن الساعة عندما سأله: فمتَى السَّاعةُ؟قالَ ما المسئولُ عنها بأعلمَ منَ السَّائلِ {\footnotesize (صحيح البخاري)}. ولهذا فقد أمر جل جلاله نبيه ببيان هذا الأمر وهو العلم الغيبي الذي لم يأذن الله بمعرفته في قوله تعالى:
\quranayah*[6][50]{\footnotesize \surahname*[6]}. وقوله تعالى:
\quranayah*[7][188]{\footnotesize \surahname*[7]}.

وأما العلم الغيبي الذي أذن الله بمعرفته فمنه ما علمه الله لأنبياءه ورسله بالوحي ومن ذلك الكتب المنزلة التي فيها من القصص الفائتة كما في قوله تعالى:
\quranayah*[11][49]{\footnotesize \surahname*[11]}. وغير ذلك من الأخبار الفائتة كفترة مكوث أهل الكهف وعددهم، وقصص الأنبياء عليهم السلام. ومن ذلك أيضا الأحداث القادمة كما في قوله تعالى:
\quranayah*[34][3]{\footnotesize \surahname*[34]}. وغير ذلك من الآيات والأحاديث التي تخبر بالأمور التي تحدث في المستقبل وفي آخر الزمان كنزول عيسى عليه السلام، وخروج الدابة، وإنهيار سد ذي القرنين، وخروج يأجوج ومأجوج، وخروج الشمس من مغربها. فكل ما سبق فيه الدليل الواضح على أن الله سبحانه وتعالى أذن بمعرفة بعض العلم الغيبي لمن شاء من خلقه ولم يأذن بمعرفة بعضه الآخر. والعلم الغيبي الذي لم يأذن الله بمعرفته أكثر مما علم الخلق كما في قوله تعالى: 
\quranayah*[17][85][9]{\footnotesize \surahname*[17]}. 

وعلم الغيب الذي أذن الله بمعرفته هو علم نسبي يتفاوت الخلق بمعرفته كل بحسب ما أضهره الله له وأذن له أن يعلم، فمنه ما يدركه الإنسان دون غيره إن علم أسبابه، ومنه ما يدركه الجن دون الإنسان، ومنه ما يدركه الدواب دون الجن والإنس، ومنه ما تدركه الملائكة دون الجن والإنس وسائر الدواب الأخرى. ومن ذلك أن الإنسان لا يرى الجن حيث قال تعالى: \quranayah*[7][27][17-25] {\footnotesize \surahname*[7]}. فدل ذلك أن الجن لهم بعد لا يراه الإنسان وهو البعد المكاني الرابع والذي يمكِّنهم من رؤية الإنس ويحجب الإنس عن رؤيتهم لأن الإنس لا يرون إلا الأبعاد المكانية الثلاثة. ولقد تبث بالحساب أن من كان لديه القدرة على الوصول لأبعاد مكانية أعلى، كانت له القدرة على الضهور والتشكل وإختراق ما دونها من الأبعاد عند طريق ذلك البعد. وهذا فيه أن الجن لهم القدرة على أن يتشكلون بصور مختلفة في الأبعاد الثلاثة وإختراقها وذلك لوجودهم في البعد الرابع. ولهذا كان للشياطين القدرة في الدخول في أجساد الإنس واتخاذ مجرى الدم فيها طريقا كما صح عن النبي ﷺ: إن الشيطانَ يجري من ابنِ آدمَ مجرَى الدمِ {\footnotesize (صحيح أبي داود وصححه الألباني)}. ومن ذلك أن الشياطين من الجن لهم القدرة على مس الإنس بمشيئة الله وقد شبه الله تعالى آكل الربا بذلك فقال جل في علاه: \quranayah*[2][275][1-13] {\footnotesize \surahname*[2]}. والأدلة في تشكل الشياطين من الجن كثيرة ومنها قصة أبو هريرة رضي الله مع الشيطان الذي أمسكه في صورة إنس في رمضان وكان يسرق الطعام من زكاة الفطر فقال له النبي ﷺ: تعلم مَن تخاطبُ منذ ثلاثِ ليالٍ يا أبا هريرةَ؟ قلتُ: لا، قال: ذاك الشيطانُ {\footnotesize (صحيح البخاري)}. ومن ذلك أيضا الجنازة فقد صح عن النبي ﷺ أنه قال: إذَا وُضِعَتِ الجِنَازَةُ، واحْتَمَلَهَا الرِّجَالُ علَى أَعْنَاقِهِمْ، فإنْ كَانَتْ صَالِحَةً، قالَتْ: قَدِّمُونِي، وإنْ كَانَتْ غيرَ صَالِحَةٍ، قالَتْ: يا ويْلَهَا أَيْنَ يَذْهَبُونَ بهَا؟ يَسْمَعُ صَوْتَهَا كُلُّ شيءٍ إلَّا الإنْسَانَ، ولو سَمِعَهُ صَعِقَ {\footnotesize (صحيح البخاري)}. ومن ذلك أيضا عذاب القبر فقد حجب عن الجن والإنس دون سائر الدواب ولهذا قال النبي ﷺ: فيضربُهُ بِها ضربةً يسمَعُها ما بينَ المشرقِ والمغربِ إلَّا الثَّقلينِ {\footnotesize (أخرجه أبو داود وصححه الألباني)}. ومن ذلك أيضا أن الثقلين لا يسمعان الملائكة التي تنادي وقت الشروق والغروب كل يوم حيث قال النبي ﷺ: ما طلعت شمسٌ قط إلا بُعِثَ بجنبتَيْها مَلَكَانِ يُناديانِ، يُسْمِعَانِ أهلَ الأرضِ إلا الثَّقليْنِ، يا أيها الناسُ هلمُّوا إلى ربكم، فإن ما قلَّ و كفى خيرٌ مما كثُرَ و ألهى، و لا آبت شمسٌ قط إلا بُعِثَ بجنبتيْها مَلَكَانِ يُناديانِ، يُسْمِعَانِ أهلَ الأرضِ إلا الثَّقلينِ، أللهم أعطِ منفقًا خلفًا، و أعطِ ممسكًا مالًا تلفًا {\footnotesize (صححه الألباني في السلسلة الصحيحة)}. ومن ذلك أيضا ذهاب الشمس للسجود تحت العرش كل يوم عند غروبها كما صح ذلك عن النبي ﷺ أنه قال: حِينَ غَرَبَتِ الشَّمْسُ: أتَدْرِي أيْنَ تَذْهَبُ؟ قُلتُ: اللَّهُ ورَسولُهُ أعْلَمُ، قالَ: فإنَّهَا تَذْهَبُ حتَّى تَسْجُدَ تَحْتَ العَرْشِ ، فَتَسْتَأْذِنَ، فيُؤْذَنُ لَهَا، ويُوشِكُ أنْ تَسْجُدَ، فلا يُقْبَلَ منها، وتَسْتَأْذِنَ فلا يُؤْذَنَ لَهَا، يُقَالُ لَهَا: ارْجِعِي مِن حَيْثُ جِئْتِ، فَتَطْلُعُ مِن مَغْرِبِهَا، فَذلكَ قَوْلُهُ تَعَالَى: (وَالشَّمْسُ تَجْرِي لِمُسْتَقَرٍّ لَهَا ذَلِكَ تَقْدِيرُ الْعَزِيزِ الْعَلِيمِ) [يس: 38] {\footnotesize (صحيح البخاري)}. وكل هذا فيه أن الثقلين من الإنس والجن يحجب عنهم العديد من الأمور الغيبة.

ومن الأمور التي حجبت على الإنس والجن هي الروح فهي من العلم الذي لا يدركه لا الإنس ولا الجن ولا سائر الدواب الأخرى ولهذا فقد قال تعالى: \quranayah*[17][85]{\footnotesize \surahname*[17]}. ولكن يدركه من المخلوقات ملك الموت الموكل بالروح ولهذا قال تعالى: \quranayah*[32][11]{\footnotesize \surahname*[32]}. وهذا فيه أن الملائكة يصلون إلى أرواح الجن والإنس وسائر ذوات الأرواح الأخرى فدل على قدرتهم للوصول إلى البعد الثالث والرابع وما فوقها من الأبعاد إلى ما شاء الله كل بحسب ما وكلهم الله به. ولهذا فإن للملائكة أيضا القدرة على التشكل في صور مختلفة في الأبعاد الثلاثة كما جاء ذلك في قوله تعالى عن جبريل عليه السلام: 
\quranayah*[19][17][5]{\footnotesize \surahname*[19]}. ومن ذلك أيضا أن جبريل عليه السلام كان يأتي النبي ﷺ في صورة رجل كما صح ذلك عن عمر بن الخطاب أنه قال: بينما نحن عند رسولِ اللهِ ذات يومٍ، إذ طلع علينا رجلٌ شديدُ بياضِ الثيابِ، شديدُ سوادِ الشعرِ، لا يُرى عليه أثرُ السفرِ ولا يعرفه منا أحدٌ، حتى جلس إلى رسولِ اللهِ، فأسند ركبتيه إلى ركبتيه، ووضع كفيه على فخِذيه، ثم قال : يا محمدُ أخبرني عن الإسلامِ؟ الحديث، إلى أن قال النبي ﷺ: يا عمرُ هل تدري من السائلُ قلتُ: اللهُ ورسولُه أعلمُ قال: فإنه جبريلُ عليه السلام أتاكم ليعلمَكم أمرَ دينِكم {\footnotesize (صحيح النسائي، صححه الألباني)}. وقد صح عن أبو هريرة وأبو ذر أنهم رأوا هذا الرجل أي جبريل عليه السلام في صورة الصحابي دحية بن خليفة الكلبي رضي الله عنه لحسن صورته. فسألوا النبي ﷺ عن ذلك فقال لهم: إنه لجبريلُ عليه السلامُ نزل في صورة دِحيةَ الكلبيِّ {\footnotesize (صحيح النسائي، صححه الألباني)}. وكان جبريل يخاطب النبي من بعده فيكشف نفسه للنبي ﷺ فيراه ولا يراه غيره ومن ذلك ما صح عن عائشة أم المؤمنين أنَّ النبيَّ صَلَّى اللهُ عليه وسلَّمَ، قالَ لَهَا: يا عَائِشَةُ هذا جِبْرِيلُ يَقْرَأُ عَلَيْكِ السَّلَامَ، فَقالَتْ: وعليه السَّلَامُ ورَحْمَةُ اللَّهِ وبَرَكَاتُهُ، تَرَى ما لا أرَى، تُرِيدُ النبيَّ صَلَّى اللهُ عليه وسلَّمَ {\footnotesize (صحيح البخاري)}.

ومن الأمور الغيبية التي أذن الله لنبيه رؤيتها كالأمور الغيبية في المستقبل من البعد الزماني كما جاء في المرأة التي كانت تلقُطُ القَذى منَ المسجِدِ فتوُفِّيَت، فقال النبي ﷺ فيها: إنِّي رأَيْتُها في الجنَّةِ {\footnotesize (حسنه المنذري في الترغيب والترهيب، وذكر بمعناه الألباني في السلسلة الضعيفة)}. ومن ذلك أيضا أخبار النبي ﷺ برؤيته لبعض المشركين في النار ومن ذلك قوله ﷺ: رَأَيْتُ عَمْرَو بنَ عامِرٍ الخُزاعِيَّ يَجُرُّ قُصْبَهُ في النَّارِ، وكانَ أوَّلَ مَن سَيَّبَ السُّيُوبَ {\footnotesize (صحيح مسلم)}. 

ومن المعلوم أن العلم الغيبي لا يمكن إدراكه بالكلية بالعلم السببي وإنما يدرك منه فقط ما أذن الله بمعرفته فأظهره وجعل أسبابه وعلامته واضحة ومنتظمة لكل عاقل لما في ذلك من مصالح دينية ودنيوية. فمثلا يتنبأ بالمطر من الغيم الأسود، وبالإنجاب من علامات الحمل، وبنهاية الشهر بمنازل القمر، وبطلوع الزرع بعد نزول المطر، وبطريق السير من سير النجم، وغير ذلك من الأسباب التي جعلها الله دلالات واضحة على ما ستئول إليه الأشياء والأحوال في المستقبل ولو على وجه التقريب. وهنا يأتي علم الحساب حيث به تحسب المقادير والأوزان والأحجام والأشكال والألوان والأصوات والحركات والأمكنة والأزمنة وغير ذلك من الأشياء التي تعرف بها الأسباب والقوانين التي جعلها سبحانه في هذا الكون. وتتفاوت هذه الأمور في إمكانية حسابها فمنها ما يستحيل حسابه كالساعة سواء الكبرى أو الصغيرى، ومنها ما يصعب حسابه كالمطر، ومنها ما يسهل ويعرف حسابه كأيام الشهر وساعات اليوم. 

فكل هذه الأسباب مرجعها للعلم السببي ويدرك بها فقط جزء من العلم الغيبي الذي أذن الله به ومثال ذلك أن ذي القرنين بخبرته بأسباب الحديد علم أن مئال سده إلى الإنهيار لما علمه من تآكل الحديد كما في قوله تعالى: 
\quranayah*[18][98]{\footnotesize \surahname*[18]}. ومن ذلك أيضا علم التسيير الذي به يتنبأ بالمكان والزمان لمعرفة الطريق بإستخدام مواضع النجوم كما في قوله تعالى: 
\quranayah*[6][97]{\footnotesize \surahname*[6]}. وقد فسر السعدي ذلك أن الله جعل النجوم هداية للخلق إلى السبل، التي يحتاجون إلى سلوكها لمصالحهم، وتجاراتهم، وأسفارهم. منها: نجوم لا تزال ترى، ولا تسير عن محلها، ومنها: ما هو مستمر السير، يعرف سيرَه أهل المعرفة بذلك، ويعرفون به الجهات والأوقات. ودلت هذه الآية ونحوها، على مشروعية تعلم سير الكواكب ومحالّها الذي يسمى علم التسيير، فإنه لا تتم الهداية ولا تمكن إلا بذلك \cite{tafsir_Saadi}. وكل هذا فيه الدلالة الواضحة على قدرة الله جل جلاله وعظيم سلطانه، ولهذا فقد بين جل جلاله عظم مواقع النجوم فقال تعالى:  
\quranayah*[56][75-76]{\footnotesize \surahname*[56]}. وقد جاء في تفسير الطبري عن قتادة ومجاهد أن مواقع النجوم هي منازلها ومساقطها، ولقد أوَّل ذلك ابن عباس وعكرمة ومجاهد أن مواقع النجوم هي آيات القرآن نزلت متفرقة.\footnote{ولقد رجح الطبري القول الأول وهو أن مواقع النجوم هي منازلها ومساقطها في السماء، وهذا بخلاف القول الثاني أن مواقع النجوم هو نزول آيات القرآن متفرقة، وإن كان القولات لا يتعارضان بالضرورة.}

ولهذا فإن التنبأ بالمستقبل بالحساب لا يكون دائما صحيحا 
أو دقيقا وبالأخص في الأمور التي يصعب حسابها، فأمر الله الواقع قد يحجب عن خلقه لحكمته ومثال ذلك قوم هود إذ ظنوا أن الغيم الأسود علامة للمطر كما هو معتاد ولكنه كان عذاب الله كما في قوله تعالى: 
\quranayah*[46][24-25]{\footnotesize \surahname*[46]}. لهذا فإن الوقائع المستقبلية يختلف حسابها بحسب ما أذن الله بعرفته ومعرفة أسبابه ومقاديره، فالتي لا تخضع لنمط معين ومعروف لا يمكن التنبأ بها إلا على وجه التقريب. وتزداد دقة الحساب مع الزيادة في معرفة هذه الأسباب والمقادير والتي تتأتى بالبحث والتجربة والقياس والملاحظة وكل ذلك ممكن بالرجوع لآيات الله الكونية التي منها يتعلم الحساب. 

ومن فضل الله ومنه أنه اختص بهذا العلم الغيبي من شاء من أنبياءه ورسله كل حسب حاله ومقامه. ومن ذلك ما فتح الله به على الخضر عليه السلام حيث علم من الأمور الغيبية ما لم يعلمه موسى عليه السلام. فكان الخضر عليه السلام أعلم من موسى في العلم الغيبي وكان موسى عليه السلام أعلم من الخضر في العلم الديني، وكل منهما كان نبيا ويأتيه الوحي من الله جل جلاله ولكن تفاوتوا في نوع العلم والفضل فموسى عليه السلام من أولى العزم من الرسل وكلم الله موسى تكليما، فكل بحسب مقامه وما فضله الله به كما في قوله تعالى:
\quranayah*[2][253][1-21]{\footnotesize \surahname*[2]}.

\comment{
وقد فسر النبي ﷺ قصة موسى مع الخضر عليهم السلام كما صح ذلك عن أبي بن كعب عندما سأل بن عباس رضي الله عنهم كما جاء ذلك في صحيح البخاري. وتبدأ هذه القصة أنه ذات يوم قَامَ مُوسَى خَطِيبًا في بَنِي إسْرَائِيلَ، فَسُئِلَ: أيُّ النَّاسِ أعْلَمُ؟فَقالَ: أنَا، فَعَتَبَ اللَّهُ عليه؛ إذْ لَمْ يَرُدَّ العِلْمَ إلَيْهِ، فأوْحَى اللَّهُ إلَيْهِ: إنَّ لي عَبْدًا بمَجْمَعِ البَحْرَيْنِ هو أعْلَمُ مِنْكَ. فلما إلتقى موسى بالخضر 
فَسَلَّمَ عليه مُوسَى، فَقالَ الخَضِرُ: وأنَّى بأَرْضِكَ السَّلَامُ؟قالَ: أنَا مُوسَى، قالَ: مُوسَى بَنِي إسْرَائِيلَ؟قالَ: نَعَمْ، أتَيْتُكَ لِتُعَلِّمَنِي ممَّا عُلِّمْتَ رَشَدًا، قالَ: (إِنَّكَ لَنْ تَسْتَطِيعَ مَعِيَ صَبْرًا) [الكهف: 67]، يا مُوسَى، إنِّي علَى عِلْمٍ مِن عِلْمِ اللَّهِ عَلَّمَنِيهِ، لا تَعْلَمُهُ أنْتَ، وأَنْتَ علَى عِلْمٍ مِن عِلْمِ اللَّهِ عَلَّمَكَهُ اللَّهُ، لا أعْلَمُهُ. \footnote{وهذا فيه أن الخضر لديه من العلم الغيبي الذي لا يعلمه موسى وأن موسى لديه من العلم الديني الذي لا يعلمه الخضر.} 

فأصر موسى على اتباعه فَقالَ: (سَتَجِدُنِي إِنْ شَاءَ اللَّهُ صَابِرًا وَلَا أَعْصِي لَكَ أَمْرًا) [الكهف: 69]، فَقالَ له الخَضِرُ: (فَإِنِ اتَّبَعْتَنِي فَلَا تَسْأَلْنِي عَنْ شَيْءٍ حَتَّى أُحْدِثَ لَكَ مِنْهُ ذِكْرًا) [الكهف: 70]، فَانْطَلَقَا يَمْشِيَانِ علَى سَاحِلِ البَحْرِ، فَمَرَّتْ سَفِينَةٌ، فَكَلَّمُوهُمْ أنْ يَحْمِلُوهُمْ، فَعَرَفُوا الخَضِرَ، فَحَمَلُوهُمْ بغيرِ نَوْلٍ، فَلَمَّا رَكِبَا في السَّفِينَةِ لَمْ يَفْجَأْ إلَّا والخَضِرُ قدْ قَلَعَ لَوْحًا مِن ألْوَاحِ السَّفِينَةِ بالقَدُومِ، فَقالَ له مُوسَى: قَوْمٌ قدْ حَمَلُونَا بغيرِ نَوْلٍ عَمَدْتَ إلى سَفِينَتِهِمْ فَخَرَقْتَهَا (لِتُغْرِقَ أَهْلَهَا لَقَدْ جِئْتَ شَيْئًا إِمْرًا * قَالَ أَلَمْ أَقُلْ إِنَّكَ لَنْ تَسْتَطِيعَ مَعِيَ صَبْرًا * قَالَ لَا تُؤَاخِذْنِي بِمَا نَسِيتُ وَلَا تُرْهِقْنِي مِنْ أَمْرِي عُسْرًا) [الكهف: 71 - 73]، قالَ: وقالَ رَسولُ اللَّهِ صَلَّى اللهُ عليه وسلَّمَ: وكَانَتِ الأُولَى مِن مُوسَى نِسْيَانًا، قالَ: وجَاءَ عُصْفُورٌ، فَوَقَعَ علَى حَرْفِ السَّفِينَةِ، فَنَقَرَ في البَحْرِ نَقْرَةً، فَقالَ له الخَضِرُ: ما عِلْمِي وعِلْمُكَ مِن عِلْمِ اللَّهِ إلَّا مِثْلُ ما نَقَصَ هذا العُصْفُورُ مِن هذا البَحْرِ. \footnote{وهذا فيه أن الإنسان لم يؤتى من العلم إلا قليلا كما تقدم بيانه.}

ثُمَّ خَرَجَا مِنَ السَّفِينَةِ، فَبيْنَا هُما يَمْشِيَانِ علَى السَّاحِلِ إذْ أبْصَرَ الخَضِرُ غُلَامًا يَلْعَبُ مع الغِلْمَانِ، فأخَذَ الخَضِرُ رَأْسَهُ بيَدِهِ، فَاقْتَلَعَهُ بيَدِهِ فَقَتَلَهُ، فَقالَ له مُوسَى: (أَقَتَلْتَ نَفْسًا زَاكِيَةً) (بِغَيْرِ نَفْسٍ لَقَدْ جِئْتَ شَيْئًا نُكْرًا * قَالَ أَلَمْ أَقُلْ لَكَ إِنَّكَ لَنْ تَسْتَطِيعَ مَعِيَ صَبْرًا) [الكهف: 74 - 75]، قالَ: وهذِه أشَدُّ مِنَ الأُولَى، قالَ: (قَالَ إِنْ سَأَلْتُكَ عَنْ شَيْءٍ بَعْدَهَا فَلَا تُصَاحِبْنِي قَدْ بَلَغْتَ مِنْ لَدُنِّي عُذْرًا * فَانْطَلَقَا حَتَّى إِذَا أَتَيَا أَهْلَ قَرْيَةٍ اسْتَطْعَمَا أَهْلَهَا فَأَبَوْا أَنْ يُضَيِّفُوهُمَا فَوَجَدَا فِيهَا جِدَارًا يُرِيدُ أَنْ يَنْقَضَّ) [الكهف: 76، 77]، قالَ: مَائِلٌ، فَقَامَ الخَضِرُ فأقَامَهُ بيَدِهِ، فَقالَ مُوسَى: قَوْمٌ أتَيْنَاهُمْ فَلَمْ يُطْعِمُونَا ولَمْ يُضَيِّفُونَا، (لَوْ شِئْتَ لَاتَّخَذْتَ عَلَيْهِ أَجْرًا) [الكهف: 77]، قالَ: (قَالَ هَذَا فِرَاقُ بَيْنِي وَبَيْنِكَ) إلى قَوْلِهِ: (ذَلِكَ تَأْوِيلُ مَا لَمْ تَسْطِعْ عَلَيْهِ صَبْرًا) [الكهف: 78 - 82]، فَقالَ رَسولُ اللَّهِ صَلَّى اللهُ عليه وسلَّمَ: وَدِدْنَا أنَّ مُوسَى صَبَرَ حتَّى يَقُصَّ اللَّهُ عَلَيْنَا مِن خَبَرِهِما. 
}

وقد جاء في تفسيير بن كثير عن بن عباس أن النبي ﷺ قال أن موسى قال للخضر: جئتك لتعلمني مما علمت رشدا. قال: يكفيك التوراة بيدك، وأن الوحي يأتيك. يا موسى، إن لي علما لا ينبغي لك أن تعلمه، وإن لك علما لا ينبغي لي أن أعلمه.\footnote{وهذا فيه أن الخضر لديه من العلم الغيبي الذي لا يعلمه موسى وأن موسى لديه من العلم الديني الذي لا يعلمه الخضر.} وقال بن عباس رضي الله عنه عن الخضر عليه السلام أنه كان رجلا يعلم علم الغيب قد علم ذلك - فقال موسى: بلى. قال: (وكيف تصبر على ما لم تحط به خبرا)؟أي: إنما تعرف ظاهر ما ترى من العدل، ولم تحط من علم الغيب بما أعلم [هـ]. وجاء أيضا ما يأكد هذا المعنى في تفسير القرطبي: وعلمناه من لدنا علما أي علم الغيب، وقال ابن عطية: كان علم الخضر علم معرفة بواطن قد أوحيت إليه، لا تعطي ظواهر الأحكام أفعاله بحسبها;وكان علم موسى علم الأحكام والفتيا بظاهر أقوال الناس وأفعالهم [هـ].
وبينما الخضر وموسى عليهم السلام على السفينة، جَاءَ عُصْفُورٌ، فَوَقَعَ علَى حَرْفِ السَّفِينَةِ، فَنَقَرَ في البَحْرِ نَقْرَةً، فَقالَ له الخَضِرُ: ما عِلْمِي وعِلْمُكَ مِن عِلْمِ اللَّهِ إلَّا مِثْلُ ما نَقَصَ هذا العُصْفُورُ مِن هذا البَحْرِ {\footnotesize (صحيح البخاري)}.\footnote{وهذا فيه أن الإنسان لم يؤتى من العلم إلا قليلا كما تقدم في معنى قوله تعالى:
\quranayah*[17][85]{\footnotesize \surahname*[17]}.} 

وكل هذا فيه أن الخضر علم ما ستؤول إليه الأمور بما علَّمه الله له من علم الغيب فعلم بذلك أن السفينة لو لم تعاب لأخذها الملك غصبا من المساكين، وأن الغلام لو لم يقتل فسوف يرهق أبواه المؤمنين طغيانا وكفرا، وأن الجدار لو لم يقام لسقط ولسرق كنز الغلامين اليتيمين أبناء الرجل الصالح. وما تصرف الخضر في هذه الأمور إلا لعلمه أن جميع هذه الأمور سيقع في علم الغيب كما أوحى الله إليه ذلك. ولكن من رحمة الله ومنه فقد أذن للخضر أن يصلح ذلك ولهذا فقد قال: 
\quranayah*[18][82][25]{\footnotesize \surahname*[18]}. وبهذا يتبين أن الخضر إنما أوتي هذا العلم العظيم لمصلحة الناس وليس للإضرار بهم أي للإصلاح الدنيوي وهذا فيه بيان الرشد والذي به تجلب المصالح الدينية والدنيوية معا. وهذا ما أراد موسى تعلمه لهذا: 
\quranayah*[18][66]{\footnotesize \surahname*[18]}. وهذا فيه أن المصالح الدنيوية لا تجلب فقط بالعلم الديني الذي به يكون الإصلاح الديني بل إن ذلك يتطلب العلم الذي به يكون الإصلاح الدنيوي ولهذا فقد قال الخضر عليه السلام: 
 \quranayah*[18][67-68]{\footnotesize \surahname*[18]}. وهذا فيه أن الخضر عليه السلام اختص بالإصلاح الدنيوي مع ما كان لديه من الإصلاح الديني بينما موسى عليه السلام اختص بالإصلاح الديني فقط. وفيه أيضا أنه بالعلم والخبرة تجلب المصالح الدنيوية وبالعلم الديني تجلب المصالح الدينية وسيأتي تفصييل ذلك في فصل الحكم الرشيد بإذن الله. وفي هذه القصة بيان تدبير الله جل جلاله فإنه يعلم سبحانه ما كان وما سيكون وما لم يكن لو كان كيف يكون فسبحان الله الذي وسع كل شئ وأحاط به علما.

\section{أقسام الميزان الشرعي}

\subsection{الميزان الفطري}

فالميزان الفطري أو العلم الفطري فهو الذي يدرك بالفطرة السليمة الموافقة للعقل والتي فطر الله الناس عليها فيعرف به الخير من الشر والعدل من الظلم والإسلام من الكفر وغير ذلك من الأمور التي فطر الله الناس عليها والدليل على هذا أن النبيُّ صَلَّى اللهُ عليه وسلَّمَ قال: ما مِن مَوْلُودٍ إلَّا يُولَدُ علَى الفِطْرَةِ، فأبَوَاهُ يُهَوِّدَانِهِ أوْ يُنَصِّرَانِهِ، أوْ يُمَجِّسَانِهِ، كما تُنْتَجُ البَهِيمَةُ بَهِيمَةً جَمْعَاءَ، هلْ تُحِسُّونَ فِيهَا مِن جَدْعَاءَ، ثُمَّ يقولُ أبو هُرَيْرَةَ رَضِيَ اللَّهُ عنْه: (فِطْرَتَ اللَّهِ الَّتي فَطَرَ النَّاسَ عَلَيْهَا) [الروم: 30] الآيَةَ {\footnotesize (صحيح البخاري)}. ومن الأمور التي تخالف الميزان الفطري مثل الشرك (وبالأخص شرك الربوبية) والمجاهرة بالمعاصي والقتل واللواط والسرقة والغش في الكيل فهي أمور تدرك بالفطرة السليمة الموافقة للعقل ويمكن إثباتها لكل ذي عقل حتى بدون وحي. 
ومن ذلك أن أغلب الأمم مسلمة كانت أو كافرة اتفقت على فرض عقوبات على السرقة والغش على سبيل المثال لموافقة ذلك للفطرة السليمة. 

ولهذا فإن مخالفة الميزان الفطري هي أشد جرما من مخالفة الميزان الديني لأنها تعارض فطرة الله التي فطر الناس عليها والتي يمكن إدراكها حتى بدون وحي. ويعتبر الميزان الفطري أدنى مراتب العدل وفي معارضة هذا الميزان تعجيل سخط الله وعقوبته في الدنيا قبل الآخرة. والميزان الفطري ناقص وهو أدنى مرتبة من الميزان الديني حيث لا يمكن به إدراك العديد من الأمور الشرعية التي يحتاج إلى الوحي لإدراكها ومنها العقيدة والعبادات والأحكام الشرعية وغيرها من الأمور التي لا يمكن إدراكها بالفطرة السليمة فقط. ولهذا فقد أرسل الله جل جلاله الرسل وأنزل الكتب لبيان الميزان الديني والذي به يكتمل بيان الميزان الشرعي الذي أمر الله عباده به.

والميزان الفطري فيه الحجة لإدراك دين الإسلام لموافقته الفطرة كما سيأتي في بيان الميزان الديني. فقد جاء في الحديث القدسي عنِ اللهِ تعالى: إني خلقتُ عبادي حنفاءَ فاجتالتْهم الشياطينُ فحرَّمتْ عليهم ما أحللتُ لهم وأمرتهم أن يشركوا بي ما لم أُنزِّل به سلطانًا {\footnotesize (صحيح، مجموع الفتاوى لابن تيمية)}.
ولهذا فإن الشياطين لا تسعى لإفساد الميزان الديني فقط وأنما تسعى لإفساد الميزان الفطري والديني معا كما في قوله تعالى عن ابليس: 
\quranayah*[4][119]{\footnotesize \surahname*[4]}. وقد جاء بيان ذلك في تفسير السعدي أن الله تعالى خلق عباده حنفاء مفطورين على قبول الحق وإيثاره، فجاءتهم الشياطين فاجتالتهم عن هذا الخلق الجميل، وزينت لهم الشر والشرك والكفر والفسوق والعصيان. فإن كل مولود يولد على الفطرة ولكن أبواه يهوِّدانه أو ينصِّرانه أو يمجِّسانه، ونحو ذلك مما يغيرون به ما فطر الله عليه العباد من توحيده وحبه ومعرفته. فافترستهم الشياطين في هذا الموضع افتراس السبع والذئاب للغنم المنفردة، فخسروا الدنيا والآخرة، ورجعوا بالخيبة والصفقة الخاسرة. ولولا لطف الله وكرمه بعباده المخلصين لجرى عليهم ما جرى على هؤلاء المفتونين \cite{tafsir_Saadi}.

وقد جاء في تفسير ابن كثير أن ابن عباس قال: 
أتى علي زمان وأنا أقول: أولاد المسلمين مع أولاد المسلمين، وأولاد المشركين مع المشركين. حتى حدثني فلان عن فلان: أن رسول الله ﷺ سئل عنهم فقال: "الله أعلم بما كانوا عاملين". فأمسكت عن قولي [هـ]. وهذا فيه أن ابن عباس امسك عن قوله بفصل أولاد المسلمين عن أولاد المشركين في اللعب عندما علم قول الرسول ﷺ أن أولاد المشركين أيضا على الفطرة السمحة التي فطر الله الناس عليها. وهذا ما يوافق باقي الأحاديث والآيات كما تقدم. 

وجاء أيضا في تفسير ابن كثير عن الفطرة أنه لا يولد أحد إلا على ذلك، ولا تفاوت بين الناس في ذلك [هـ]. وبهذا يعلم أن المكلفين قد تساوا في الميزان الفطري عند نشأتهم وهذا من عدل الله إذ أعطاهم سبحانه الفطرة السليمة الموافقة للعقل حتى يدركوا بذلك الميزان الديني. ولكن هذه الفطرة قد تفسد فيضل صاحبها عند البلوغ فإن شاء الله هداه وإن شاء أزاغه وكل ذلك بهداية الله الكونية. ونقل هذا المعنى القرطبي في تفسيره عن شيخه أبو العباس قوله: قال شيخنا في عبارته: إن الله تعالى خلق قلوب بني آدم مؤهلة لقبول الحق، كما خلق أعينهم وأسماعهم قابلة للمرئيات والمسموعات، فما دامت باقية على ذلك القبول وعلى تلك الأهلية أدركت الحق ودين الإسلام وهو الدين الحق. وقد دل على صحة هذا المعنى قوله: كما تنتج البهيمة بهيمة جمعاء هل تحسون فيها من جدعاء يعني أن البهيمة تلد ولدها كامل الخلقة سليما من الآفات، فلو ترك على أصل تلك الخلقة لبقي كاملا بريئا من العيوب، لكن يتصرف فيه فيجدع أذنه ويوسم وجهه، فتطرأ عليه الآفات والنقائص فيخرج عن الأصل;وكذلك الإنسان، وهو تشبيه واقع ووجهه واضح [هـ]. فرحم الله علماء قرطبة من الأندلس الأسبانية الذين بينوا هذا المعنى العظيم.

\subsection{الميزان الديني}

وأما القسم الثاني من الميزان الشرعي فهو الميزان الديني أو العلم الديني وهو موافق للميزان الفطري ومكمل له. ويدرك العلم الديني بالوحي بالمنزل من عند الله تبارك وتعالى على الأنبياء والمرسلين عليهم الصلاة والسلام. ولهذا فقد أثبت سبحانه موافقة دينه للفطرة التي فطر الناس عليها في قوله تعالى: 
\quranayah*[30][30]{\footnotesize \surahname*[30]}. وهذه فيه أن الميزان الديني الذي أنزله الله كان ولا يزال موافقا للفطرة ومكملا لها وهو دين الإسلام الذي أرسلت به كل الرسل والأنبياء عليهم الصلاة والسلام من آدم عليه السلام إلى محمد ﷺ.

ومن الأمور التي تخالف الميزان الديني مثل الشرك (وبالأخص شرك الألوهية)، ومنع الزكاة، والحكم بغيير ما أنزل الله كتحريم ما أحل الله أو تحليل ما حرم الله وغير ذلك من الأمور التي تخالف أمر الله ورسوله والتي يمكن إدراكها بالوحي المنزل وبالحجة الواضحة والبينة إستنادا إلى جاء في كتاب الله عز وجل، أو صح في سنه نبيه الكريم، أو ثبت عن سبيل المؤمنين من السلف الصالحين. 

ولهذا فقد أمر الله عز وجل جميع الأنبياء لدعوة المشركين لعبادة الله وحده لا شريك له (أي إلى توحيد الألوهية) وإقامة الحجة عليهم بالميزان الفطري أي بإيمانهم بأن الله هو من خلقهم وخلق السموات والأرض وهو مدبر الكون (أي بإيمانهم بتوحيد الربوبية) كما في قوله تعالى:
\quranayah*[39][38]{\footnotesize \surahname*[39]}.
وفي قوله تعالى: 
\quranayah*[43][87]{\footnotesize \surahname*[43]}. فوصف الله جل جلاله هؤلاء بأن أكثرهم لا يعقلون في قوله:
\quranayah*[29][63]{\footnotesize \surahname*[29]}.
ووصفهم جل جلاله أيضا بأنهم لا يعلمون في قوله:
\quranayah*[31][25]{\footnotesize \surahname*[31]}.
؛ أولا لا يعقلون لمخالفتهم الميزان الفطري الموافق للعقل وثانيا لا يعلمون لمخالفتهم الميزان الديني وهو أمر الله المنزل من عنده. 

ولهذا فقد بين ذلك إبراهيم عليه السلام لأبيه آزر كما في قوله تعالى:
\quranayah*[19][42-45]{\footnotesize \surahname*[19]}. فحاجه أولا بالميزان الفطري الذي يقام بالحجة العقلية على بطلان عبادة ما لا يسمع ولا يبصر ولا ينفع ولا يضر، وحاجه ثانيا بالميزان الديني الذي يقام بالعلم الديني الصحيح المنزل من الله تبارك وتعالى وهو الوحي الموافق للفطرة والمكمل لها. وبين له الحكم الجزائي للشرك الموجب لعذاب الله فأقام عليه بذلك الحجة الكاملة والواضحة. 

والعلم الديني هو ما قام عليه الدليل من كتاب الله أو سنة نبيه ﷺ وهو الحق وهو الهدى الذي يهدي إلى الطريق المستقيم الذي يرضي الله جل جلاله ولهذا فقد قال النبي ﷺ: تركتُ فيكم أمرينِ؛ لن تَضلُّوا ما إن تمسَّكتُم بهما: كتابَ اللَّهِ وسُنَّتي، ولن يتفَرَّقا حتَّى يرِدا عليَّ الحوضَ {\footnotesize (صحيح الترغيب)}. وقد ضرب النبي ﷺ مثلا عن نفسه في بيان العلم الديني الذي جاء به فقال: مَثَلُ ما بَعَثَنِي اللَّهُ به مِنَ الهُدَى والعِلْمِ، كَمَثَلِ الغَيْثِ الكَثِيرِ أصابَ أرْضًا، فَكانَ مِنْها نَقِيَّةٌ، قَبِلَتِ الماءَ، فأنْبَتَتِ الكَلَأَ والعُشْبَ الكَثِيرَ، وكانَتْ مِنْها أجادِبُ، أمْسَكَتِ الماءَ، فَنَفَعَ اللَّهُ بها النَّاسَ، فَشَرِبُوا وسَقَوْا وزَرَعُوا، وأَصابَتْ مِنْها طائِفَةً أُخْرَى، إنَّما هي قِيعانٌ لا تُمْسِكُ ماءً ولا تُنْبِتُ كَلَأً، فَذلكَ مَثَلُ مَن فَقُهَ في دِينِ اللَّهِ، ونَفَعَهُ ما بَعَثَنِي اللَّهُ به فَعَلِمَ وعَلَّمَ، ومَثَلُ مَن لَمْ يَرْفَعْ بذلكَ رَأْسًا، ولَمْ يَقْبَلْ هُدَى اللَّهِ الذي أُرْسِلْتُ بهِ {\footnotesize (صحيح البخاري)}.

 وخير هذا العلم هو القرآن كتاب الله فقد قال النبي ﷺ: إنَّ أفْضَلَكُمْ مَن تَعَلَّمَ القُرْآنَ وعَلَّمَهُ {\footnotesize (صحيح البخاري)}. وقد كان ﷺ يعلم أصحابه القرآن ويدعوا لهم فعن عبد الله بن عباس قال: ضَمَّنِي إلَيْهِ النبيُّ صَلَّى اللهُ عليه وسلَّمَ، وقالَ: اللَّهُمَّ عَلِّمْهُ الكِتابَ، وفي رواية: اللَّهُمَّ عَلِّمْهُ الحِكْمَةَ {\footnotesize (صحيح البخاري)}. وقد كان النبي ﷺ يرغب أصحابه في حفظ كتاب الله عز وجل فقال: لا حَسَدَ إلَّا في اثْنَتَيْنِ: رَجُلٌ عَلَّمَهُ اللَّهُ القُرْآنَ، فَهو يَتْلُوهُ آناءَ اللَّيْلِ، وآناءَ النَّهارِ، فَسَمِعَهُ جارٌ له، فقالَ: لَيْتَنِي أُوتِيتُ مِثْلَ ما أُوتِيَ فُلانٌ، فَعَمِلْتُ مِثْلَ ما يَعْمَلُ، ورَجُلٌ آتاهُ اللَّهُ مالًا فَهو يُهْلِكُهُ في الحَقِّ، فقالَ رَجُلٌ: لَيْتَنِي أُوتِيتُ مِثْلَ ما أُوتِيَ فُلانٌ، فَعَمِلْتُ مِثْلَ ما يَعْمَلُ {\footnotesize (صحيح البخاري)}.

ولقد كلف سبحانه عباده بالإجتهاد بهذا العلم مع العمل 
\quranayah*[4][135][1-9]{\footnotesize \surahname*[4]}
\quranayah*[5][8][1-9]{\footnotesize \surahname*[5]}
وعدم التقلبد فيه وحجة لك أو عليك فقال جل جلال: 
\quranayah*[33][67]{\footnotesize \surahname*[33]}

والعلم الديني نافع وحجة لصاحبه مع العمل الموافق لأمر الله الشرعي، وغير نافع وحجة على صاحبه مع العمل المخالف لأمر الله الشرعي. ومن أعظم العمل المخالف لأمر الله للرياء كما جاء عن النبي ﷺ أنه قال لأبي هريرة: أولئكَ الثلاثةُ أولُ خلقِ اللهِ تُسعَّرُ بهمُ النارُ يومَ القيامةِ. وقال ابنُ رَجَب: (أوَّلُ من تُسَعَّرُ به النَّارُ مِن الموحِّدينَ العُبَّادُ المُراؤون بأعمالِهم، أوَّلُهم العالِمُ والمجاهِدُ والمتصَدِّقُ للرِّياءِ؛ لأنَّ يَسيرَ الرِّياءِ شِركٌ) {\footnotesize (كلمة الإخلاص)}. وقال ابن القيم: إنَّ النَّارَ أوَّلُ ما تسعَّرُ بالعالمِ والمنفِقِ والمقتولِ في الجِهادِ إذا فعلوا ذلِكَ ليقال {\footnotesize (زاد المعاد)}. يقول الشيخ العثيمين رحمه الله في ذلك: لا بُدَّ مِن العَملِ بالعِلمِ؛ لأنَّ ثمرةَ العِلمِ العَملُ؛ لأنَّه إذا لم يعمَلْ بعِلمِه صار مِن أوَّلِ مَن تُسعَّرُ بهم النَّارُ يومَ القيامةِ، وقد قيل: وعالِمٌ بعِلمِه لم يعمَلَنْ مُعذَّبٌ مِن قَبْلِ عُبَّادِ الوَثَنِ، فإذا لم يعمَلْ بعِلمِه أُورِث الفَشلَ في العِلمِ وعَدمَ البركةِ ونِسيانَ العِلمِ؛ لقولِ اللهِ تعالى:
\quranayah*[5][13]{\footnotesize \surahname*[13]} [هـ].

والناس يتفاوتون في هذا العلم الشرعي، وجتى الصحابة رضوان الله عليهم وما ذكره القرطبي في تفسير قوله تعالى: 
\quranayah*[27][22]{\footnotesize \surahname*[27]}
حيث قال: أي علمت ما لم تعلمه من الأمر فكان في هذا رد على من قال : إن الأنبياء تعلم الغيب [.] وفي الآية دليل على أن الصغير يقول للكبير والمتعلم للعالم: عندي ما ليس عندك، إذا تحقق ذلك وتيقنه. هذا عمر بن الخطاب رضي الله عنه مع علمه لم يكن عنده علم بالاستئذان. وكان علم التيمم عند عمار وغيره ، وغاب عن عمر وابن مسعود حتى قالا: لا يتيمم الجنب. وكان حكم الإذن في أن تنفر الحائض عند ابن عباس ولم يعلمه عمر ولا زيد بن ثابت. وكان غسل رأس المحرم معلوما عند ابن عباس وخفي عن المسور بن مخرمة . ومثله كثير فلا يطول به. 

والعلم الديني هو العلم الذي يرفع قبل قيام الساعة كما ثبت عن عبد الله بن مسعود وأبو موسى الأشرعي أن النبي ﷺ قال: 
إنَّ بيْنَ يَدَيِ السَّاعَةِ أيَّامًا، يُرْفَعُ فيها العِلْمُ، ويَنْزِلُ فيها الجَهْلُ، ويَكْثُرُ فيها الهَرْجُ والهَرْجُ: القَتْلُ {\footnotesize (صحيح البخاري)}. وفي حديث آخر عن أنس بن مالك أن النبي ﷺ قال: إنَّ مِن أشْراطِ السَّاعَةِ: أنْ يُرْفَعَ العِلْمُ ويَثْبُتَ الجَهْلُ، ويُشْرَبَ الخَمْرُ، ويَظْهَرَ الزِّنا {\footnotesize (صحيح البخاري)}. وعن أبي هريرة أن النبي ﷺ قال: لَا تَقُومُ السَّاعَةُ حتَّى يُقْبَضَ العِلْمُ، وتَكْثُرَ الزَّلَازِلُ، ويَتَقَارَبَ الزَّمَانُ، وتَظْهَرَ الفِتَنُ، ويَكْثُرَ الهَرْجُ - وهو القَتْلُ القَتْلُ - حتَّى يَكْثُرَ فِيكُمُ المَالُ فَيَفِيضَ {\footnotesize (صحيح البخاري)}. وعن أنس بن مالك أن النبي ﷺ قال: 
لَا تَقُومُ السَّاعَةُ وإمَّا قالَ: مِن أَشْرَاطِ السَّاعَةِ، أَنْ يُرْفَعَ العِلْمُ، وَيَظْهَرَ الجَهْلُ، وَيُشْرَبَ الخَمْرُ، وَيَظْهَرَ الزِّنَا، وَيَقِلَّ الرِّجَالُ، وَيَكْثُرَ النِّسَاءُ حتَّى يَكونَ لِلْخَمْسِينَ امْرَأَةً القَيِّمُ الوَاحِدُ {\footnotesize (صحيح البخاري)}. وعن عبدالله بن عمرو أن النبي ﷺ قال: إنَّ اللَّهَ لا يَقْبِضُ العِلْمَ انْتِزَاعًا يَنْتَزِعُهُ مِنَ العِبَادِ، ولَكِنْ يَقْبِضُ العِلْمَ بقَبْضِ العُلَمَاءِ، حتَّى إذَا لَمْ يُبْقِ عَالِمًا اتَّخَذَ النَّاسُ رُؤُوسًا جُهَّالًا، فَسُئِلُوا فأفْتَوْا بغيرِ عِلْمٍ، فَضَلُّوا وأَضَلُّوا {\footnotesize (صحيح البخاري)}. وهذا فيه أن العلم الذي يرفع إنما هو العلم الديني وليس العلم السببي.

\section{الغاية من إرسال الرسل وإنزال الكتب}

أرسل الله عز وجل رسله بالكتاب أولا لبيان الحق وهو العلم الشرعي الصحيح ومن ثم لإقامة الميزان الشرعي بالقسط والعدل بين الناس حيث يقول جل جلاله:
\quranayah*[57][25]{\footnotesize \surahname*[57]}. فالكتاب هو الحق كما في قوله تعالى: \quranayah*[2][144][21]{\footnotesize \surahname*[2]}. والقسط في الكيل والوزن والعدل بين الناس من إقامة الميزان الشرعي وهو من الأمور التي أوصى الله تعالى بها. وقد دلنا سبحانه في هذه الآية على الحديد والأخذ به لنصرة الله جل جلاله ورسله ونصرة الحق الذي جاءوا به. 

وقد جاء في تفسير الطبري: عن قتادة (الْكِتَابَ وَالْمِيزَانَ) قال: الميزان: العدل [.] وقال ابن زيد، في قوله: (وَأَنـزلْنَا مَعَهُمُ الْكِتَابَ وَالْمِيزَانَ) بالحق؛ قال: الميزان: ما يعمل الناس، ويتعاطون عليه في الدنيا من معايشهم التي يأخذون ويعطون، يأخذون بميزان، ويعطون بميزان، يعرف ما يأخذ وما يعطي. قال: والكتاب فيه دين الناس الذي يعملون ويتركون، فالكتاب للآخرة، والميزان للدنيا. وقوله: (لِيَقُومَ النَّاسُ بِالْقِسْطِ) يقول تعالى ذكره: ليعمل الناس بينهم بالعدل. وقوله: (وَأَنـزلْنَا الْحَدِيدَ فِيهِ بَأْسٌ شَدِيدٌ) يقول تعالى ذكره: وأنـزلنا لهم الحديد فيه بأس شديد، يقول: فيه قوّة شديدة، ومنافع للناس، وذلك ما ينتفعون به منه عند لقائهم العدوّ، وغير ذلك من منافعه [.] وقوله: (وَلِيَعْلَمَ اللَّهُ مَنْ يَنْصُرُهُ وَرُسُلَهُ بِالْغَيْبِ) يقول تعالى ذكره: أرسلنا رسلنا إلى خلقنا وأنـزلنا معهم هذه الأشياء ليعدلوا بينهم، وليعلم حزب الله من ينصر دين الله ورسله بالغيب [هـ]. 

وجاء في تفسير ابن كثير: (وأنزلنا معهم الكتاب) وهو: النقل المصدق (والميزان) وهو: العدل قاله مجاهد، وقتادة، وغيرهما. وهو الحق الذي تشهد به العقول الصحيحة المستقيمة المخالفة للآراء السقيمة [.] ولهذا قال في هذه الآية: (ليقوم الناس بالقسط) أي: بالحق والعدل وهو: اتباع الرسل فيما أخبروا به، وطاعتهم فيما أمروا به، فإن الذي جاءوا به هو الحق الذي ليس وراءه حق، كما قال: (وتمت كلمة ربك صدقا وعدلا) [الأنعام: 115] أي: صدقا في الإخبار، وعدلا في الأوامر والنواهي \.ولهذا يقول المؤمنون إذا تبوءوا غرف الجنات، والمنازل العاليات، والسرر المصفوفات: (الحمد لله الذي هدانا لهذا وما كنا لنهتدي لولا أن هدانا الله لقد جاءت رسل ربنا بالحق) [الأعراف: 43]. وقوله: (وأنزلنا الحديد فيه بأس شديد) أي: وجعلنا الحديد رادعا لمن أبى الحق وعانده بعد قيام الحجة عليه [.] ولهذا قال تعالى: (فيه بأس شديد) يعني: السلاح كالسيوف، والحراب، والسنان، والنصال، والدروع، ونحوها \.(ومنافع للناس) أي: في معايشهم كالسكة، والفأس، والقدوم، والمنشار، والإزميل، والمجرفة، والآلات التي يستعان بها في الحراثة، والحياكة، والطبخ، والخبز، وما لا قوام للناس بدونه، وغير ذلك. [.] وقوله: (وليعلم الله من ينصره ورسله بالغيب) أي: من نيته في حمل السلاح نصرة الله ورسله، (إن الله قوي عزيز) أي: هو قوي عزيز، ينصر من نصره من غير احتياج منه إلى الناس، وإنما شرع الجهاد ليبلو بعضكم ببعض [هـ].

 وفي هذا دليل على أن نصرة الله ورسله تكون بثلاثة أمور وهي: (1) إقامة الحق في نفوس الناس بالعلم الشرعي الصحيح، (2) إقامة الميزان الشرعي بين الناس بالعدل والقسط، (3) الأخذ بأسباب القوى كالحديد وما يلزم ذلك من علوم كالحساب والطب والفيزياء وغيرها من العلوم السببية التي تكمن المسلمين من دحر الأعداء ونشر الحق ونصرته. وهذه الأمور الثلاثة هي أركان الحكم الرشيد التي بها يكون التمكين كما سيأتي بيان ذلك في فصل الحكم الرشيد.

فالله جل جلاله أنزل كتابه لتحقيق هذه الغاية العظيمة وهي إقامة الحق والميزان الشرعي كما بين ذلك في قوله تعالى:
\quranayah*[42][17]{\footnotesize \surahname*[42]}.
فذكر الله الميزان إلحاقا بالحق لان الحق لا يكون إلا بالعلم الشرعي الصحيح وهو يقتضي الميزان الشرعي الذي لا يكون إلا بالعمل الصحيح الموافق للحق ومنه العدل والقسط كما أمر تعالى ومن ذلك بلا شك الحساب الصحيح. فإقامة الميزان الشرعي من الوصايا العشر من سورة الأنعام في قوله تعالى:
\quranayah*[6][152][12]{\footnotesize \surahname*[6]}. فقرن الله عز وجل في هذه الآيات بين الكيل والميزان والعدل في القول والوفاء بالعهد. وفيه أن الميزان والكيل لا يكون إلا بالقسط وهو العدل الظاهر. وفيه أن العدل ذكر مع ذا القربى ولذلك يكون العدل بين الناس.
وأيضا من الوصايا التي ذكرها الله في سورة الإسراء في قوله تعالى:
\quranayah*[17][35]{\footnotesize \surahname*[17]}. يقول السعدي في تفسيره:
وهذا أمر بالعدل وإيفاء المكاييل والموازين بالقسط من غير بخس ولا نقص. ويؤخذ من عموم المعنى النهي عن كل غش في ثمن أو مثمن أو معقود عليه والأمر بالنصح والصدق في المعاملة \cite{tafsir_Saadi}. ومن ذلك بلا الشك الحساب ولذلك وجب الوفاء والصدق فيه من غير غش ولا تضليل وإقامة الميزان فيه بالقسط كما في الكيل.

وإقامة الحق تكون بالعلم الشرعي الصحيح بيانه والعمل به في النفوس وفيه صلاح الآخرة وإقامة الميزان تكون بالعدل فيما بين النفوس وفيه صلاح الدنيا. ولهذا فقد قال ابن القيم رحمه الله: «أصل كل خير في الدنيا والآخرة هو العلم والعدل، وأصل كل شرٍ في الدنيا والآخرة الجهل والظلم [هـ] {\footnotesize (إغاثة اللهفان في مصايد الشيطان)}.

\section{تفاوت الرسل في العلم والفضل ودعوتهم واحدة}

لا شك أن الأنبياء والرسل قد تفاوتوا في العلم والفضل والدليل قوله تعالى: 
\quranayah*[2][253]{\footnotesize \surahname*[2]}. وقوله تعالى:
\quranayah*[17][55]{\footnotesize \surahname*[17]}. ومن أمثلة ذلك ما تقدم في التفاوت في العلم والفضل بين موسى عليه السلام والخضر عليه السلام. 

ولكن جميع الرسل دعوتهم واحدة وهي الدعوة إلى إقامة الحق والميزان ومن أعظم ذلك الدعوة إلى عبادة الله وحده لا شريك له كما في قوله تعالى: 
\quranayah*[16][36]{\footnotesize \surahname*[16]}. وقوله تعالى:
\quranayah*[21][25]{\footnotesize \surahname*[21]}. وهذه الدعوة إنما هي دين الإسلام الذي أرسل الله به جميع الرسل والأنبياء عليهم الصلاة والسلام. وهذا فيه أن الدعوة إلى الإسلام واحدة وهي دعوة إلى توحيد الألوهية والربوبية والأسماء والصفات والعبادات والأحكام الشرعية وغيرها من الأمور التي تدعو إليها الرسل والأنبياء عليهم الصلاة والسلام كما في قوله تعالى: 
\quranayah*[42][13]{\footnotesize \surahname*[42]}.

وبهذا يتبين أنه لا يمكن التفريق أو التفضيل بين دعوة الرسل فدعوتهم واحدة كما قال جل جلاله: \quranayah*[2][136]{\footnotesize \surahname*[2]}. وقال تعالى:
\quranayah*[2][285]{\footnotesize \surahname*[2]}.

ولهذا فقد نهى النبي ﷺ عن التفضيل بينه وبين الأنبياء على سبيل الإستنقاص أو على وجه التعصب أو التفريق أو التفاخر لما في ذلك بطر للحق ومن أعظم ذلك عدم إرجاع هذا الفضل لله تبارك وتعالى والذي بيده هذا التفضيل كما تقدم دليل ذلك، ومن ذلك حديث أبي هريرة رضي الله عنه أنه قال: بيْنَما يَهُودِيٌّ يَعْرِضُ سِلْعَةً له أُعْطِيَ بهَا شيئًا، كَرِهَهُ -أَوْ لَمْ يَرْضَهُ فقالَ: لَا، وَالَّذِي اصْطَفَى مُوسَى عليه السَّلَامُ علَى البَشَرِ، قالَ: فَسَمِعَهُ رَجُلٌ مِنَ الأنْصَارِ فَلَطَمَ وَجْهَهُ، قالَ: تَقُولُ: وَالَّذِي اصْطَفَى مُوسَى عليه السَّلَامُ علَى البَشَرِ وَرَسولُ اللهِ صَلَّى اللَّهُ عليه وَسَلَّمَ بيْنَ أَظْهُرِنَا؟! قالَ: فَذَهَبَ اليَهُودِيُّ إلى رَسولِ اللهِ صَلَّى اللَّهُ عليه وَسَلَّمَ، فَقالَ: يا أَبَا القَاسِمِ، إنَّ لي ذِمَّةً وَعَهْدًا، وَقالَ: فُلَانٌ لَطَمَ وَجْهِي، فَقالَ رَسولُ اللهِ صَلَّى اللَّهُ عليه وَسَلَّمَ: لِمَ لَطَمْتَ وَجْهَهُ؟قالَ: قالَ يا رَسولَ اللهِ: وَالَّذِي اصْطَفَى مُوسَى عليه السَّلَامُ علَى البَشَرِ، وَأَنْتَ بيْنَ أَظْهُرِنَا! قالَ: فَغَضِبَ رَسولُ اللهِ صَلَّى اللَّهُ عليه وَسَلَّمَ حتَّى عُرِفَ الغَضَبُ في وَجْهِهِ، ثُمَّ قالَ: لا تُفَضِّلُوا بيْنَ أَنْبِيَاءِ اللهِ؛ فإنَّه يُنْفَخُ في الصُّورِ، فَيَصْعَقُ مَن في السَّمَوَاتِ وَمَن في الأرْضِ إلَّا مَن شَاءَ اللَّهُ، قالَ: ثُمَّ يُنْفَخُ فيه أُخْرَى، فأكُونُ أَوَّلَ مَن بُعِثَ فَإِذَا مُوسَى عليه السَّلَامُ آخِذٌ بالعَرْشِ، فلا أَدْرِي أَحُوسِبَ بصَعْقَتِهِ يَومَ الطُّورِ، أَوْ بُعِثَ قَبْلِي، وَلَا أَقُولُ: إنَّ أَحَدًا أَفْضَلُ مِن يُونُسَ بنِ مَتَّى عليه السَّلَامُ {\footnotesize (صحيح مسلم، وصححه الألباني)}. وفي رواية اخرى: لا تُخَيِّروني على موسى؛ فإن الناسَ يُصْعَقون، فأكونُ أولَ مَن يَفِيقُ، فإذا موسى باطشٌ في جانبِ العرشِ، فلا أدري أكانَ ممَن صُعِقَ فأفاق قبلي، أو كان ممَن استَثْنَى اللهُ عزَّ وجلَّ {\footnotesize (صحيح أبي داود، وصححه الألباني)}. 

فهذا النهي جاء لأنه لم يكن لبيان فضل الله على أنبياءه ورسله بما فضلهم الله به وإنما كان على سبيل الإستنقاص أو التعصب أو التفريق أو التفاخر أو غير ذلك من الأمور التي تعارض ما جائوا به كأنما دعوتهم ليست بواحدة وهذا بخلاف التفضيل الذي فاضل الله به بين أنبياءه ورسله والذي فيه عدل الله تبارك وتعالى حيث اثبت سبحانه التفاوت بينهم في العلم والفضل ونفى التفريق بينهم في الدعوة. وبهذا يكون المعني الصحيح لقوله ﷺ (لا تفضلوا بين أنبياء الله) أي لا تفضلوا بين أنبياء الله تفضيلا يخالف التفضيل الذي فاضلهم الله به. وكذلك يقال بين نبينا ﷺ وبين موسى عليه السلام أو بين نبينا ﷺ وبين يونس عليه السلام. والله أعلى وأعلم.

\comment{
صحيح البخاري أنس بن مالك
1. السماء الأولى السماء الدنيا
آدم، نهران النيل والفرات، قصر من لؤلؤ وزبرجد وهو الكوثر
2. السماء الثانية
ادريس
3. السماء الثالثة

4. السماء الرابعة
هارون
5. السماء الخامسة
؟
6. السماء السادسة
ابراهيم
7. السماء السابعة
موسى
8. فوق السماء السابعة
سدرة المنتهى

صحيح البخاري أبو ذر الغفاري
1. السماء الأولى السماء الدنيا
آدم
2-5 السماء الثانية
ادريس، وموسى، وعيسى ويحي، 
6. السماء السادسة
ابراهيم
7. السماء السابعة
سدرة المنتهى والجنة 
8. فوق السماء السابعة
العرش، صريف الأقلام، فرضت الصلاة

صحيح البخاري مالك بن صعصة الأنصاري
1. السماء الأولى السماء الدنيا
آدم
2. السماء الثانية
عيسى ويحي
3. السماء الثالثة
يوسف
4. السماء الرابعة
إدريس
5. السماء الخامسة
هارون
6. السماء السادسة
موسى
7. السماء السابعة
ابراهيم، البيت المعمور، سدرة المنتهى، 
8. فوق السماء السابعة
العرش

https://dorar.net/aqeeda/1463/%D8%A7%D9%84%D9%85%D8%B7%D9%84%D8%A8-%D8%A7%D9%84%D8%A3%D9%88%D9%84-%D8%AA%D8%B9%D9%8A%D9%8A%D9%86-%D8%A3%D9%88%D9%84%D9%8A-%D8%A7%D9%84%D8%B9%D8%B2%D9%85
}


\section{الإصلاح وأنواعه}
قد تقدم معنا تفاوت الأنبياء في العلم والفضل ولكن دعوتهم واحدة وهي دعوة إلى إقامة الحق والميزان ومن ذلك عبادة الله وحده لا شريك له. وهذه الدعوة تحتاج إلى إصلاح الناس وهذا الإصلاح يكون بالعلم النافع والعمل الصالح والأمر بالمعروف والنهي عن المنكر. وهذا الإصلاح يكون إما إصلاح ديني أو إصلاح دنيوي أو كلاهما ومثال ذلك قصة موسى عليه السلام مع الخضر عليه السلام. فالخضر عليه السلام اختصه الله بعلم الغيب للإصلاح الدنيوي وأما موسى عليه السلام فاختصه الله بالعلم الديني للإصلاح الديني. فكان للخضر الرشاد والذي فيه الإصلاح الديني والدنيوي معا وكان لموسى الحكمة التي فيها الإصلاح الديني فقط. بينما فضل الله جل جلاله موسى عليه السلام في الفضل والعلم لما كان معه من العلم والحكمة في أمر الله الشرعي. 

فالإنبياء جاءوا بالحق والميزان لحكمة الله عز وجل ومن ذلك الإصلاح الديني والدنيوي. ولكن أغلب الرسل اختصهم الله جل جلاله للإصلاح الديني لأن الإصلاح الديني فيه صلاح العباد والذي به تدرك مصالح الدنيا والآخرة ومن ذلك الصدق والأمانة وغيرها من الخصال التي توافق الفطرة والدين وتعود بالنفع الديني والدنيوي. إلا أن العديد من المنافع الدنيوية الآخرى لا تدرك بالعلم الديني فقط وإنما تدرك بالعلم السببي كعلم الطب والهندسة وغيرها من العلوم النافعة التي ينتفع بها الناس في أمور دنياهم وآخرتهم. وقد من الله سبحانه وتعالى على كل البشر فجعل لهم كل ما يحتاجونه من عقل وفطرة لإدراك هذا العلم السببيي كما تقدم بيانه. 

ومن ذلك أن الله اختص النبي ﷺ بالعلم الشرعي للإصلاح الديني وليس بالعلم السببي ومن ذلك ما رواه العديد من الصحابة رضوان الله عليهم في قصة تلقيح النخل، ومنها أن أنس بن مالك قال: أنَّ النبيَّ صَلَّى اللَّهُ عليه وَسَلَّمَ مَرَّ بقَوْمٍ يُلَقِّحُونَ، فَقالَ: لو لَمْ تَفْعَلُوا لَصَلُحَ قالَ: فَخَرَجَ شِيصًا، فَمَرَّ بهِمْ فَقالَ: ما لِنَخْلِكُمْ؟قالوا: قُلْتَ كَذَا وَكَذَا، قالَ: أَنْتُمْ أَعْلَمُ بأَمْرِ دُنْيَاكُمْ {\footnotesize (صحيح مسلم)}. 
وحديث طلحة بن عبيدالله حيث قال: مَرَرْتُ مع رَسولِ اللهِ ﷺ بقَوْمٍ علَى رُؤُوسِ النَّخْلِ، فَقالَ: ما يَصْنَعُ هَؤُلَاءِ؟فَقالوا: يُلَقِّحُونَهُ؛ يَجْعَلُونَ الذَّكَرَ في الأُنْثَى فيَلْقَحُ، فَقالَ رَسولُ اللهِ صَلَّى اللَّهُ عليه وَسَلَّمَ: ما أَظُنُّ يُغْنِي ذلكَ شيئًا، قالَ: فَأُخْبِرُوا بذلكَ فَتَرَكُوهُ، فَأُخْبِرَ رَسولُ اللهِ صَلَّى اللَّهُ عليه وَسَلَّمَ بذلكَ، فَقالَ: إنْ كانَ يَنْفَعُهُمْ ذلكَ فَلْيَصْنَعُوهُ؛ فإنِّي إنَّما ظَنَنْتُ ظَنًّا، فلا تُؤَاخِذُونِي بالظَّنِّ، وَلَكِنْ إذَا حَدَّثْتُكُمْ عَنِ اللهِ شيئًا، فَخُذُوا به؛ فإنِّي لَنْ أَكْذِبَ علَى اللهِ عَزَّ وَجَلَّ {\footnotesize (صحيح مسلم)}. وحديث رافع بن خديج حيث قال: قَدِمَ نَبِيُّ اللهِ صَلَّى اللَّهُ عليه وَسَلَّمَ المَدِينَةَ وَهُمْ يَأْبُرُونَ النَّخْلَ، يقولونَ: يُلَقِّحُونَ النَّخْلَ، فَقالَ: ما تَصْنَعُونَ؟قالوا: كُنَّا نَصْنَعُهُ، قالَ: لَعَلَّكُمْ لو لَمْ تَفْعَلُوا كانَ خَيْرًا، فَتَرَكُوهُ، فَنَفَضَتْ -أَوْ فَنَقَصَتْ- قالَ: فَذَكَرُوا ذلكَ له، فَقالَ: إنَّما أَنَا بَشَرٌ، إذَا أَمَرْتُكُمْ بشَيءٍ مِن دِينِكُمْ، فَخُذُوا به، وإذَا أَمَرْتُكُمْ بشَيءٍ مِن رَأْيِي، فإنَّما أَنَا بَشَرٌ {\footnotesize (صحيح مسلم)}. 

وكل ما تقدم فيه أن نبينا ﷺ أقر لهم بأنهم أعلم بأمور دنياهم أي بالعلم السببي لهذا فقد قال: (إن كان ينفعهم ذلك فَلْيَصْنَعُوهُ)، وفرق عليه ﷺ بين أمر الله جل جلاله الذي به يكون الإصلاح الديني وبين أمور الدنيا التي بها يكون الإصلاح الدنيوي. فجعل كلامه ﷺ في أمور الدنيا ظن وقال: (لا تُؤَاخِذُونِي به) ولكن كلامه في أمر الله حق. وبين النبي ﷺ أنه بشر يخطئ ويصيب في أمور الدنيا كغيره من البشر ولهذا قال:  (إنَّما أَنَا بَشَرٌ، إذَا أَمَرْتُكُمْ بشَيءٍ مِن دِينِكُمْ، فَخُذُوا به، وإذَا أَمَرْتُكُمْ بشَيءٍ مِن رَأْيِي، فإنَّما أَنَا بَشَرٌ). وهذا من تواضعه وصدقه ﷺ فهو الصادق الأمين الذي لا يكذب على الله ولا على الناس عليه الصلاة والسلام. 

ولقد قال جل جلاله في كتابه: 
\quranayah*[18][110]{\footnotesize \surahname*[18]}. وقد جاء في تفسير الطبري أن معنى ذلك: أي قل لهؤلاء المشركين يا محمد: إنما أنا بشر مثلكم من بني آدم لا علم لي إلا ما علمني الله وإن الله يوحي إليّ أن معبودكم الذي يجب عليكم أن تعبدوه ولا تشركوا به شيئا، معبود واحد لا ثاني له، ولا شريك (فَمَنْ كَانَ يَرْجُوا لِقَاءَ رَبِّهِ) يقول: فمن يخاف ربه يوم لقائه، ويراقبه على معاصيه، ويرجو ثوابه على طاعته (فَلْيَعْمَلْ عَمَلا صَالِحًا) يقول: فليخلص له العبادة، وليفرد له الربوبية [هـ].


\section{الحكمة والرشاد}

ذكر جل جلاله الحكمة والرشاد في مواضع مختلفة في كتابه الكريم. فالحكمة فيها الإصلاح الديني بالعلم الشرعي الصحيح الموافق للحق وفيها العمل الموافق للميزان الشرعي. وأما الرشاد فهو أعم ومنه الحكمة بالإضافة إلى العلم بالأسباب والأخذ بها، أي أن الرشاد يكون بمعنى الإصلاح الديني فقط أو الإصلاح الدنيوي فقط أو كلاهما معا. فالرشاد الديني هي الهداية التي بها تجلب المصالح الشرعية، ويطلق الرشاد أيضا على الأمور التي بها تجلب المصالح الدنيوية مثل العلم السببي. فيكون بذلك الرشاد الكامل يشمل إقامة الحق بالعلم الشرعي الصحيح وإقامة الميزان بالعدل والعلم بالأسباب والأخذ بها. ولا يلزم أن الرشاد أفضل من الحكمة على وجه الإطلاق وإنما التفاضل يرجع للعلم والعمل بأمر الله الشرعي.

ولقد اختص جل جلاله من خلقه من يشاء بالحكمة فقال سبحانه وتعالى: 
\quranayah*[2][269]{\footnotesize \surahname*[2]}. وقد ذكر السعدي في تفسيره: 
لما أمر تعالى بهذه الأوامر العظيمة المشتملة على الأسرار والحكم وكان ذلك لا يحصل لكل أحد، بل لمن منَّ عليه وآتاه الله الحكمة، وهي العلم النافع والعمل الصالح ومعرفة أسرار الشرائع وحكمها، وإن من آتاه الله الحكمة فقد آتاه خيرا كثيرا وأي خير أعظم من خير فيه سعادة الدارين والنجاة من شقاوتهما! وفيه التخصيص بهذا الفضل وكونه من ورثة الأنبياء، فكمال العبد متوقف على الحكمة، إذ كماله بتكميل قوتيه العلمية والعملية فتكميل قوته العلمية بمعرفة الحق ومعرفة المقصود به، وتكميل قوته العملية بالعمل بالخير وترك الشر، وبذلك يتمكن من الإصابة بالقول والعمل وتنزيل الأمور منازلها في نفسه وفي غيره، وبدون ذلك لا يمكنه ذلك، ولما كان الله تعالى قد فطر عباده على عبادته ومحبة الخير والقصد للحق، فبعث الله الرسل مذكرين لهم بما ركز في فطرهم وعقولهم، ومفصلين لهم ما لم يعرفوه، انقسم الناس قسمين قسم أجابوا دعوتهم فتذكروا ما ينفعهم ففعلوه، وما يضرهم فتركوه، وهؤلاء هم أولو الألباب الكاملة، والعقول التامة، وقسم لم يستجيبوا لدعوتهم، بل أجابوا ما عرض لفطرهم من الفساد، وتركوا طاعة رب العباد، فهؤلاء ليسوا من أولي الألباب، فلهذا قال تعالى: (وما يذكر إلا أولو الألباب) \cite{tafsir_Saadi}.

ومن رحمة الله جل جلاله أنه جعل نبينا الكريم ﷺ معلما لأمته لهذه الحكمة فقال جل جلاله: 
\quranayah*[2][151]{\footnotesize \surahname*[2]}. وقد جاء في تفسير ابن كثير أنه تعالى يذكر عباده المؤمنين ما أنعم به عليهم من بعثة الرسول محمد صلى الله عليه وسلم إليهم، يتلو عليهم آيات الله مبينات ويزكيهم، أي: يطهرهم من رذائل الأخلاق ودنس النفوس وأفعال الجاهلية، ويخرجهم من الظلمات إلى النور، ويعلمهم الكتاب وهو القرآن والحكمة وهي السنة [هـ]. وقال السعدي أن معنى (وَيُعَلِّمُكُمُ الْكِتَابَ) أي: القرآن، ألفاظه ومعانيه، (وَالْحِكْمَةَ) قيل: هي السنة، وقيل: الحكمة، معرفة أسرار الشريعة والفقه فيها، وتنزيل الأمور منازلها. فيكون - على هذا - تعليم السنة داخلا في تعليم الكتاب، لأن السنة، تبين القرآن وتفسره، وتعبر عنه \cite{tafsir_Saadi}.

وأما الرشاد قد يكون بمعنى الإصلاح الديني فقط وهو الهداية لإتباع الحق كقوله تعالى: \quranayah*[2][186]{\footnotesize \surahname*[186]}. ومعنى يرشدون أي يهتدون فيحصل لهم الرشد الديني الذي هو الهداية للإيمان والأعمال الصالحة، ويزول عنهم الغي المنافي للإيمان والأعمال الصالحة كما جاء في تفسير السعدي. وأيضا قوله تعالى: \quranayah*[40][38]{\footnotesize \surahname*[40]}. وجاء معنى هذا في تفسير الطبري أي: إن اتبعتموني فقبلتم مني ما أقول لكم، بينت لكم طريق الصواب الذي ترشدون إذا أخذتم فيه وسلكتموه وذلك هو دين الله الذي ابتعث به موسى. وكل هذا من الإصلاح الديني والهداية. ويأتي الرشاد أيضا بمعنى الإصلاح الدنيوي فقط كقوله تعالى: \quranayah*[18][66]{\footnotesize \surahname*[18]}، وهذا لأن موسى عليه السلام كان لديه الإصلاح الديني وهو أعلم من الخضر عليه السلام في ذلك. ويأتي الرشاد أيضا بمعنى الإصلاح الديني والدنيوي معا كقوله تعالى: \quranayah*[18][10]{\footnotesize \surahname*[10]}، وهذا فيه أولا صلاح الدين حيث أن الله جل جلاله هداهم وزادهم هدى، وثانيا  صلاح الدنيا حيث جعل سبحانه وتعالى لهم حفظ البدن مع طول الفترة المكوث في الكهف فرارا من قومهم. وكمال الرشاد لا يدركه الإنسان بسهولة لأنه يتطلب الجمع بين العلم الشرعي والعلم السببي معا وهذا الأمر يعطيه الله لمن شاء من عباده ولهذا فقد أمر جل جلاله نبيه الكريم ﷺ فقال: \quranayah*[18][24][5]{\footnotesize \surahname*[18]}. وقال السعدي في تفسيره: فأمره أن يدعو الله ويرجوه، ويثق به أن يهديه لأقرب الطرق الموصلة إلى الرشد. وحري بعبد، تكون هذه حاله، ثم يبذل جهده، ويستفرغ وسعه في طلب الهدى والرشد، أن يوفق لذلك، وأن تأتيه المعونة من ربه، وأن يسدده في جميع أموره \cite{tafsir_Saadi}.

والرشاد إن جمع مع الهداية يكون بمعنى الإصلاح الدنيوي، وتكون الهداية بمعنى الحكمة والتي بها يكون الإصلاح الديني. ولهذا فإن نبينا الكريم ﷺ بدأ بالإصلاح الديني وسمى الخلفاء من بعده بالخلفاء الراشدين حتى يقوموا بالإصلاح الديني والدنيوي معا حيث قال ﷺ: عَليكم بِسنَّتي وسنَّةِ الخلفاءِ الرَّاشدينَ المَهْديِّينَ مِن بَعدي {\footnotesize (صححه الألباني)}. وفي رواية أخرى: المهديين الراشدين {\footnotesize (صحيح الجامع، صححه الألباني)}.

\section{مكانة أهل العلم الشرعي}

إن أهل العلم الشرعي هم القائمين بالإصلاح الديني وهم الذين يأمرون الناس بالقسط فيأمرون بالمعروف وينهون عن المنكر فهم أعلم الناس بالحق كما في قوله تعالى:
\quranayah*[34][6]{\footnotesize \surahname*[34]}. وهذا لأن الحق يهدي إلى الطريق المستقيم كما ذكر تعالى في هذه الآية وفي قوله تعالى:
\quranayah*[22][54]{\footnotesize \surahname*[22]}. فأهل العلم الشرعي يبينون للناس الحق الذي جاءت به الرسل والأنبياء ساعين إلى الهداية الشرعية للناس وراجين لهم الهداية الكونية من الله جل جلاله. وقد جعل الله جل جلاله أهل العلم حجة على الناس لما معهم من الحق فكانوا بذلك هم ورثة الأنبياء في الأرض فقد قال تعالى:
\quranayah*[17][107]{\footnotesize \surahname*[17]}.
وقوله تعالى:
\quranayah*[29][49]{\footnotesize \surahname*[29]}.


ولهذا فقد رفع الله مكانة أهل الإيمان وأهل العلم في الدنيا والأخرة لما عرفوا من الحق كما في قوله تعالى:
\quranayah*[58][11][19]{\footnotesize \surahname*[58]}.
ومن أعظم ذلك قوله تعالى:
\quranayah*[3][18]{\footnotesize \surahname*[3]}.
يقول السعدي في تفسيره:
هذا تقرير من الله تعالى للتوحيد بأعظم الطرق الموجبة له، وهي شهادته تعالى وشهادة خواص الخلق وهم الملائكة وأهل العلم [.] وأما شهادة أهل العلم فلأنهم هم المرجع في جميع الأمور الدينية خصوصا في أعظم الأمور وأجلها وأشرفها وهو التوحيد، فكلهم من أولهم إلى آخرهم قد اتفقوا على ذلك ودعوا إليه وبينوا للناس الطرق الموصلة إليه، فوجب على الخلق التزام هذا الأمر المشهود عليه والعمل به، وفي هذا دليل على أن أشرف الأمور علم التوحيد لأن الله شهد به بنفسه وأشهد عليه خواص خلقه، والشهادة لا تكون إلا عن علم ويقين، بمنزلة المشاهدة للبصر، ففيه دليل على أن من لم يصل في علم التوحيد إلى هذه الحالة فليس من أولي العلم. وفي هذه الآية دليل على شرف العلم من وجوه كثيرة، منها: أن الله خصهم بالشهادة على أعظم مشهود عليه دون الناس، ومنها: أن الله قرن شهادتهم بشهادته وشهادة ملائكته، وكفى بذلك فضلا، ومنها: أنه جعلهم أولي العلم، فأضافهم إلى العلم، إذ هم القائمون به المتصفون بصفته، ومنها: أنه تعالى جعلهم شهداء وحجة على الناس، وألزم الناس العمل بالأمر المشهود به، فيكونون هم السبب في ذلك، فيكون كل من عمل بذلك نالهم من أجره، وذلك فضل الله يؤتيه من يشاء، ومنها: أن إشهاده تعالى أهل العلم يتضمن ذلك تزكيتهم وتعديلهم وأنهم أمناء على ما استرعاهم عليه \cite{tafsir_Saadi}.


ومن أعظم المنكر قتل الأنبياء أو الذين يأمرون الناس بالقسط القائمين بالإصلاح الديني أو الإصلاح الدنيوي أو كلاهما. ولهذا فقد قال جل جلاله ذمه لأهل الكتاب لبيان هذا الجرم العظيم في قوله تعالى:
\quranayah*[3][21]{\footnotesize \surahname*[3]}. وقد جاء في تفسير السعدي أن هؤلاء الذين أخبر الله عنهم في هذه الآية، أشد الناس جرما وأي: جرم أعظم من الكفر بآيات الله التي تدل دلالة قاطعة على الحق الذي من كفر بها فهو في غاية الكفر والعناد ويقتلون أنبياء الله الذين حقهم أوجب الحقوق على العباد بعد حق الله، الذين أوجب الله طاعتهم والإيمان بهم، وتعزيرهم، وتوقيرهم، ونصرهم وهؤلاء قابلوهم بضد ذلك، ويقتلون أيضا الذين يأمرون الناس بالقسط الذي هو العدل، وهو الأمر بالمعروف والنهي عن المنكر الذي حقيقته إحسان إلى المأمور ونصح له، فقابلوهم شر مقابلة، فاستحقوا بهذه الجنايات المنكرات أشد العقوبات، وهو العذاب المؤلم البالغ في الشدة إلى غاية لا يمكن وصفها، ولا يقدر قدرها المؤلم للأبدان والقلوب والأرواح \cite{tafsir_Saadi}. 

وقد صح عن النبي ﷺ أنه قال: إن الإسلام بدأ غريبًا، وسيعودُ غريبًا كما بدأَ، فطُوبَى للغُرباءِ قيل: من هم يا رسولَ اللهِ؟قال: الذينَ يصلحونَ إذا فسدَ الناسُ {\footnotesize (صحيحه الألباني في السلسلة الصحيحة)}. وفي زيادة: وليأرَزنَّ الإسلامُ إلى ما بين المسجدَيْن كما تأرَزُ الحيَّةُ إلى جُحرِها {\footnotesize (غريب أورده ابن حجر العسقلاني في موافقة الخبر الخبر)}. وفي رواية: الذين يُصْلِحُون ما أفسَدَ الناسُ مِن بعدِي مِن سُنتي {\footnotesize (قال الألباني ضعيف جدا)}.

عن أنسٍ رضي اللَّه عنه قالَ: قيلَ يا رسولَ اللَّهِ متى نتركُ الأمرَ بالمعروفِ والنَّهيَ عنِ المنكرِ قالَ إذا ظهرَ فيكم ما ظهرَ في الأممِ السابقة وفي روايةٍ في بني إسرائيلَ قالوا يا رسولَ اللَّهِ وما ظهرَ في الأممِ قبلنا قالَ المُلكُ في صغارِكم والفاحِشةُ في كبارِكم والعلمُ في رُذالتِكم {\footnotesize (حسنه السخاوي والوادعي وضعفه الألباني)}.


فالأمر بالمعروف واجب على كل مسلم إلى أن يشاء الله. 



\section{حال الأنبياء وأتباعهم مع الميزان الشرعي}

ولما كان الأنبياء أعلم الناس بأمر الله وأحرصهم، فقد أقاموا الميزان الشرعي حق إقامته في حكمهم الرشيد بين الخلق. ومثال ذلك يوسف عليه السلام في قوله تعالى:
\quranayah*[12][55]{\footnotesize \surahname*[12]}.
وهذا فيه حرصه عليه السلام على إقامة الكيل والوزن بما يرضي الله وهذا من الإصلاح الذي أمر الله به حيث قال لإخوته عن الكيل:
\quranayah*[12][59-60]{\footnotesize \surahname*[12]}.

فلا شك أن التفريط في الكيل والوزن من أعظم البلايا التي حذرنا الله عز وجل منها في كتابه الكريم فيقول تعالى:
\quranayah*[83][1-6]{\footnotesize \surahname*[83]}. فالظلم في الكيل والوزن من الإفساد العظيم ومن أسباب تعجيل العذاب في الدنيا قبل الأخرة، وفي قصة مدين مع نبيهم شعيبا العبرة الواضحة في ذلك. يقول تعالى على لسان نبيه شعيب محذرا قومه:
\quranayah*[7][85]{\footnotesize \surahname*[7]}.
وفي موضع أخر من سورة الشعراء:
\quranayah*[26][181-183]{\footnotesize \surahname*[26]}.
وفي سورة هود:
\quranayah*[11][84-85]{\footnotesize \surahname*[11]}. وهذا فيه أن شعيبا عليه السلام دعا قومه لإقامة الحق أولا وهو التوحيد بإفراد الله بالعبادة وثانيا لإقامة الميزان الشرعي وهو الكيل والوزن بالقسط. وفيه أن بخس الناس أشيائهم والخسران والنقصان في الكيل والوزن من الظلم والفساد الموجب لسخط الله وعذابه العاجل.

ونبينا ﷺ كان من أحرص الناس في إقامة الكيل والميزان وحذر من الفساد في ذلك في العديد من المواضع منها ما ورد عن ابن عباس رضي الله عنه أنه قَالَ: قَالَ رَسُولُ اللَّهِ صَلَّى اللَّهُ عَلَيْهِ وَسَلَّمَ لِأَصْحَابِ الْكَيْلِ وَالْمِيزَانِ: «إِنَّكُمْ قَدْ وُلِّيتُمْ أَمْرَيْنِ هَلَكَتْ فِيهِمَا الْأُمَمُ السَّابِقَة قبلكُمْ».
{\footnotesize رَوَاهُ التِّرْمِذِيّ}. وقد علم الصحابة والتابعين بأهمية إقامة الميزان والمكيال وأن الفساد فيهما من أسباب سخط الله ومنه ما ورد في كتاب الموطأ للإمام مالك عَنْ يَحْيَى بْنِ سَعِيدٍ، أَنَّهُ سَمِعَ سَعِيدَ بْنَ الْمُسَيَّبِ، يَقُولُ إِذَا جِئْتَ أَرْضًا يُوفُونَ الْمِكْيَالَ وَالْمِيزَانَ فَأَطِلِ الْمُقَامَ بِهَا وَإِذَا جِئْتَ أَرْضًا يُنَقِّصُونَ الْمِكْيَالَ وَالْمِيزَانَ فَأَقْلِلِ الْمُقَامَ بِهَا.

\section{حال الأمم مع الحق والميزان الشرعي}

من عدل الله جل جلاله أن سنته في خلقه من الأمم السابقة واللاحقة ثابتة كما قال تعالى: \quranayah*[33][62]{\footnotesize \surahname*[33]}، أي أن سنة الله لا تبدل ولا تغير كما جاء في تفسير ابن كثير. والتفاوت في العقوبة أو التمكين أو الهوان للأمم السابقة أو اللاحقة في الدنيا كل بحسب حاله وما يستحقه بما شاء الله جل جلاله بعدله أو رحمته أو حكمته. وحال الأمم في الدنيا، سواء كانت كافرة أو مسلمة،  يدور مع الحق والميزان في خمسة أحوال من الأشد عقوبة إلى الأهون عقوبة، أو من الأقل تمكينا إلى الأكثر تمكينا، أو من الأكثر هوانا إلى الأقل هونا:

\begin{compactitem}
    \item الدولة الكافرة الظالمة الهالكة
    \item الدولة المسلمة الظالمة
    \item الدولة الكافرة الظالمة
    \item الدولة الكافرة العادلة
    \item الدولة المؤمنة العادلة
\end{compactitem}

وهذا الترتيب فيه تقديم الميزان على الحق في الدنيا بحسب ما قامت به الحجة. وهذا لأن الله جل جلاله قدم في الدنيا المصلحة العامة التي تكون بين الناس على المصلحة الشخصية التي تكون في النفس، فجعل سبحانه إقامة الميزان أي العدل بين الناس والإصلاح فيما بينهم في الدنيا مقدما على إقامة الحق في نفوس الناس. وهذا لأنه بالعدل في الدنيا تحفظ الأموال والأعراض والدماء، فلا تصلح الحياة إلا بالإصلاح الذي يكون بالعدل والدليل على هذا قوله تعالى: \quranayah*[11][117]{\footnotesize \surahname*[11]}. وقد جاء في تفسير الطبري أن معنى ذلك أن الله جل جلاله لم يكن ليهلكهم بشركهم بالله. وذلك قوله " بظلم " يعني: بشرك، (وأهلها مصلحون)، فيما بينهم لا يتظالمون، ولكنهم يتعاطَون الحقّ (أي العدل) بينهم، وإن كانوا مشركين، إنما يهلكهم إذا تظالموا [هـ]. وأدنى هذا العدل هو العدل الموافق للفطرة أي الميزان الفطري. وجاء أيضا تفصيل ذلك في تفسير القرطبي أن معنى وأهلها مصلحون أي فيما بينهم في تعاطي الحقوق; أي لم يكن ليهلكهم بالكفر وحده حتى ينضاف إليه الفساد، كما أهلك قوم شعيب ببخس المكيال والميزان، وقوم لوط باللواط; ودل هذا على أن المعاصي أقرب إلى عذاب الاستئصال في الدنيا من الشرك، وإن كان عذاب الشرك في الآخرة أصعب [هـ]. فكل هذا فيه أن مخالفة الميزان أهلك في الدنيا من مخالفة الحق، وأن مخالفة الميزان الفطري على وجه الخصوص بالظلم والفساد والمعاصي من أسباب تعجيل سخط الله وعقابه في الدنيا قبل الآخرة. 


ومن ذلك ما نقله شيخ الإسلام ابن تيمية: وَلِهَذَا قِيلَ: إنَّ اللَّهَ يُقِيمُ الدَّوْلَةَ الْعَادِلَةَ وَإِنْ كَانَتْ كَافِرَةً؛ وَلَا يُقِيمُ الظَّالِمَةَ وَإِنْ كَانَتْ مُسْلِمَةً. وَيُقَالُ: الدُّنْيَا تَدُومُ مَعَ الْعَدْلِ وَالْكُفْرِ وَلَا تَدُومُ مَعَ الظُّلْمِ وَالْإِسْلَامِ [.] وَذَلِكَ أَنَّ الْعَدْلَ نِظَامُ كُلِّ شَيْءٍ؛ فَإِذَا أُقِيمَ أَمْرُ الدُّنْيَا بِعَدْلِ قَامَتْ وَإِنْ لَمْ يَكُنْ لِصَاحِبِهَا فِي الْآخِرَةِ مِنْ خَلَاقٍ وَمَتَى لَمْ تَقُمْ بِعَدْلِ لَمْ تَقُمْ وَإِنْ كَانَ لِصَاحِبِهَا مِنْ الْإِيمَانِ مَا يُجْزَى بِهِ فِي الْآخِرَةِ {\footnotesize (مجموع الفتاوى 28/146)}. وقد قال الشيخ الألباني رحمه الله تعالى في توضيح هذا المعنى: ذلك لأنَّ الظلم هو سبب خراب البلاد وهلاك العباد، فإذا كانت الأمة أو الدولة كافرة ولكنها تحكم بالعدل فيما بينها، هذا العدل الذي يعرفه الناس بفِطَرهم، فإذا كانوا يحكمون بذلك فستقوم دولتهم وتستمرُّ مدَّة طويلة، والتاريخ يحفل بهذا [هـ]. وقال الشيخ ابن باز رحمه الله في ذلك: مع الظلم والفساد في الأرض تعمّ العقوبات، ولا تستقيم الدولة على ذلك، وإنما تستقيم على العدل، دولة عادلة وإن كانت كافرةً تستقيم معها أحوال رعيتها، ودولة ظالمة وإن كانت مسلمةً تختلُّ بها الأمور، ولا تنتظم بها الأمور، وقد تحصل الشَّحناء والعداوة والفتن، نسأل الله العافية [هـ].


ولقد بين رسولنا الكريم ﷺ خطر مخالفة الميزان الشرعي على الحاكم والمحكومين وهذا يشمل الميزان الفطري والميزان الديني. وجاء في حديث سعد بن تميم أنه قيل: يا رسولَ اللهِ، ما للخليفةِ مِن بعدِك؟ قال: مِثلُ الذي لي، ما عدَلَ في الحُكمِ، وقسَطَ في القِسطِ، ورَحِمَ ذا الرَّحِمِ، فمَن فعَلَ غيرَ ذلك فليس منِّي ولستُ منه {\footnotesize (صحيح، تخريج سنن أبي داود، وصححه الألباني)}. وهذا فيه أن الله أوجب على الحاكم العدل في الحكم وأن النبي ﷺ تبرأ من الحاكم الظالم. وعَنْ عَبْدِ اللَّهِ بْنِ عُمَرَ، قَالَ أَقْبَلَ عَلَيْنَا رَسُولُ اللَّهِ ـ ﷺ ـ فَقَالَ «يَا مَعْشَرَ الْمُهَاجِرِينَ خَمْسٌ إِذَا ابْتُلِيتُمْ بِهِنَّ وَأَعُوذُ بِاللَّهِ أَنْ تُدْرِكُوهُنَّ: لَمْ تَظْهَرِ الْفَاحِشَةُ فِي قَوْمٍ قَطُّ حَتَّى يُعْلِنُوا بِهَا إِلاَّ فَشَا فِيهِمُ الطَّاعُونُ وَالأَوْجَاعُ الَّتِي لَمْ تَكُنْ مَضَتْ فِي أَسْلاَفِهِمُ الَّذِينَ مَضَوْا، وَلَمْ يَنْقُصُوا الْمِكْيَالَ وَالْمِيزَانَ إِلاَّ أُخِذُوا بِالسِّنِينَ وَشِدَّةِ الْمَؤُنَةِ وَجَوْرِ السُّلْطَانِ عَلَيْهِمْ، وَلَمْ يَمْنَعُوا زَكَاةَ أَمْوَالِهِمْ إِلاَّ مُنِعُوا الْقَطْرَ مِنَ السَّمَاءِ وَلَوْلاَ الْبَهَائِمُ لَمْ يُمْطَرُوا، وَلَمْ يَنْقُضُوا عَهْدَ اللَّهِ وَعَهْدَ رَسُولِهِ إِلاَّ سَلَّطَ اللَّهُ عَلَيْهِمْ عَدُوًّا مِنْ غَيْرِهِمْ فَأَخَذُوا بَعْضَ مَا فِي أَيْدِيهِمْ، وَمَا لَمْ تَحْكُمْ أَئِمَّتُهُمْ بِكِتَابِ اللَّهِ وَيَتَخَيَّرُوا مِمَّا أَنْزَلَ اللَّهُ إِلاَّ جَعَلَ اللَّهُ بَأْسَهُمْ بَيْنَهُمْ».
{\footnotesize (أخرجه ابن ماجه وصححه الألباني)}.
فكل ما ذكره النبي ﷺ في هذا الحديث هي من المخالفات الظالمة التي تخالف الميزان الشرعي، ولكن الرسول ﷺ قدم مخالفة الميزان الفطري على الميزان الديني. فالمجاهرة بالمعاصي وإنقاص الكيل من الأعمال التي تعارض الميزان الفطري والذي يمكن إدراكه بالفطرة السليمة. وهذا فيه بيان خطورة مخالفة الفطرة للناس عموما مسلمين كانوا أو غير مسلمين. وفي تقديم هذا النوع بيان المبالغة في المعصية. وأما منع الزكاة، ومخالفة أمر الله ورسوله، والحكم بغير ما أنزل الله فهي تخالف الميزان الديني الذي يدرك بالوحي. وهذا فيه بيان خطورة مخالفة أمر الله وبالأخص للمسلمين. فكل هذه الأمور من أسباب البلاء العظيم ومنها الطاعون والأمراض والفقر والجوع وجور السلطان والهوان والفتن. وفيه الدليل على نبوته ﷺ فقد وقع ذلك كما أخبر بعد أن تهاون الكثير من المسلمين وغير المسلمين في أمر الميزان إلا من رحم الله. وبهذا يعرف أن الأمة المسلمة ينالها بمخالفة الميزان الشرعي من العقوبات في الدنيا ما لا يناله غيرها وذلك لما عرفت من الحق وبحسب ما قامت به من الظلم كما هو موضح في الجدول \ref{tab:violations}. وسيأتي بيان ذلك في حال الأمة المسلمة الظالمة.

\begin{table}[h]
\centering
\begin{tabular}{|c|c|c|c|}
\hline
\textbf{المخالفة الظالمة} & \textbf{العقوبة} & \textbf{نوع الميزان} & \textbf{الأمة}  \\
\hline
ظهور الفاحشة والجهر بها & الطاعون والأوجاع & الميزان الفطري & مسلمة أو كافرة \\
\hline
نقص المكيال والميزان & السنين وشدة المؤنة وجور السلطان & الميزان الفطريي & مسلمة أو كافرة \\
\hline
منع الزكاة & منع القطر من السماء  & الميزان الديني & مسلمة \\
\hline
نقض عهد الله ورسوله & تسلط العدو وأخذه بعض ما في أيديهم & الميزان الديني & مسلمة \\
\hline
الحكم بغير كتاب الله & القتال والفتن بينهم & الميزان الديني & مسلمة \\
\hline
\end{tabular}
\caption{مخالفات الميزان الشرعي وعقوباتها بحسب ما أخبر النبي ﷺ}
\label{tab:violations}
\end{table}


\subsection{الدولة الكافرة الظالمة الهالكة}

الدولة أو الأمة الكافرة الظالمة الهالكة هي التي أرسل الله لها رسولا على وجه الخصوص من الأمم السابقة فلم تقبل الحق بكفرها ولم تقم الميزان بظلمها رغم قيام الحجة عليهم. وهي الأشد عقوبة في الدنيا سواءا كان لها التمكين أو لم يكن، وهذا لأن الحجة قامت عليها ولكن لم تقم الحق ولم تقم أدنى درجات العدل وهو الميزان الفطري فلم يقيموا العدل الذي به تحفظ الأموال (كالغش في الكيل مثل قوم شعيب)، أو الأعراض (كالزنى واللواط مثل قوم لوط)، أو الدماء (كالجور والقتل بغير حق مثل فرعون)، وغيرها من الأمور التي تخالف الميزان الفطري، والتي جاء الأنبياء بتحريمها.

وقد جرت سنة الله في الأزمان الخالية من الأمم الكافرة الظالمة الهالكة المكذبة لرسلها أن ينزل الله عليها عذابه العاجل بمجرد إخراجهم لرسلهم ظلما وعدوانا وكفرا كما قال تعالى:  
\quranayah*[17][77]{\footnotesize \surahname*[17]}. وقد جاء في تفسير ابن كثير أنه أي: هكذا عادتنا في الذين كفروا برسلنا وآذوهم، يخرج الرسول من بين أظهرهم ويأتيهم العذاب [هـ]. وقد تنوع فيهم العقاب كل بحسب حاله كما قال تعالى: 
\quranayah*[29][40]{\footnotesize \surahname*[29]}. وكل هذا فيه العبرة والتذكير من الله جل جلاله لإتباع أمره والإنقياد له لكل من استخلفه الله ولهذا فقد قال تعالى:
\quranayah*[10][13-14]{\footnotesize \surahname*[10]}.
وهذا فيه أن الله جل جلاله ناظر إلى أعمالنا وأعمال الأمم ومجازيها بعدله سبحانه وتعالى.

\subsection{الدولة المسلمة الظالمة}

الدولة أو الأمة المسلمة الظالمة هي التي عرفت الحق بقبولها للإسلام دينا لها ولم تقم الميزان بظلمها فلم يعمل غالب أفرادها بتعاليم وأوامر الإسلام. فعقوبتها في الدنيا تكون بالهوان وعدم التمكين وهذا لأنها عرفت الحق ولم تعمل به على الوجه المطلوب منها فلم تقم العدل الذي أمرها الله به وهو الميزان الشرعي، أي أن أفرادها، حكاما كانوا أو محكومين، لم يقيموا العدل الذي به تحفظ الأموال، أو الأعراض، أو الدماء، وغيرها من الأمور التي تخالف الميزان الفطري، بالإضافة إلى مخالفة الميزان الديني كمنع الزكاة، أو الحكم بغير ما أنزل الله، وغيرها من الأوامر الدينية التي تخالف الميزان الشرعي، وهذا يشمل الميزان الفطري والميزان الديني معا.

فعدل الدولة المسلمة لا يكون فقط بموافقة الميزان الفطري كما هو الحال بالنسبة للدولة الكافرة، وإنما يكون عدل الدولة المسلمة بموافقة الميزان الشرعي، أي الميزان الفطري والديني معا. وظلم الدولة المسلمة يكون بمخالفة الميزان الشرعي وهذا لما عرفت من الحق والدليل قوله تعالى: 
\quranayah*[3][23]{\footnotesize \surahname*[3]}. وقد جاء في تفسير السعدي أن الله تعالى يخبر عن حال أهل الكتاب الذين أنعم الله عليهم بكتابه، فكان يجب أن يكونوا أقوم الناس به وأسرعهم انقيادا لأحكامه، فأخبر الله عنهم أنهم إذا دعوا إلى حكم الكتاب تولى فريق منهم وهم يعرضون، تولوا بأبدانهم، وأعرضوا بقلوبهم، وهذا غاية الذم، وفي ضمنها التحذير لنا أن نفعل كفعلهم، فيصيبنا من الذم والعقاب ما أصابهم، بل الواجب على كل أحد إذا دعي إلى كتاب الله أن يسمع ويطيع وينقاد، كما قال تعالى \quranayah*[24][51]{\footnotesize \surahname*[24]} \cite{tafsir_Saadi}.

وقد قال شيخ الإسلام ابن تيمية رحمه الله: ولهذا يروى ان الله ينصر الدولة العادلة وان كانت كافرة، ولا ينصر الدولة الظالمة وان كانت مؤمنة {\footnotesize (مجموع الفتاوى 28/63)}. والأصح أن يقال: "ولا ينصر الدولة الظالمة وإن كانت مسلمة"، وهذا لان الظلم ينافي الغاية من الإيمان وكماله وهو بلا شك معصية لله ورسوله والدليل قوله تعالى: \quranayah*[49][14]{\footnotesize \surahname*[49]}. فالدولة المؤمنة هي التي تطيع الله ورسوله ومن ذلك الحكم بالعدل، فإن خالفت أمر الله فقد خالفت الغاية من الإيمان وتكون أدنى مرتبة في الدين وهي مرتبة الإسلام وليس مرتبة الإيمان. فالله جل جلاله لم يكتب للأمة المسلمة الظالمة التمكين  بل كتب لهم الهوان، ووعد سبحانه الذين ءامنوا بالتمكين كما سيأتي بيانه في الأمة المؤمنة العادلة. ولهذا فقد جاء أيضا عن شيخ الإسلام هذا التصحيح فقال في موضع آخر: وَلِهَذَا قِيلَ: إنَّ اللَّهَ يُقِيمُ الدَّوْلَةَ الْعَادِلَةَ وَإِنْ كَانَتْ كَافِرَةً؛ وَلَا يُقِيمُ الظَّالِمَةَ وَإِنْ كَانَتْ مُسْلِمَةً. وَيُقَالُ: الدُّنْيَا تَدُومُ مَعَ الْعَدْلِ وَالْكُفْرِ وَلَا تَدُومُ مَعَ الظُّلْمِ وَالْإِسْلَامِ {\footnotesize (مجموع الفتاوى 28/146)}.

والدولة المسلمة الظالمة ينالها من العقوبات والهوان ما لا ينال غيرها وهذا لأنها عرفت الحق أو انتسبت إليه ولم تعمل به أو تسعى للعمل به فينالها كل العقوبات الخمسة التي ذكرها النبي ﷺ كما تقدم في الجدول \ref{tab:violations}. ومن ذلك الذل والهوان وتسلط العدو لنقض عهد الله ورسوله أي مخالفة أمر الله كالربا والعينة وترك الجهاد فعَنِ ابْنِ عُمَرَ، قَالَ سَمِعْتُ رَسُولَ اللَّهِ ﷺ يَقُولُ "إِذَا تَبَايَعْتُمْ بِالْعِينَةِ وَأَخَذْتُمْ أَذْنَابَ الْبَقَرِ وَرَضِيتُمْ بِالزَّرْعِ وَتَرَكْتُمُ الْجِهَادَ سَلَّطَ اللَّهُ عَلَيْكُمْ ذُلاًّ لاَ يَنْزِعُهُ حَتَّى تَرْجِعُوا إِلَى دِينِكُمْ" {\footnotesize (صححه الألباني)}. وهذا الذل يكون بتسلط الأعداء الكفار عليها وهذا لنقضها عهد الله وهو الميزان الشرعي الذي كلفهم الله به حبا في الدنيا، فهذا كله من أسباب الذل والهوان في الدنيا قبل الأخرة ومن أسباب تعجيل عقاب الله وتسلط الأعداء  كما بين ذلك النبي ﷺ فعَنْ ثَوْبَانَ، قَالَ قَالَ رَسُولُ اللَّهِ ﷺ "يُوشِكُ الأُمَمُ أَنْ تَدَاعَى عَلَيْكُمْ كَمَا تَدَاعَى الأَكَلَةُ إِلَى قَصْعَتِهَا". فَقَالَ قَائِلٌ وَمِنْ قِلَّةٍ نَحْنُ يَوْمَئِذٍ قَالَ "بَلْ أَنْتُمْ يَوْمَئِذٍ كَثِيرٌ وَلَكِنَّكُمْ غُثَاءٌ كَغُثَاءِ السَّيْلِ وَلَيَنْزِعَنَّ اللَّهُ مِنْ صُدُورِ عَدُوِّكُمُ الْمَهَابَةَ مِنْكُمْ وَلَيَقْذِفَنَّ اللَّهُ فِي قُلُوبِكُمُ الْوَهَنَ". فَقَالَ قَائِلٌ يَا رَسُولَ اللَّهِ وَمَا الْوَهَنُ قَالَ "حُبُّ الدُّنْيَا وَكَرَاهِيَةُ الْمَوْتِ" {\footnotesize (صححه الألباني)}. 

وكل هذا فيه أيضا أن الله يمكن الأمم الأخرى كالدولة الكافرة على المسلمين ولو كانوا ظالمين لهم وهذا لأن المسلمين خالفوا أمر الله الشرعي فيمسهم من الظلم والهوان ما لا يمس غيرهم حتى يرجعوا إلى دينهم الذي كلفهم الله به. ولم يستثنيي الرسول ﷺ من الأمم الأخرى وهذا بالعموم فتكون بذلك الأمة الكافرة الظالمة والأمة الكافرة العادلة مسلطة على الأمة المسلمة الظالمة. وهذا إنما لحكمة الله وعدله ورحمته فلو كان التميكن للأمة المسلمة الظالمة لطغت على أمر الله ولتجبرت ولتركت هذا الدين العظيم ظلما وعدوانا وغرورا ولكان حالها كحال الأمم السابقة التي فرحت بما كان لديها من العلم السببي  ولتركت أمر الله الشرعي كما هي عادة البشر في من سبق من الذين خلوا من قبل. وهذا لا يكون لأن الله جل جلاله أراد لهذا الدين أن يبقى إلى قبل قيام الساعة. وسيأتي بيان ذلك في دولة الحكم الرشيد وهي الدولة المؤمنة العادلة.

وقد نهى النبي ﷺ عن الخروج على الدولة المسلمة الظالمة وبالأخص لما يترتب على ذلك من ظلم الذي يخالف الميزان الفطري والذي به يكون فساد المصالح العامة في الدنيا كسفك الدماء ونهب الأموال وهتك الأعراض، والتي هي أشد ظلما في الدنيا من الظلم الذي يكون بمخالفة الميزان الديني كمنع الزكاة أو الحكم بغير ما أنزل الله من باب الهوى. وقد تقدم معنا أن الله جل جلاله قدم في الدنيا إقامة الميزان بين الناس بالعدل على إقامة الحق في نفوس الناس لتقديم المصلحة العامة على الخاصة.  ولذلك فقد نهى النبي ﷺ عن الخروج على ولاة الأمر المسلمين ولو كانوا ظالمين وعاصين لله ولرسوله  ما أقاموا فينا الصلاة وكفى بنا أن نبغضهم في الله على ما عصوا به الله ورسوله كما جاء عن عوف بن مالك الأشجعي أن النبي ﷺ قال: خِيارُ أئِمَّتِكُمُ الَّذِينَ تُحِبُّونَهُمْ ويُحِبُّونَكُمْ، ويُصَلُّونَ علَيْكُم وتُصَلُّونَ عليهم، وشِرارُ أئِمَّتِكُمُ الَّذِينَ تُبْغِضُونَهُمْ ويُبْغِضُونَكُمْ، وتَلْعَنُونَهُمْ ويَلْعَنُونَكُمْ، قيلَ: يا رَسولَ اللهِ، أفَلا نُنابِذُهُمْ بالسَّيْفِ؟ فقالَ: لا، ما أقامُوا فِيكُمُ الصَّلاةَ، وإذا رَأَيْتُمْ مِن وُلاتِكُمْ شيئًا تَكْرَهُونَهُ، فاكْرَهُوا عَمَلَهُ، ولا تَنْزِعُوا يَدًا مِن طاعَةٍ، وفي رواية أخرى، قالَ: لا، ما أقامُوا فِيكُمُ الصَّلاةَ، لا، ما أقامُوا فِيكُمُ الصَّلاةَ، ألا مَن ولِيَ عليه والٍ، فَرَآهُ يَأْتي شيئًا مِن مَعْصِيَةِ اللهِ، فَلْيَكْرَهْ ما يَأْتي مِن مَعْصِيَةِ اللهِ، ولا يَنْزِعَنَّ يَدًا مِن طاعَةٍ {\footnotesize (صحيح مسلم، وصححه الألباني في تخريج كتاب السنة)}. للمزيد من التفاصيل، راجع فصل \fullautoref{sec:app_rebellion}.

ولهذا يجب على الدولة المسلمة الظالمة التي تنتسب إلى الإسلام الرجوع لأمر الله جل جلاله بتحكيم شرع الله والذي به تحفظ الدماء والأموال والأعراض بما يرضي الله جل جلاله. وقد قال الشيخ ابن باز رحمه الله: والواجب على كل دولة إسلامية أن تحذر نقمة الله، وأن تبادر بتحكيم شريعة الله، وأن تتقي الله في ذلك، كل دولة تنتسب للإسلام ثم تتساهل في هذا الأمر فقد أتت أمرًا عظيمًا، وإذا كان تساهلها عن اعتقاد الجواز وإنه لا يجب عليها تحكيم شريعة الله فهذه دولة كافرة كفرًا أكبر، نعوذ بالله، إذا اعتقدت أنه لا يلزمها الحكم بشريعة الله، وأنه يجوز لها الحكم بهذه القوانين فهذا كفر أكبر، وردة عظمى [.] ولا حول ولا قوة إلا بالله. نسأل الله السلامة والعافية {\footnotesize (أحوال الحكم بغير ما أنزل الله، التعليقات على ندوات الجامع الكبير)}. 


\comment{


قال النبي ﷺ: "لَتُتْبَعُنَّ سُنَنُ مَنْ كَانَ قَبْلَكُمْ شِبْرًا بِشِبْرٍ وَذِرَاعًا بِذِرَاعٍ حَتَّى لَوْ دَخَلُوا جُحْرَ ضَبٍّ لاَتَّبَعْتُمُوهُمْ قُلْنَا يَا رَسُولَ اللَّهِ الْيَهُودُ وَالنَّصَارَى قَالَ فَمَنْ" {\footnotesize (صحيح مسلم)}. 

يقول الشيخ العثيمين رحمه الله في بيان العينة:
أن يبيع شيئا بثمن مؤجل ثم يشتريه ممن باعه عليه بأقل منه نقدًا [.] وسُمي بذلك لأن المشتري لم يُرد السلعة وإنما أراد العين أي: النقد، لينتفع به، ودليل ذلك: أنه اشتراها بثمن زائد مؤجل، ثم باعها على من اشتراها منه بنقد، فكأنه لم يقصد هذه السلعة وإنما قصد الثمن الدراهم، فلهذا سمي بيع عينة [.] والغالب أن هذا ملازم لهذا، يعني أن الذي ينهمك في طلب الدنيا ويتحيل على الحصول عليها حتى بما حرم الله، الغالب أنه يترك الجهاد، لأن قلبه انشغل بالدنيا [هـ].
}

\subsection{الدولة الكافرة الظالمة}

الدولة أو الأمة الكافرة الظالمة هي التي لم تقبل الحق بكفرها ولم تقم الميزان بظلمها فيمسها من العقوبات في الدنيا بحسب ما قامت به الحجة عليهم سواءا كان لها التمكين أو لم يكن ولكن الله جل جلاله يؤخذها في الدنيا بمخالفة أدنى درجات العدل وهو الميزان الفطري. فعدل الدولة الكافرة يكون بموافقة الميزان الفطري، أي أن أفرادها، حكاما كانوا أو محكومين، لم يقيموا العدل الذي به تحفظ الأموال (كالغش في الكيل)، أو الأعراض (كالزنى واللواط)، أو الدماء (كالجور والقتل بغير حق)، وغيرها من الأمور التي تخالف الميزان الفطري. 

وهذا يكون للأمم أو الدول الكافرة الظالمة اللاحقة التي لم يبعث لها رسول على وجه الخصوص فلها من العذاب في الدنيا بحسب ما قامت به عليهم الحجة ومن ذلك قوله تعالى: 
\quranayah*[17][15]{\footnotesize \surahname*[17]}. فالله جل جلاله يؤاخذهم في الدنيا بمخالفتهم للميزان الفطري ولهذا فقد عمم رسولنا ﷺ العقوبة في المخالفات التي تخالف الفطرة السليمة كما تقدم معنا بيانه في الجدول \ref{tab:violations} كالمجاهرة بالمعاصي والغش في الكيل. ومن حكمة الله وعدله أن الله يسلط الأمم الأخرى أي الأمم الكافرة على الدولة المسلمة الظالمة حتى ترجع إلى دينها ولهذا فقد قال النبي ﷺ: "يُوشِكُ الأُمَمُ أَنْ تَدَاعَى عَلَيْكُمْ كَمَا تَدَاعَى الأَكَلَةُ إِلَى قَصْعَتِهَا، ألى أن قال: وَلَيَنْزِعَنَّ اللَّهُ مِنْ صُدُورِ عَدُوِّكُمُ الْمَهَابَةَ مِنْكُمْ وَلَيَقْذِفَنَّ اللَّهُ فِي قُلُوبِكُمُ الْوَهَنَ {\footnotesize (صححه الألباني)}. وقد تقدم بيان ذلك في حال الدولة المسلمة الظالمة وكيف تتكالب عليها الأمم الأخرى.

\subsection{الدولة الكافرة العادلة}

الدولة أو الأمة الكافرة العادلة هي التي لم تقبل الحق بكفرها ولم تقم الميزان الشرعي بمخالفة الميزان الديني ولكن أقامت أدنى درجات العدل وهو الميزان الفطري فتسلم بذلك في الدنيا من العقوبات بحسب ما وافقت به الفطرة وينزل عليها العذاب لمخالفة الحق والميزان الديني بحسب ما قامت به الحجة عليهم. ولكن الله يجعل لها من التمكين والنصر بحسب ما قامت به من العدل. وهذا لأن عدل الدولة الكافرة يكون بموافقة الميزان الفطري، أي أن أفرادها، حكاما كانوا أو محكومين، أقاموا العدل الذي به تحفظ الأموال، أو الأعراض، أو الدماء، وغيرها من الأمور التي توافق الميزان الفطري. وقد تقدم معنا ما نقله شيخ الإسلام ابن تيمية: وَلِهَذَا قِيلَ: إنَّ اللَّهَ يُقِيمُ الدَّوْلَةَ الْعَادِلَةَ وَإِنْ كَانَتْ كَافِرَةً؛ وَلَا يُقِيمُ الظَّالِمَةَ وَإِنْ كَانَتْ مُسْلِمَةً. وَيُقَالُ: الدُّنْيَا تَدُومُ مَعَ الْعَدْلِ وَالْكُفْرِ وَلَا تَدُومُ مَعَ الظُّلْمِ وَالْإِسْلَامِ {\footnotesize (مجموع الفتاوى 28/146)}.

وهذا ينطبق اليوم على حال كثير من دول الغرب التي على ما فيها من الكفر بدين الإسلام إلا أن القانون فيها نافذ على جميع مواطنيها فتعطى الحقوق وتقام الحدود بدون تفريق على ضعيفهم وشريفهم وبما اتفقوا عليه في برلمناتهم التشريعية ولهذا فقد سلموا في غالبهم من جور سلاطينهم عليهم. فمن ذلك على سبيل المثال الأنظمة الإجتماعية التي فيها تجمع الأموال من العاملين والشركات كالضرائب فتعطى لفقرائهم ومساكينهم وتمول بها البنى التحتية من خدمات صحية وغيرها والتي فيها صلاح معيشتهم. ومن ذلك أيضا فصل الأجهزة الرقابية والتشريعية عن الجيهات التنفيذية لمراقبتها ومحاسبتها ومراجعتها وإنفاذ القانون على من يستغل السلطة لسرقة المال العام أو التلاعب به وإهداره. وكل هذا لما لهم من الأخلاق والمحاسن الإنسانية الموافقة للفطرة من الصدق في القول والأمانة في العمل وهذا لا يخفى على كل من عاش في بلادهم. وقد أدرك ذلك الصحابي عمرو بن العاص بعدما سمع المستورد بن شداد يقول أن النبي ﷺ قال: تَقُومُ السَّاعَةُ والرُّومُ أكْثَرُ النَّاسِ. فَقالَ له عَمْرٌو بن العاص: أبْصِرْ ما تَقُولُ، قالَ: أقُولُ ما سَمِعْتُ مِن رَسولِ اللهِ صَلَّى اللَّهُ عليه وسلَّمَ، قالَ: لَئِنْ قُلْتَ ذلكَ، إنَّ فيهم لَخِصالًا أرْبَعًا: إنَّهُمْ لأَحْلَمُ النَّاسِ عِنْدَ فِتْنَةٍ، وأَسْرَعُهُمْ إفاقَةً بَعْدَ مُصِيبَةٍ، وأَوْشَكُهُمْ كَرَّةً بَعْدَ فَرَّةٍ وخَيْرُهُمْ لِمِسْكِينٍ ويَتِيمٍ وضَعِيفٍ، وخامِسَةٌ حَسَنَةٌ جَمِيلَةٌ: وأَمْنَعُهُمْ مِن ظُلْمِ المُلُوكِ. وفي رواية أخرى: إنَّهُمْ لأَحْلَمُ النَّاسِ عِنْدَ فِتْنَةٍ، وأَجْبَرُ النَّاسِ عِنْدَ مُصِيبَةٍ، وخَيْرُ النَّاسِ لِمَساكِينِهِمْ وضُعَفائِهِمْ {\footnotesize (صحيح مسلم)}. 

وكل هذا فيه أن الدولة الكافرة العادلة لها وعليها، فيذم كفرها ويحمد عدلها، ولا يرد عليها كل أمرها، بل يحمد ما فيها من العدل والإنصاف والمحاسن الإنسانية الموافقة للفطرة، ويذم ما فيها من كفر وفسق وعدوان على دين الله ورسله وهذا ما أوصانا به جل جلاله في كتابه العظيم فقال: 
\quranayah*[5][8]{\footnotesize \surahname*[5]}. وقال القرطبي في تفسيره: ودلت الآية أيضا على أن كفر الكافر لا يمنع من العدل عليه. وقال ابن كثير في تفسيره: وقوله: (ولا يجرمنكم شنآن قوم على ألا تعدلوا) أي: لا يحملنكم بغض قوم على ترك العدل فيهم، بل استعملوا العدل في كل أحد، صديقا كان أو عدوا. للمزيد من التفاصيل، راجع فصل \fullautoref{sec:app_justice}.

\subsection{الدولة المؤمنة العادلة}

الدولة أو الأمة المؤمنة العادلة هي التي عرفت الحق بقبولها للإسلام دينا لها وأقامت الميزان الشرعي التي أمرها الله به فعمل غالب أفرادها بتعاليم وأوامر الإسلام بإقامة الحق في أنفسهم وإقامة الميزان الشرعي فيما بينهم. فجزائها في الدنيا يكون بالتمكين على سائر الأمم الأخرى وهذا لأنها عرفت الحق وعملت به على الوجه المطلوب منها فأقامت العدل الذي أمرها الله به وهو الميزان الشرعي، أي أن أفرادها، حكاما كانوا أو محكومين، أقاموا العدل الذي به تحفظ الأموال، والأعراض، والدماء، وغيرها من الأمور التي توافق الميزان الفطري، بالإضافة إلى إقامة الميزان الديني كأداء الزكاة، والوفاء بعهد الله ورسوله، والحكم بما أنزل الله، وغيرها من الأوامر الدينية التي توافق الميزان الشرعي، وهذا يشمل الميزان الفطري والميزان الديني معا.

فالدولة المؤمنة العادلة هي التي آمنت بالحق وهو العلم الشرعي الصحيح من كتاب الله وسنة نبيه الثابتة والصحيحة وأقامت الميزان الشرعي بالعدل والقسط إتباعا لأمر الله جل جلاله. وقد وعدها جل جلاله بالتمكين فقال: \quranayah*[24][55]{\footnotesize \surahname*[24]}. وهذا فيه أن الله نعتهم بالإيمان لأنهم آمنوا بالحق وعملوا الصالحات الموافقة للميزان الشرعي الذي يرضي الله جل جلاله ومن أجل ذلك التوحيد وهو عبادة الله وحده لا شريك له، فوافقوا بذلك المعنى الحقيقي لكلمة التوحيد بإقامة الحق في أنفسهم وإقامة العدل فيما بينهم. وهذا بخلاف ما ينشر الأن أن التمكين يأتي بإقامة التوحيد في أنفس العباد فقط دون الإشارة إلى ما يلزم ذلك من إقامة العدل فيما بين العباد وهذا بخلاف الأدلة والبراهين الواضحة من كتاب الله جل جلاله. 

والأمة المؤمنة العادلة منصورة بوعد الله عز وجل وبوعد الرسول ﷺ إلى آخر الزمان حتى يقبضهم الله جل جلاله قبل قيام الساعة فعن عقبة بن نافع أنه سمع رَسولَ اللهِ ﷺ يقولُ: لا تَزَالُ عِصَابَةٌ مِن أُمَّتي يُقَاتِلُونَ علَى أَمْرِ اللهِ، قَاهِرِينَ لِعَدُوِّهِمْ، لا يَضُرُّهُمْ مَن خَالَفَهُمْ، حتَّى تَأْتِيَهُمُ السَّاعَةُ وَهُمْ علَى ذلكَ. فَقالَ عبدُ اللهِ: أَجَلْ، ثُمَّ يَبْعَثُ اللَّهُ رِيحًا كَرِيحِ المِسْكِ مَسُّهَا مَسُّ الحَرِيرِ، فلا تَتْرُكُ نَفْسًا في قَلْبِهِ مِثْقَالُ حَبَّةٍ مِنَ الإيمَانِ إلَّا قَبَضَتْهُ، ثُمَّ يَبْقَى شِرَارُ  النَّاسِ عليهم تَقُومُ السَّاعَةُ {\footnotesize (صحيح مسلم)}.

والدولة المؤمنة العادلة هي دولة الحكم الرشيد وهي الدولة التي أسسها النبي ﷺ وبدأها بالإصلاح الديني أولا ومن ثم قام عليها الخلفاء المهديين الراشدين من بعده بالإصلاح الديني والدنيوي معا حيث أقاموا الحق والميزان الشرعي مع الأخذ بالأسباب فكتب الله لهم التمكين وجعلهم ملوكا في الأرض حتى بلغ الإسلام الهند والصين شرقا وبلاد الأندلس وأروبا غربا. وسيأتي تفصيل ذلك في فصل الحكم الرشيد.

ولقد وضع النبي ﷺ لنا أسس هذه الدولة التي يحكم فيها بكتاب الله عز وجل وسنة نبيه ويكون فيها الشورى، والرحمة، واللين، والحكمة، والعدل، والرشاد. ففي هذه الدولة  يتساوى فيها المسلمين في الحقوق والواجبات الأساسية ومن أعظم ذلك حرمة الدم والمال والعرض وإن اختلفت ألوانهم وأشكالهم فقال ﷺ: يا أيُّها النَّاسُ، ألَا إنَّ ربَّكم واحِدٌ، وإنَّ أباكم واحِدٌ، ألَا لا فَضْلَ لِعَربيٍّ على عَجَميٍّ، ولا لعَجَميٍّ على عَرَبيٍّ، ولا أحمَرَ على أسوَدَ، ولا أسوَدَ على أحمَرَ؛ إلَّا بالتَّقْوى، إلى أن قال: فإنَّ اللهَ قد حَرَّمَ بيْنَكم دِماءَكم وأمْوالَكم وأعْراضَكم كحُرْمةِ يَومِكم هذا، في شَهرِكم هذا، في بَلَدِكم هذا، {\footnotesize (صحيح، تخريج المسند لشعيب، الصحيح المسند)}. ومن ذلك أيضا أن المسلمين يتساوون أيضا في الحدود وهذا من عدل الإسلام إذ تطبق الحدود على الشريف والضعيف على حد السواء بدون تفريق فقد صح عن النبي ﷺ قال: لو أنَّ فَاطِمَةَ بنْتَ مُحَمَّدٍ سَرَقَتْ لَقَطَعْتُ يَدَهَا {\footnotesize (صحيح مسلم)}. 

وكل هذا فيه أن دولة الحكم الرشيد هي الدولة المؤمنة العادلة التي تقام بإقامة الحق في النفوس وإقامة الميزان بين النفوس مع الأخذ بالأسباب. وهي الدولة التي تحفظ فيها الدماء والأموال والأعراض بما يرضي الله جل جلاله، فيقام فيها الكيل بالقسط، وتؤدى فيها الزكاة، وتقام فيها الحدود، وتتبع فيها أوامر الله جل جلاله من الحكام والمحكومين، كل بما أوتمن عليه من الأمانة، وما فرض عليه من العهد والميثاق، فكل ذلك سيئل عليه الجميع يوم الحساب. وهي الدولة التي حكامها يتخيرون في حكمهم مما أنزل الله جل جلاله حتى لا يكون حالهم كحال الدولة المسلمة الظالمة كما حذر من ذلك النبي ﷺ  كما تقدم. 

\subsection{ملخص حال الأمم}

بالتتبع والإستقراء يتبين أن الله يقيم الأمم التي تحكم بالعدل الذي يوافق الفطرة وإن كانت كافرة. فإن في ذلك سلامة من عذاب الله في الدنيا كما تقدم. فإن كانت مؤمنة وتحكم بالعدل كانت حكما راشدا ووعدها الله بالتمكين في الدنيا والفوز في الأخرة. وإن كانت مسلمة ولا تحكم بالعدل فقد خالفت حكمة الله والغاية من إيمانها الذي يقتضي إقامة العدل والميزان ويصدق فيها قوله تعالى:
\quranayah*[49][14]{\footnotesize \surahname*[49]}. وهذا هو حال أغلب أمة الإسلام في يومنا هذا كما هو معروف. فتكون بذلك الأمة الكافرة التي تحكم بالعدل قائمة فوق الأمة المسلمة التي لا تحكم بالعدل. وأما الأمة المؤمنة التي تقيم الحق وأجله التوحيد وما يقتضيه ذلك من إقامة الميزان ومنه العدل بين الناس مع الأخذ بالأسباب تكون هي فوقهم جميعا كما دلت على ذلك الآيات والأحاديث. وهذا فيه الحكمة البالغة من الله عز وجل ومنه أن الله لا يرضى لعباده الظلم. ولا يزال الذل والهوان هو حال الأمة المسلمة الظالمة حتى تقيم في غالبها، حكاما ومحكومين، الميزان الشرعي بالعدل الذي أمر الله به. وقد قال تعالى:
\quranayah*[13][11][12]{\footnotesize \surahname*[13]}.

وبهذا يعلم أن الأمم إنما تقام بإقامة الميزان بالعدل بين الناس فإن تحقق ذلك سلمت سخط الله وعذابه في الدنيا وإن كانت كافرة. فإن لم تقم الميزان بالعدل بين الناس، تكون بذلك قد جنت على نفسها عقاب الله العاجل في الدنيا من فقر وجوع وذل وجور السلطان وإن كانت مسلمة. وأما إن كانت مؤمنة وأقامت الحق مع إقامة الميزان والأخذ بالأسباب كما أمر الله تعالى كانت حكما راشدا وتحقق لها التمكين في الدنيا والفوز في الأخرة. وأما إقامة التوحيد دون إقامة الميزان وما يقتضيه من العدل بين الناس فهذا ينافي حكمة الله وأمره الذي بينه في كتابه وعلى لسان نبيه ﷺ. ولهذا وجب على المسلمين ودعاتهم الرجوع إلى أمر الله وعدم التهاون في ذلك بالعناية بإقامة الحق وأجلُّه التوحيد وإقامة الميزان وأجلُّه العدل بين الناس على حد السواء، مع الأخذ بالأسباب حتى يكون لهم التمكين الذي وعد الله به. ولمعرفة الأسباب وفهمها وجب علينا العناية بعلم الحساب بحثا وتطبيقا سعيا لتحقيق هذه الغاية العظيمة التي أمرنا الله بها، والتي عليها يبنى الحكم الرشيد.

والدولة المؤمنة العادلة لا تأتي بالعواطف والأوهام، إنما تكون عن علم وإخلاص في القول والعمل وصبر على أقدار الله ويقين بوعد الله عز وجل. وهذا يتطلب الجهاد في سبيل الله جهاد النفس وجهاد الغير سعيا لإقامة الحق والميزان الشرعي وليس لإبتغاء السلطة أو العلو في الأرض وإنما لنشر الحق والعدل الذي أمر الله به. وهذا لا يقوم به إلا من أخلص لله حقا وباع نفسه وماله لله عن صدق ورضي أن يدخل مع الله في هذه التجارة الرابحة ولهذا فقد قال جل جلاله: \quranayah*[9][111]{\footnotesize \surahname*[9]}. وهذه التجارة مع الله لإبتغاء مرضاته والسعي لبلوغ جنته التي وعد المتقين وللنجاة من عذابه وسخطه يوم الحساب. ولهذا يأتي بيان يوم الحساب قبل بيان الحكم الرشيد حتى يعلم الإنسان ما فيه نجاته في ذلك اليوم العظيم، وأن الغاية ليست الحكم الرشيد أو الخلافة الراشدة في حد ذاتها كما يتغنى بذلك الكثير من الواهمين، إنما الحكم الرشيد والخلافة الراشدة هي وسيلة لغاية عظيمة وهي إقامة الحق والميزان مع الأخذ بالأسباب كما بين جل جلاله ذلك في كتابه العظيم.

\section{الحساب من صور الميزان والجهل به من الأمية}

تقدم معنا أن الميزان يأتي بمعنى العدل، ولكن الميزان له صور مختلفة. فعلى سبيل الميثال الكيل المعروف الذي به توزن الأشياء هو صورة من صور الميزان ولكنه ميزان حسي. وكذلك الحساب فهو أيضا صورة من صور الميزان ولكنه ميزان معنوي، أي أنه لا يلزم أن يكون في صورة حسية معينة ولكن قد يجرى الحساب كتابيا أو ذهنيا أو بإستخدام الآلة الحاسبة أو أجهزة الكمبيوتر أو بإستخدام مجموعة كبيرة من الأجهزة المعقدة. وهذا الحساب يتنوع ويتدرج من السهل والبسيط إلى الصعب والمعقد. والحساب حاله كحال الكيل يبنى على الميزان والوزن الصحيح ويجب الوفاء به والصدق فيه فهذا من الإصلاح والعدل الذي أمر الله به والذي فطر الناس عليه كما تقدم بيانه في الميزان الشرعي بأقسامه. وعليه فإن الحساب لا يكون صحيحا إلا بالقسط والعدل كما في الكيل الوزن تماما. ومن رحمة الله أنه فطر الناس على هذا وجعل لهم كل ما يحتاجونه من عقل وسمع وبصر لفهم الحساب والعدد وليبنوا عليه مصالحهم الدينية والدنيوية. ولهذا فإن الآيات الشرعية التي تأمر بالعدل في الكيل وإقامة الميزان بالقسط فهي بلا شك تشمل الحساب وهو الميزان المعنوي كما هو الحال مع الكيل وهو الميزان الحسي. والغش في الحساب من مخالفة الميزان ومن الظلم والفساد الذي لا يرضى الله به ومن أسباب تعجيل سخط الله في الدنيا قبل الأخرة.

وبهذا يعلم أن الحساب الصحيح يكون سبيلا إلى الحكم بالعدل والحكم الرشيد. فإن وافق الحساب الميزان الفطري (أي الفطرة السليمة) كان ذلك نجاة من سخط الله وعذابه في العاجلة (اي في الدنيا). ومثال ذلك أن يكون حساب البائع حسابا صحيحا مع زبائنه بدون غش أو نقص. وإن وافق الحساب الميزان الديني كان ذلك نجاة من سخط الله وعذابه في الأخرة. ومثال ذلك حساب المواريث والزكاة الحساب الصحيح. وإن وافق الحساب الميزان الشرعي (أي الميزان الفطري والميزان الديني معا) كان ذلك نجاة من سخط الله وعذابه في الدنيا والآخرة معا. ومثال ذلك إخراج الزكاة صحيحة وكاملة من تجارة لا غش فيها ولا ظلم. وفي غالب الأحيان يكون الحساب الموافق للميزان الديني موافقا أيضا للميزان الفطري وهذا لأن الدين جاء موافقا للفطرة ومكملا لها. ولهذا كان الحساب الصحيح لإقامة الحق والميزان الشرعي من بنيان الحكم الرشيد. وعليه يكون علم الحساب من الدين بالضرورة وليس بخلاف ذلك إذ يتعذر إقامة الميزان حق إقامته من دون حساب صحيح. وهو أيضا مفتاح العلوم السببية، إذ يتعذر معرفة الأسباب وفهمها من دون حساب صحيح.

ومن أعظم البلايا في زماننا هذا أن المسلمين قد أضاعوا هذا العلم العظيم ظنا منهم أنه ليس من الدين بعد أن كانوا روادا فيه ووضعوا أسسه وقواعده في زمن هارون الرشيد فسبقوا بذلك كل الأمم الأخرى كما سيأتي. فاعتنت وتسابقت وتهالفت عليه الأمم الأخرى وكان سببا في نهوضها وإزدهارها بل وأيضا تسلطها على أمة الإسلام. فتضييع علم الحساب من الأمية التي جاء الإسلام بالحث على خلافها من طلب العلم ونشره وإقامة الحق والميزان. فالأمية لا تكون فقط بعدم القدرة على القراءة والكتابة كما هو شائع، وإنما ايضا بعدم القدرة على الحساب.
ومما يأكد هذا الطرح قوله ﷺ عندما سأل عن عدد الأيام في الشهر فقال ﷺ:
إنَّا أُمَّةٌ أُمِّيَّةٌ، لا نَكْتُبُ ولَا نَحْسُبُ، الشَّهْرُ هَكَذَا وهَكَذَا. يَعْنِي مَرَّةً تِسْعَةً وعِشْرِينَ، ومَرَّةً ثَلَاثِينَ" {\footnotesize (صحيح البخاري)}. فجعل ﷺ الجهل بعلم الحساب في زمانه من الأمية.

ولهذا جاءت الشريعة بالحث أولا على القراءة ومن ثم الحساب. فكان أول ما أنزل الله "إقرا" وفيه الحث لأمة الإسلام على تعلم القراءة والكتابة وطلب العلم ونشره.
يقول تعالى:
\quranayah*[96][1-5]{\footnotesize \surahname*[96]}.
ومن ثم جاء الحث على التأمل في آيات الله الكونية في مواضع كثيرة ليس فقط لمجرد التفكر في خلق الله ولكن أيضا لتعلم العدد والحساب كما جاء في قوله تعالى:
\quranayah*[10][5]{\footnotesize \surahname*[10]}. ولهذا فإن كل إنسان لم يتعلم الحساب مع القراءة والكتابة يكون أميا كما بين ذلك النبي ﷺ. والله حث هذه الأمة الأمية في كتابه العظيم على العلم الذي يتأتى بالقراءة والكتابة حتى تقيم الحق وعلى تعلم العدد والحساب حتى تقيم الميزان الشرعي بالعدل والقسط.

\comment{
ومن المعلوم أن الرسول ﷺ كان أميا كما قال تعالى:
\quranayah*[7][158][21]{\footnotesize \surahname*[7]}.
وكذلك الصحابة رضى الله عنهم أجمعين وصفهم الله بالأمية في قوله تعالى:
\quranayah*[62][2]{\footnotesize \surahname*[62]}. فالرسول ﷺ والصحابة رضى الله عنهم في غالبهم كانوا أمة أمية لا يقرأون ولا يكتبون ولا بحسبون كما ورد هذا في كتاب الله وسنة نبيه ﷺ.
يقول الشيخ اين باز رحمه الله تعالى:
فكان النبي ﷺ مثلما قال الله: وَوَجَدَكَ ضَالًّا فَهَدَى [الضحى:7]، جاهلاً بالعلوم التي جاء بها الوحي، لم يكن عنده علم بما شرع الله له في كتابه العظيم، ولم يكن عنده علم بعلوم الأولين المرسلين، ولم يكن يكتب ويخط حتى جاءه هذا الخير العظيم والوحي العظيم ﷺ، فكل إنسان لم يتعلم ولم يكتب يقال له: أمي، والأمة العربية هكذا كان الغالب عليها أنها أمية لا تكتب ولا تقرأ، هذا الغالب على أمة محمد ﷺ
[هـ].
}

\section{الحساب الصحيح هو الميزان}

والحساب الصحيح لا يقام إلا بالميزان فعَنْ أَبِي سَعِيدٍ وَأَبِي هُرَيْرَةَ: أَنَّ رَسُولَ اللَّهِ صَلَّى اللَّهُ عَلَيْهِ وَسلم اسْتَعْمَلَ رَجُلًا عَلَى خَيْبَرَ فَجَاءَهُ بِتَمْرٍ جَنِيبٍ فَقَالَ: «أَكُلُّ تَمْرِ خَيْبَرَ هَكَذَا؟» قَالَ: لَا وَاللَّهِ يَا رَسُولَ اللَّهِ, إِنَّا لَنَأْخُذُ الصَّاعَ مِنْ هَذَا بِالصَّاعَيْنِ وَالصَّاعَيْنِ بِالثَّلَاثِ فَقَالَ: «لَا تَفْعَلْ بِعِ الْجَمْعَ بِالدَّرَاهِمِ ثُمَّ ابْتَعْ بِالدَّرَاهِمِ جَنِيبًا». وَقَالَ: «فِي الْمِيزَانِ مِثْلَ ذَلِكَ» {\footnotesize (مُتَّفَقٌ عَلَيْهِ)}.
وهذا فيه حرص النبي ﷺ حيث انه من المعلوم أن من أخذ صاعا اضافيا لا يثبت قيمة البيع. فيكون من أخذ صاعين بدل صاع فقد اشترى بنصف قيمة ما باع، بينما من أخذ ثلاثة بدل اثنين فقد اشترى بثلثي قيمة ما باع. وهذا من الظلم الذي لا يقع إلا خطأ أو جهلا أو غشا. فأخبر النبي ﷺ أن هذا بخلاف الميزان وهو الحساب الصحيح في البيع والشراء, بل ونهى عن ذلك وأمر بأخذ القيمة عند البيع ومن ثم الشراء حتى تثبت القيمة. وفيه أيضا أن الرسول ﷺ سمى الحساب الصحيح ميزانا في قوله "في الميزان مثل ذلك". وهذا فيه دليل على نبوته ﷺ فهو أمي لا يحسب ولكن لا ينطق إلا بالحق كما أخبر ذلك الله عز وجل في كتابه العظيم:
\quranayah*[53][3-4]{\footnotesize \surahname*[53]}.

\comment{

ومما يؤكد ما سبق أيضا قوله ﷺ:
الذهب بالذهب وزنًا بوزن، مثلًا بمثل، سواءً بسواء، يدًا بيد.
يقول الشيخ ابن باز رحمه الله تعالى في بيان معنى هذا الحديث: ولا فرق بين كونه جديدًا أو قديمًا، أو كون هذا أطيب وهذا أطيب، ما دام جنس الذهب لابد أن يكونا متساويين في الوزن يدًا بيد، يقبض في الحال [هـ]. وعن أبي سعيد الخدري رضي الله عنه أن رسول الله ﷺ قال: لا تبيعوا الذهب بالذهب إلا مثلا بمثل ولا تشفوا -أي تفاضلوا- بعضها على بعض، ولا تبيعوا منها غائبا بناجز
{\footnotesize (متفق عليه)}.

والحساب يبنى على التقدير العددي والتقدير الفكري. أما التقدير العددي فهو يضبط بالوزن والتقدير فكري يضبط بالحق. ولهذا يكون تقدير المخلوق محدود وناقص بما توفر لديه من علم وإدراك. وأما تقدير الخالق فهو تقدير كامل لا نقص فيه لأن الله هو العليم بكل شئ والقادر على كل شئ. ومن تقدير الله التقدير الكوني والتقدير الشرعي.ولما كان تقدير الله هو الحق وهو الميزان الذي لا نقص فيه, وافق تقديره الكوني سبحانه الميزان الكومي وتقديره الشرعي الميزان الشرعي.
}

ولهذا فإن الحساب الصحيح يبنى على التقدير العددي والوزن وهو ما نعرفه اليوم بالتساوي (أي علامة = في الحساب). فقد سماه الخوارزمي رحمه الله تعالى بالجبر والمقابلة بحيث يبنى الحساب على التساوي بين المتغيرات لجبر ما اختل من الميزان, عليه يمكن حساب ما جهل منها. فقد قال ابن تيمية رحمه الله عن هذا:
وأما حساب الفرائض ومعرفة أصول المسائل وتصحيحها والمناسخات وقسمة التركات، وهذا الثاني كله علم معقول يُعلم بالعقل، كسائر حساب المعاملات وغير ذلك من الأنواع التي يحتاج إليها الناس [.] ثم قد ذكروا حساب المجهول الملقب بحساب الجبر والمقابلة في ذلك، وهو علم قديم [.] أول من عرف أنه أدخله فيها محمد بن موسى الخوارزمي. وبعض الناس يذكر عن علي بن أبي طالب أنه تكلم فيه، وأنه تعلم ذلك من يهودي، وهذا كذب على علي {\footnotesize (مجموع الفتاوى 9/214)}.
ويقول اين تيمية فيه أيضا:
وكذلك كثير من متأخري أصحابنا يشتغلون وقت بطالتهم بعلم الفرائض والحساب والجبر والمقابلة والهندسة ونحو ذلك، لأن فيه تفريحاً للنفس، وهو علم صحيح لا يدخل فيه غلط {\footnotesize (مجموع الفتاوى ج9/ص129)}. وكل هذا فيه اهتمام السلف بهذا العلم العظيم في أوقات فراغهم رغم إنشغالهم بالأمور العظيمة الأخرى في بيان الحق ورد البدع والشبهات التي عصفت في ذلك الزمن كعلم الكلام والفلفسة التي تخالف صريح كتاب الله جل جلاله وسنة نبيه ﷺ.

\comment{
ويقول الشيخ الفوزان حفظه الله:
أما ما كان من علم الحساب الذي ينتفع به في معرفة المواقيت ومعرفة القبلة فهذا مباح وهو ما يسمى علم التسيير [.] وهو معرفة الحساب الذي به ينتفع الناس في مواقيت عباداتهم ومعاملاتهم ومواقيت زروعهم وغرس أشجارهم ويستدلون به على القبلة فهذا مباح وقد يجب تعلمه إذا كان يعين على أداء العبادات في مواقيتها [هـ].
}

والناس يتفاوتون في قدراتهم الحسابية كل بحسب إجتهاده وبما أودعه الله جل جلاله فيهم من إدراك وعقل. ولهذا فإن حساب الإنسان يكون بحسب إدراكه وعقله وتقديره. وهذا فيه أن الإنسان قد يخفى عليه العديد من الأمور وقد يتخلل حسابه النقص وبالأخص في الأمور التي يصعب حسابها. ولكن الله جعل الناس متوافقين في الحساب بفطرتهم. والله جل جلاله له كمال العدل في حسابه وهذا لأن تقديره تقدير كامل لا نقص فيه لأن الله هو العليم بكل شئ والقادر على كل شئ. وفي معرفة حساب الله العديد من الفوائد منها أن الله عدل وقائم بالقسط في الدنيا والآخرة، وفي الميزان الكوني والميزان الشرعي. وفي معرفة ذلك الإستعداد ليوم الحساب الذي فيه يحاسب الله المكلفين كما سيأتي بيانه في الفصل القادم. وفي هذه المعرفة أيضا الترغيب في إقامة الحق والميزان مع الأخذ بالأسباب والتي بها يمكن تحقيق الحكم الرشيد كما سيأتي بيانه في الفصل الذي يلي الفصل القادم.
