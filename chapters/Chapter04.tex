\chapter{الحساب الكوني}


\section{مقدمة}

خلق الله كل شئ بقدر معلوم ووضع الميزان فجعل هذا الكون موزونا ومتناسقا سبحانه. ومن حكمته أنه سبحانه فطر الناس على فهم ذلك الميزان وجعل لهم كل ما يحتاجونه في ذلك من عقل وسمع وبصر. ومن فضله علي الناس أنه أرسل إليهم الرسل وأنزل الكتاب وما فرط فيه من شئ. ومن ذلك أنه أرشد سبحانه التأمل في آياته الكونية لتعلم العدد والحساب. ولهذا فإن أفضل طريقة لفهم علم الحساب هو التأمل والتفكر في الظواهر الطبيعية التي خلقها الله وومحاولة حسابها. فهي المرجع لنا حتى نتحقق من صحة وسلامة الحساب. وهذا النهج هو نهج القرآن وهو أفضل الطرق وأحسنها. ويمكن دراسة علم الحساب مجردا من أي تطبيقات وهذا أيضا نهج معروف. ولكن الجمع بين علم الفيزياء والحساب لمحاولة محاكاة الظواهر الطبيعية من أفضل الطرق لتطوير علم الحساب. وهذا معروف حيث تفوق الباحثين الذين جمعوا بين الحساب والفيزياء مثل نيوتن وأنشتاين وفورير على غيرهم ممن درس علم الرياضيات المجرد فكانوا روادا في ذلك.

ولهذا يكون الطريق للبحث وفهم علم الحساب كما يلي:
\begin{compactenum}
  \item التفكر في آيات الله الكونية والتأمل فيها ومحاولة فهمها وربطها ببعضها البعض على وجه الإجمال.
  \item حساب هذه الظواهر على إنفراد ومع التدرج في التعقيد حتى يمكن حسابها بالدقة المطلوبة ومن عدة طرق وجوانب.
  \item الجمع بين الظواهر المترابطة ومحاولة فهم ترابطها وتأتيرها على بعضها البعض وحساب ذلك لبناء فهم أكثر شمولا ودقة.
  \item تلخيص القواعد الحسابية بناءا على ما سبق وتطبيقها في فهم ظواهر أخرى أكثر تعقيدا أو تصحيح الحساب فيما ينفع الناس في أمور دينهم ودنياهم.
\end{compactenum}



\section{جداول}
فيما يلي مثال على جدول:

\begin{table}[h]
  \centering
  \begin{tabular}{|c|c|c|}
    \hline
    العنوان 1 & العنوان 2 & العنوان 3 \\
    \hline
    الخلية 1  & الخلية 2  & الخلية 3  \\
    \hline
    الخلية 4  & الخلية 5  & الخلية 6  \\
    \hline
  \end{tabular}
  \caption{مثال على جدول}
\end{table}

\newpage

% Example of a reference using BibTeX
\section{مراجع باستخدام BibTeX}
لإضافة مراجع باستخدام BibTeX، يمكن استخدام الملف التالي `references.bib`:

\begin{verbatim}
@book{example,
  author    = "المؤلف",
  title     = "عنوان الكتاب",
  publisher = "دار النشر",
  year      = "السنة"
}
\end{verbatim}

ثم تضمين المراجع في المستند الرئيسي:

\begin{verbatim}
\bibliographystyle{plain}
\bibliography{references}
\end{verbatim}

\newpage

\section{نص الفصل الثاني - الصفحة الثانية}

هذه الصفحة الثانية للفصل الثاني تحتوي على نص إضافي لاختبار تقسيم الصفحات وظهور الرؤوس والأقدام بشكل صحيح في النصوص العربية.

\newpage

\section{نص الفصل الثاني - الصفحة الثالثة}

هذه الصفحة الثالثة للفصل الثاني تحتوي على المزيد من النصوص لاختبار تقسيم الصفحات وظهور الرؤوس والأقدام بشكل صحيح في النصوص العربية.
