\chapter{الذكاء الإصطناعي}


\section{مقدمة}





\section{المعرفة والوعي والإدراك}
هناك نقاش مهم يتعلق بموضوع ما إذا كانت نماذج الذكاء الإصطناعي العام ستكون قادرة على المعرفة والإدراك والوعي في المستقبل القريب (تقريبا مع حلول 2030). وحبيت أن أنوه على عدة أمور وخاصة للمسلمين.



أغلب الباحثين في مجال الذكاء الإصطناعي يعتقدون أن نماذج الذكاء الإصطناعي العام والتي ستكون لها قدرات معرفية تتحاوز قدرة الإنسان في جميع المجالات, ستكون قادرة على الإدراك وسيكون لها وعي. وبطريقة غير مباشرة هذا ما هو إلا مقدمة للقول بأن هذه النماذج لها أرواح مثلها مثل البشر.



بالمختصر هذا أمر خطير يمس عقيدة المسلمين. والشرح كالأتي.



أولا يجب التنويه لأمر هام جدا ألا وهو أن المعرفة والإدراك والوعي كلها تطلب مشاعر وهي أمور تختص بها الروح والتي هي من أمر الله التي بها يبث الله الحياة في خلقه. وهذا معروف بما نعلمه عن الروح بما دلت عليه البراهين في الكتاب والسنة من أن الروح تنفخ في الجنين في بطن أمه وتصعد إلى السماء وتهبط وتقبض في النوم وترسل وتعذب في القبر وتضرب ضربة يسمعها كل شي إلا الثقلين. فالأنس والجن والحيوانات كلها ذات أرواح وتكون حية فقط عند وجود الروح في الجسد وتموت بمفارقة الروح الجسد. وهذا موضوع بحث فيه الإمام ابن القيم في كتابه الروح لمن أراد أن يطلع. وقد ذكر الشيخ ابن باز رحمه الله أن الحياة نوعاة: حياة روح وحركة مثل الأنسان والحيوان والجن وحياة نمو فقط مثل النباتات وهي ليس لها خصائص حياة الروح مثل السمع والبصر والحركة.

أنظر الرابط أول التعليق.



ولهذا فإن الإدعاء بأن نماذج الذكاء الإصطناعي العام سيكون لها وعي وإدراك بمعنى القدرة على الإحساس والإستقلال بذاتها كما هو معروف من خصائص الروح هو أمر ينافي عقيدة المسلمين.





القدرات المعرفية الخارقة المجردة من الإحساس لا تلزم الوعي والإدراك. فالحاسب الإلي له قدرة كبيرة لحفظ ومعالجة وتحليل كمبيات كبيرة من البيانات وتزداد هذه القدرة مع زيادة القدرة الحسابية والتخزينية. ولكن هذا لا يجعل الحاسب واعي أو مدرك.