
\chapter{الحكم الرشيد}

\section{مقدمة}


قال تعالى:


يَا دَاوُودُ إِنَّا جَعَلْنَاكَ خَلِيفَةً فِي الْأَرْضِ فَاحْكُم بَيْنَ النَّاسِ بِالْحَقِّ وَلَا تَتَّبِعِ الْهَوَىٰ فَيُضِلَّكَ عَن سَبِيلِ اللَّهِ ۚ إِنَّ الَّذِينَ يَضِلُّونَ عَن سَبِيلِ اللَّهِ لَهُمْ عَذَابٌ شَدِيدٌ بِمَا نَسُوا يَوْمَ الْحِسَابِ [ص : 26]


تفسير ابن كثير:

هذه وصية من الله - عز وجل - لولاة الأمور أن يحكموا بين الناس بالحق المنزل من عنده تبارك وتعالى ولا يعدلوا عنه فيضلوا عن سبيله وقد توعد [ الله ] تعالى من ضل عن سبيله ، وتناسى يوم الحساب ، بالوعيد الأكيد والعذاب الشديد .

قال ابن أبي حاتم : حدثنا أبي حدثنا هشام بن خالد حدثنا الوليد ، حدثنا مروان بن جناح ، حدثني إبراهيم أبو زرعة - وكان قد قرأ الكتاب - أن الوليد بن عبد الملك قال له : أيحاسب الخليفة فإنك قد قرأت الكتاب الأول ، وقرأت القرآن وفقهت ؟ فقلت : يا أمير المؤمنين أقول ؟ قال : قل في أمان . قلت يا أمير المؤمنين أنت أكرم على الله أو داود ؟ إن الله - عز وجل - جمع له النبوة والخلافة ثم توعده في كتابه فقال : ( يا داود إنا جعلناك خليفة في الأرض فاحكم بين الناس بالحق ولا تتبع الهوى فيضلك عن سبيل الله إن الذين يضلون ) الآية .

وقال عكرمة : ( لهم عذاب شديد بما نسوا يوم الحساب ) هذا من المقدم والمؤخر لهم عذاب شديد يوم الحساب بما نسوا .

وقال السدي : لهم عذاب شديد بما تركوا أن يعملوا ليوم الحساب .

وهذا القول أمشى على ظاهر الآية فالله أعلم .

\section{أركان الحكم الرشيد}

يقام الحكم الرشيد على ثلاثة أركان وهي: (1) إقامة الحق بالعلم الصحيح، (2) إقامة الميزان الشرعي بالعدل والقسط، (3) الأخذ بأسباب القوى كالحديد وما يلزم ذلك من علوم كالحساب والطب وغيرها من العلوم التي تكمن المسلمين من دحر الأعداء ونشر الحق ونصرته. فهذا كله من نصر الله ورسله وقد وعد سبحانه بنصر من ينصره كما في قوله تعالى:
\quranayah*[47][7]{\footnotesize \surahname*[47]}. 

وبهذه الأمور الثلاثة التي يبنى عليها الحكم الرشيد يكون التمكين الذي وعد الله به كما ذكر سبحانه في قصة ذي القرنين في قوله تعالى: 
\quranayah*[18][84-85]{\footnotesize \surahname*[18]}. وقد بين معنى ذلك الشيخ العثيمين رحمه الله أن معنى "من كل شئ سببا" أن الله أتاه كل الأسباب التي بها يكون التمكين في الأرض من قوة السلطة وتمام الملك فانتفع بما أعطاه الله من الأسباب. فهذا التمكين جاء بتسخير الله وهذا لأن ذي القرنين أخذ بالأسباب التي أعطاها الله له مع إقامة الحق وإقامة العدل كما في قوله تعالى: 
\quranayah*[18][87-89]{\footnotesize \surahname*[18]}. وقد جاء في تفسير السعدي رحمه الله أن هذا يدل على كونه من الملوك الصالحين الأولياء، العادلين العالمين، حيث وافق مرضاة الله في معاملة كل أحد، بما يليق بحاله [هـ]. وقد أثبت  سبحانه له التمكين والرشد لما له من الخبرة في إتباع الأسباب كما في قوله تعالى: 
\quranayah*[18][91-92]{\footnotesize \surahname*[18]}. ومعنى ذلك أي: أحطنا بما عنده من الخير والأسباب العظيمة كما جاء في تفسير السعدي. ومن ذلك أنه كان لديه من الأسباب العلمية ما يمكنه من فهم العديد من العلوم التي تمكنه من الإنتقال إلى مشارق الأرض ومغاربها وفهم اللغات الأخرى، ومن ذلك ما فقه به ألسنة أولئك القوم (الذين لا يفقهون قولا) الذين اشتكوا إليه ضرر يأجوج ومأجوج [.] إفسادهم في الأرض، فلم يكن ذو القرنين ذا طمع، ولا رغبة في الدنيا، ولا تاركا لإصلاح أحوال الرعية، بل كان قصده الإصلاح، فلذلك أجاب طلبتهم لما فيها من المصلحة، ولم يأخذ منهم أجرة، وشكر ربه على تمكينه واقتداره [هـ]. فلم يطلب منهم إلا أن يعينوه على حمل زبر أي قطع الحديد ووضعه في مكانه بين الجبلين وإشعال النار له بالمنافيخ الشديدة والآلات العظيمة لإذابة النحاس حتى يكون سائلا فيصبه عليها ليستحكم السد استحكاما هائلا يعجز يأجوج ومأجوج على الصعود فوقه فضلا عن ثقبه. 

وقد علم ذا القرنين أن كل ذلك من فضل الله عليه حيث قال تعالى: 
\quranayah*[18][98]{\footnotesize \surahname*[18]}. وما أجمل ما أورده السعدي في تفسيره هذه الآية حيث قال: 
فلما فعل هذا الفعل الجميل والأثر الجليل، أضاف النعمة إلى موليها وقال: { هَذَا رَحْمَةٌ مِنْ رَبِّي ْ} أي: من فضله وإحسانه عليَّ، وهذه حال الخلفاء الصالحين، إذا من الله عليهم بالنعم الجليلة، ازداد شكرهم وإقرارهم، واعترافهم بنعمة الله كما قال سليمان عليه السلام، لما حضر عنده عرش ملكة سبأ مع البعد العظيم، قال: 
\quranayah*[27][40][19]{\footnotesize \surahname*[27]}.
بخلاف أهل التجبر والتكبر والعلو في الأرض فإن النعم الكبار، تزيدهم شرا وبطرا. كما قال قارون، لما آتاه الله من الكنوز، ما إن مفاتحه لتنوء بالعصبة أولي القوة، قال: 
\quranayah*[28][78][1-6]{\footnotesize \surahname*[28]}. 

وقد علم أيضا ذا القرنين بما لديه من الخبرة بأسباب الحديد وما قد يطرأ عليه من صدإ وتآكل بعد زمن أنه سياتي يوم وينهار هذا السد العظيم ويخرج يأجوج ومأجوج في آخر الزمان كما في قوله تعالى: 
\quranayah*[18][98][6]{\footnotesize \surahname*[18]}. وجاء في تفسير السعده رحمه الله أن قوله: { فَإِذَا جَاءَ وَعْدُ رَبِّي ْ} أي: لخروج يأجوج ومأجوج { جَعَلَهُ ْ} أي: ذلك السد المحكم المتقن { دَكَّاءَ ْ} أي: دكه فانهدم، واستوى هو والأرض { وَكَانَ وَعْدُ رَبِّي حَقًّا ْ} [هـ]. ولهذا فقد أخبر سبحانه بوقوع ذلك لا محالة في قوله تعالى: 
\quranayah*[21][96]{\footnotesize \surahname*[21]}. وقد قال السعدي رحمه الله: أنه في آخر الزمان، ينفتح السد عنهم، فيخرجون إلى الناس في هذه الحالة والوصف، الذي ذكره الله من كل من مكان مرتفع، وهو الحدب ينسلون أي: يسرعون. وفي هذا دلالة على كثرتهم الباهرة، وإسراعهم في الأرض، إما لبذواتهم، وإما لما خلق الله لهم من الأسباب التي تقرب لهم البعيد، وتسهل عليهم الصعب، وأنهم يقهرون الناس، ويعلون عليهم في الدنيا، وأنه لا يد لأحد بقتالهم. 

وكل ما تقدم فيه أن ذا القرنين لم يكن فقط يأخذ بالأسباب وإنما كان يقيم الحق والعدل مع الأخذ بالأسباب والعلم بها وذلك من فضل الله عليه وتوفيقه له رحمه الله تعالى ورضي عنه. وقد قال عنه الشيخ ابن باز رحمه الله: ذو القرنين ملك عظيم صاحب خير، وإحسان، وإصلاح، واختلف الناس في نبوته، والمشهور أنه ملك صالح. وفي موضع آخر رجح الشيخ ابن باز رحمه الله أن ذا القرنين نبيا من الأنبياء لأنه كان يتبع أمر الله في الأرض وظاهر الآيات أنه كان يتلقى هذه الأوامر والتوجيهات من ربه جل جلاله وهذا شأن النبي.

\section{شروط الحكم الرشيد}

ولقد وضع النبي ﷺ لنا أسس هذه الدولة التي يحكم فيها بكتاب الله عز وجل وسنة نبيه ويكون فيها الشورى، والرحمة، واللين، والحكمة، والعدل، والرشاد. ففي هذه الدولة  يتساوى فيها المسلمين في الحقوق والواجبات الأساسية ومن أعظم ذلك حرمة الدم والمال والعرض وإن اختلفت ألوانهم وأشكالهم فقال ﷺ: يا أيُّها النَّاسُ، ألَا إنَّ ربَّكم واحِدٌ، وإنَّ أباكم واحِدٌ، ألَا لا فَضْلَ لِعَربيٍّ على عَجَميٍّ، ولا لعَجَميٍّ على عَرَبيٍّ، ولا أحمَرَ على أسوَدَ، ولا أسوَدَ على أحمَرَ؛ إلَّا بالتَّقْوى، أبَلَّغتُ؟ قالوا: بَلَّغَ رسولُ اللهِ، ثم قال: أيُّ يَومٍ هذا؟ قالوا: يَومٌ حَرامٌ، ثم قال: أيُّ شَهرٍ هذا؟ قالوا: شَهرٌ حَرامٌ، قال: ثم قال: أيُّ بَلَدٍ هذا؟ قالوا: بَلَدٌ حَرامٌ، قال: فإنَّ اللهَ قد حَرَّمَ بيْنَكم دِماءَكم وأمْوالَكم وأعْراضَكم كحُرْمةِ يَومِكم هذا، في شَهرِكم هذا، في بَلَدِكم هذا، أبَلَّغتُ؟ قالوا: بَلَّغَ رسولُ اللهِ، قال: لِيُبلِّغِ الشَّاهِدُ الغائِبَ {\footnotesize (صحيح، تخريج المسند لشعيب، الصحيح المسند)}.

ومن ذلك أيضا أن المسلمين يتساوون أيضا في الحدود وهذا من عدل الإسلام إذ تطبق الحدود على الشريف والضعيف على حد السواء بدون تفريق فقد صح عن عائشة أم المؤمنين رضي الله عنها أنها قالت: أنَّ قُرَيْشًا أَهَمَّهُمْ شَأْنُ المَرْأَةِ الَّتي سَرَقَتْ في عَهْدِ النبيِّ صَلَّى اللَّهُ عليه وسلَّمَ في غَزْوَةِ الفَتْحِ، فَقالوا: مَن يُكَلِّمُ فِيهَا رَسولَ اللهِ صَلَّى اللَّهُ عليه وسلَّمَ؟ فَقالوا: وَمَن يَجْتَرِئُ عليه إلَّا أُسَامَةُ بنُ زَيْدٍ، حِبُّ رَسولِ اللهِ صَلَّى اللَّهُ عليه وسلَّمَ ، فَأُتِيَ بهَا رَسولُ اللهِ صَلَّى اللَّهُ عليه وسلَّمَ، فَكَلَّمَهُ فِيهَا أُسَامَةُ بنُ زَيْدٍ، فَتَلَوَّنَ وَجْهُ رَسولِ اللهِ صَلَّى اللَّهُ عليه وسلَّمَ ، فَقالَ: أَتَشْفَعُ في حَدٍّ مِن حُدُودِ اللهِ؟ فَقالَ له أُسَامَةُ: اسْتَغْفِرْ لي يا رَسولَ اللهِ، فَلَمَّا كانَ العَشِيُّ، قَامَ رَسولُ اللهِ صَلَّى اللَّهُ عليه وسلَّمَ، فَاخْتَطَبَ، فأثْنَى علَى اللهِ بما هو أَهْلُهُ، ثُمَّ قالَ: أَمَّا بَعْدُ، فإنَّما أَهْلَكَ الَّذِينَ مِن قَبْلِكُمْ أنَّهُمْ كَانُوا إذَا سَرَقَ فِيهِمِ الشَّرِيفُ تَرَكُوهُ، وإذَا سَرَقَ فِيهِمِ الضَّعِيفُ أَقَامُوا عليه الحَدَّ، وإنِّي وَالَّذِي نَفْسِي بيَدِهِ، لو أنَّ فَاطِمَةَ بنْتَ مُحَمَّدٍ سَرَقَتْ لَقَطَعْتُ يَدَهَا، ثُمَّ أَمَرَ بتِلْكَ المَرْأَةِ الَّتي سَرَقَتْ، فَقُطِعَتْ يَدُهَا. وقالَتْ عَائِشَةُ: فَحَسُنَتْ تَوْبَتُهَا بَعْدُ، وَتَزَوَّجَتْ، وَكَانَتْ تَأتِينِي بَعْدَ ذلكَ فأرْفَعُ حَاجَتَهَا إلى رَسولِ اللهِ صَلَّى اللَّهُ عليه وسلَّمَ {\footnotesize (صحيح مسلم)}. 

\section{واجبات الحكم الرشيد}


تقديم المصلحة العامة على المصلحة الخاصة:

لولا قَومُكِ حَديثو عَهدٍ بكُفرٍ لنَقَضتُ الكَعبةَ فجَعَلتُ لها بابَينِ: بابٌ يَدخُلُ مِنه النَّاسُ، وبابٌ يَخرُجونَ

قال الشبخ ابن باز: فترك ﷺ نقض الكعبة وإدخال حجر إسماعيل فيها خشية الفتنة، وهذا يدل على وجوب مراعاة المصالح العامة وتقديم المصلحة العليا، وهي تأليف القلوب وتثبيتها على الإسلام على المصلحة التي هي أدنى منها وهي إعادة الكعبة على قواعد إبراهيم.

الرحمة:

قال ﷺ :مَنْ لَا يَرْحَمِ النَّاسَ لَا يُرْحَمْ، وَمَنْ لَا يُرْحَمْ لَا يُغْفَرْ لَهُ {\footnotesize (صحيح الترمذي)}.


التيسير:

قال ﷺ :يَسِّرُوا وَلَا تُعَسِّرُوا، وَبَشِّرُوا وَلَا تُنَفِّرُوا {\footnotesize (صحيح البخاري)}.

إنَّ الدِّينَ يُسرٌ، و لا يُشادُّ الدِّينَ أحدٌ إلا غَلَبَهُ، فَسَدِّدُوا و قارِبُوا و أبْشِرُوا، و اسْتَعِينُوا بِالغَدْوَةِ و الرَّوْحَةِ وشيءٍ من الدُّلَجَةِ
الراوي : أبو هريرة | المحدث : الألباني | المصدر : صحيح الجامع
الصفحة أو الرقم : 1611 | خلاصة حكم المحدث : صحيح 

جاء عن سعد بن أبي وقاص أنه اسْتَأْذَنَ عُمَرُ علَى رَسولِ اللَّهِ صلَّى اللهُ عليه وسلَّمَ وعِنْدَهُ نِساءٌ مِن قُرَيْشٍ يُكَلِّمْنَهُ ويَسْتَكْثِرْنَهُ، عالِيَةً أصْواتُهُنَّ، فَلَمَّا اسْتَأْذَنَ عُمَرُ قُمْنَ يَبْتَدِرْنَ الحِجابَ، فأذِنَ له رَسولُ اللَّهِ صلَّى اللهُ عليه وسلَّمَ ورَسولُ اللَّهِ صلَّى اللهُ عليه وسلَّمَ يَضْحَكُ، فقالَ عُمَرُ: أضْحَكَ اللَّهُ سِنَّكَ يا رَسولَ اللَّهِ، قالَ: عَجِبْتُ مِن هَؤُلاءِ اللَّاتي كُنَّ عِندِي، فَلَمَّا سَمِعْنَ صَوْتَكَ ابْتَدَرْنَ الحِجابَ! قالَ عُمَرُ: فأنْتَ يا رَسولَ اللَّهِ كُنْتَ أحَقَّ أنْ يَهَبْنَ، ثُمَّ قالَ: أيْ عَدُوَّاتِ أنْفُسِهِنَّ؛ أتَهَبْنَنِي ولا تَهَبْنَ رَسولَ اللَّهِ صلَّى اللهُ عليه وسلَّمَ؟! قُلْنَ: نَعَمْ، أنْتَ أفَظُّ وأَغْلَظُ مِن رَسولِ اللَّهِ صلَّى اللهُ عليه وسلَّمَ، قالَ رَسولُ اللَّهِ صلَّى اللهُ عليه وسلَّمَ: والَّذي نَفْسِي بيَدِهِ، ما لَقِيَكَ الشَّيْطانُ قَطُّ سالِكًا فَجًّا إلَّا سَلَكَ فَجًّا غيرَ فَجِّكَ {\footnotesize (صحيح البخاري)}.

عن أبي هريرة قال: بيْنَا الحَبَشَةُ يَلْعَبُونَ عِنْدَ النبيِّ صَلَّى اللهُ عليه وسلَّمَ بحِرَابِهِمْ، دَخَلَ عُمَرُ فأهْوَى إلى الحَصَى فَحَصَبَهُمْ بهَا، فَقالَ: دَعْهُمْ يا عُمَرُ. [وفي رِوايةٍ زادَ]: في المَسْجِدِ. {\footnotesize (صحيح البخاري)}.

وعن أم المؤمنين عائشة رضي الله عنها قالت: رَأَيْتُ النبيَّ صَلَّى اللهُ عليه وسلَّمَ يَسْتُرُنِي برِدائِهِ، وأنا أنْظُرُ إلى الحَبَشَةِ يَلْعَبُونَ في المَسْجِدِ، حتَّى أكُونَ أنا الَّتي أسْأَمُ، فاقْدُرُوا قَدْرَ الجارِيَةِ الحَدِيثَةِ السِّنِّ، الحَرِيصَةِ علَى اللَّهْوِ.
{\footnotesize (صحيح البخاري)}.

وعن أم المؤمنين عائشة رضي الله عنها قالت: أنَّ أبَا بَكْرٍ رَضِيَ اللَّهُ عنْه، دَخَلَ عَلَيْهَا، وعِنْدَهَا جَارِيَتَانِ في أيَّامِ مِنًى تُغَنِّيَانِ، وتُدَفِّفَانِ، وتَضْرِبَانِ، والنبيُّ صَلَّى اللهُ عليه وسلَّمَ مُتَغَشٍّ بثَوْبِهِ، فَانْتَهَرَهُما أبو بَكْرٍ، فَكَشَفَ النبيُّ صَلَّى اللهُ عليه وسلَّمَ عن وجْهِهِ، فَقالَ: دَعْهُما يا أبَا بَكْرٍ، فإنَّهَا أيَّامُ عِيدٍ. وتِلْكَ الأيَّامُ أيَّامُ مِنًى . وَقالَتْ عَائِشَةُ: رَأَيْتُ النبيَّ صَلَّى اللهُ عليه وسلَّمَ يَسْتُرُنِي، وأَنَا أنْظُرُ إلى الحَبَشَةِ، وهُمْ يَلْعَبُونَ في المَسْجِدِ، فَزَجَرَهُمْ فَقالَ النبيُّ صَلَّى اللهُ عليه وسلَّمَ: دَعْهُمْ، أمْنًا بَنِي أرْفِدَةَ يَعْنِي مِنَ الأمْنِ. {\footnotesize (صحيح البخاري)}.

الأمانة:
قال ﷺ :ما مِنْ عبدٍ يسترْعيه اللهُ رعيَّةً، يموتُ يومَ يموتُ، وهوَ غاشٌّ لرعِيَّتِهِ، إلَّا حرّمَ اللهُ عليْهِ الجنَّةَ {\footnotesize (صحيح الجامع وصححه الألباني)}. وفي رواية: ما مِن عَبْدٍ اسْتَرْعاهُ اللَّهُ رَعِيَّةً، فَلَمْ يَحُطْها بنَصِيحَةٍ، إلَّا لَمْ يَجِدْ رائِحَةَ الجَنَّةِ {\footnotesize (صحيح البخاري)}.

النصيحة من الرعية: 

وفي صحيح الترمذي من حديث أبي بكر الصديق - رضي الله عنه - قال: سمعت رسول الله - ﷺ - يقول: إن الناس إذا رأوا الظالم فلم يأخذوا على يديه أوشك أن يعمهم الله بعقاب من عنده.

الشورى: 

قال تعالى: \quranayah*[42][38]{\footnotesize \surahname*[42]} يقول الشيخ الألباني رحمه الله: فنسأل الله عز وجل أن يرحم عباده المسلمين وأن يلهمهم الرجوع إلى الدين على الفهم الصحيح، وأن لا يتعصبوا لحاكم، وأن يعطلوا كلمة شاعت في العصر الحاضر: "ولي الأمر هكذا يريد"، ولي الأمر من هو؟ هو عمر بن الخطاب!، هو رجل من الناس، ولي الأمر هذا واجب عليه من قديم أنه يشكل مجلس شورى وهو أحوج إلى هذا المجلس من عمر بن الخطاب، عمر بن الخطاب لو كان يريد أن يعتد برأيه وبشخصه وبعلمه وبخاصة بعد أن سمع تلك الشهادة ممن لا ينطق عن الهوى، إن هو إلا وحي يوحى: "يابن الخطاب، ما سلكت فجا إلا سلك الشيطان فجا غير فجك"، كان هو بيستقل! افعلوا، لا تفعلوا، افعلوا، اهجموا، امسكوا، إلى غيره، لكن لا، هو يعرف، كما أنزل الله على قلب محمد عليه السلام: "وشاورهم في الأمر"، ورسول الله أولى أن لا يشاور فضلا عن عمر، عمر أولى أن يشاور من الرسول، والرسول أولى من عمر أن لا يشاور، لأنه ما بتكلم إلا بوحي السماء ولكن جعلها قاعدة شرعية أبدية: "وأمرهم شورى بينهم". فكل دولة مسلمة تدعي بأنها تحكِّم شريعة الله وتحكم بما أنزل الله، قبل كل شئ يجب أن يكون لديها مجلس شورى، هذا المجلس يجب أن يكون فيه نخبة العلماء، أولا علماء في الشرع، ثانيا علماء في كل العلوم اللي بحاجة لهذا المجتمع إن كان مثلا إقتصاد، ان كان اجتماع، ان كان سياسة، ان كان جيش، إلى اخره. هذا المجلس إذا طرأ على البلاد الإسلامية طارئ يستشار، بعد ذلك يقال رأى ولي الأمر كذا. أما ولي الأمر ما استشار قيل له افعل كذا ففعل ثم يفرض على أهل العلم ان يبرروا وأن يسوغوا هذا الواقع، هذا ليس من الإسلام في شئ أبدا. 

\comment{
https://baheth.ieasybooks.com/ar/media/%D9%81%D8%AA%D8%A7%D9%88%D9%89-%D8%A7%D9%84%D8%A3%D9%84%D8%A8%D8%A7%D9%86%D9%8A-1490-%D8%AA%D9%83%D9%84%D9%85-%D8%B9%D9%86-%D8%A7%D9%84%D8%B0%D9%8A%D9%86-%D9%8A%D8%AC%D9%8A%D8%B2%D9%88%D9%86-%D8%A7%D9%84%D8%A7%D8%B3%D8%AA%D8%B9%D8%A7%D9%86%D8%A9-%D8%A8%D8%A7%D9%84%D9%85%D8%B4%D8%B1%D9%83%D9%8A%D9%86?cue=5486147
}

\section{الحكم الرشيد في زمن الصحابة}

\comment{

فقد روى الإمام أحمد في "المسند" (30 / 355) عَنِ حُذَيْفَةُ، قال: قَالَ رَسُولُ اللهِ صَلَّى اللهُ عَلَيْهِ وَسَلَّمَ: (تَكُونُ النُّبُوَّةُ فِيكُمْ مَا شَاءَ اللهُ أَنْ تَكُونَ، ثُمَّ يَرْفَعُهَا إِذَا شَاءَ أَنْ يَرْفَعَهَا، ثُمَّ تَكُونُ خِلَافَةٌ عَلَى مِنْهَاجِ النُّبُوَّةِ، فَتَكُونُ مَا شَاءَ اللهُ أَنْ تَكُونَ، ثُمَّ يَرْفَعُهَا إِذَا شَاءَ اللهُ أَنْ يَرْفَعَهَا، ثُمَّ تَكُونُ مُلْكًا عَاضًّا، فَيَكُونُ مَا شَاءَ اللهُ أَنْ يَكُونَ، ثُمَّ يَرْفَعُهَا إِذَا شَاءَ أَنْ يَرْفَعَهَا، ثُمَّ تَكُونُ مُلْكًا جَبْرِيَّةً، فَتَكُونُ مَا شَاءَ اللهُ أَنْ تَكُونَ، ثُمَّ يَرْفَعُهَا إِذَا شَاءَ أَنْ يَرْفَعَهَا، ثُمَّ تَكُونُ خِلَافَةٌ عَلَى مِنْهَاجِ نُبُوَّةٍ).

}


البيان الواضح لسنين الخلافة الراشدة


قال صلى الله عليه وسلم 

خلافةُ النُّبوَّةِ ثلاثون سنةً ، ثم يُؤتي اللهُ الملكَ مَن يشاءُ

الراوي : سفينة مولى رسول الله صلى الله عليه وسلم | المحدث : الألباني | المصدر : صحيح الجامع | الصفحة أو الرقم : 3257 | خلاصة حكم المحدث : صحيح
   
قال سعيدٌ: قال لي سَفينةُ: أمسِكْ عليكَ : أبو بكرٍ سنتين، وعمرُ عشرًا، وعثمانُ اثنتي عشرةَ، وعليُّ كذا، قال سعيدٌ : قلتُ لسفينةَ : إنَّ هؤلاء يزعمون أنَّ عليًّا لم يكن بخليفةٍ، قال : كذبَتْ أستاهُ بني الزرقاءِ – يعني : بني مرْوانَ -الراوي : سفينة مولى رسول الله صلى الله عليه وسلم | المحدث : الألباني | المصدر : صحيح أبي داود
الصفحة أو الرقم: 4646 | خلاصة حكم المحدث : حسن

وجاء في تفسير ابن كثير 
وجاء في تفسيير القرطبي ان الشعبي قال: كان بين عمر وأبي خصومة ، فتقاضيا إلى زيد بن ثابت ، فلما دخلا عليه أشار لعمر إلى وسادته ، فقال عمر : هذا أول جورك ، أجلسني وإياه مجلسا واحدا ، فجلسا بين يديه .


لَمَّا تُوُفِّيَ النبيُّ صَلَّى اللهُ عليه وسلَّمَ واسْتُخْلِفَ أبو بَكْرٍ، وكَفَرَ مَن كَفَرَ مِنَ العَرَبِ، قالَ عُمَرُ: يا أبا بَكْرٍ، كيفَ تُقاتِلُ النَّاسَ، وقدْ قالَ رَسولُ اللَّهِ صَلَّى اللهُ عليه وسلَّمَ: أُمِرْتُ أنْ أُقاتِلَ النَّاسَ حتَّى يقولوا: لا إلَهَ إلَّا اللَّهُ، فمَن قالَ: لا إلَهَ إلَّا اللَّهُ، فقَدْ عَصَمَ مِنِّي مالَهُ ونَفْسَهُ إلَّا بحَقِّهِ، وحِسابُهُ علَى اللَّهِ قالَ أبو بَكْرٍ: واللَّهِ لَأُقاتِلَنَّ مَن فَرَّقَ بيْنَ الصَّلاةِ والزَّكاةِ، فإنَّ الزَّكاةَ حَقُّ المالِ، واللَّهِ لو مَنَعُونِي عَناقًا كانُوا يُؤَدُّونَها إلى رَسولِ اللَّهِ صَلَّى اللهُ عليه وسلَّمَ لَقاتَلْتُهُمْ علَى مَنْعِها قالَ عُمَرُ: فَواللَّهِ ما هو إلَّا أنْ رَأَيْتُ أنْ قدْ شَرَحَ اللَّهُ صَدْرَ أبِي بَكْرٍ لِلْقِتالِ، فَعَرَفْتُ أنَّه الحَقُّ.
عرض مختصر..
الراوي : أبو هريرة | المحدث : البخاري | المصدر : صحيح البخاري
الصفحة أو الرقم : 6924 | خلاصة حكم المحدث : [صحيح]

أنَّ أَبَا بَكْرٍ رَضيَ اللهُ عنه خَرَجَ، وعُمَرُ رَضيَ اللهُ عنه يُكَلِّمُ النَّاسَ، فَقالَ: اجْلِسْ، فأبَى، فَقالَ: اجْلِسْ، فأبَى، فَتَشَهَّدَ أَبُو بَكْرٍ رَضيَ اللهُ عنه، فَمَالَ إلَيْهِ النَّاسُ، وتَرَكُوا عُمَرَ، فَقالَ: أَمَّا بَعْدُ، فمَن كانَ مِنكُم يَعْبُدُ مُحَمَّدًا صلَّى اللهُ عليه وسلَّمَ، فإنَّ مُحَمَّدًا صلَّى اللهُ عليه وسلَّمَ قدْ مَاتَ، ومَن كانَ يَعْبُدُ اللَّهَ، فإنَّ اللَّهَ حَيٌّ لا يَمُوتُ، قالَ اللَّهُ تَعَالَى: {وَمَا مُحَمَّدٌ إِلَّا رَسُولٌ قَدْ خَلَتْ مِنْ قَبْلِهِ الرُّسُلُ} إلى {الشَّاكِرِينَ} [آل عمران: 144]، واللَّهِ لَكأنَّ النَّاسَ لمْ يَكونُوا يَعْلَمُونَ أنَّ اللَّهَ أَنْزَلَهَا حتَّى تَلاهَا أَبُو بَكْرٍ رَضيَ اللهُ عنه، فَتَلَقَّاهَا منه النَّاسُ، فَما يُسْمَعُ بَشَرٌ إلَّا يَتْلُوهَا.
عرض مختصر..
الراوي : عبدالله بن عباس | المحدث : البخاري | المصدر : صحيح البخاري
الصفحة أو الرقم : 1242 | خلاصة حكم المحدث : [صحيح] 

قال أبو بكرٍ ، بعد أن حمِد اللهَ وأثنَى عليه : يا أيُّها النَّاسُ ، إنَّكم تقرءون هذه الآيةَ ، وتضعونها على غيرِ موضعِها عَلَيْكُمْ أَنْفُسَكُمْ لَا يَضُرُّكُمْ مَنْ ضَلَّ إِذَا اهْتَدَيْتُمْ وإنَّا سمِعنا النَّبيَّ صلَّى اللهُ عليه وسلَّم يقولُ : إنَّ النَّاسَ إذا رأَوُا الظَّالمَ فلم يأخُذوا على يدَيْه أوشك أن يعُمَّهم اللهُ بعقابٍ وإنِّي سمِعتُ رسولَ اللهِ صلَّى اللهُ عليه وسلَّم يقولُ : ما من قومٍ يُعمَلُ فيهم بالمعاصي ، ثمَّ يقدِرون على أن يُغيِّروا ، ثمَّ لا يُغيِّروا إلَّا يوشِكُ أن يعُمَّهم اللهُ منه بعقابٍ
الراوي : أبو بكر الصديق | المحدث : الألباني | المصدر : صحيح أبي داود
الصفحة أو الرقم: 4338 | خلاصة حكم المحدث : صحيح


أنَّ أَبَا بَكْرٍ رَضيَ اللهُ عنه خَرَجَ، وعُمَرُ رَضيَ اللهُ عنه يُكَلِّمُ النَّاسَ، فَقالَ: اجْلِسْ، فأبَى، فَقالَ: اجْلِسْ، فأبَى، فَتَشَهَّدَ أَبُو بَكْرٍ رَضيَ اللهُ عنه، فَمَالَ إلَيْهِ النَّاسُ، وتَرَكُوا عُمَرَ، فَقالَ: أَمَّا بَعْدُ، فمَن كانَ مِنكُم يَعْبُدُ مُحَمَّدًا صلَّى اللهُ عليه وسلَّمَ، فإنَّ مُحَمَّدًا صلَّى اللهُ عليه وسلَّمَ قدْ مَاتَ، ومَن كانَ يَعْبُدُ اللَّهَ، فإنَّ اللَّهَ حَيٌّ لا يَمُوتُ، قالَ اللَّهُ تَعَالَى: {وَمَا مُحَمَّدٌ إِلَّا رَسُولٌ قَدْ خَلَتْ مِنْ قَبْلِهِ الرُّسُلُ} إلى {الشَّاكِرِينَ} [آل عمران: 144]، واللَّهِ لَكأنَّ النَّاسَ لمْ يَكونُوا يَعْلَمُونَ أنَّ اللَّهَ أَنْزَلَهَا حتَّى تَلاهَا أَبُو بَكْرٍ رَضيَ اللهُ عنه، فَتَلَقَّاهَا منه النَّاسُ، فَما يُسْمَعُ بَشَرٌ إلَّا يَتْلُوهَا.
عرض مختصر..
الراوي : عبدالله بن عباس | المحدث : البخاري | المصدر : صحيح البخاري
الصفحة أو الرقم : 1242 | خلاصة حكم المحدث : [صحيح] 

كُنْتُ جَالِسًا عِنْدَ النَّبيِّ صَلَّى اللهُ عليه وسلَّمَ، إذْ أقْبَلَ أبو بَكْرٍ آخِذًا بطَرَفِ ثَوْبِهِ حتَّى أبْدَى عن رُكْبَتِهِ، فَقَالَ النَّبيُّ صَلَّى اللهُ عليه وسلَّمَ: أمَّا صَاحِبُكُمْ فقَدْ غَامَرَ ، فَسَلَّمَ وقَالَ: إنِّي كانَ بَيْنِي وبيْنَ ابْنِ الخَطَّابِ شَيءٌ، فأسْرَعْتُ إلَيْهِ ثُمَّ نَدِمْتُ، فَسَأَلْتُهُ أنْ يَغْفِرَ لي فأبَى عَلَيَّ، فأقْبَلْتُ إلَيْكَ، فَقَالَ: يَغْفِرُ اللَّهُ لكَ يا أبَا بَكْرٍ، ثَلَاثًا، ثُمَّ إنَّ عُمَرَ نَدِمَ، فأتَى مَنْزِلَ أبِي بَكْرٍ، فَسَأَلَ: أثَّمَ أبو بَكْرٍ؟ فَقالوا: لَا، فأتَى إلى النَّبيِّ صَلَّى اللهُ عليه وسلَّمَ فَسَلَّمَ، فَجَعَلَ وجْهُ النَّبيِّ صَلَّى اللهُ عليه وسلَّمَ يَتَمَعَّرُ، حتَّى أشْفَقَ أبو بَكْرٍ، فَجَثَا علَى رُكْبَتَيْهِ، فَقَالَ: يا رَسولَ اللَّهِ، واللَّهِ أنَا كُنْتُ أظْلَمَ، مَرَّتَيْنِ، فَقَالَ النَّبيُّ صَلَّى اللهُ عليه وسلَّمَ: إنَّ اللَّهَ بَعَثَنِي إلَيْكُمْ فَقُلتُمْ: كَذَبْتَ، وقَالَ أبو بَكْرٍ: صَدَقَ، ووَاسَانِي بنَفْسِهِ ومَالِهِ، فَهلْ أنتُمْ تَارِكُوا لي صَاحِبِي؟ مَرَّتَيْنِ، فَما أُوذِيَ بَعْدَهَا.
عرض مختصر..
الراوي : أبو الدرداء | المحدث : البخاري | المصدر : صحيح البخاري
الصفحة أو الرقم : 3661 | خلاصة حكم المحدث : [صحيح]

لما بويع أبو بكر بالخلافة بعد بيعة السقيفة تكلم أبو بكر، فحمد الله وأثنى عليه ثم قال:
"أما بعد أيها الناس فإني قد وليت عليكم ولست بخيركم، فإن أحسنت فأعينوني، وإن أسأت فقوموني، الصدق أمانة، والكذب خيانة، والضعيف فيكم قوي عندي حتى أريح عليه حقه إن شاء الله، والقوى فيكم ضعيف حتى آخذ الحق منه إن شاء الله، لا يدع قوم الجهاد في سبيل الله إلا ضربهم الله بالذل، ولا تشيع الفاحشة في قوم قط إلا عمهم الله بالبلاء، أطيعوني ما أطعت الله ورسوله، فإذا عصيت الله ورسوله فلا طاعة لي عليكم".

(يا أيُّها الناس، قد وُلِّيت عليكم ولست بخيركم، فإن رأيتموني على حقٍّ فأعينوني، وإن رأيتموني على باطل فسدِّدوني. أطيعوني ما أطعتُ الله فيكم، فإذا عصيتُه فلا طاعة لي عليكم. ألا إنَّ أقواكم عندي الضعيف حتى آخذ الحقَّ له، وأضعفكم عندي القويُّ حتى آخذ الحقَّ منه. أقول قولي هذا وأستغفر الله لي ولكم).

أنَّ رجلًا، قال لرسولِ اللهِ صلَّى اللهُ عليه وسلَّم : رأيتُ كأنَّ ميزانًا دُلِّيَ منَ السماءِ، فوُزِنتَ بأبي بكرٍ فرجَحتَ بأبي بكرٍ، ثم وُزِن أبو بكرٍ بعُمرَ، فرجَح أبو بكرٍ، ثم وُزِن عُمرُ بعثمانَ، فرَجَح عُمرُ، ثم رُفِع الميزانُ، فاستَهَلَّها رسولُ اللهِ صلَّى اللهُ عليه وسلَّم خلافةَ نبوةٍ، ثم يؤتي اللهُ المُلكَ مَن يشاءُ

الراوي : سفينة مولى رسول الله صلى الله عليه وسلم | المحدث : البوصيري | المصدر : إتحاف الخيرة المهرة
الصفحة أو الرقم : 5/ 11 | خلاصة حكم المحدث : إسناده صحيح | أحاديث مشابهة | شرح حديث مشابه

أنَّ رجلًا قال : يا رسولَ اللهِ رأيتُ كأنَّ مِيزانًا دُلِّي مِنَ السماءِ فَوُزِنْتَ فيه أنت وأبوبكرٍ فَرَجَحْتَ بأبي بكرٍ ثم وُزِنَ فيه أبوبكرٍ وعمرُ فَرَجَحَ أبو بكرٍ بعمرَ ثم وُزِنَ فيه عمرُ وعثمانُ فَرَجَحَ عمرُ بعثمانَ ثم رُفِعَ الميزانُ فاسْتآلهَا يعني تَأَوَّلَها ثم قال : خِلافَةُ نُبُوَّةٍ ثم يُؤتِي اللهُ الملكَ مَنْ يَشاءُ

الراوي : أبو بكرة نفيع بن الحارث | المحدث : الألباني | المصدر : تخريج كتاب السنة
الصفحة أو الرقم : 1135 | خلاصة حكم المحدث : صحيح 

\section{مختصر سيرة معاوية بن أبي سفيان}


وثبت من طرق عن عبد الله بن عمر رضي الله عنهما أنه قال: ما رأيت أحدا أسود من معاوية، قال الراوي: ولا عمر؟ قال: كان عمر خيرا منه، وكان معاوية أسود منه41.
قال الإمام أحمد: "معنى أسود: أي أسخى، وقال: السيّد: الحليم، والسيّد: المعطي، أعطى معاويةُ أهلَ المدينة عَطايا ما أعطاها خليفةٌ قد كان قبلَه".

قال عنه عبدالله بن عباس -رَضِيَ اللَّهُ عَنْهُما-: ما رأيتُ رجلًا كان أخْلَقَ للمُلْك من معاوية.

قال قبيصة بن جابر: ما رأيت أحدًا أعظم حلمًا، ولا أكثر سؤددًا، ولا أبعد أناةً، ولا ألين مخرجًا، ولا أرحب باعًا بالمعروف من معاوية.

وعن عبدالله بن عمرو بن العاص -رَضِيَ اللَّهُ عَنْهُ- أنه قال: ما رأيت أحدًا أسود (من السيادة) من معاوية.

وعن أبي الدرداء -رضي الله عنه- أنه قال لأهل الشام: ما رأيت أحدًا أشبه صلاة بصلاة رسول الله -صَلَّى اللَّهُ عَلَيْهِ وَسَلَّمَ- من إمامكم هذا (يعني: معاوية).

أخرج الطبراني عن سعيد المقبري، قال: قال عمر بن الخطاب: تذكرون كسرى وقيصر ودهاءهما وعندكم معاوية!

قال عمر بن الخطاب -رَضِيَ اللَّهُ عَنْهُ- لما ولَّاه الشام: لا تذكروا معاوية إلا بخير؛ فإني سمعت رسول الله -صَلَّى اللَّهُ عَلَيْهِ وَسَلَّمَ- يقول: (اللهم اهد به).

روى البخاري في الصحيح أنه قيل لابن عباس: هل لك في أمير المؤمنين معاوية فإنه ما أوتر إلا بواحدة، قال: إنه فقيه.

ولقد دعى النبي ﷺ  لمعاوية تعلم الكتاب والحساب معا فعن العرباض بن سارية أنه قال سمِعتُ النَّبيَّ ﷺ وهوَ يَدعو إلى السَّحورِ في شهرِ رمضانَ: هَلُمَّ إلى الغداءِ المبارَكِ ثمَّ سمعتُه يقولُ: اللَّهمَّ علِّمْ مُعاويةَ الكِتابَ، والحِسابَ ، وقِهِ العَذابَ \href{https://shamela.ws/book/25794/13683#p2}{\faExternalLink}\href{https://shamela.ws/book/22669/1909#p2}{\faExternalLink}\href{https://shamela.ws/book/4445/6628#p1}{\faExternalLink}\href{https://shamela.ws/book/9442/5496#p12}{\faExternalLink} \cite{ahmid}\cite{dahabi_Siyar}\cite{ibnKathir_AlBidayah}\cite{albani_Sahiha}.\footnote{أحمد: 17152، وأورده الذهبي في سير أعلام النبلاء وبن كثير في البداية والنهاية، والإمام الألباني في السلسلة الصحيحة.} وفي رواية أخرى: "اللهم علمه الكتاب ومكن له في البلاد وقه العذاب". ولعل هذا فيه الإشارة الكافية لأهمية علم الحساب وأنه من الدين وأنه من أسباب التمكين لأن به يقام العدل في الحكم. ولقد إستجاب الله جل جلاله هذا الدعاء لنبيه ﷺ فقد كان معاوية رضي الله عنه حكما عدلا حتى عرف بالمهدي عند أئمة التابعين فقد جاء عن مجاهد رحمه الله أنه قال: "لو رأيتم معاوية لقلتم هذا المهدي" \href{https://shamela.ws/book/1077/702#p1}{\faExternalLink}. وقال قتادة رحمه الله: "لو أصبحتم في مثل عمل معاوية لقال أكثركم: هذا المهدي" \href{https://shamela.ws/book/1077/701#p1}{\faExternalLink}. وقد ذكر عند الأعمش عمر بن عبد العزيز وعدله، فقال الأعمش: "فكيف لو أدركتم معاوية؟ قالوا: يا أبا محمد، يعني في حلمه؟ قال: لا والله، ألا بل في عدله" \href{https://shamela.ws/book/1077/700#p1}{\faExternalLink}\href{https://shamela.ws/book/927/3063#p3}{\faExternalLink}.


وفيما شجر بين معاوية وعلي رضي الله عنهم يقول الشيخ ابن باز رحمه الله: وأهل السنة يقولون: إنَّ عليًّا وأصحابه هم المصيبون، وأن معاوية قد أخطأ، وأن عمل معاوية عمل بَغْيٍ، ولكنه مجتهد؛ للمُطالبة بدم قتلة عثمان، فاجتهد فأخطأ فله أجر الاجتهاد، وسمَّاه النبيُّ: باغيًا، قال في عمار: تقتل عمّارًا الفئةُ الباغيةُ، وهم معاوية وأصحابه، قتلوه، فعليٌّ وأصحابه هم أهل العدل، ولهم البيعة الصحيحة، والذين مع معاوية بيعتهم غير صحيحةٍ، مجتهدون لهم أجرٌ واحدٌ، رضي الله عنهم وأرضاهم، والواجب الكفُّ عن مساوئهم، والتَّرضي عنهم، والإيمان بأنهم مجتهدون: مَن أصاب فله أجران، ومَن أخطأ فله أجرٌ.


\section{مختصر سيرة عمر بن عبد العزيز}

تبدأ قصة عمر بن عبد العزيز مع الخليفة الراشد أمير المؤمنين عمر بن الخطاب رضي الله عندما نهى فِي خِلَافَته عَن مذق اللَّبن بِالْمَاءِ، فَخرج ذَات لَيْلَة فِي حَوَاشِي الْمَدِينَة فَإِذا بإمرأة تَقول لإبنة لَهَا أَلا تمذقين لبنك فقد أَصبَحت. فَقَالَت الْجَارِيَة كَيفَ أمذق وَقد نهى أَمِير الْمُؤمنِينَ عَن المذق. فَقَالَت قد مذق النَّاس فامذقي فَمَا يدْرِي أَمِير الْمُؤمنِينَ. فَقَالَت إِن كَانَ عمر لَا يعلم فإله عمر يعلم، مَا كنت لأفعله وَقد نهى عَنهُ. فَوَقَعت مقالتها من عمر، فَلَمَّا أصبح، دَعَا عَاصِمًا ابْنه فَقَالَ يَا بني اذْهَبْ إِلَى مَوضِع كَذَا وَكَذَا فاسأل عَن الْجَارِيَة ووصفها لَهُ. فَذهب عَاصِم، فَإِذا هِيَ جَارِيَة من بني هِلَال. فَقَالَ لَهُ عمر اذْهَبْ يَا بني فَتَزَوجهَا، فَمَا أحراها أَن تَأتي بِفَارِس يسود الْعَرَب. فَتَزَوجهَا عَاصِم بن عمر فَولدت لَهُ بنت عَاصِم بن عمر بن الْخطاب، فَتَزَوجهَا عبد الْعَزِيز بن مَرْوَان بن الحكم فَأَتَت بعمر بن عبد الْعَزِيز. وجاء أيضا أن عمر بن الخطاب رضي الله عنه أنه ذات يوم استيقظ من نَومه فَمسح النّوم عَن وَجهه وفرك عَيْنَيْهِ وَهُوَ يَقُول من هَذَا الَّذِي من ولد عمر يُسمى عمر يسير بسيرة عمر يُرَدِّدهَا مَرَّات \cite{ibnAbdAlHakam_OmarIbnAbdAlAziz}. وقيل إن عمر بن الخطاب قال إن من ولدي رجلاً بوجهه شتر يملأ الأرض عدلاً \cite{dahabi_Siyar}.

وَولد عمر بن عبد الْعَزِيز بِالْمَدِينَةِ فَلَمَّا شب وعقل وَهُوَ غُلَام كَانَ يَأْتِي عبد الله بن عمر كثيرا لقرابة أمه مِنْهُ (خال أمه) وكان يحبه ويحب التشبه به \cite{ibnAbdAlHakam_OmarIbnAbdAlAziz}. وروى ضمام بن إسماعيل عن أبي قبيل أن عمر بن عبد العزيز بكى وهو غلام صغير فأرسلت إليه أمه وقالت ما يبكيك؟ قال ذكرت الموت قال وكان يومئذ قد جمع القرآن فبكت أمه حين بلغها ذلك \cite{dahabi_Siyar}. ونقل الزبير بن بكار عن العتبي أن أول ما استبين من عمر بن عبد العزيز أن أباه ولي مصر وهو حديث السن يشك في بلوغه فأراد إخراجه فقال يا أبت أو غير ذلك؟ لعله أن يكون أنفع لي ولك أن أبقى في المدينة فأقعد إلى فقهاء أهلها وأتأدب بآدابهم فوجهه إلى المدينة فاشتهر بها بالعلم والعقل مع حداثة سنه \cite{dahabi_Siyar}. ولما أرادت أم عمر أن تهاجر إلى مصر لتلحق بزوجها عبد العزيز بن مروان الذي كان واليا لمصر، راجعت خالها عبد الله بن عمر رضي الله عنه فطلب منها عبد الله بن عمر ترك عمر بن عبد العزيز في المدينة وقال لها: خَلْفي هَذَا الْغُلَام عندنَا يُرِيد عمر فَإِنَّهُ أشبهكم بِنَا أهل الْبَيْت. فخلفته عِنْده. فنشأ في المدينة وتعلم العلم الشرعي النافع مع ما كان له من خشية الله التي عرف بها. وعند وفاة أبيه، بعث إليه عمه عبد الملك بن مروان وخلطه بولده وقدمه على كثير منهم وزوجه بابنته فاطمة \cite{dahabi_Siyar}. جاء عنه أنه دائما ما يفقده رفقاءه فيجدونه يجلس باكيا. وسأله عن ذلك أمير المؤمنين وابن عمه سليمان بن عبد الملك فقال: مَا يبكيك يَا أَبَا حَفْص، فقَالَ عمر بن عبد العزيز: أبكاني يَا أَمِير الْمُؤمنِينَ أَنِّي ذكرت يَوْم الْقِيَامَة، من قدم شَيْئا وجده، وَلم أقدم شَيْئا فَلم أجد شَيْئا. وَخرج سُلَيْمَان بن عبد الْملك وَمَعَهُ عمر بن عبد الْعَزِيز إِلَى الْحَج فَأَصَابَهُمْ مطر شَدِيد ورعد وبرق فَقَالَ سُلَيْمَان هَل رَأَيْت مثل هَذَا يَا أَبَا حَفْص فَقَالَ يَا أَمِير الْمُؤمنِينَ هَذَا فِي حِين رَحمته فَكيف بِهِ فِي حِين غَضَبه \cite{ibnAbdAlHakam_OmarIbnAbdAlAziz}. 

وجاء أيضا أنه ذَات لَيْلَة خرج عمر بن عبد العزيز على مركب لَهُ يسير وَحده وَتَبعهُ مُزَاحم فَتقدم عمر وَتَأَخر مُزَاحم فَنظر مُزَاحم فَإِذا هُوَ بِرَجُل يُسَايِر عمر وَعَهده بِهِ وَحده وَقد وضع الرجل يَده على عاتق عمر قَالَ مُزَاحم فَقلت فِي نَفسِي من هَذَا إِن هَذَا لذُو دَالَّة عَلَيْهِ فحركت للحوق بِهِ فَأَدْرَكته فَإِذا هُوَ وَحده لَا أرى مَعَه أحدا غَيره فَقلت لَهُ رَأَيْت مَعَك رجلا آنِفا قد وضع يَده على عاتقك وَهُوَ يسايرك فَقلت فِي نَفسِي من هَذَا إِن هَذَا لذُو دَالَّة عَلَيْهِ فلحقتكما فَلم أر أحدا غَيْرك فَقَالَ عمر أوقد رَأَيْته يَا مُزَاحم قَالَ نعم قَالَ إِنِّي لأحسبك رجلا صَالحا ذَلِك يَا مُزَاحم الْخضر أعلمني أَنِّي سألي هَذَا الْأَمر وأعان عَلَيْهِ \cite{ibnAbdAlHakam_OmarIbnAbdAlAziz}. وذكر ذلك الذهبي فقال:  عن رياح بن عبيدة قال خرج عمر بن عبد العزيز إلى الصلاة وشيخ متوكئ على يده فقلت في نفسي هذا شيخ جاف فلما صلى ودخل لحقته فقلت له من الشيخ الذي كان يتكئ على يدك فقال يا رياح رأيته؟ قلت: نعم قال ما أحسبك إلا رجلاً صالحاً ذاك أخي الخضر أتاني فأعلمني أني سألي أمر الأمة وإني سأعدل فيها \cite{dahabi_Siyar}.

وولي عمر بن عبد العزيز أميرا على المدينة بأمر الخليفة قبله سُلَيْمَان بن عبد الْملك حيث أن سليمان لم يستطع توريث الملك لأبناءه لصغرهم فكان يشكوا في مرضه ويقول: (إِن بني صبية صغَار، أَفْلح من كَانَ لَهُ كبار)و ،(إِن بني صبية صيفيون، أَفْلح من كَانَ لَهُ ربعيون). وكان عمر عبد العزيز يرد عليه بقوله: يا أمير الْمُؤمنِينَ، يَقُول الله تبَارك وَتَعَالَى (قد أَفْلح من تزكّى وَذكر اسْم ربه فصلى) عارضا عليه أن يحتسب ذلك لله جل جلاله إن ولى غيرهم. فحدث سليمان بن عبد الملك نَفسه بِولَايَة عمر بن عبد الْعَزِيز لما كَانَ يعرف من حَاله، فَشَاور رَجَاء فِيمَن يعْقد لَهُ فَأَشَارَ عَلَيْهِ رَجَاء بعمر وسدد لَهُ رَأْيه فِيهِ، فَوَافَقَ ذَلِك رَأْي سُلَيْمَان وَقَالَ لأعقدن عقدا لَا يكون للشَّيْطَان فِيهِ نصيب فاسْتخْلف فِيهِ عمر بن عبد الْعَزِيز وَيزِيد بن عبد الْملك من بعد عمر. فلما لَقِي سُلَيْمَان ربه وَقضى الله عَلَيْهِ الْمَوْت فَقَامَ عمر حَتَّى جلس على الْمِنْبَر فنعى للنَّاس سُلَيْمَان وَفتح الْكتاب فَإِذا فِيهِ اسْتِخْلَاف عمر وَيزِيد بن عبد الْملك فَسمع النَّاس وأطاعوا وَقَامُوا فَبَايعُوا لعمر عبد العزيز. وجاء أيضا أن رجلا من أهل المدينة قد رأى فِي مَنَامه كَأَن قَائِلا من السَّمَاء ينظر إِلَيْهِ يَقُول أَتَاكُم الْعدْل واللين وَإِظْهَار الْعَمَل الصَّالح فِي الْمُصَلِّين فَقَالَ لَهُ الرجل من هُوَ يَرْحَمك الله فَنزل إِلَى الأَرْض وَكتب بِيَدِهِ عمر فاستخلف عمر فِي يَوْم تِلْكَ اللَّيْلَة \cite{ibnAbdAlHakam_OmarIbnAbdAlAziz}.

وجاء أيضا في كتاب سيرة عمر عبد العزيز \cite{ibnAbdAlHakam_OmarIbnAbdAlAziz} أنه لَمَّا دفن سُلَيْمَان دَعَا عمر بن عبد العزيز بِدَوَاةٍ وَقِرْطَاس فَكتب ثَلَاثَة كتب لم يَسعهُ فِيمَا بَينه وَبَين الله عز وَجل أَن يؤخرها فأمضاها من فوره فَأخذ النَّاس فِي كِتَابه إِيَّاهَا هُنَالك فِي همزه يَقُولُونَ مَا هَذِه العجلة أما كَانَ يصبر إِلَى أَن يرجع إِلَى منزله هَذَا حب السُّلْطَان. فإذا به يعجل بالحكم بالعدل في ثلاث مسائل وهي: 
\begin{compactitem}
    \item كَانَ سُلَيْمَان قد أمر مسلمة بن عبد الْملك بحصار الْقُسْطَنْطِينِيَّة برا وبحرا وأشفى على فتحهَا ثمَّ خدع عَنْهَا حَتَّى أحرزوا طعامهم وحوائجهم ثمَّ أغلقوها دونه بعد الإشفاء عَلَيْهَا فَبلغ ذَلِك سُلَيْمَان فَغَضب مِمَّا فعل بِهِ فَحلف أَن لَا يقفله مِنْهَا مَا دَامَ حَيا فَاشْتَدَّ عَلَيْهِم الْمقَام وجاعوا حَتَّى أكلُوا الدَّوَابّ من الْجهد والجوع حَتَّى يتَنَحَّى الرجل عَن دَابَّته فتقطع بِالسُّيُوفِ فَبلغ رَأس الدَّابَّة كَذَا وَكَذَا درهما ولج سُلَيْمَان فِي أَمرهم فَكَانَ ذَلِك يغم عمر فَلَمَّا ولي رأى أَنه لَا يَسعهُ فِيمَا بَينه وَبَين الله عز وَجل أَن يَلِي شَيْئا من أُمُور الْمُسلمين ثمَّ يُؤَخر قفلهم سَاعَة فَذَلِك الَّذِي حمله على تَعْجِيل الْكتاب.
    \item كتب بعزل أُسَامَة بن زيد التنوخي وَكَانَ على خراج مصر وَأمر بِهِ أَن يحبس فِي كل جند سنة ويقيد وَيحل عَن الْقَيْد عِنْد كل صَلَاة ثمَّ يرد فِي الْقَيْد وَكَانَ غاشما ظلوما معتديا فِي الْعُقُوبَات بِغَيْر مَا أنزل الله عز وَجل يقطع الْأَيْدِي فِي خلاف مَا يُؤمر بِهِ ويشق أَجْوَاف الدَّوَابّ فَيدْخل فِيهَا القطاع ويطرحهم للتماسيح فحبس بِمصْر سنة ثمَّ نقل إِلَى أَرض فلسطين فحبس بهَا سنة ثمَّ مَاتَ عمر رَحمَه الله وَولي يزِيد بن عبد الْملك فَرد أُسَامَة على مصر.
    \item كتب بعزل يزِيد بن أبي مُسلم عَن إفريقية وَكَانَ عَامل سوء يظْهر التأله والنفاذ لكل مَا أَمر بِهِ السُّلْطَان مِمَّا جلّ أَو صغر من السِّيرَة بالجور والمخالفة للحق وَكَانَ فِي هَذَا يكثر الذّكر وَالتَّسْبِيح وَيَأْمُر بالقوم فيكونون بَين يَدَيْهِ يُعَذبُونَ وَهُوَ يَقُول سُبْحَانَ الله وَالْحَمْد لله شدّ يَا غُلَام مَوضِع كذاوكذا لبَعض مَوَاضِع الْعَذَاب وَهُوَ يَقُول لَا إِلَه إِلَّا الله وَالله أكبر شدّ يَا غُلَام مَوضِع كذاوكذا فَكَانَت حَالَته تِلْكَ شَرّ الْحَالَات فَكتب بعزله.
\end{compactitem}

وكل هذا فيه أن عمر بن عبد العزيز كان رجلا عادلا لا يرضى بالظلم حيث عجل بالعدل والإنصاف عندما تولى الأمر رحمه الله. وجاء عن ميمون بن مهران أنه قال: سمعت عمر بن عبد العزيز يقول لو أقمت فيكم خمسين عاماً ما استكملت فيكم العدل إني لأريد الأمر من أمر العامة فأخاف ألا تحمله قلوبهم فأخرج معه طمعا من طمع الدنيا \cite{dahabi_Siyar}. وجاء عن عبد الله بن محمد عن الأوزاعي أنه قال كتب إلينا عمر بن عبد العزيز رسالة لم يحفظها غيري وغير مكحول: أما بعد فإنه من أكثر ذكر الموت رضي من الدنيا باليسير ومن عد كلامه من عمله قل كلامه إلا فيما ينفعه والسلام \cite{dahabi_Siyar}. وقال الأوزاعي كان عمر بن عبد العزيز إذا أراد أن يعاقب رجلا حبسه ثلاثا ثم عاقبه كراهية أن يعجل في أول غضبه \cite{dahabi_Siyar}. وجاء أيضا أن عمر بن عبد العزيز كتب إلى عدي بن أرطاة عامله على العراق: إذا أمكنتك القدرة على المخلوق فاذكر قدرة الخالق القادر عليك، واعلم أن ما لك عند الله أكثر مما لك عند الناس \cite{ibnAbdRabbih_AlIqd}. وكتب عبد الحميد بن عبد الرحمن إلى عمر: إنّ رجلا شتمك فأردت أن أقتله، فكتب إليه عمر عبد العزيز: لو قتلته لأقدتك به، فإنه لا يقتل أحد بشتم أحد إلا رجل شتم نبيّا \cite{ibnAbdRabbih_AlIqd}.

وبعد أن دفن سُلَيْمَان وَقَامَ عمر بن عبد الْعَزِيز بالأمر قربت إِلَيْهِ المراكب، فَقَالَ مَا هَذِه، فَقَالُوا مراكب لم تركب قطّ يركبهَا الْخَلِيفَة أول مَا يَلِي، فَتَركهَا وَخرج يلْتَمس بغلته، وَقَالَ يَا مُزَاحم ضم هَذَا إِلَى بَيت مَال الْمُسلمين. ونصبت لَهُ سرادقات وَحجر لم يجلس فِيهَا أحد قطّ كَانَت تضرب للخلفاء أول مَا يلون، فَقَالَ مَا هَذِه، فَقَالُوا سرادقات وَحجر لم يجلس فِيهَا أحد قطّ يجلس فِيهَا الْخَلِيفَة أول مَا يَلِي، قَالَ يَا مُزَاحم ضم هَذِه إِلَى أَمْوَال الْمُسلمين ثمَّ ركب بغلته. وَانْصَرف إِلَى الْفرش والوطاء الَّذِي لم يجلس عَلَيْهِ أحد يفرش للخلفاء أول مَا يلون، فَجعل يدْفع ذَلِك بِرجلِهِ حَتَّى يُفْضِي إِلَى الْحَصِير، ثمَّ قَالَ يَا مُزَاحم ضم هَذَا لأموال الْمُسلمين. فَلَمَّا أصبح عمر، قَالَ لَهُ أهل سُلَيْمَان هَذَا لَك وَهَذَا لنا، قَالَ وَمَا هَذَا وَمَا هَذَا، قَالُوا هَذَا مِمَّا لبس الْخَلِيفَة من الثِّيَاب وَمَسّ من الطّيب فَهُوَ لوَلَده، وَمَا لم يمس وَلم يلبس فَهُوَ للخليفة بعده وَهُوَ لَك، قَالَ عمر مَا هَذَا لي وَلَا لِسُلَيْمَان وَلَا لكم وَلَكِن يَا مُزَاحم ضم هَذَا كُله إِلَى بَيت مَال الْمُسلمين. فتشاور عليه وزراءه فيما بينهم فَقَالُوا أما المراكب والسرادقات وَالْحجر والشوار والوطاء فَلَيْسَ فِيهِ رَجَاء بعد أَن كَانَ مِنْهُ فِيهِ مَا قد علمْتُم، وَبقيت خصْلَة هِيَ الْجَوَارِي نعرضهن عَلَيْهِ فَعَسَى أَن يكون مَا تُرِيدُونَ فِيهِنَّ، فَإِن كَانَ وَإِلَّا فَلَا طمع لكم عِنْده. فَأتي بالجواري فعرضن عَلَيْهِ كأمثال الدمى فَلَمَّا نظر إلَيْهِنَّ جعل يسألهن وَاحِدَة وَاحِدَة من أَنْت، وَلمن كنت، وَمن بعث بك، فتخبره الْجَارِيَة بأصلها وَلمن كَانَت وَكَيف أخذت، فيأمر بردهن إِلَى أهليهن ويحملن إِلَى بلادهن حَتَّى فرغ مِنْهُنَّ. فَلَمَّا رَأَوْا ذَلِك أيسوا مِنْهُ وَعَلمُوا أَنه سيحمل النَّاس على الْحق. وأحتجب من الناس ثلاثا أيام لَا يدْخل عَلَيْهِ أحد، ووجوه بني مَرْوَان وَبني أُميَّة وأشراف الْجنُود وَالْعرب وغيرهم بِبَابِهِ ينظرُونَ مَا يخرج عَلَيْهِم مِنْهُ، فَجَلَسَ للنَّاس بعد ثَلَاث أيام وَحَملهمْ على شَرِيعَة من الْحق فعرفوها، فَرد الْمَظَالِم، وَأَحْيَا الْكتاب وَالسّنة، وَسَار بِالْعَدْلِ، ورفض الدُّنْيَا وزهد فِيهَا، وتجرد لإحياء أَمر الله عز وجل فَلم يزل على ذَلِك حَتَّى قَبضه الله عزوجل فرحمه الله \cite{ibnAbdAlHakam_OmarIbnAbdAlAziz}.

وجاء أيضا أنه لما انصرف عمر بن عبد العزيز من دفن سليمان بن الملك تبعه الأمويون، فلما دخل إلى منزله قال له الحاجب: الأمويون بالباب. قال: وما يريدون؟ قال: ما عوّدتهم الخلفاء قبلك. قال ابنه عبد الملك وهو إذ ذاك ابن أربع عشرة سنة: ائذن لي في إبلاغهم عنك. قال: وما تبلغهم؟ قال: أقول: أبي يقرئكم السلام ويقول لكم إِنِّي أَخافُ إِنْ عَصَيْتُ رَبِّي عَذابَ يَوْمٍ عَظِيمٍ \cite{ibnAbdRabbih_AlIqd}. وجاء عن ابنه عبد الملك بن عمر بن عبد العزيز أنه قال لأبيه: يا أبت، مالك لا تنفذ الأمور؟ فو الله ما أبالي لو أن القدور غلت بي وبك في الحق! قال له عمر: لا تعجل يا بنيّ؛ فإنّ الله ذمّ الخمر في القرآن مرتين وحرّمها في الثالثة، وأنا أخاف أن أحمل الحق على الناس جملة فيدفعونه جملة ويكون من ذلك فتنة \cite{ibnAbdRabbih_AlIqd}. وهذا فيه أن عبد الملك بن عمر بن عبد العزيز كان ابنا صالحا، وقد جاء أنه مات قبل أبوه. فلما نزل بعبد الملك بن عمر بن عبد العزيز الموت قال له عمر: كيف تجدك يا بنيّ؟ قال أجدني في الموت، فاحتسبني، فثواب الله خير لك مني، فقال: يا بني، والله لأن تكون في ميزاني أحبّ إليّ من أن أكون في ميزانك. قال: أما والله لأن يكون ما تحب، أحبّ إليّ من أن يكون ما أحب! ثم مات، فلما فرغ من دفنه وقف على قبره وقال: يرحمك الله يا بنيّ فلقد كنت سارّا مولودا، وبارّا ناشئا، وما أحب أني دعوتك فأجبتني؛ فرحم الله كل عبد، من حر أو عبد، ذكر أو أنثى دعا لك برحمة! فكان الناس يترحمون على عبد الملك ليدخلوا في دعوة عمر؛ ثم انصرف، فدخل الناس يعزونه، فقال: إن الذي نزل بعبد الملك أمر لم نزل نعرفه، فلما وقع لم ننكره \cite{ibnAbdRabbih_AlIqd}.

وجاء عن عمر عبد العزيز أنه كان لا يحب من يقوم له، وَلما ولي عمر بن عبد الْعَزِيز قَامَ النَّاس بَين يَدَيْهِ فَقَالَ يَا معشر النَّاس إِن تقوموا نقم وَإِن تقعدوا نقعد فَإِنَّمَا يقوم النَّاس لرب الْعَالمين إِن الله فرض فَرَائض وَسن سننا من أَخذ بهَا لحق وَمن تَركهَا محق وَمن أَرَادَ أَن يصحبنا فليصحبنا بِخمْس يُوصل إِلَيْنَا حَاجَة من لَا تصل إِلَيْنَا حَاجته ويدلنا من الْعدْل إِلَى مَا لَا نهتدي إِلَيْهِ وَيكون عونا لنا على الْحق وَيُؤَدِّي الْأَمَانَة إِلَيْنَا وَإِلَى النَّاس وَلَا يغتب عندنَا أحدا وَمن لم يفعل فَهُوَ فِي حرج من صحبتنا وَالدُّخُول علينا \cite{ibnAbdAlHakam_OmarIbnAbdAlAziz}. وجاء عنه أنه كان يرد الإطراء فعن عمرو بن عثمان الحمصي حدثنا خالد بن يزيد عن جعونة قال دخل رجل على عمر بن عبد العزيز فقال يا أمير المؤمنين إن من قبلك كانت الخلافة لهم زينا وأنت زين الخلافة فأعرض عنه \cite{dahabi_Siyar}. وقال ابن عيينة قال رجل لعمر بن عبد العزيز جزاك الله عن الإسلام خيراً قال بل جزى الله الإسلام عني خيراً \cite{dahabi_Siyar}.  وذات يوم ناداه رجل فَقَالَ يَا خَليفَة الله فِي الأَرْض فَقَالَ لَهُ عمر مَه إِنِّي لما ولدت اخْتَار لي أَهلِي اسْما فسموني عمر فَلَو ناديتني يَا عمر أَجَبْتُك [.]  فَلَمَّا وليتموني أُمُوركُم سميتموني أَمِير الْمُؤمنِينَ فَلَو ناديتني يَا أَمِير الْمُؤمنِينَ أَجَبْتُك وَأما خَليفَة الله فِي الأَرْض فلست كَذَلِك وَلَكِن خلفاء الله فِي الأَرْض دَاوُد النَّبِي عَلَيْهِ السَّلَام وَشبهه قَالَ الله تبَارك وَتَعَالَى: (يَا دَاوُد إِنَّا جعلناك خَليفَة فِي الأَرْض) \cite{ibnAbdAlHakam_OmarIbnAbdAlAziz}.

وَكَانَ عمر بن عبد الْعَزِيز إِذْ كَانَ واليا على الْمَدِينَة إِذا بَات على ظهر الْمَسْجِد مَسْجِد رَسُول الله صلى الله عَلَيْهِ وَسلم لم تقربه امْرَأَة إعظاما لمَسْجِد رَسُول الله صلى الله عَلَيْهِ وَسلم \cite{ibnAbdAlHakam_OmarIbnAbdAlAziz}.
ومن حكمه أنه أرجع مبدأ الشورى، فلَمَّا قَدِمَ عُمَرُ بْنُ عبد العزيز المدينة ونزل دار مروان دخل عليه الناس فسلموا، فلما صلى الظهر دعا عشرة من فقهاء المدينة: عروة بن الزبير، وعبيد الله بن عبد الله بن عتبة، وأبا بكر بن عبد الرحمن، وأبا بكر بن سُلَيْمَان بن ابى حثمه، وسليمان بن يسار، وو القاسم بْن محمد، وسالم بْن عَبْد اللَّهِ بن عُمَرَ، وعبد اللَّه بن عبد الله ابن عمرو، وعبد الله بن عامر بن ربيعة، وخارجة بن زيد، فدخلوا عليه فجلسوا، فَحَمِدَ اللَّهَ وَأَثْنَى عَلَيْهِ بِمَا هُوَ أَهْلُهُ، ثم قال: إني إنما دعوتكم لأمر تؤجرون عليه، وتكونون فيه أعوانا على الحق، ما أريد أن أقطع أمرا إلا برأيكم أو برأي من حضر منكم، فإن رأيتم أحدا يتعدى، أو بلغكم عن عامل لي ظلامة، فأحرج الله على من بلغه ذلك إلا بلغني. فخرجوا يجزونه خيرا، وافترقوا \cite{ibnJareerTabari_Tareekh}. وجاء أيضا أنه لما قدم أنس بن مَالك خَادِم النَّبِي صلى الله عَلَيْهِ وَسلم من الْعرَاق إِلَى الْمَدِينَة، كَانَت تعجبه صَلَاة عمر بن عبد الْعَزِيز وَكَانَ عمر أميرها، فصلى أنس خَلفه فَقَالَ مَا صليت خلف إِمَام بعد رَسُول الله صلى الله عَلَيْهِ وَسلم أشبه صَلَاة بِصَلَاة رَسُول الله صلى الله عَلَيْهِ وَسلم من إمامكم هَذَا. وَكَانَ عمر بن عبد الْعَزِيز رَضِي الله عَنهُ يتم الرُّكُوع وَالسُّجُود ويخفف الْقعُود وَالْقِيَام \cite{ibnJareerTabari_Tareekh}. وعن عطاء قال كان عمر بن عبد العزيز يجمع كل ليلة الفقهاء فيتذاكرون الموت والقيامة والآخرة ويبكون \cite{dahabi_Siyar}.

وكان يكره كل الظالمين وبالأخص الحجاج ومن ذلك أن الْحجَّاج قد ولي الْمَوْسِم، فَكتب عمر إِلَى الْخَلِيفَة يستعفيه أَن يمر عَلَيْهِ بِالْمَدِينَةِ. فَكتب أمير المؤمنين إِلَى الْحجَّاج إِن عمر بن عبد الْعَزِيز كتب إِلَيّ يستعفيني من ممرك عَلَيْهِ فَلَا عَلَيْك أَن لَا تمر بِمن كرهك فَتنحّى عَن الْمَدِينَة  \cite{ibnAbdAlHakam_OmarIbnAbdAlAziz}. جاء أيضا حَمَّدُ بْنُ عَلِيٍّ، ثنا أَبُو الْعَبَّاسِ بْنُ قُتَيْبَةَ، ثنا إِبْرَاهِيمُ بْنُ هِشَامِ بْنِ يَحْيَى، حَدَّثَنِي أَبِي، عَنْ جَدِّي قَالَ: قَالَ عُمَرُ: " مَا حَسَدْتُ الْحَجَّاجَ عَدُوَّ اللهِ عَلَى شَيْءٍ حَسَدِي إِيَّاهُ عَلَى حُبِّهِ الْقُرْآنَ وَإِعْطَائِهِ أَهْلَهُ، وَقَوْلِهِ حِينَ حَضَرَتْهُ الْوَفَاةُ: اللهُمَّ اغْفِرْ لِي، فَإِنَّ النَّاسَ يَزْعُمُونَ أَنَّكَ لَا تَفْعَلُ \cite{abuNuaim_Hilya}.

وكان عمر عبد العزيز متبعا لكتاب الله وسنة نبيه فقال: سنّ رَسُول الله صلى الله عَلَيْهِ وَسلم وولاة الْأَمر من بعده سننا الْأَخْذ بهَا اعتصام بِكِتَاب الله وَقُوَّة على دين الله وَلَيْسَ لأحد تبديلها وَلَا تغييرها وَلَا النّظر فِي أَمر خالفها من اهْتَدَى بهَا فَهُوَ مهتد وَمن استنصر بهَا فَهُوَ مَنْصُور وَمن تَركهَا وَاتبع غير سَبِيل الْمُؤمنِينَ ولاه الله مَا تولى وأصلاه جَهَنَّم وَسَاءَتْ مصيرا \cite{ibnAbdAlHakam_OmarIbnAbdAlAziz}.  وكان عمر بن عبد العزيز يخطب في قومه بالوعظ والتذكير بأمر الله فتارة يخطب عن التقوى وتارة يخطب عن البعث والتوبة والإستغفار وكان يبكي على المنبر ويبكي من حوله من الناس. وخطب أيضا: أيها الناس، لا تستصغروا الذنوب، والتمسوا تمحيص ما سلف منها بالتوبة منها؛ إِنَّ الْحَسَناتِ يُذْهِبْنَ السَّيِّئاتِ، ذلِكَ ذِكْرى لِلذَّاكِرِينَ، وقال عز وجل: وَالَّذِينَ إِذا فَعَلُوا فاحِشَةً أَوْ ظَلَمُوا أَنْفُسَهُمْ ذَكَرُوا اللَّهَ فَاسْتَغْفَرُوا لِذُنُوبِهِمْ وَمَنْ يَغْفِرُ الذُّنُوبَ إِلَّا اللَّهُ وَلَمْ يُصِرُّوا عَلى ما فَعَلُوا وَهُمْ يَعْلَمُونَ \cite{ibnAbdRabbih_AlIqd}. وكانت خطبه قصيرة جدا ولكن بالغة ومؤثرة ومن ذلك أنه ذات يوم خطب بالنَّاس فَقَالَ أَيهَا النَّاس إِنَّه لَيْسَ بعد نَبِيكُم نَبِي وَلَيْسَ بعد الْكتاب الَّذِي أنزل عَلَيْكُم كتاب فَمَا أحل الله على لِسَان نبيه فَهُوَ حَلَال إِلَى يَوْم الْقِيَامَة وَمَا حرم الله على لِسَان نبيه فَهُوَ حرَام إِلَى يَوْم الْقِيَامَة أَلا إِنِّي لست بقاض وَإِنَّمَا أَنا منفذ لله وَلست بِمُبْتَدعٍ وَلَكِنِّي مُتبع أَلا إِنَّه لَيْسَ لأحد أَن يطاع فِي مَعْصِيّة الله عز وَجل لست بِخَيْرِكُمْ وَإِنَّمَا أَنا رجل مِنْكُم أَلا وَإِنِّي أثقلكم حملا يَا أَيهَا النَّاس إِن أفضل الْعِبَادَة أَدَاء الْفَرَائِض وَاجْتنَاب الْمَحَارِم أَقُول قولي هَذَا وَأَسْتَغْفِر الله الْعَظِيم لي وَلكم \cite{ibnAbdAlHakam_OmarIbnAbdAlAziz}. فأحبه الناس لما كان فيه من خير، ومن ذلك أن سهيل بن أبي صالح قال: كنت مع أبي غداة عرفة فوقفنا لننظر لعمر ابن عبد العزيز وهو أمير الحاج فقلت يا أبتاه والله إني لأرى الله يحب عمر قال لم؟ قلت: لما أراه دخل له في قلوب الناس من المودة وأنت سمعت ابا هريرة يقول قال رسول الله ﷺ: "إذا أحب الله عبدا نادى جبريل إن الله قد أحب فلانا فأحبوه" {\footnotesize (صحيح البخاري، صحيح مسلم)} \cite{dahabi_Siyar}.

وكان عمر بن عبد العزيز يخاف الله، وسئلت فَاطِمَة بنت عبد الْملك زَوْجَة عمر بن عبد الْعَزِيز عَن عبَادَة عمر فَقَالَت وَالله مَا كَانَ بِأَكْثَرَ النَّاس صَلَاة وَلَا أَكْثَرهم صياما وَلَكِن وَالله مَا رَأَيْت أحدا أخوف لله من عمر لقد كَانَ يذكر الله فِي فرَاشه فينتفض انتفاض العصفور من شدَّة الْخَوْف حَتَّى نقُول ليصبحن النَّاس وَلَا خَليفَة لَهُم \cite{ibnAbdAlHakam_OmarIbnAbdAlAziz}. وَقَرَأَ عمر بن عبد الْعَزِيز بِالنَّاسِ ذَات لَيْلَة (وَاللَّيْل إِذا يغشى) فَلَمَّا بلغ (فأنذرتكم نَارا تلظى) خنقته الْعبْرَة فَلم يسْتَطع أَن ينفذها فَرجع حَتَّى إِذا بلغَهَا خنقته الْعبْرَة فَلم يسْتَطع أَن ينفذها فَتَركهَا وقرأ سورة غَيرهَا \cite{ibnAbdAlHakam_OmarIbnAbdAlAziz}. وكان رضي الله عنه رحيما بالمسلمين، ومن ذلك أنه كتب على أبي بكر بن حزم إِن كل من هلك وَعَلِيهِ دين لم يكن دينه فِي خرقه فَاقْض عَنهُ دينه من بَيت مَال الْمُسلمين \cite{ibnAbdAlHakam_OmarIbnAbdAlAziz}. كان بتعاهد رعيته بالنصيحة، وعند وقوع الزلازل كتب عمر بن عبد الْعَزِيز إِلَى أهل الْأَمْصَار إِن هَذِه الرجفة شَيْء يُعَاتب الله بِهِ الْعباد وَقد كنت كتبت إِلَى أهل بلد كذاوكذا أَن يخرجُوا يَوْم كذاو كَذَا فَمن اسْتَطَاعَ أَن يتَصَدَّق فيلفعل فَإِن الله عز وَجل يَقُول (قد أَفْلح من تزكّى) وَقَالَ قُولُوا كَمَا قَالَ أبوكم آدم (رَبنَا ظلمنَا أَنْفُسنَا وَإِن لم تغْفر لنا وترحمنا لنكونن من الخاسرين) وَقُولُوا كَمَا قَالَ نوح (وَإِلَّا تغْفر لي وترحمني أكن من الخاسرين) وَقُولُوا كَمَا قَالَ مُوسَى (رب إِنِّي ظلمت نَفسِي فَاغْفِر لي) \cite{ibnAbdAlHakam_OmarIbnAbdAlAziz}.

ولقد عرف الناس الغنى في زمن عمر بن عبد العزيز ومن ذلك أن يحيى بن سعيد قال بَعَثَنِي عمر بن عبد الْعَزِيز على صدقَات إفريقية فاقتضيتها وَطلبت فُقَرَاء نعطيها لَهُم فَلم نجد بهَا فَقِيرا وَلم نجد من يَأْخُذهَا مني قد أغْنى عمر بن عبد الْعَزِيز النَّاس فاشتريت بهَا رقابا فأعتقتهم وولاؤهم للْمُسلمين \cite{ibnAbdAlHakam_OmarIbnAbdAlAziz}. ومن عدله أنه قال لا تهدموا كنيسة ولا بيعة ولا بيت نار صولحتم عليه، ولا تحدثن كنيسة ولا بيت نار، ولا تجر الشاة إلى مذبحها، ولا تحدوا الشفرة على رأس الذبيحة، ولا تجمعوا بين الصلاتين إلا من عذر \cite{ibnJareerTabari_Tareekh}. 


قال معاوية بن يحيى: حدثنا أرطاة قال: قيل لعمر بن عبد العزيز: لو جعلت على طعامك أمينا لا تغتال وحرسيا إذا صليت وتنح عن الطاعون، قال اللهم إن كنت تعلم أني أخاف يوما دون يوم القيامة، فلا تؤمن خوفي \cite{dahabi_Siyar}. وقد سمم عمر عبد العزيز وشاع بين الناس أنه سحر، فعن معروف بن مشكان عن مجاهد قال لي عمر بن عبد العزيز ما يقول في الناس؟ قلت: يقولون مسحور، قال ما أنا بمسحور، ثم دعا غلاما له فقال ويحك ما حملك على أن سقيتني السم؟ قال ألف دينار أعطيتها وعلى أن أعتق، قال هاتها، فجاء بها، فألقاها في بيت المال وقال اذهب حيث لا يراك أحد \cite{dahabi_Siyar}. وجاء عن زياد عن مالك قال: دخل مسلمة بن عبد الملك على عمر بن عبد العزيز في المرضة التي مات فيها، فقال له: يا أمير المؤمنين، إنك فطمت أفواه ولدك عن هذا المال، وتركتهم عالة. ولا بدّ لهم من شيء يصلحهم، فلو أوصيت بهم إليّ أو إلى نظرائك من أهل بيتك لكفيتك مئونتهم إن شاء الله. فقال عمر أجلسوني. فأجلسوه، فقال: الحمد لله، أبالفقر تخوّفني يا مسلمة؟ أما ما ذكرت أني فطمت أفواه ولدي عن هذا المال وتركتهم عالة، فإني لم أمنعهم حقا هو لهم، ولم أعطهم حقا هو لغيرهم؛ وأما ما سألت من الوصاة إليك أو إلى نظرائك من أهل بيتي، فإن وصيتي بهم إلى الله الذي نزّل الكتاب وهو يتولّى الصالحين؛ وإنما بنو عمر أحد رجلين: رجل اتقى الله فجعل الله له من أمره يسرا ورزقه من حيث لا يتحسب، ورجل غيرّ وفجر، فلا يكون عمر أول من أعانه على ارتكابه. ادعوا لي بنيّ، فدعوهم، وهم يومئذ اثنا عشر غلاما، فجعل يصعّد بصره فيهم ويصوّبه حتى اغرورقت عيناه بالدمع ثم قال: بنفسي فتية تركتهم ولا مال لهم! يا بني، إني قد تركتكم من الله بخير، إنكم لا تمرّون على مسلم ولا معاهد إلا ولكم عليه حق واجب إن شاء الله، يا بنيّ، ميّلت رأيي بين أن تفتقروا في الدنيا وبين أن يدخل أبوكم النار، فكان أن تفتقروا إلى آخر الأبد خيرا من دخول أبيكم يوما واحدا في النار؛ قوموا يا بنيّ عصمكم الله ورزقكم! قال: فما احتاج أحد من أولاد عمر ولا افتقر \cite{ibnAbdRabbih_AlIqd}. وعن عبيد بن حسان قال لما احتضر عمر بن عبد العزيز قال اخرجوا عني فقعد مسلمة وفاطمة على الباب فسمعوه يقول مرحبا بهذه الوجوه ليست بوجوه إنس ولا جان ثم تلا: \quranayah*[28][83]{\footnotesize \surahname*[28]}  ثم هدأ الصوت فقال مسلمة لفاطمة قد قبض صاحبك فدخلوا فوجدوه قد قبض \cite{dahabi_Siyar}. وروى عن يوسف بن ماهك قال بينا نحن نسوي التراب على قبر عمر بن عبد العزيز إذاسقط علينا كتاب رق من السماء فيه بسم الله الرحمن الرحيم أمان من الله لعمر بن عبد العزيز من النار. وقال الذهبي عن هذا: مثل هذه الآية لو تمت لنقلها أهل ذاك الجمع ولما انفرد بنقلها مجهول مع أن قلبي منشرح للشهادة لعمر أنه من أهل الجنة \cite{dahabi_Siyar}.

مات عمر بن عبد العزيز يوم الجمعة لخمس بقين من رجب بدير سمعان في سنة إحدى ومائة، وهو ابن تسع وثلاثين سنة وأشهر، وكانت خلافته سنتين وخمسه اشهر، ومات بدير سمعان \cite{ibnJareerTabari_Tareekh}. وقال هشام لما جاء نعيه إلى الحسن قال مات خير الناس \cite{dahabi_Siyar}.فترك ورائه سيرة عطرة رحمه الله، فقال عنه الذهي في سير أعلام النبلاء: وكان من أئمة الاجتهاد ومن الخلفاء الراشدين رحمة الله عليه. ولخص سيرته فقال: قد كان هذا الرجل حسن الخلق والخلق، كامل العقل، حسن السمت، جيد السياسة حريصا على العدل بكل ممكن، وافر العلم، فقيه النفس، ظاهر الذكاء والفهم، أواهاً منيباً قانتاً لله حنيفاً، زاهداً مع الخلافة، ناطقا بالحق مع قلة المعين وكثرة الأمراء الظلمة الذين ملوه وكرهوا محاققته لهم ونقصه أعطياتهم وأخذه كثيراً مما في أيديهم مما أخذوه بغير حق، فما زالوا به حتى سقوه السم فحصلت له الشهادة والسعادة، وعد عند أهل العلم من الخلفاء الراشدين والعلماء العاملين \cite{dahabi_Siyar}. وقال عنه ميمون بن مهران إن الله كان يتعاهد الناس بنبي بعد نبي وإن الله تعاهد الناس بعمر بن عبد العزيز \cite{dahabi_Siyar}. وقال حرملة سمعت الشافعي يقول الخلفاء خمسة أبو بكر وعمر وعثمان وعلي وعمر بن عبد العزيز وفي رواية الخلفاء الراشدون وورد عن أبي بكر بن عياش نحوه وروى عباد بن السماك عن الثوري مثله \cite{dahabi_Siyar}. وعن عبد الرحمن بن زيد عن عمر بن أسيد قال والله ما مات عمر بن عبد العزيز حتى جعل الرجل يأتينا بالمال العظيم فيقول اجعلوا هذا حيث ترون فما يبرح يرجع بماله كله قد أغنى عمر الناس \cite{dahabi_Siyar}. وقال فيه حماد بن واقد سمعت مالك بن دينار يقول الناس يقولون عني زاهد إنما الزاهد عمر بن عبد العزيز الذي أتته الدنيا فتركها \cite{dahabi_Siyar}. وفي الزهد لابن المبارك أخبرنا إبراهيم بن نشيط حدثنا سليمان بن حميد عن أبي عبيدة بن عقبة بن نافع أنه دخل على فاطمة بنت عبد الملك فقال ألا تخبريني عن عمر؟ قالت ما أعلم أنه اغتسل من جنابة ولا احتلام منذ استخلف \cite{dahabi_Siyar}. وعن ابن المبارك عن هشام بن الغاز عن مكحول لو حلفت لصدقت ما رأيت أزهد ولا أخوف لله من عمر بن عبد العزيز \cite{dahabi_Siyar}.

وولي يزيد بن عبد الملك بن مروان بن الحكم، وأمه عاتكة بنت يزيد بن معاوية، يوم الجمعة لخمس بقين من رجب سنة إحدى ومائة \cite{ibnAbdRabbih_AlIqd}. فزاد الظلم بعد ذلك. 

\section{الحساب في زمن الحكم الرشيد}

البحث والعناية بعلم الحساب هو غاية جليلة ومهمة عظيمة أعتنى بها المسلمون اللاحقون في زمن الخليفة الراشد هارون الرشيد التي أسس دار الحكمة في بغداد العراق حتى أصبح المسلمين في ذلك الوقت روادا في علم الحساب والذى كان مفتاحا لهم لشتى العلوم الأخرى حتى عرف ذلك الزمان بالعصر الإسلامي الذهبي. ومن أبرز من بحث وألف في علم الحساب هو العالم الفذ محمد بن موسى الخوارزمي رحمه الله تعالى والذي وصل صيته أقطاب الأرض حتى دخل أسمه معاجم وقواميس كافة اللغات الأخرى. فاللوغرتميات جاءت من الترجمة اللاتينية لإسمه وهو ما عرف عند العرب المتأخرين بالخوارزميات. وهذا مفهوم يبنى عليه كافة الحسابات المركبة والمعقدة التي نراها اليوم من انظمة الحساب والمنطق بشتى أنواعها بما فيها أنظمة الصواريخ والطيران وحتى انظمة الذكاء الإصطناعي. وقد ألف الخوارزمي كتابه "المختصر في الجبر والمقابلة" وكان هذا الكتاب نافعا للمسلمين وغيرهم وهو أساس تقدم البشرية في شتى المجالا إلى يومنا هذا. ولهذا سمي علم الموازنة والمقابلة بعلم "الجبر" كما سماه الخوارزمي بذلك وتمت اضافة كلمة "الجبر" أيضا إلى كافة معاجم اللغات الأخرى. ويعتبر الخوارزمي إلى يومنا هو مؤسس علم الجبر والحساب والخوارزميات ومن أهم علماء الحساب في تاريخ البشرية.

وللأسف فقد غاب وغيب على أغلب المسلمين في زماننا هذا أهمية ميراث الخوارزمي في علم الجبر والحساب. وهو ميراث حري بنا جمع شتاته وإعادة بناء أركانه لتقوم الأمة بالميزان الذي أمرنا الله به. فقد جهل الكثير من المسليمن ميراث الخوارزمي حتى بخس قدره ونسي علمه فكان بين مفرط أو مدلس. ومن ذلك ضياع كتابه في الجبر والمقابلة من المسلمين حتى تمت طباعة أول نسحة عربية منه في عام 1939م (1357هـ) بناء على النسخة الأصلية الوحيدة التي سرقت من مصر ونقلت إلى بريطانيا والتي يرجع تاريخها إلى عام 1439م (843هـ) أي بعد وفاة الخوارزمي بحوالي 500 عام شمسية.
ليرجع لنا كتاب الخوارزمي بعد حوالي ألف عام من تأليفه.
وفي كل هذه الأعوام ترجم كتابه إلى شتى اللغات ومنها الأنجليزية والألمانية والفرنسية وأصبحت مرجعا لجميع الحضارات الأوروبية وغيرها.
ليتفاجأ المسلمين بوجود كلمات عربية في هذه الثقافات ومنها algorithms والتي تعني الخوارزميات وكلمة algebra وهي الجبر في معجم اللغة الانجليزية على سبيل المثال لا الحصر.

ومن التدليس الذي تعرض له الخوارزمي في تقديم كتابه هو نسبة عمله إلى الحضارة المصرية في طرح مخالف للطرح الذي وضعه الخوارزمي في كتابه. وهذا ليس إلا إحقاقا للحق ولا يجب أن يحمل هذا على محمل الإستنقاص لمن نقل هذا العمل لنا تقديما وتعليقا فجزاهم الله خير الجزاء. ومن التدليس أيضا طرح كتابه في الحساب مجردا من الغاية التي كتب لها ومنه عدم ذكر سبب تأليف كتابه في الجبر والذي كان في الأساس سعيا منه رحمه الله لتحقيق الحكم الرشيد بناء على الحساب الصحيح في الميراث والبيع والشراء والكراء وما بتعلق بذلك من حساب المسافات والأرض. وليتبين طرح الخوارزمي نضع مقدمة كتابه رحمه الله والتي جاء فيها:\footnote{مع تصرف يسير من حذف لكلمات التي تخالف السياق وفي الغالب قد يظن انها أخطاء خلال النسخ.}

\newpage
\fbox{\small
    \begin{minipage}{34em}
        \begin{center}
            بسم الله الرحمن الرحيم
        \end{center}
        هذا كتاب وضعه محمد بن موسى الخوارزمي افتتحه بأن قال:

        الحمد الله على نعمه بما هو أهله من محامده التي بأداء ما افترض منها على من يعبده من خلقه يقع اسم الشكر ويستوجب المزيد إقرارا بروبويته وتذللا لعزته وخشوعا لعظمته. بعث محمدا صلى الله عليه وعلى آله وسلم بالنبوة على حين فترة من الرسل نورا من الحق ودروس من الهدي فبصر به من العمى واستنقذ به من الهلكة وكثر به بعد قلة وألف به بعد الشتات.

        تبارك الله ربنا وتعالى جده وتقدست أسماؤه ولا إله غيره, وصلى الله على محمد النبي وآله وسلم. ولم تزل العلماء في الأزمنة الخالية والأمم الماضية يكتبون الكتب مما يصنفون من صنوف العلم ووجوه الحكمة نظرا لمن بعدهم واحتسابا للأجر بقدر الطاقة ورجاء أن يلحقهم من أجر ذلك وذخره وذكره ويبقى لهم من لسان صدق ما يصغر في جنبه كثير مما كانوا يتكفلونه من المؤونة ويحملونه على أنفسهم من المشقة في كشف أسرار العلم وغامضه. إما رجل سبق إلى مالم يكن مستخرجا قبله فورثه من بعده. وإما رجل شرح مما أبقى الأولون ما كان مستغلقا فأوضح طريقه وسهل مسلكه وقرب مأخذه. وإما رجل وجد في بعض الكتب خللا فلم شعثه وأقام أوده وأحسن الظن بصاحبه غير راد عليه ولا مفتخر بذلك من فعل نفسه.

        وقد شجعني ما فضل الله به الامام المأمون أمير المؤمنين مع الخلافة التي حاز له إرثها وأكرمه بلباسها وحلاه بزينتها, من الرغبة في الأدب وتقريب أهله وإدنائهم وبسط كنفه لهم ومعونته إياهم على إيضاح ما كان مستبهما وتسهيل ما كان مستوعرا. على أن ألفت من كتاب الجبر والمقابلة كتابا مختصرا حاصرا للطيف الحساب وجليله لما يلزم الناس من الحاجة إليه في مواريثهم ووصياهم وفي مقاسمتهم وأحكامهم وتجارتهم, وفي جميع ما بتعاملون به بينهم من مساحة الأرضين وكرى الأنهار والهندسة وغير ذلك من وجوهه وفنونه, مقدما لحسن النية فيه وراجيا لأن ينزله أهل الأدب بفضل ما استودعوا من نعم الله تعالى وجليل آلائه وجميل بلائه عندهم منزلته وبالله توفيقي في هذا لا في غيره عليه توكلت وهو رب العرش العظيم وصلى الله على جميع الأنبياء والمرسلين.
    \end{minipage}
}

وعليه يلعم أن الخوارزمي رحمه الله إنما ألف كتابه هذا لتوضيح علم الحساب الصحيح الذي يحتاج إليه الناس في أمور دينهم ودنياهم. فقد إفتتح الخوارزمي رحمه الله كتابه بالبسملة متبعا سنة الأنبياء في ذلك. وكان رحمه الله حريصا وراجيا بأن يعتنى أهل الأدب بهذا الكتاب ويعطونه حقه وينزلونه منزلته لما علم ما فيه من أسس وقواعد لا غنى عنها في علم الحساب الصحيح. وختم مقدمته سائلا الله التوفيق في ذلك ومتوكلا عليه. وبهذا يتبين حسن مقصد الخوارزمي من تأليف كتابه فنسأل الله العلي العظيم أن يرحمه رحمة واسعة وأن يرفغ قدره في الجنة وأن يجزيه عنا خير الجزاء.

فبدأو بالعناية بعلم الحساب وجمع مؤلفاته من كافة أقطاب الدنيا فعكفوا على ترجمتها حتى فهموها وعقلوها وعرفوا ما شابها من خطأ ونقصان. فأسسوا نظام الأرقام الذي نعرفه اليوم فقسموا الأرقام إلى ارقام فردية وأسسوا علم الجبر وحساب المثلثات وغيرها من علوم الحساب بشكل لم تعرفه البشرية من قبل. وكان ذلك سببا في تحقيق الحكم الرشيد في المعاملات والبيع والشراء والكراء. فكان علم الحساب مفتاحا في تطور المسلمين في شتى مجالات الدنيا ومنها مجال الهندسة والطب في العصر الإسلامي الذهبي.

يعتبر المسلمين هم من وضع أسس العلم الحديث في زمن هارون الرشيد الذي أسس دار الحكمة في بغداد لتكون في ذلك الزمان عاصمة الحضارة في العلم حيث عرف هذا العصر بالعصر الإسلامي الذهبي. وبفضل التأمل والنظر في آيات القرآن الكريم الدالة على تعلم العدد والحساب وإقامة الوزن بالقسط وأن كل شئ خلق بقدر، استطاع المسلمين التفوق على غيرهم من الأمم الأخرى في علم الحساب فكانوا هم أول من أسس علم الجبر والمقابلة وكان ذلك على يدي العالم الجليل محمد بن موسى الخوارزمي -رحمه الله تعالى- الذي نُسِيَ فضله وضُيِعَ علمه بين إفراط وتفريط. ولقد بين شيخ الإسلام بن تيمية أن أهل السنة في زمانه قد اعتنوا بالنظر في علم الجبر والمقابلة الذي أسسه الخوارزمي وأشار إلى أهمية هذا العلم في العلم الشرعي ومن ذلك قوله: 
"وَكَذَلِكَ كَثِيرٌ مِنْ مُتَأَخِّرِي أَصْحَابِنَا يَشْتَغِلُونَ وَقْتَ بَطَالَتِهِمْ بِعِلْمِ الْفَرَائِضِ وَالْحِسَابِ وَالْجَبْرِ وَالْمُقَابَلَةِ وَالْهَنْدَسَةِ وَنَحْوِ ذَلِكَ؛ لِأَنَّ فِيهِ تَفْرِيحًا لِلنَّفْسِ وَهُوَ عِلْمٌ صَحِيحٌ لَا يَدْخُلُ فِيهِ غَلَطٌ. وَقَدْ جَاءَ عَنْ عُمَرَ بْنِ الْخَطَّابِ أَنَّهُ قَالَ: إذَا لَهَوْتُمْ فَالْهُوا بِالرَّمْيِ وَإِذَا تَحَدَّثْتُمْ فَتَحَدَّثُوا بِالْفَرَائِضِ. فَإِنَّ حِسَابَ الْفَرَائِضِ عِلْمٌ مَعْقُولٌ مَبْنِيٌّ عَلَى أَصْلٍ مَشْرُوعٍ فَتَبْقَى فِيهِ رِيَاضَةُ الْعَقْلِ وَحِفْظُ الشَّرْعِ" \href{https://shamela.ws/book/7289/4394#p1}{\faExternalLink} \cite{ibnTaimia_Majmoo}.\footnote{مجموع الفتاوى 9/129.} وكل هذا فيه اهتمام السلف بهذا العلم العظيم في أوقات فراغهم رغم إنشغالهم بالأمور العظيمة الأخرى في بيان الحق ورد البدع والشبهات التي عصفت في ذلك الزمن كعلم الكلام والفلفسة التي تخالف صريح كتاب الله جل جلاله وسنة نبيه ﷺ.

وينسب لشيخ الإسلام قوله عن الخوارزمي: "وإن كان علمه صحيحا إلا إن العلوم الشرعية مستغنية عنه وعن غيره". ولكن هذا القول لم يثبت عن شيخ الإسلام في حق الخوارزمي. بل هذا من التدليس والتفريط في هذا العلم العظيم الذي ابتلينا به في زماننا. وإنما كان شبخ الإسلام يرد على من غلى في علم الحساب وأراد أن يجعل الشريعة متوقفة عليه من غلاة المنطق، فقال في ذلك: "وَقَدْ بَيَّنَّا أَنَّهُ يُمْكِنُ الْجَوَابُ عَنْ كُلِّ مَسْأَلَةٍ شَرْعِيَّةٍ جَاءَ بِهَا الرَّسُولُ صَلَّى اللَّهُ عَلَيْهِ وَسَلَّمَ بِدُونِ حِسَابِ الْجَبْرِ وَالْمُقَابَلَةِ. وَإِنْ كَانَ حِسَابُ الْجَبْرِ وَالْمُقَابَلَةِ صَحِيحًا فَنَحْنُ قَدْ بَيَّنَّا أَنَّ شَرِيعَةَ الْإِسْلَامِ وَمَعْرِفَتَهَا لَيْسَتْ مَوْقُوفَةً عَلَى شَيْءٍ يُتَعَلَّمُ مِنْ غَيْرِ الْمُسْلِمِينَ أَصْلًا وَإِنْ كَانَ طَرِيقًا صَحِيحًا. بَلْ طُرُقُ الْجَبْرِ وَالْمُقَابَلَةِ فِيهَا تَطْوِيلٌ. يُغْنِي اللَّهُ عَنْهُ بِغَيْرِهِ كَمَا ذَكَرْنَا فِي الْمَنْطِقِ. وَهَكَذَا كُلُّ مَا بُعِثَ بِهِ النَّبِيُّ صَلَّى اللَّهُ عَلَيْهِ وَسَلَّمَ مِثْلَ الْعِلْمِ بِجِهَةِ الْقِبْلَةِ وَالْعِلْمِ بِمَوَاقِيتِ الصَّلَاةِ وَالْعِلْمِ بِطُلُوعِ الْفَجْرِ وَالْعِلْمِ بِالْهِلَالِ؛ فَكُلُّ هَذَا يُمْكِنُ الْعِلْمُ بِهِ بِالطُّرُقِ الَّتِي كَانَ الصَّحَابَةُ وَالتَّابِعُونَ لَهُمْ بِإِحْسَانِ يَسْلُكُونَهَا وَلَا يَحْتَاجُونَ مَعَهَا إلَى شَيْءٍ آخَرَ\comment{. وَإِنْ كَانَ كَثِيرٌ مِنْ النَّاسِ قَدْ أَحْدَثُوا طُرُقًا أُخَرَ؛ وَكَثِيرٌ مِنْهُمْ يَظُنُّ أَنَّهُ لَا يُمْكِنُ مَعْرِفَةُ الشَّرِيعَةِ إلَّا بِهَا. وَهَذَا مِنْ جَهْلِهِمْ كَمَا يَظُنُّ طَائِفَةٌ مِنْ النَّاسِ أَنَّ الْعِلْمَ بِالْقِبْلَةِ لَا يُمْكِنُ إلَّا بِمَعْرِفَةِ أَطْوَالِ الْبِلَادِ وَعُرُوضِهَا. وَهُوَ وَإِنْ كَانَ عِلْمًا صَحِيحًا حِسَابِيًّا يُعْرَفُ بِالْعَقْلِ لَكِنَّ مَعْرِفَةَ الْمُسْلِمِينَ بِقِبْلَتِهِمْ لَيْسَتْ مَوْقُوفَةً عَلَى هَذَا [.] فَلِهَذَا كَانَ قُدَمَاءُ عُلَمَاءِ " الْهَيْئَةِ " كَبَطْلَيْمُوسَ صَاحِبِ الْمَجِسْطِي وَغَيْرِهِ لَمْ يَتَكَلَّمُوا فِي ذَلِكَ بِحَرْفِ وَإِنَّمَا تَكَلَّمَ فِيهِ بَعْضُ الْمُتَأَخِّرِينَ مِثْلَ كوشيار الدَّيْلَمِيَّ وَنَحْوِهِ لَمَّا رَأَوْا الشَّرِيعَةَ جَاءَتْ بِاعْتِبَارِ الرُّؤْيَةِ. فَأَحَبُّوا أَنْ يَعْرِفُوا ذَلِكَ بِالْحِسَابِ فَضَلُّوا وَأَضَلُّوا.}" \href{https://shamela.ws/book/7289/4480#p1}{\faExternalLink} \cite{ibnTaimia_Majmoo}.\footnote{مجموع الفتاوى 9/215.}

وقد ذكر شيخ الإسلام إعتناء أهل السنة في زمانه بالعلوم الصادقة ومنها الجبر والمقابلة وخص الخوارزمي في ذلك فقال: 
"لِهَذَا يَرْغَبُ كَثِيرٌ مِنْ عُلَمَاءِ السُّنَّةِ فِي النَّظَرِ فِي الْعُلُومِ الصَّادِقَةِ الدَّقِيقَةِ كَالْجَبْرِ وَالْمُقَابَلَةِ وَعَوِيصِ الْفَرَائِضِ وَالْوَصَايَا وَالدُّورِ وَهُوَ عِلْمٌ صَحِيحٌ فِي نَفْسِهِ [.] وَأَمَّا "حِسَابُ الْفَرَائِضِ" فَمَعْرِفَةُ أُصُولِ الْمَسَائِلِ وَتَصْحِيحُهَا وَالْمُنَاسَخَاتُ وَقِسْمَةُ التَّرِكَاتِ. وَهَذَا الثَّانِي كُلُّهُ عِلْمٌ مَعْقُولٌ يُعْلَمُ بِالْعَقْلِ كَسَائِرِ حِسَابِ الْمُعَامَلَاتِ وَغَيْرِ ذَلِكَ مِنْ الْأَنْوَاعِ الَّتِي يَحْتَاجُ إلَيْهَا النَّاسُ. ثُمَّ قَدْ ذَكَرُوا حِسَابَ الْمَجْهُولِ الْمُلَقَّبَ بِحِسَابِ الْجَبْرِ وَالْمُقَابَلَةِ فِي ذَلِكَ وَهُوَ عِلْمٌ قَدِيمٌ لَكِنَّ إدْخَالَهُ فِي الْوَصَايَا وَالدَّوْرِ وَنَحْوِ ذَلِكَ أَوَّلُ مَنْ عُرِفَ أَنَّهُ أَدْخَلَهُ فِيهَا مُحَمَّدُ بْنُ مُوسَى الخوارزمي. وَبَعْضُ النَّاسِ يَذْكُرُ عَنْ عَلِيِّ بْنِ أَبِي طَالِبٍ أَنَّهُ تَكَلَّمَ فِيهِ وَأَنَّهُ تَعَلَّمَ ذَلِكَ مِنْ يَهُودِيٍّ وَهَذَا كَذِبٌ عَلَى عَلِيٍّ."
\href{https://shamela.ws/book/7289/4479#p2}{\faExternalLink} \cite{ibnTaimia_Majmoo}.\footnote{مجموع الفتاوى 9/214.}

سيرة هارون الرشيد
اللحيدان
%\href{https://www.youtube.com/watch?v=0QDv__BL1VQ}{video}

الفوزان
%\href{https://www.alfawzan.af.org.sa/ar/node/7212}{text}


كلام الفوزان في المأمون
%\href{https://www.youtube.com/watch?v=dwUnqfHH7eI}{video}
غرر به المعتزلة

ومن الحكم الرشيد مصالحة الكفار لدرء المفاسد كما في العهد المكي
وفيه ايضا ان النجاشي لم يكن مسلم ولا  يظلم عنده أحد فهو حقق العدل
%\href{https://www.youtube.com/watch?v=bZH3NTKYoOQ}{video}



\section{عودة الحكم الرشيد في آخر الزَّمانِ}

فقال: يا أبا بكرة، حدثني بشيء سمعته من رسول الله صلى الله عليه وسلم، فقال: كان رسول الله صلى الله عليه وسلم يعجبه الرؤيا الصالحة ويسأل عنها، فقال رسول الله صلى الله عليه وسلم ذات يوم: " أيكم رأى رؤيا؟ " فقال رجل: أنا يا رسول الله، رأيت كأن ميزانا دلي من السماء، فوزنت أنت بأبي بكر فرجحت بأبي بكر، ثم وزن أبو بكر بعمر، فرجح أبو بكر بعمر، ثم وزن عمر بعثمان، فرجح عمر بعثمان، ثم رفع الميزان، فاستاء لها رسول الله صلى الله عليه وسلم، فقال: " خلافة نبوة، ثم يؤتي الله الملك من يشاء "، قال عفان فيه: فاستآلها، (١) وقال حماد: فساءه ذلك (٢)
\href{https://shamela.ws/book/25794/16870#p1}{\faExternalLink}


فقال حذيفة: قال رسول الله صلى الله عليه وسلم: " تكون النبوة فيكم ما شاء الله أن تكون، ثم يرفعها إذا شاء (٢) أن يرفعها، ثم تكون خلافة على منهاج النبوة، فتكون ما شاء الله أن تكون، ثم يرفعها إذا شاء الله أن يرفعها، ثم تكون ملكا عاضا، فيكون ما شاء الله أن يكون، ثم يرفعها إذا شاء أن يرفعها، ثم تكون ملكا جبرية، فتكون ما شاء الله أن تكون، ثم يرفعها إذا شاء أن يرفعها، ثم تكون خلافة على منهاج نبوة " (٣) ثم سكت،

\href{https://shamela.ws/book/25794/14930#p2}{\faExternalLink}

قالَ: كانَتْ بَنُو إسْرائِيلَ تَسُوسُهُمُ الأنْبِياءُ، كُلَّما هَلَكَ نَبِيٌّ خَلَفَهُ نَبِيٌّ، وإنَّه لا نَبِيَّ بَعْدِي، وسَيَكونُ خُلَفاءُ فَيَكْثُرُونَ. قالوا: فَما تَأْمُرُنا؟ قالَ: فُوا ببَيْعَةِ الأوَّلِ فالأوَّلِ، أعْطُوهُمْ حَقَّهُمْ؛ فإنَّ اللَّهَ سائِلُهُمْ عَمَّا اسْتَرْعاهُمْ.

يَكونُ في آخِرِ الزَّمانِ خَلِيفَةٌ يَقْسِمُ المالَ ولا يَعُدُّهُ.
الراوي : أبو سعيد الخدري وجابر بن عبدالله | المحدث :مسلم | المصدر : صحيح مسلم
الصفحة أو الرقم: 2913 | خلاصة حكم المحدث : [صحيح]


مِنْ خُلَفائِكُمْ خَلِيفَةٌ يَحْثُو المالَ حَثْيًا، لا يَعُدُّهُ عَدَدًا. وفي رِوايَةِ ابْنِ حُجْرٍ: يَحْثِي المالَ.
الراوي : أبو سعيد الخدري | المحدث : مسلم | المصدر : صحيح مسلم | الصفحة أو الرقم : 2914 | خلاصة حكم المحدث : [صحيح]
   





\section{معادلات}
فيما يلي مثال على معادلة رياضية:

\begin{equation}
    E = mc^2
\end{equation}

ومثال آخر على معادلة معقدة:

\begin{equation}
    \int_0^\infty e^{-x^2} \, dx = \frac{\sqrt{\pi}}{2}
\end{equation}

\newpage

\section{نص الفصل الأول - الصفحة الثانية}

هذه الصفحة الثانية للفصل الأول تحتوي على نص إضافي لتوضيح كيفية تنسيق النصوص في كتب اللاتكس باللغة العربية.

\newpage

\section{نص الفصل الأول - الصفحة الثالثة}

هذه الصفحة الثالثة للفصل الأول تحتوي على المزيد من النصوص لاختبار تقسيم الصفحات وظهور الرؤوس والأقدام بشكل صحيح في النصوص العربية.
