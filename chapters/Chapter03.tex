
\chapter{التفكر في آيات الله الكونية}

\section{مقدمة}
هذه هي المقدمة للفصل الأول.

\section{التفكر في آيات الله الكونية}
التفكر في آيات الله الكونية هو عبادة عظيمة حث الله عز وجل عليها في كتابه في غير موضع. منها ما قرن مع العدد والحساب كما في قوله تعالى:
\quranayah*[17][12]{\footnotesize \surahname*[17]}.
وفي قوله تعالى: \quranayah*[10][5]{\footnotesize \surahname*[10]}.

\quranayah*[3][190-191]{\footnotesize \surahname*[3]}.

\quranayah*[45][3-5]{\footnotesize \surahname*[45]}.

التفكر في كتاب الله:
\quranayah*[59][21]{\footnotesize \surahname*[59]}.
\quranayah*[16][44]{\footnotesize \surahname*[16]}.

التفكر في السموات والأرض
\quranayah*[45][13]{\footnotesize \surahname*[45]}.

التفكر في الأرض, والجبال, والأنهار, والثمرات, والأزواج, والليل والنهار:
\quranayah*[13][3]{\footnotesize \surahname*[13]}.

التفكر في النحل:

\quranayah*[16][69]{\footnotesize \surahname*[16]}.

التفكر في الحياة الدنيا:

\quranayah*[10][24]{\footnotesize \surahname*[10]}.

التفكر في الزرع والزيتون والنخيل والأعناب, وكل الثمرات:

\quranayah*[16][11]{\footnotesize \surahname*[16]}.

التفكر في الزواج:

\quranayah*[30][21]{\footnotesize \surahname*[30]}.

التفكر في الموت:

\quranayah*[39][42]{\footnotesize \surahname*[39]}.

النهي

\quranayah*[22][46]{\footnotesize \surahname*[22]}.
