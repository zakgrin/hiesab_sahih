\chapter{الملحق}
% \addcontentsline{toc}{chapter}{\protect\numberline{}الملحق}
\label{sec:appendix}

\section{مسألة أول ما خلق الله}
\label{sec:app_first_creation}


لقد تبث عن النبي ﷺ أنه الله لم بخلق السموات والأرض إلا بعد كتابة المقادير وكان عرشه على الماء سبحانه. ولكن اختلف أهل العلم في أول ما خلق الله قبل أن يخلق السموات والأرض. فمنهم من قدم العرش والماء على القلم واللوح المحفوظ، ومنهم من قدم القلم واللوح المحفوظ على العرش والماء. 

\subsection{الأحاديث الخاصة بالمسألة}

حدثنا عُمَرُ بْنُ حَفْصِ بْنِ غِيَاثٍ، حَدَّثَنَا أَبِي، حَدَّثَنَا الْأَعْمَشُ، حَدَّثَنَا جَامِعُ بْنُ شَدَّادٍ، عَنْ صَفْوَانَ بْنِ مُحْرِزٍ، أَنَّهُ حَدَّثَهُ عَنْ عِمْرَانَ بْنِ حُصَيْنٍ رضي الله عنه قَالَ: دَخَلْتُ عَلَى النَّبِيِّ ﷺ وَعَقَلْتُ نَاقَتِي بِالْبَابِ، فَأَتَاهُ نَاسٌ مِنْ بَنِي تَمِيمٍ فَقَالَ: "اقْبَلُوا البُشْرَى يَا بَنِي تَمِيمٍ"، قَالُوا: قَدْ بَشَّرْتَنَا فَأَعْطِنَا - مَرَّتَيْنِ - ثُمَّ دَخَلَ عَلَيْهِ نَاسٌ مِنْ أَهْلِ اليَمَنِ فَقَالَ: "اقْبَلُوا البُشْرَى يَا أَهْلَ اليَمَنِ إِذْ لَمْ يَقْبَلْهَا بَنُو تَمِيمٍ"، قَالُوا: قَدْ قَبِلْنَا يَا رَسُولَ اللَّهِ. قَالُوا: جِئْنَاكَ نَسْأَلُكَ عَنْ هَذَا الأَمْرِ؟ قَالَ: "كَانَ اللَّهُ وَلَمْ يَكُنْ شَيْءٌ غَيْرُهُ، وَكَانَ عَرْشُهُ عَلَى المَاءِ، وَكَتَبَ فِي الذِّكْرِ كُلَّ شَيْءٍ، وَخَلَقَ السَّمَاوَاتِ وَالأَرْضَ". فَنَادَى مُنَادٍ: ذَهَبَتْ نَاقَتُكَ يَا ابْنَ الحُصَيْنِ، فَانْطَلَقْتُ فَإِذَا هِيَ يَقْطَعُ دُونَهَا السَّرَابُ، فَوَاللَّهِ لَوَدِدْتُ أَنِّي كُنْتُ تَرَكْتُهَا \href{https://shamela.ws/book/1284/2020#p2}{\faExternalLink} \cite{bukhari}.\footnote{صحيح البخاري: 3199}

وجاء أيضا في صحيح البخاري نفس الحديث بسند آخر:
حدثنا عَبْدَانُ، عَنْ أَبِي حَمْزَةَ، عَنِ الْأَعْمَشِ، عَنْ جَامِعِ بْنِ شَدَّادٍ، عَنْ صَفْوَانَ بْنِ مُحْرِزٍ، عَنْ عِمْرَانَ بْنِ حُصَيْنٍ قَالَ: إِنِّي عِنْدَ النَّبِيِّ ﷺ إِذْ جَاءَهُ قَوْمٌ مِنْ بَنِي تَمِيمٍ فَقَالَ: "اقْبَلُوا الْبُشْرَى يَا بَنِي تَمِيمٍ"، قَالُوا: بَشَّرْتَنَا فَأَعْطِنَا، فَدَخَلَ نَاسٌ مِنْ أَهْلِ الْيَمَنِ فَقَالَ: "اقْبَلُوا الْبُشْرَى يَا أَهْلَ الْيَمَنِ إِذْ لَمْ يَقْبَلْهَا بَنُو تَمِيمٍ"، قَالُوا: قَبِلْنَا، جِئْنَاكَ لِنَتَفَقَّهَ فِي الدِّينِ، وَلِنَسْأَلَكَ عَنْ أَوَّلِ هَذَا الْأَمْرِ مَا كَانَ؟ قَالَ: "كَانَ اللَّهُ وَلَمْ يَكُنْ شَيْءٌ قَبْلَهُ، وَكَانَ عَرْشُهُ عَلَى الْمَاءِ، ثُمَّ خَلَقَ السَّمَاوَاتِ وَالْأَرْضَ، وَكَتَبَ فِي الذِّكْرِ كُلَّ شَيْءٍ"، ثُمَّ أَتَانِي رَجُلٌ فَقَالَ: يَا عِمْرَانُ، أَدْرِكْ نَاقَتَكَ فَقَدْ ذَهَبَتْ، فَانْطَلَقْتُ أَطْلُبُهَا فَإِذَا السَّرَابُ يَنْقَطِعُ دُونَهَا، وَايْمُ اللَّهِ، لَوَدِدْتُ أَنَّهَا قَدْ ذَهَبَتْ وَلَمْ أَقُمْ \href{https://shamela.ws/book/1284/4624#p6}{\faExternalLink} \cite{bukhari}.\footnote{صحيح البخاري: 7413} وهذا فيه تقديم السموات والأرض على كتابة المقادير وهذا يتعارض مع أغلب الأحاديث الأخرى.

حدَّثنا جعفرُ بنُ مسافر الهُذلىُّ، حدَّثنا يحيى بن حسَّان، حدَّثنا الوليدُ بن رباحٍ، عن إبراهيمَ بن أبي عَبْلَة، عن أبي حفصةَ، قال: قال عبادةُ بن الصَّامت لابنه: يا بُنيَّ إنَّك لن تَجِدَ طعمَ حقيقةِ الإيمان حتّى تَعلمَ أن ما أصابَكَ لم يكُنْ ليُخْطئَكَ، وما أخطَاكَ لم يكُنْ ليُصيبَك، سمعتُ رسول الله ﷺ يقول: إنَّ أولَ ما خلق اللهُ القلمُ، فقال لهُ: اكتبْ، قال: ربِّ وماذا أكتبُ؟ قال: اكتُبْ مقاديرَ كلِّ شيءٍ حتى تقومَ الساعةُ. وفي رواية: اكتُبْ القدَرَ، ما كان و ما هو كائِنٌ إلى الأبَدِ. ومن مات على غيرِ هذا فليسَ مِني \href{https://shamela.ws/book/117359/3975#p2}{\faExternalLink} \cite{SunanAbiDawood}.\footnote{أبي داود: 4700 واللفظ له، أحمد: 22705، الترمذي: 3319، وصححه الألباني في صحيح الجامع.} وهذا فيه أن الله خلق القلم أولا ثم خلق اللوح المحفوظ وأمر جل جلاله القلم أن يكتب على اللوح المحفوظ المقادير كلها إلى قيام الساعة. وكل هذا كان قبل خلق السموات والأرض بخمسين ألف سنة وكان عرشه جل جلاله على الماء كما صح ذلك عن النبي ﷺ أنه قال:  كَتَبَ اللَّهُ مَقَادِيرَ الخَلَائِقِ قَبْلَ أَنْ يَخْلُقَ السَّمَوَاتِ وَالأرْضَ بِخَمْسِينَ أَلْفَ سَنَةٍ، وَعَرْشُهُ علَى المَاءِ \href{https://shamela.ws/book/1727/6683#p2}{\faExternalLink} \cite{muslim}.\footnote{صحيح مسلم: 2653، وصححه الألباني في شرح الطحاوية.}

حَدَّثَنَا يَزِيدُ بْنُ هَارُونَ، أَخْبَرَنَا حَمَّادُ بْنُ سَلَمَةَ، عَنْ يَعْلَى بْنِ عَطَاءٍ، عَنْ وَكِيعِ بْنِ عُدُسٍ، عَنْ عَمِّهِ أَبِي رَزِينٍ، قَالَ: قُلْتُ: يَا رَسُولَ اللهِ، أَيْنَ كَانَ رَبُّنَا عَزَّ وَجَلَّ قَبْلَ أَنْ يَخْلُقَ خَلْقَهُ؟ قَالَ: " كَانَ فِي عَمَاءٍ مَا تَحْتَهُ هَوَاءٌ، وَمَا فَوْقَهُ هَوَاءٌ، ثُمَّ خَلَقَ عَرْشَهُ عَلَى الْمَاءِ" \href{https://shamela.ws/book/25794/12650#p1}{\faExternalLink} \cite{ahmid}.\footnote{أحمد: 16188 واللفظ له، الترمذي: 3109، ابن ماجه: 182، ضعفه الألباني.}.  ولكن هذا الحديث ضعفه الشيخ الألباني وقال فيه نظر؛ لأن وكيعاً هذا مجهول \href{https://shamela.ws/book/9442/5278#p10}{\faExternalLink} \cite{albani_Sahiha}، وقال شعيب الأرناؤوط في تخريج مسند الإمام أحمد إسناده ضعيف لنفس السبب.
\subsection{أقوال من قدم العرش والماء}

\subsection{أقوال من قدم القلم واللوح}

قال شيخ الإسلام ابن تيمية في مجموع الفتاوى \href{https://shamela.ws/book/7289/9349#p1}{\faExternalLink}:

وَمِنْ هَذَا: الْحَدِيثُ الَّذِي رَوَاهُ أَبُو دَاوُد وَالتِّرْمِذِي وَغَيْرُهُمَا عَنْ عبادة بْنِ الصَّامِتِ عَنْ النَّبِيِّ صَلَّى اللَّهُ عَلَيْهِ وَسَلَّمَ أَنَّهُ قَالَ: " {أَوَّلُ مَا خَلَقَ اللَّهُ الْقَلَمَ فَقَالَ لَهُ: اُكْتُبْ قَالَ: وَمَا أَكْتُبُ. قَالَ: مَا هُوَ كَائِنٌ إلَى يَوْمِ الْقِيَامَةِ} " فَهَذَا الْقَلَمُ خَلَقَهُ لِمَا أَمَرَهُ بِالتَّقْدِيرِ الْمَكْتُوبِ قَبْلَ خَلْقِ السَّمَوَاتِ وَالْأَرْضِ بِخَمْسِينَ أَلْفَ سَنَةٍ وَكَانَ مَخْلُوقًا قَبْلَ خَلْقِ السَّمَوَاتِ وَالْأَرْضِ وَهُوَ أَوَّلُ مَا خُلِقَ مِنْ هَذَا الْعَالَمِ وَخَلَقَهُ بَعْدَ الْعَرْشِ كَمَا دَلَّتْ عَلَيْهِ النُّصُوصُ وَهُوَ قَوْلُ جُمْهُورِ السَّلَفِ كَمَا ذَكَرْتُ أَقْوَالَ السَّلَفِ فِي غَيْرِ هَذَا الْمَوْضِعِ.

وأورد شيخ الإسلام خمسة عشرة وجها، وقال في الوجه الرابع عشر: 

وَلِهَذَا قَالَ {عُمَرُ بْنُ الْخَطَّابِ رَضِيَ اللَّهُ عَنْهُ قَامَ فِينَا رَسُولُ اللَّهِ صَلَّى اللَّهُ عَلَيْهِ وَسَلَّمَ مَقَامًا فَأَخْبَرَنَا عَنْ بَدْءِ الْخَلْقِ حَتَّى دَخَلَ أَهْلُ الْجَنَّةِ مَنَازِلَهُمْ وَأَهْلُ النَّارِ مَنَازِلَهُمْ} " رَوَاهُ الْبُخَارِيُّ. فَالنَّبِيُّ صَلَّى اللَّهُ عَلَيْهِ وَسَلَّمَ أَخْبَرَهُمْ بِبَدْءِ الْخَلْقِ إلَى دُخُولِ أَهْلِ الْجَنَّةِ وَالنَّارِ مَنَازِلَهُمَا. وَقَوْلُهُ: " بَدَأَ الْخَلْقَ " مِثْلُ قَوْلِهِ فِي الْحَدِيثِ الْآخَرِ: " {قَدَّرَ اللَّهُ مَقَادِيرَ الْخَلَائِقِ قَبْلَ أَنْ يَخْلُقَ السَّمَوَاتِ وَالْأَرْضَ بِخَمْسِينَ أَلْفَ سَنَةٍ} " فَإِنَّ الْخَلَائِقَ هُنَا الْمُرَادُ بِهَا الْخَلَائِقُ الْمَعْرُوفَةُ الْمَخْلُوقَةُ بَعْدَ خَلْقِ الْعَرْشِ وَكَوْنِهِ عَلَى الْمَاءِ. وَلِهَذَا كَانَ التَّقْدِيرُ لِلْمَخْلُوقَاتِ هُوَ التَّقْدِيرُ لِخَلْقِ هَذَا الْعَالَمِ كَمَا فِي حَدِيثِ الْقَلَمِ: {إنَّ اللَّهَ لَمَّا خَلَقَهُ قَالَ: اُكْتُبْ قَالَ: وَمَاذَا أَكْتُبُ؟ قَالَ: اُكْتُبْ مَا هُوَ كَائِنٌ إلَى يَوْمِ الْقِيَامَةِ} . وَكَذَلِكَ فِي الْحَدِيثِ الصَّحِيحِ: " {إنَّ اللَّهَ قَدَّرَ مَقَادِيرَ الْخَلَائِقِ قَبْلَ أَنْ يَخْلُقَ السَّمَوَاتِ وَالْأَرْضَ بِخَمْسِينَ أَلْفَ سَنَةٍ وَكَانَ عَرْشُهُ عَلَى الْمَاءِ} " وَقَوْلُهُ فِي الْحَدِيثِ الْآخَرِ الصَّحِيحِ: " {كَانَ اللَّهُ وَلَا شَيْءَ قَبْلَهُ وَكَانَ عَرْشُهُ عَلَى الْمَاءِ وَكَتَبَ فِي الذِّكْرِ كُلَّ شَيْءٍ ثُمَّ خَلَقَ السَّمَوَاتِ وَالْأَرْضَ} " يُرَادُ بِهِ أَنَّهُ كَتَبَ كُلَّ مَا أَرَادَ خَلْقَهُ مِنْ ذَلِكَ؛ فَإِنَّ لَفْظَ كُلِّ شَيْءٍ يَعُمُّ فِي كُلِّ مَوْضِعٍ بِحَسَبِ مَا سِيقَتْ لَهُ كَمَا فِي قَوْلِهِ: {بِكُلِّ شَيْءٍ عَلِيمٌ} و {عَلَى كُلِّ شَيْءٍ قَدِيرٌ} وَقَوْلِهِ: {اللَّهُ خَالِقُ كُلِّ شَيْءٍ} وَ {تُدَمِّرُ كُلَّ شَيْءٍ} {وَأُوتِيَتْ مِنْ كُلِّ شَيْءٍ} وَ {فَتَحْنَا عَلَيْهِمْ أَبْوَابَ كُلِّ شَيْءٍ} {وَمِنْ كُلِّ شَيْءٍ خَلَقْنَا زَوْجَيْنِ} وَأَخْبَرَتْ الرُّسُلُ بِتَقَدُّمِ أَسْمَائِهِ وَصِفَاتِهِ كَمَا فِي قَوْلِهِ: {وَكَانَ اللَّهُ عَزِيزًا حَكِيمًا} . {سَمِيعًا بَصِيرًا} . {غَفُورًا رَحِيمًا} وَأَمْثَالَ ذَلِكَ.

قال شيخ الإسلام ابن تيمية في مجموع الفتاوى \href{https://shamela.ws/book/7289/2988#p1}{\faExternalLink}:

فَقَدْ أَخْبَرَ أَنَّ عَرْشَهُ كَانَ عَلَى الْمَاءِ قَبْلَ أَنْ يَخْلُقَ السَّمَوَاتِ وَالْأَرْضَ كَمَا قَالَ تَعَالَى: {وَهُوَ الَّذِي خَلَقَ السَّمَاوَاتِ وَالْأَرْضَ فِي سِتَّةِ أَيَّامٍ وَكَانَ عَرْشُهُ عَلَى الْمَاءِ}. وَقَدْ ثَبَتَ فِي صَحِيحِ الْبُخَارِيِّ وَغَيْرِهِ عَنْ عِمْرَانَ بْنِ حُصَيْنٍ عَنْ النَّبِيِّ صَلَّى اللَّهُ عَلَيْهِ وَسَلَّمَ أَنَّهُ قَالَ: {كَانَ اللَّهُ وَلَمْ يَكُنْ شَيْءٌ غَيْرُهُ وَكَانَ عَرْشُهُ عَلَى الْمَاءِ وَكَتَبَ فِي الذِّكْرِ كُلَّ شَيْءٍ وَخَلَقَ السَّمَوَاتِ وَالْأَرْضَ} وَفِي رِوَايَةٍ لَهُ {كَانَ اللَّهُ وَلَمْ يَكُنْ شَيْءٌ قَبْلَهُ وَكَانَ عَرْشُهُ عَلَى الْمَاءِ ثُمَّ خَلَقَ السَّمَوَاتِ وَالْأَرْضَ وَكَتَبَ فِي الذِّكْرِ كُلَّ شَيْءٍ} وَفِي رِوَايَةٍ لِغَيْرِهِ صَحِيحَةٍ: {كَانَ اللَّهُ وَلَمْ يَكُنْ شَيْءٌ مَعَهُ وَكَانَ عَرْشُهُ عَلَى الْمَاءِ ثُمَّ كَتَبَ فِي الذِّكْرِ كُلَّ شَيْءٍ} وَثَبَتَ فِي صَحِيحِ مُسْلِمٍ عَنْ عَبْدِ اللَّهِ بْنِ عَمْرٍو عَنْ النَّبِيِّ صَلَّى اللَّهُ عَلَيْهِ وَسَلَّمَ أَنَّهُ قَالَ: {إنَّ اللَّهَ قَدَّرَ مَقَادِيرَ الْخَلَائِقِ قَبْلَ أَنْ يَخْلُقَ السَّمَوَاتِ وَالْأَرْضَ بِخَمْسِينَ أَلْفِ سَنَةٍ وَكَانَ عَرْشُهُ عَلَى الْمَاءِ} . وَهَذَا التَّقْدِيرُ بَعْدَ وُجُودِ الْعَرْشِ وَقَبْلَ خَلْقِ السَّمَوَاتِ وَالْأَرْضِ بِخَمْسِينَ أَلْفِ سَنَةٍ. 

قال ابن كثير في البداية والنهاية \href{https://shamela.ws/book/30097/106#p7}{\faExternalLink}:

واختلف هؤلاء في أيها خُلِق أولًا؟ فقال قائلون: خلق القلم قبل هذه الأشياء كلها، وهذا هو اختيار ابن جرير، وابن الجوزي، وغيرهما. قال ابن جرير: وبعد القلم السحاب الرقيق، وبعده العرش.

واحتجوا بالحديث الذي رواه [الإمام] أحمد، وأبو داود والترمذي، عن عُبادة بن الصامت، ﵁، قال: قال رسولُ اللَّه ﷺ: "إنَّ أوَّلَ ما خَلَقَ اللَّهُ القَلَمُ. ثُمَّ قالَ لَهُ اكْتُبْ، فجرى في تلك السَّاعة بما هُوَ كائنٌ إلى يَوْمِ القيَامَةِ" لفظ أحمد. وقال الترمذي: حسن صحيح غريب.

والذي عليه الجمهور، فيما نقله الحافظ أبو العلاء الهَمَذَاني وغيره: أنَّ العرش مخلوق قبل ذلك، وهذا هو الذي رواه ابن جرير من طريق الضحاك عن ابن عباس، كما دلَّ على ذلك الحديث الذي رواه مسلم في "صحيحه" حيث قال: حدّثني أبو الطاهر أحمدُ بْن عمرو بن السَّرْح، حدَّثنا ابن وهب، أخبرني أبو هانئ الخَوْلاني، عن أبي عبد الرحمن الحُبُلي، عن عبد اللَّه بن عمرو بن العاص قال: سمعتُ رسولَ اللَّه ﷺ يقول: "كَتَبَ اللَّهُ مَقَاديْرَ الخَلائِقِ قَبْلَ أنْ يَخْلُقَ السَّموَاتِ وَالأرْضَ بخَمْسين ألْفَ سَنةٍ، قالَ: وعَرْشُهُ على المَاءِ"، قالوا: فهذا التقدير هو كتابته بالقلم المقاديرَ.

وقد دلَّ هذا الحديثُ أنَّ ذلك بعد خلق العرش، فثبتَ تقدُّم العرش على القلم الذي كتبت به المقادير كما ذهب إلى ذلك الجماهير. ويُحمل حديثُ القلم على أنَّه أوَّلُ المخلوقات من هذا العالم.

ويؤيد هذا ما رواه البخارى، عن عِمْران بن حصين: قال: قال أهلُ اليمن لرسول اللَّه ﷺ: جِئْنَاكَ لنَتَفقَّه في الدِّيْن وَلنَسْألَكَ عَنْ أوَّلِ الأمْرِ. فَقَالَ: "كانَ اللَّهُ وَلَمْ يَكُنْ شَيْءٌ قَبْلَهُ -وفي رواية: معه، وفي رواية: غيره- وكَان عَرْشُهُ عَلَى الماءِ، وكَتَبَ في الذِّكْرِ كُلَّ شَيْءٍ، وخَلَقَ السَّمواتِ والأرْضَ"، وفي لفظ: " ثمَّ خَلقَ السَّمواتِ والأرْضَ". فسألوه عن ابتداء خلق السَّمواتِ والأرض، ولهذا قالوا: جئناك نسألك عن أول هذا الأمر، فأجابهم عمَّا سألوا فقط. ولهذا لم يخبرهم بخلق العرش كما أخبر به في حديث أبي رَزين المتقدم.

قال ابن جرير: وقال آخرون: بل خلقَ اللَّهُ ﷿ الماءَ قَبْلَ العَرْشِ. رواه السُّدّي عن أبي مالك، وعن أبي صالح عن ابن عباس، وعن مُرَّة عن ابن مسعود، وعن ناس من أصحاب النبي ﷺ قالوا: إنَّ اللَّه كان عرشُه على الماء، ولم يخلق شيئًا غير ما خلق قبل الماء.

وقال أبو حجر العسقلاني في كتاب فتح البارئ بشرح صحيح البخاري \href{https://shamela.ws/book/1673/3483#p2}{\faExternalLink}: 



قَوْلُهُ: (كَانَ اللَّهُ وَلَمْ يَكُنْ شَيْءٌ غَيْرُهُ) فِي الرِّوَايَةِ الْآتِيَةِ فِي التَّوْحِيدِ: وَلَمْ يَكُنْ شَيْءٌ قَبْلَهُ، وَفِي رِوَايَةِ غَيْرِ الْبُخَارِيِّ: وَلَمْ يَكُنْ شَيْءٌ مَعَهُ، وَالْقِصَّةُ مُتَّحِدَةٌ فَاقْتَضَى ذَلِكَ أَنَّ الرِّوَايَةَ وَقَعَتْ بِالْمَعْنَى، وَلَعَلَّ رَاوِيهَا أَخَذَهَا مِنْ قَوْلِهِ ﷺ فِي دُعَائِهِ فِي صَلَاةِ اللَّيْلِ - كَمَا تَقَدَّمَ مِنْ حَدِيثِ ابْنِ عَبَّاسٍ - أَنْتَ الْأَوَّلُ فَلَيْسَ قَبْلَكَ شَيْءٌ لَكِنَّ رِوَايَةَ الْبَابِ أَصْرَحُ فِي الْعَدَمِ، وَفِيهِ دَلَالَةٌ عَلَى أَنَّهُ لَمْ يَكُنْ شَيْءٌ غَيْرُهُ لَا الْمَاءُ وَلَا الْعَرْشُ وَلَا غَيْرُهُمَا؛ لِأَنَّ كُلَّ ذَلِكَ غَيْرُ اللَّهِ تَعَالَى، وَيَكُونُ قَولهُ وَكَانَ عَرْشُهُ عَلَى الْمَاءِ مَعْنَاهُ أَنَّهُ خَلَقَ الْمَاءَ سَابِقًا ثُمَّ خَلَقَ الْعَرْشَ عَلَى الْمَاءِ، وَقَدْ وَقَعَ فِي قِصَّةِ نَافِعِ بْنِ زَيْدٍ الْحِمْيَرِيِّ بِلَفْظِ: كَانَ عَرْشُهُ عَلَى الْمَاءِ ثُمَّ خَلَقَ الْقَلَمَ، فَقَالَ: اكْتُبْ مَا هُوَ كَائِنٌ، ثُمَّ خَلَقَ السَّمَاوَاتِ وَالْأَرْضَ وَمَا فِيهِنَّ فَصَرَّحَ بِتَرْتِيبِ الْمَخْلُوقَاتِ بَعْدَ الْمَاءِ وَالْعَرْشِ.

قَوْلُهُ: (وَكَانَ عَرْشُهُ عَلَى الْمَاءِ، وَكَتَبَ فِي الذِّكْرِ كُلَّ شَيْءٍ، وَخَلَقَ السَّمَاوَاتِ وَالْأَرْضَ) هَكَذَا جَاءَتْ هَذِهِ الْأُمُورُ الثَّلَاثَةُ مَعْطُوفَةً بِالْوَاوِ، وَوَقَعَ فِي الرِّوَايَةِ الَّتِي فِي التَّوْحِيدِ: ثُمَّ خَلَقَ السَّمَاوَاتِ وَالْأَرْضَ وَلَمْ يَقَعْ بِلَفْظِ ثُمَّ إِلَّا فِي ذِكْرِ خَلْقِ السَّمَاوَاتِ وَالْأَرْضِ، وَقَدْ رَوَى مُسْلِمٌ مِنْ حَدِيثِ عَبْدِ اللَّهِ بْنِ عَمْرٍو مَرْفُوعًا: أَنَّ اللَّهَ قَدَّرَ مَقَادِيرَ الْخَلَائِقِ قَبْلَ أَنْ يَخْلُقَ السَّمَاوَاتِ وَالْأَرْضَ بِخَمْسِينَ أَلْفَ سَنَةٍ وَكَانَ عَرْشُهُ عَلَى الْمَاءِ وَهَذَا الْحَدِيثُ يُؤَيِّدُ رِوَايَةَ مَنْ رَوَى: ثُمَّ خَلَقَ السَّمَاوَاتِ وَالْأَرْضَ بِاللَّفْظِ الدَّالِّ عَلَى التَّرْتِيبِ.

قَوْلُهُ: (وَكَانَ عَرْشُهُ عَلَى الْمَاءِ) قَالَ الطِّيبِيُّ: هُوَ فَصْلٌ مُسْتَقِلٌّ؛ لِأَنَّ الْقَدِيمَ مَنْ لَمْ يَسْبِقْهُ شَيْءٌ، وَلَمْ يُعَارِضْهُ فِي الْأَوَّلِيَّةِ، لَكِنْ أَشَارَ بِقَوْلِهِ: وَكَانَ عَرْشُهُ عَلَى الْمَاءِ إِلَى أَنَّ الْمَاءَ وَالْعَرْشَ كَانَا مَبْدَأُ هَذَا الْعَالَمِ لِكَوْنِهِمَا خُلِقَا قَبْلَ خَلْقِ السَّمَاوَاتِ وَالْأَرْضِ، وَلَمْ يَكُنْ تَحْتَ الْعَرْشِ إِذْ ذَاكَ إِلَّا الْمَاءُ. وَمُحَصَّلُ الْحَدِيثِ أَنَّ مُطْلَقَ قَوْلِهِ وَكَانَ عَرْشُهُ عَلَى الْمَاءِ مُقَيَّدٌ بِقَوْلِهِ وَلَمْ يَكُنْ شَيْءٌ غَيْرُهُ، وَالْمُرَادُ بـ كَانَ فِي الْأَوَّلِ الْأَزَلِيَّةَ وَفِي الثَّانِيِ الْحُدُوثَ بَعْدَ الْعَدَمِ. وَقَدْ رَوَى أَحْمَدُ، وَالتِّرْمِذِيُّ وَصَحَّحَهُ مِنْ حَدِيثِ أَبِي رَزِينٍ الْعُقَيْلِيِّ مَرْفُوعًا: أَنَّ الْمَاءَ خُلِقَ قَبْلَ الْعَرْشِ، وَرَوَى السُّدِّيُّ فِي تَفْسِيرِهِ بِأَسَانِيدَ مُتَعَدِّدَةٍ: أَنَّ اللَّهَ لَمْ يَخْلُقْ شَيْئًا مِمَّا خَلَقَ قَبْلَ الْمَاءِ، وَأَمَّا مَا رَوَاهُ أَحْمَدُ، وَالتِّرْمِذِيُّ وَصَحَّحَهُ مِنْ حَدِيثِ عُبَادَةَ بْنِ الصَّامِتِ مَرْفُوعًا: أَوَّلُ مَا خَلَقَ اللَّهُ الْقَلَمَ، ثُمَّ قَالَ: اكْتُبْ، فَجَرَى بِمَا هُوَ كَائِنٌ إِلَى يَوْمِ الْقِيَامَةِ فَيُجْمَعُ بَيْنَهُ وَبَيْنَ مَا قَبْلَهُ بِأَنَّ أَوَّلِيَّةَ الْقَلَمِ بِالنِّسْبَةِ إِلَى مَا عَدَا الْمَاءَ وَالْعَرْشَ أَوْ بِالنِّسْبَةِ إِلَى مَا مِنْهُ صَدَرَ مِنَ الْكِتَابَةِ، أَيْ أَنَّهُ قِيلَ لَهُ اكْتُبْ أَوَّلَ مَا خُلِقَ، وَأَمَّا حَدِيثُ: أَوَّلُ مَا خَلَقَ اللَّهُ الْعَقْلُ فَلَيْسَ لَهُ طَرِيقٌ ثَبْتٌ، وَعَلَى تَقْدِيرِ ثُبُوتِهِ فَهَذَا التَّقْدِيرُ الْأَخِيرُ هُوَ تَأْوِيلُهُ وَاللَّهُ أَعْلَمُ.

وَحَكَى أَبُو الْعَلَاءِ الْهَمْدَانِيُّ أَنَّ لِلْعُلَمَاءِ قَوْلَيْنِ فِي أَيِّهِمَا خُلِقَ أَوَّلًا الْعَرْشُ أَوِ الْقَلَمُ؟ قَالَ: وَالْأَكْثَرُ عَلَى سَبْقِ خَلْقِ الْعَرْشِ، وَاخْتَارَ ابْنُ جَرِيرٍ وَمَنْ تَبِعَهُ الثَّانِي، وَرَوَى ابْنُ أَبِي حَازِمٍ مِنْ طَرِيقِ سَعِيدِ بْنِ جُبَيْرٍ، عَنِ ابْنِ عَبَّاسٍ قَالَ: خَلَقَ اللَّهُ اللَّوْحَ الْمَحْفُوظَ مَسِيرَةَ خَمْسِمِائَةِ عَامٍ، فَقَالَ لِلْقَلَمِ قَبْلَ أَنْ يَخْلُقَ الْخَلْقَ وَهُوَ عَلَى الْعَرْشِ: اكْتُبْ، فَقَالَ: وَمَا أَكْتُبُ؟ قَالَ: عِلْمِي فِي خَلْقِي إِلَى يَوْمِ الْقِيَامَةِ، ذَكَرَهُ فِي تَفْسِيرِ سُورَةِ سُبْحَانَ، وَلَيْسَ فِيهِ سَبْقُ خَلْقِ الْقَلَمِ عَلَى الْعَرْشِ، بَلْ فِيهِ سَبْقُ الْعَرْشِ. وَأَخْرَجَ الْبَيْهَقِيُّ فِي الْأَسْمَاءِ وَالصِّفَاتِ مِنْ طَرِيقِ الْأَعْمَشِ، عَنْ أَبِي ظَبْيَانَ، عَنِ ابْنِ عَبَّاسٍ قَالَ: أَوَّلُ مَا خَلَقَ اللَّهُ الْقَلَمَ، فَقَالَ لَهُ: اكْتُبْ، فَقَالَ: يَا رَبِّ وَمَا أَكْتُبُ؟ قَالَ: اكْتُبِ الْقَدَرَ، فَجَرَى بِمَا هُوَ كَائِنٌ مِنْ ذَلِكَ الْيَوْمِ إِلَى قِيَامِ السَّاعَةِ وَأَخْرَجَ سَعِيدُ بْنُ مَنْصُورٍ، عَنْ أَبِي عَوَانَةَ، عَنْ أَبِي بِشْرٍ عَنْ مُجَاهِدٍ قَالَ: بَدْءُ الْخَلْقِ الْعَرْشُ وَالْمَاءُ وَالْهَوَاءُ، وَخُلِقَتِ الْأَرْضُ مِنَ الْمَاءِ وَالْجَمْعُ بَيْنَ هَذِهِ الْآثَارِ وَاضِحٌ.

قَوْلُهُ: (وَكَتَبَ) أَيْ قَدَّرَ (فِي الذِّكْرِ) أَيْ فِي مَحَلِّ الذِّكْرِ، أَيْ فِي اللَّوْحِ الْمَحْفُوظِ (كُلَّ شَيْءٍ) أَيْ مِنَ الْكَائِنَاتِ، وَفِي الْحَدِيثِ جَوَازُ السُّؤَالِ عَنْ مَبْدَإِ الْأَشْيَاءِ وَالْبَحْثُ عَنْ ذَلِكَ وَجَوَازُ جَوَابِ الْعَالِمِ بِمَا يَسْتَحْضِرُهُ مِنْ ذَلِكَ، وَعَلَيْهِ الْكَفُّ إِنْ خَشِيَ عَلَى السَّائِلِ مَا يَدْخُلُ عَلَى مُعْتَقِدِهِ. وَفِيهِ أَنَّ جِنْسَ الزَّمَانِ وَنَوْعَهُ حَادِثٌ، وَأَنَّ اللَّهَ أَوْجَدَ هَذِهِ الْمَخْلُوقَاتِ بَعْدَ أَنْ لَمْ تَكُنْ، لَا عَنْ عَجْزٍ عَنْ ذَلِكَ بَلْ مَعَ الْقُدْرَةِ. وَاسْتَنْبَطَ بَعْضُهُمْ مِنْ سُؤَالِ الْأَشْعَرِيِّينَ عَنْ هَذِهِ الْقِصَّةِ أَنَّ الْكَلَامَ فِي أُصُولِ الدِّينِ وَحُدُوثِ الْعِلْمِ مُسْتَمِرَّانِ فِي ذُرِّيَّتِهِمْ حَتَّى ظَهَرَ ذَلِكَ مِنْهُمْ فِي أَبِي الْحَسَنِ الْأَشْعَرِيِّ، أَشَارَ إِلَى ذَلِكَ ابْنُ عَسَاكِرَ.


\subsection{خلاصة الأقوال والراجح منها}

\section{مسألة يدين الله}

يَطْوِي اللَّهُ عزَّ وجلَّ السَّمَواتِ يَومَ القِيامَةِ، ثُمَّ يَأْخُذُهُنَّ بيَدِهِ اليُمْنَى، ثُمَّ يقولُ: أنا المَلِكُ، أيْنَ الجَبَّارُونَ؟ أيْنَ المُتَكَبِّرُونَ. ثُمَّ يَطْوِي الأرَضِينَ بشِمالِهِ، ثُمَّ يقولُ: أنا المَلِكُ أيْنَ الجَبَّارُونَ؟ أيْنَ المُتَكَبِّرُونَ؟
صحيح مسلم

قال الشيخ ابن باز رحمه الله مجيبا على: ما معنى حديث "وكلتا يدي الرحمن يمين"؟

الحديث ثابتٌ، ورواه مسلم، ومسلم رحمه الله توخَّى الأحاديث الصَّحيحة، وإذا كان جرح عمر بن حمزة بعض الناس فمُسلم لم يجرحه، وروى عنه، ووثَّقه ابنُ حبان، وصحح له الحاكم.
فالمقصود أن الحديث لا بأس به، وهي شمال في الاسم، وأما في الفضل فهي يمين، ولهذا في الحديث الصحيح: كلتا يدي ربي يمين مباركة، فكلاهما يمين مباركة في الشرف والفضل، وتُسمَّى إحداهما: يمينًا، كما قال تعالى: وَالسَّمَاوَاتُ مَطْوِيَّاتٌ بِيَمِينِهِ [الزمر:67]، وتُسمَّى الأخرى: شمالًا، وهي يمينٌ في الفضل والبركة والشرف، وإن سُمِّيَتْ شمالًا، لكنها في الفضل والشرف لها ما لليمين باليُمن والخير والبركة والشرف، ولا منافاة، فالحديث كلتا يدي ربي يمين مباركة، يُبين فضلها وشرفها، وأنه لا نقصَ فيها، والتَّسمية بتسميتها شمالًا لا يدل على النقص، بل إنما هي مجرد أسماء فقط، كما أن تسمية يده: يد، وتسميته قدمه: قدم، وعين، وسمع، وبصر، كل هذا لا يتضمن المشابهة والتَّمثيل، فكلها صفات تليق بالله، وكلها كاملة، ليس فيها نقصٌ، تليق بالله جلَّ وعلا، لا يُماثل فيها خلقه .

\section{مسألة أثقل المخلوقات}

قال شيخ الإسلام ابن تيمية في مجموع الفتاوى \href{https://shamela.ws/book/7289/2991#p1}{\faExternalLink}:

وَقَدْ ثَبَتَ فِي صَحِيحِ مُسْلِمٍ {عَنْ جُوَيْرِيَّةَ بِنْتِ الْحَارِثِ: أَنَّ النَّبِيَّ صَلَّى اللَّهُ عَلَيْهِ وَسَلَّمَ دَخَلَ عَلَيْهَا وَكَانَتْ تُسَبِّحُ بِالْحَصَى مِنْ صَلَاةِ الصُّبْحِ إلَى وَقْتِ الضُّحَى فَقَالَ: لَقَدْ قُلْت بَعْدَك أَرْبَعَ كَلِمَاتٍ لَوْ وُزِنَتْ بِمَا قلتيه لَوَزَنَتْهُنَّ: سُبْحَانَ اللَّهِ عَدَدَ خَلْقِهِ سُبْحَانَ اللَّهِ زِنَةَ عَرْشِهِ سُبْحَانَ اللَّهِ رِضَى نَفْسِهِ سُبْحَانَ اللَّهِ مِدَادَ كَلِمَاتِهِ} . فَهَذَا يُبَيِّنُ أَنَّ زِنَةَ الْعَرْشِ أَثْقَلُ الْأَوْزَانِ.

\subsection{مسألة تفاوت الزمان}


وفي تفاوت الزمان، يقول شيخ الإسلام ابن تيمية:

وَالرُّسُلُ أَخْبَرَتْ بِخَلْقِ الْأَفْلَاكِ وَخَلْقِ الزَّمَانِ الَّذِي هُوَ مِقْدَارُ حَرَكَتِهَا (أي حركة الأفلاك) مَعَ إخْبَارِهَا بِأَنَّهَا خُلِقَتْ مِنْ مَادَّةٍ قَبْلَ ذَلِكَ وَفِي زَمَانٍ قَبْلَ هَذَا الزَّمَانِ؛ فَإِنَّهُ سُبْحَانَهُ أَخْبَرَ أَنَّهُ خَلَقَ السَّمَوَاتِ وَالْأَرْضَ فِي سِتَّةِ أَيَّامٍ وَسَوَاءٍ قِيلَ: أَنَّ تِلْكَ الْأَيَّامَ بِمِقْدَارِ هَذِهِ الْأَيَّامِ الْمُقَدَّرَةِ بِطُلُوعِ الشَّمْسِ وَغُرُوبِهَا؛ أَوْ قِيلَ: إنَّهَا أَكْبَرُ مِنْهَا كَمَا قَالَ بَعْضُهُمْ: إنَّ كُلَّ يَوْمٍ قَدْرُهُ أَلْفُ سَنَةٍ فَلَا رَيْبَ أَنَّ تِلْكَ الْأَيَّامَ الَّتِي خُلِقَتْ فِيهَا السَّمَوَاتُ وَالْأَرْضُ غَيْرُ هَذِهِ الْأَيَّامِ وَغَيْرُ الزَّمَانِ الَّذِي هُوَ مِقْدَارُ حَرَكَةِ هَذِهِ الْأَفْلَاكِ. وَتِلْكَ الْأَيَّامُ مُقَدَّرَةٌ بِحَرَكَةِ أَجْسَامٍ مَوْجُودَةٍ قَبْلَ خَلْقِ السَّمَوَاتِ وَالْأَرْضِ. وَقَدْ أَخْبَرَ سُبْحَانَهُ أَنَّهُ {اسْتَوَى إلَى السَّمَاءِ وَهِيَ دُخَانٌ فَقَالَ لَهَا وَلِلْأَرْضِ ائْتِيَا طَوْعًا أَوْ كَرْهًا قَالَتَا أَتَيْنَا طَائِعِينَ} فَخُلِقَتْ مِنْ الدُّخَانِ وَقَدْ جَاءَتْ الْآثَارُ عَنْ السَّلَفِ إنَّهَا خُلِقَتْ مِنْ بُخَارِ الْمَاءِ؛ وَهُوَ الْمَاءُ الَّذِي كَانَ الْعَرْشُ عَلَيْهِ الْمَذْكُورُ فِي قَوْلِهِ: {وَهُوَ الَّذِي خَلَقَ السَّمَاوَاتِ وَالْأَرْضَ فِي سِتَّةِ أَيَّامٍ وَكَانَ عَرْشُهُ عَلَى الْمَاءِ}.

\section{مسألة العدل مع الكفار}
\label{sec:app_justice}

الدولة الكافرة العادلة لها وعليها، فيذم كفرها ويحمد عدلها، ولا يرد عليها كل أمرها، بل يحمد ما فيها من العدل والإنصاف والمحاسن الإنسانية الموافقة للفطرة، ويذم ما فيها من كفر وفسق وعدوان على دين الله ورسله وهذا ما أوصانا به جل جلاله في كتابه العظيم فقال: 
\quranayah*[5][8]{\footnotesize \surahname*[5]}. وقال القرطبي في تفسيره: ودلت الآية أيضا على أن كفر الكافر لا يمنع من العدل عليه. وقال ابن كثير في تفسيره: وقوله: (ولا يجرمنكم شنآن قوم على ألا تعدلوا) أي: لا يحملنكم بغض قوم على ترك العدل فيهم، بل استعملوا العدل في كل أحد، صديقا كان أو عدوا [هـ]. 


وقول شهادة الحق في الدولة الكافرة لا يعني موالاتها وإن كانت عادلة، بل هذا ما هوا إلا شهادة الحق وقد تقدم بيان ذم ما فيها من كفر وفسق وعصيان لدين الله ورسله. وهذا لأن الله جل جلاله أمرنا بالعدل في القول ولو على أنفسنا فقال جل في علاه:
\quranayah*[4][135]{\footnotesize \surahname*[4]}. وقد جاء في تفسير ابن كثير: وقوله (فلا تتبعوا الهوى أن تعدلوا) أي : فلا يحملنكم الهوى والعصبية وبغضة الناس إليكم، على ترك العدل في أموركم وشؤونكم، بل الزموا العدل على أي حال كان، كما قال تعالى : (ولا يجرمنكم شنآن قوم على ألا تعدلوا اعدلوا هو أقرب للتقوى) [المائدة: 8] [هـ]. ومن ذلك ما صح عن جابِرُ بنُ عبدِ اللهِ رضِيَ اللهُ عنهما أنه قال: أفاءَ اللهُ عزَّ وجلَّ خَيبرَ على رسولِ اللهِ صلَّى اللهُ عليه وسَلَّم، فأقَرَّهُم رسولُ اللهِ صلَّى اللهُ عليه وسَلَّم كما كانوا، وجَعَلَها بَينَه وبَينَهُم، فبَعَثَ عَبدَ اللهِ بنَ رَواحةَ، فخَرَصَها عليهم، ثُمَّ قال لهم: يا مَعشَرَ اليَهودِ، أنتُم أبغَضُ الخَلْقِ إليَّ، قتَلتُم أنبياءَ اللهِ عزَّ وجلَّ، وكَذَبتُم على اللهِ، وليس يَحمِلُني بُغْضي إيَّاكم على أنْ أَحيفَ عليكم، قد خَرَصتُ عِشرينَ ألْفَ وَسْقٍ مِن تَمرٍ، فإنْ شِئتُم فلكُم، وإنْ أبَيتُم فلي، فقالوا: بهذا قامَتِ السَّمَواتُ والأرضُ، قد أخَذْنا، فاخْرُجوا عنَّا.{\footnotesize (صحيح على شرط مسلم، تخريج المسند لشعيب، تخريج سنن الدارقطني)}. وهذا فيه أن اليهود عرفوا أنه بالعدل قامت السموات والأرض وهذا ما سبق بيانه في الميزان الكوني، وأن عبد الله بن رواحة رضي الله عنه أقام فيهم الميزان الشرعي وأقر لهم بذلك بعدله معهم. 

وقد جاء في تفسير الطبري عن ابن عباس قوله: "كونوا قوامين بالقسط شهداء لله ولو على أنفسكم أو الوالدين والأقربين"، قال: أمر الله المؤمنين أن يقولوا الحقَّ ولو على أنفسهم أو آبائهم أو أبنائهم، ولا يحابوا غنيًّا لغناه، ولا يرحموا مسكينًا لمسكنته، وذلك قوله: "إن يكن غنيًّا أو فقيرًا فالله أولى بهما فلا تتبعوا الهوى أن تعدلوا "، فتذروا الحق، فتجوروا [هـ]. وأيضا جاء في تفسير الطبري: حدثنا سعيد، عن قتادة: "يا أيها الذين آمنوا كونوا قوامين بالقسط شهداء لله" الآية، هذا في الشهادة. فأقم الشهادة، يا ابن آدم، ولو على نفسك، أو الوالدين، أو على ذوي قرابتك، أو شَرَفِ قومك. فإنما الشهادة لله وليست للناس، وإن الله رضي العدل لنفسه، والإقساط والعدل ميزانُ الله في الأرض، به يردُّ الله من الشديد على الضعيف، ومن الكاذب على الصادق، ومن المبطل على المحق. وبالعدل يصدِّق الصادقَ، ويكذِّب الكاذبَ، ويردُّ المعتدي ويُرَنِّخُه، تعالى ربنا وتبارك. وبالعدل يصلح الناس، يا ابن آدم "إن يكن غنيًّا أو فقيرًا فالله أولى بهما"، يقول: أولى بغنيكم وفقيركم. قال: وذكر لنا أن نبيَّ الله موسى عليه السلام قال: "يا ربِّ، أي شيء وضعت في الأرض أقلّ؟"، قال: " العدلُ أقلُّ ما وضعت في الأرض". فلا يمنعك غِنى غنيّ ولا فقر فقير أن تشهد عليه بما تعلم، فإن ذلك عليك من الحق، وقال جل ثناؤه: " فالله أولى بهما " [هـ].

\section{مسألة الخروج على ولي أمر المسلمين}
\label{sec:app_rebellion}

إن من المسائل المهمة لأمة الإسلام بالعموم هي مسئلة الخروج على ولي أمر المسلمين. فهذه مسألة خطيرة وعظيمة يجب ألا يتكلم فيها إلا بعلم. 
وقد نهى النبي ﷺ عن الخروج على الدولة المسلمة الظالمة وبالأخص لما يترتب على ذلك من ظلم الذي يخالف الميزان الفطري والذي به يكون فساد المصالح العامة في الدنيا كسفك الدماء ونهب الأموال وهتك الأعراض، والتي هي أشد ظلما في الدنيا من الظلم الذي يكون بمخالفة الميزان الديني كمنع الزكاة أو الحكم بغير ما أنزل الله من باب الهوى. وقد تقدم معنا أن الله جل جلاله قدم في الدنيا إقامة الميزان بين الناس بالعدل على إقامة الحق في نفوس الناس لتقديم المصلحة العامة على الخاصة. ولذلك فقد نهى النبي ﷺ عن الخروج على ولاة الأمر المسلمين ولو كانوا ظالمين وعاصين لله ولرسوله  ما أقاموا فينا الصلاة وكفى بنا أن نبغضهم في الله على ما عصوا به الله ورسوله كما جاء عن عوف بن مالك الأشجعي أن النبي ﷺ قال: خِيارُ أئِمَّتِكُمُ الَّذِينَ تُحِبُّونَهُمْ ويُحِبُّونَكُمْ، ويُصَلُّونَ علَيْكُم وتُصَلُّونَ عليهم، وشِرارُ أئِمَّتِكُمُ الَّذِينَ تُبْغِضُونَهُمْ ويُبْغِضُونَكُمْ، وتَلْعَنُونَهُمْ ويَلْعَنُونَكُمْ، قيلَ: يا رَسولَ اللهِ، أفَلا نُنابِذُهُمْ بالسَّيْفِ؟ فقالَ: لا، ما أقامُوا فِيكُمُ الصَّلاةَ، وإذا رَأَيْتُمْ مِن وُلاتِكُمْ شيئًا تَكْرَهُونَهُ، فاكْرَهُوا عَمَلَهُ، ولا تَنْزِعُوا يَدًا مِن طاعَةٍ ، وفي رواية أخرى، قالَ: لا، ما أقامُوا فِيكُمُ الصَّلاةَ، لا، ما أقامُوا فِيكُمُ الصَّلاةَ، ألا مَن ولِيَ عليه والٍ، فَرَآهُ يَأْتي شيئًا مِن مَعْصِيَةِ اللهِ، فَلْيَكْرَهْ ما يَأْتي مِن مَعْصِيَةِ اللهِ، ولا يَنْزِعَنَّ يَدًا مِن طاعَةٍ {\footnotesize (صحيح مسلم، وصححه الألباني في تخريج كتاب السنة)}. 


وعن حذيفة بن اليمان رضي الله أن النبي ﷺ قال: يكونُ بَعْدِي أَئِمَّةٌ لا يَهْتَدُونَ بهُدَايَ، وَلَا يَسْتَنُّونَ بسُنَّتِي، وَسَيَقُومُ فيهم رِجَالٌ قُلُوبُهُمْ قُلُوبُ الشَّيَاطِينِ في جُثْمَانِ إنْسٍ، قُلتُ: كيفَ أَصْنَعُ يا رَسولَ اللهِ، إنْ أَدْرَكْتُ ذلكَ؟ قالَ: تَسْمَعُ وَتُطِيعُ لِلأَمِيرِ، وإنْ ضُرِبَ ظَهْرُكَ، وَأُخِذَ مَالُكَ، فَاسْمَعْ وَأَطِعْ {\footnotesize (صحيح مسلم)}. وقد أوصى بذلك النبي ﷺ في حجة الوداع فعن أم الحصين الأحمسية أنها قالت: سمعتُ رسولَ اللَّهِ صلَّى اللَّهُ عليْهِ وسلَّمَ يخطبُ في حجَّةِ الوداعِ  \comment{، وعليْهِ بردٌ قدِ التفعَ بِهِ من تحتِ إبطِهِ قالت فأنا أنظرُ إلى عضلةِ عضدِهِ تزتَجُّ، سمعتُهُ} يقولُ: يا أيُّها النَّاسُ اتَّقوا اللَّهَ وإن أمِّرَ عليْكم عبدٌ حبشيٌّ مجدَّعٌ فاسمعوا لَهُ وأطيعوا ما أقامَ لَكم كتابَ اللَّهِ  {\footnotesize (صحيح الترمذي، وصححه الألباني)}. وأيضا حديث العرباض بن سارية رضي الله عنه أنه قال: وعظَنا رسولُ اللَّهِ صلَّى اللَّهُ عليْهِ وسلَّمَ يومًا بعدَ صلاةِ الغداةِ موعِظةً بليغةً ذرِفَت منْها العيونُ ووجِلَت منْها القلوبُ، فقالَ رجلٌ إنَّ هذِهِ موعظةُ مودِّعٍ فماذا تعْهدُ إلينا يا رسولَ اللَّهِ، قالَ أوصيكم بتقوى اللَّهِ والسَّمعِ والطَّاعةِ وإن عبدٌ حبشيٌّ فإنَّهُ من يعِش منْكم يرَ اختلافًا كثيرًا وإيَّاكم ومحدَثاتِ الأمورِ فإنَّها ضَلالةٌ فمن أدرَكَ ذلِكَ منْكم فعليْهِ بِسُنَّتي وسنَّةِ الخلفاءِ الرَّاشدينَ المَهديِّينَ عضُّوا عليْها بالنَّواجذِ {\footnotesize (صحيح الترمذي، وصححه الألباني)}.

فالأدلة في نهي الرسول ﷺ على الخروج على الولاة العصاة كثيرة جدا، فلا يسعنا الخروج على ولاة الأمر الظالمين والعاصين لله ورسوله ليس مجاملة أو حبا لهم ولا مداهنة في دين الله وإنما إلتزاما بأمر النبي ﷺ حقنا للدماء وتقديما للمصلحة العامة على الخاصة، ولكن نبغضهم في الله على ما عصوا به الله ورسوله ولا نصدقهم ولا نعينهم على ظلمهم كما صح ذلك عن النبي ﷺ أنه قال لكَعبِ بنِ عُجْرةَ: أعاذَكَ اللهُ من إمارةِ السُّفهاءِ، قال: وما إمارةُ السُّفهاءِ؟ قال: أُمراءُ يكونونَ بَعْدي، لا يَقتَدونَ بهَدْيي، ولا يَستَنُّونَ بسُنَّتي، فمَن صدَّقَهم بكذِبِهم، وأعانَهم على ظُلْمِهم، فأولئك ليسوا منِّي، ولستُ منهم، ولا يَرِدوا عليَّ حَوْضي، ومَن لم يُصدِّقْهم بكذِبِهم، ولم يُعِنْهم على ظُلْمِهم، فأولئك منِّي وأنا منهم، وسيَرِدوا عليَّ حَوْضي {\footnotesize (صحيح ابن حبان)}. ويكفي ولي الأمر المسلم الظالم ذلا وخسرانا أن أن النبي ﷺ قد تبرأ منه كما جاء في حديث سعد بن تميم أنه قيل: يا رسولَ اللهِ، ما للخليفةِ مِن بعدِك؟ قال: مِثلُ الذي لي، ما عدَلَ في الحُكمِ، وقسَطَ في القِسطِ، ورَحِمَ ذا الرَّحِمِ، فمَن فعَلَ غيرَ ذلك فليس منِّي ولستُ منه {\footnotesize (صحيح، تخريج سنن أبي داود، وصححه الألباني)}.


\comment{
فلا يسعنا الخروج على ولاة الأمر وإن ظلموا ولكن لا نصدقهم ولا نعينهم على ظلمهم كما أخبر بذلك جابر بن عبد الله أن النبي ﷺ قال لكَعبِ بنِ عُجْرةَ: أعاذَكَ اللهُ من إمارةِ السُّفهاءِ، قال: وما إمارةُ السُّفهاءِ؟ قال: أُمراءُ يكونونَ بَعْدي، لا يَقتَدونَ بهَدْيي، ولا يَستَنُّونَ بسُنَّتي، فمَن صدَّقَهم بكذِبِهم، وأعانَهم على ظُلْمِهم، فأولئك ليسوا منِّي، ولستُ منهم، ولا يَرِدوا عليَّ حَوْضي، ومَن لم يُصدِّقْهم بكذِبِهم، ولم يُعِنْهم على ظُلْمِهم، فأولئك منِّي وأنا منهم، وسيَرِدوا عليَّ حَوْضي، يا كَعبُ بنَ عُجْرةَ، الصومُ جُنَّةٌ، والصَّدقةُ تُطفِئُ الخَطيئةُ، والصَّلاةُ قُربانٌ -أو قال: بُرهانٌ- يا كَعبَ بنَ عُجْرةَ، إنَّه لا يدخُلُ الجَنَّةَ لَحمٌ نبَتَ من سُحتٍ، النَّارُ أَوْلى به، يا كَعبُ بنَ عُجْرةَ، النَّاسُ غَاديانِ: فمُبتاعٌ نفْسَه فمُعتِقُها، وبائعٌ نفْسَه فموبِقُها {\footnotesize (صحيح ابن حبان)}. \comment{{\footnotesize (صحيح ابن حبان، حسنه الوادعي في الصحيح المسند، والمنذري في الترغيب والترهيب، وقال شعيب الأرناؤوط إسناده قوي على شرط مسلم ف تخريج المسند لشعيب، وقال غيرهم رجاله رجال الصحيح)}. }
}
ويفرق بين النصح لولي الأمر الظالم وبين إنكار المنكر بالعموم، فإنكار المنكر بالعموم واجب على كل مسلم، وبالأخص رد الظالمين لمن استطاع أن يغير ويصلح بدون أن يترتب على ذلك مفسدة أعظم، فقد جاء عن أبوبكر الصديق أنه قال بعد أن حمِد اللهَ وأثنَى عليه: يا أيُّها النَّاسُ، إنَّكم تقرءون هذه الآيةَ، وتضعونها على غيرِ موضعِها (عَلَيْكُمْ أَنْفُسَكُمْ لَا يَضُرُّكُمْ مَنْ ضَلَّ إِذَا اهْتَدَيْتُمْ)، وإنَّا سمِعنا النَّبيَّ صلَّى اللهُ عليه وسلَّم يقولُ: إنَّ النَّاسَ إذا رأَوُا الظَّالمَ فلم يأخُذوا على يدَيْه أوشك أن يعُمَّهم اللهُ بعقابٍ وإنِّي سمِعتُ رسولَ اللهِ صلَّى اللهُ عليه وسلَّم يقولُ: ما من قومٍ يُعمَلُ فيهم بالمعاصي، ثمَّ يقدِرون على أن يُغيِّروا، ثمَّ لا يُغيِّروا إلَّا يوشِكُ أن يعُمَّهم اللهُ منه بعقابٍ {\footnotesize (صحيح أبي داود، وصححه الألباني)}. ولقد بايع النبي ﷺ أصحابه على السمع والطاعة والنصح لكل مسلم كما جاء ذلك عن جرير بن عبدالله أنه قال: بايَعْتُ النبيَّ صَلَّى اللَّهُ عليه وسلَّمَ علَى السَّمْعِ والطَّاعَةِ، فَلَقَّنَنِي، فِيما اسْتَطَعْتُ، والنُّصْحِ لِكُلِّ مُسْلِمٍ {\footnotesize (صحيح مسلم)}. فالنصح يكون لكل مسلم سواء كان ولي الأمر وغير ولي الأمر ويكون بحسب الحاجة وبالحكمة والموعظة الحسنة. ومن المصلحة في أغلب الأحوال أن تكون النصيحة لولي الأمر بالسر لما قد يترتب على الجهر بها من الفتن أو التحريض. ولهذا فقد قال النبي ﷺ: مَن أرادَ أن ينصحَ لذي سلطانٍ في أمرٍ فلا يُبدِهِ عَلانيةً ولَكِن ليأخذْ بيدِهِ فيَخلوَ بهِ فإن قبِلَ منهُ فذاكَ وإلَّا كانَ قد أدَّى الَّذي علَيهِ لَهُ {\footnotesize (صححه الألباني في تخريج كتاب السنة)}. فلو كانت هذه النصيحة لسلطان ظالم فهذا من أفضل الجهاد كما جاء ذلك عن أبو سعيد الخدري أن النبي ﷺ قال: أفضَلُ الجِهادِ كلمةُ عدلٍ، وفي رواية: كلمة حق، عندَ سُلطانٍ جائرٍ {\footnotesize (صحيح ابن ماجه، وصححه الألباني)}.

وأما الخروج على ولاة الأمور فشرطه أن يكون عندهم كفرا بواحا ظاهرا لا شك فيه. فقد جاء عن عبادة بن الصامت أنه قال: دَعَانَا رَسولُ اللهِ صَلَّى اللَّهُ عليه وسلَّمَ فَبَايَعْنَاهُ، فَكانَ فِيما أَخَذَ عَلَيْنَا: أَنْ بَايَعَنَا علَى السَّمْعِ وَالطَّاعَةِ في مَنْشَطِنَا وَمَكْرَهِنَا، وَعُسْرِنَا وَيُسْرِنَا، وَأَثَرَةٍ عَلَيْنَا، وَأَنْ لا نُنَازِعَ الأمْرَ أَهْلَهُ، قالَ: إلَّا أَنْ تَرَوْا كُفْرًا بَوَاحًا عِنْدَكُمْ مِنَ اللهِ فيه بُرْهَانٌ {\footnotesize (صحيح مسلم)}. ومن المعلوم أنه ليس كل حكم بغير ما أنزل الله كفر ومن ذلك بلا شك القوانين الوضعية التي لا تعارض كتاب الله وسنة نبيه ﷺ. ومثال ذلك الأمور التي فيها مصالح الناس كالأمور التنظيمة المباحة فهذا أمر مطلوب ولازم وبه يؤجر ولي الأمر لما في ذلك من نفع عام لجميع المسلمين، كتنظيم طرق سير السيارات، وقوانين حماية البيانات، وغيرها من القوانين التي بها تحفظ الدماء، والأموال، والأعراض. والكفر البواح لا يكون بالحكم بغير ما أنزل الله مع الإقرار بالذنب دون الإعتقاد بجواز ذلك والجهر به كمن يفعل ذلك من باب الهوى. وإنما الكفر البواح هو الإعتقاد مع الجهر أن الحكم المخالف لشرع الله وكتابه هو حكم جائز على وجه التفضيل أو المساواة أو الرد أو غير ذلك. ومن ذلك من يعتقد بأفضلية حكم غير الله على حكم الله أو مساواة حكم غير الله مع حكم الله أو جواز حكم غير الله أو رد حكم الله، المخالف لشرع الله وكتابه والجهر بذلك. ولقد بين ذلك الشيخ ابن باز رحمه الله في بيان القوانين الوضعية والآراء البشرية \textbf{التي تخالف شرع الله} فقال: الحكم بغير ما أنزل الله [بالقوانين التي تخالف شرع الله] أقسام، تختلف أحكامهم بحسب اعتقادهم وأعمالهم، فمن حكم بغير ما أنزل الله يرى أن ذلك أحسن من شرع الله فهو كافر عند جميع المسلمين، وهكذا من يحكّم القوانين الوضعية بدلا من شرع الله ويرى أن ذلك جائز، ولو قال: إن تحكيم الشريعة أفضل فهو كافر لكونه استحل ما حرم الله. أما من حكم بغير ما أنزل الله اتباعا للهوى أو لرشوة أو لعداوة بينه وبين المحكوم عليه أو لأسباب أخرى وهو يعلم أنه عاص لله بذلك وأن الواجب عليه تحكيم شرع الله [وإنما خالفها فعلا لا عقيدة لهوى] فهذا يعتبر من أهل المعاصي والكبائر ويعتبر قد أتى كفرا أصغر وظلما أصغر وفسقا أصغر كما جاء هذا المعنى عن ابن عباس رضي الله عنهما وعن طاووس وجماعة من السلف الصالح وهو المعروف عند أهل العلم. والله ولي التوفيق {\footnotesize (مجموع فتاوى ومقالات الشيخ ابن باز: 4/416)}. 

\comment{
ومن ذلك ما أخبر به حذيفة بن اليمان رضي الله أن النبي ﷺ قال:
كان الناسُ يسألون رسولَ اللهِ صلَّى اللهُ عليهِ وسلَّم عن الخيرِ، وكنتُ أسألُه عن الشرِّ، مخافةَ أن يُدركني، فقلتُ يارسولَ اللهِ، إنَّا كنا في جاهليةٍ وشرٍّ، فجاءنا اللهُ بهذا الخيرِ [فنحنُ فيه]، [وجاء بك]، فهل بعد هذا الخيرِ من شرٍّ [كما كان قبلَه؟]، [قال ياحذيفةُ تعلَّمْ كتابَ اللهِ، واتَّبِعْ ما فيه، (ثلاثَ مراتٍ). قال: قلتُ: يارسولَ اللهِ أبعد هذا الشرِّ من خيرٍ؟] قال نعم، [قلتُ: فما العِصمةُ منه؟ قال: السيفُ]، قلتُ وهل بعد ذلك الشرِّ من خيرٍ؟(وفي طريقٍ: قلتُ: وهل بعدَ السيفِ بقيَّةٌ؟) قال: نعم، وفيه (وفي طريقٍ: تكونُ إمارةٌ (وفي لفظٍ: جماعةٌ) على أقذاءٍ، وهُدنةٍ على) دَخَنٍ، قال: قلتُ: وما دَخَنُه؟ قال: قومٌ، وفي طريقٍ أخرى: يكونُ بعدي أئمةٌ [يستنُّونَ بغيرِ سُنَّتِي، و] يَهدُون بغيرِ هديِي، تعرفُ منهم وتُنكرُ، [وسيقومُ فيهم رجالٌ قلوبُهم قلوبُ الشياطينِ، في جثمانِ إنسٍ] (وفي أخرى: الهُدنةُ على دَخَنٍ ما هيَ؟قال: لا ترجعُ قلوبُ أقوامٍ على الذي كانت عليه) فقلتُ: هل بعد ذلك الخيرِ من شرٍّ؟قال: نعم، [فتنةٌ عمياءُ صمَّاءُ عليها] دعاةٌ على أبوابِ جهنمَ، من أجابَهُم إليها قذفوهُ فيها فقلتُ: يا رسولَ اللهِ، صِفْهُمْ لنا؟قال: هم من جِلدَتِنا، و يتكلمون بألسِنَتِنا، قلتُ: [يا رسولَ اللهِ]، فما تأمُرني إذا أدركني ذلك؟قال: تلزمُ جماعةَ المسلمين، وإمامَهم [تسمعُ وتُطيعُ الأميرَ، وإن ضرب ظهرَك، وأخذ مالَك، فاسمع وأطِعْ] فقلتُ: فإن لم يكن لهم جماعةٌ ولا إمامٌ؟ قال: فاعتزل تلك الفِرقَ كلَّها، ولو أن تعضَّ على أصلِ شجرةٍ، حتى يُدرِكَك الموتُ وأنت على ذلك. (وفي طريقٍ) فإن تَمُتْ يا حذيفةُ وأنت عاضٌّ على جذلٍ خيرٌ لك من أن تتبعَ أحدًا منهم. (وفي أخرى) فإن رأيتَ يومئذٍ للهِ عزَّ وجلَّ في الأرضِ خليفةً، فالزَمْهُ وإن ضرب ظهرَك وأخذ مالَك، فإن لم ترَ خليفةً فاهرب [في الأرضِ] حتى يُدرِكَك الموتُ وأنت عاضٌّ على جذلِ شجرةٍ. قال: قلتُ: ثم ماذا؟ قال: ثم يخرجُ الدجالُ. قال: قلتُ: فبم يجيءُ؟ قال: بنهرٍ أو قال: ماءٍ ونارٍ فمن دخل نهرَه حُطَّ أجرُه ووجب وِزْرُه، ومن دخل نارَه وجب أجرُه، وحُطَّ وِزْرُه . [قلتُ: يا رسولَ اللهِ: فما بعد الدجالِ؟ قال: عيسى بنُ مريمَ] قال: قلتُ: ثم ماذا؟ قال: لو أنتجتَ فرسًا لم تركب فَلُوَّها حتى تقومَ الساعةُ. {\footnotesize (جمعه الألباني من عدة طرق في صحيح البخاري وأورده في السلسلة الصحيحة)}.
}


\section{مسألة التفرق في الدين }
\label{sec:app_division}




\section{مسألة تجريح الأعيان}
\label{sec:app_division}

يقول شيخ الإسلام بن تيمية: وليعلم أن المؤمن تجب موالاته، وإن ظلمك واعتدى عليك، والكافر تجب معاداته، وإن أعطاك وأحسن إليك؛ فإن الله سبحانه بعث الرسل وأنزل الكتب؛ ليكون الدين كله لله، فيكون الحب لأوليائه، والبغض لأعدائه، والإكرام لأوليائه، والإهانة لأعدائه، والثواب لأوليائه، والعقاب لأعدائه. وإذا اجتمع في الرجل الواحد خير وشر وفجور، وطاعة ومعصية، وسنة وبدعة: استحق من الموالاة والثواب بقدر ما فيه من الخير، واستحق من المعاداة والعقاب بحسب ما فيه من الشر، فيجتمع في الشخص الواحد موجبات الإكرام والإهانة، فيجتمع له من هذا وهذا، كاللص الفقير تقطع يده لسرقته، ويعطى من بيت المال ما يكفيه لحاجته. هذا هو الأصل الذي اتفق عليه أهل السنة والجماعة، وخالفهم الخوارج، والمعتزلة، ومن وافقهم عليه، فلم يجعلوا الناس لا مستحقًّا للثواب فقط، ولا مستحقًّا للعقاب فقط. 

وقال أيضًا: معلوم أنه في كل طائفة بر، وفاجر، وصديق، وزنديق، والواجب موالاة أولياء الله المتقين من جميع الأصناف، وبعض الكفار والمنافقين من جميع الأصناف، والفاسق الِملِّيُّ يعطى من الموالاة بقدر إيمانه، ويعطى من المعاداة بقدر فسقه، فإن مذهب أهل السنة والجماعة أن الفاسق الملي له الثواب والعقاب إذا لم يعف الله عنه، وإنه لا بد أن يدخل النار من الفساق من شاء الله، وإن كان لا يخلد في النار أحد من أهل الإيمان، بل يخلد فيها المنافقون كما يخلد فيها المتظاهرون بالكفر. اهـ.

وراجع للفائدة الفتوى رقم: 113503.
