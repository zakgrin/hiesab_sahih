\chapter{الملحق}
% \addcontentsline{toc}{chapter}{\protect\numberline{}الملحق}
\label{sec:appendix}

\section{مسألة العدل مع الكفار}
\label{sec:app_justice}

الدولة الكافرة العادلة لها وعليها، فيذم كفرها ويحمد عدلها، ولا يرد عليها كل أمرها، بل يحمد ما فيها من العدل والإنصاف والمحاسن الإنسانية الموافقة للفطرة، ويذم ما فيها من كفر وفسق وعدوان على دين الله ورسله وهذا ما أوصانا به جل جلاله في كتابه العظيم فقال: 
\quranayah*[5][8]{\footnotesize \surahname*[5]}. وقال القرطبي في تفسيره: ودلت الآية أيضا على أن كفر الكافر لا يمنع من العدل عليه. وقال ابن كثير في تفسيره: وقوله: (ولا يجرمنكم شنآن قوم على ألا تعدلوا) أي: لا يحملنكم بغض قوم على ترك العدل فيهم، بل استعملوا العدل في كل أحد، صديقا كان أو عدوا [هـ]. 


وقول شهادة الحق في الدولة الكافرة لا يعني موالاتها وإن كانت عادلة، بل هذا ما هوا إلا شهادة الحق وقد تقدم بيان ذم ما فيها من كفر وفسق وعصيان لدين الله ورسله. وهذا لأن الله جل جلاله أمرنا بالعدل في القول ولو على أنفسنا فقال جل في علاه:
\quranayah*[4][135]{\footnotesize \surahname*[4]}. وقد جاء في تفسير ابن كثير: وقوله (فلا تتبعوا الهوى أن تعدلوا) أي : فلا يحملنكم الهوى والعصبية وبغضة الناس إليكم، على ترك العدل في أموركم وشؤونكم، بل الزموا العدل على أي حال كان، كما قال تعالى : (ولا يجرمنكم شنآن قوم على ألا تعدلوا اعدلوا هو أقرب للتقوى) [المائدة: 8] [هـ]. ومن ذلك ما صح عن جابِرُ بنُ عبدِ اللهِ رضِيَ اللهُ عنهما أنه قال: أفاءَ اللهُ عزَّ وجلَّ خَيبرَ على رسولِ اللهِ صلَّى اللهُ عليه وسَلَّم، فأقَرَّهُم رسولُ اللهِ صلَّى اللهُ عليه وسَلَّم كما كانوا، وجَعَلَها بَينَه وبَينَهُم، فبَعَثَ عَبدَ اللهِ بنَ رَواحةَ، فخَرَصَها عليهم، ثُمَّ قال لهم: يا مَعشَرَ اليَهودِ، أنتُم أبغَضُ الخَلْقِ إليَّ، قتَلتُم أنبياءَ اللهِ عزَّ وجلَّ، وكَذَبتُم على اللهِ، وليس يَحمِلُني بُغْضي إيَّاكم على أنْ أَحيفَ عليكم، قد خَرَصتُ عِشرينَ ألْفَ وَسْقٍ مِن تَمرٍ، فإنْ شِئتُم فلكُم، وإنْ أبَيتُم فلي، فقالوا: بهذا قامَتِ السَّمَواتُ والأرضُ، قد أخَذْنا، فاخْرُجوا عنَّا.{\footnotesize (صحيح على شرط مسلم، تخريج المسند لشعيب، تخريج سنن الدارقطني)}. وهذا فيه أن اليهود عرفوا أنه بالعدل قامت السموات والأرض وهذا ما سبق بيانه في الميزان الكوني، وأن عبد الله بن رواحة رضي الله عنه أقام فيهم الميزان الشرعي وأقر لهم بذلك بعدله معهم. 

وقد جاء في تفسير الطبري عن ابن عباس قوله: "كونوا قوامين بالقسط شهداء لله ولو على أنفسكم أو الوالدين والأقربين"، قال: أمر الله المؤمنين أن يقولوا الحقَّ ولو على أنفسهم أو آبائهم أو أبنائهم، ولا يحابوا غنيًّا لغناه، ولا يرحموا مسكينًا لمسكنته، وذلك قوله: "إن يكن غنيًّا أو فقيرًا فالله أولى بهما فلا تتبعوا الهوى أن تعدلوا "، فتذروا الحق، فتجوروا [هـ]. وأيضا جاء في تفسير الطبري: حدثنا سعيد، عن قتادة: "يا أيها الذين آمنوا كونوا قوامين بالقسط شهداء لله" الآية، هذا في الشهادة. فأقم الشهادة، يا ابن آدم، ولو على نفسك، أو الوالدين، أو على ذوي قرابتك، أو شَرَفِ قومك. فإنما الشهادة لله وليست للناس، وإن الله رضي العدل لنفسه، والإقساط والعدل ميزانُ الله في الأرض، به يردُّ الله من الشديد على الضعيف، ومن الكاذب على الصادق، ومن المبطل على المحق. وبالعدل يصدِّق الصادقَ، ويكذِّب الكاذبَ، ويردُّ المعتدي ويُرَنِّخُه، تعالى ربنا وتبارك. وبالعدل يصلح الناس، يا ابن آدم "إن يكن غنيًّا أو فقيرًا فالله أولى بهما"، يقول: أولى بغنيكم وفقيركم. قال: وذكر لنا أن نبيَّ الله موسى عليه السلام قال: "يا ربِّ، أي شيء وضعت في الأرض أقلّ؟"، قال: " العدلُ أقلُّ ما وضعت في الأرض". فلا يمنعك غِنى غنيّ ولا فقر فقير أن تشهد عليه بما تعلم، فإن ذلك عليك من الحق، وقال جل ثناؤه: " فالله أولى بهما " [هـ].

\section{مسألة الخروج على ولي أمر المسلمين}
\label{sec:app_rebellion}

إن من المسائل المهمة لأمة الإسلام بالعموم هي مسئلة الخروج على ولي أمر المسلمين. فهذه مسألة خطيرة وعظيمة يجب ألا يتكلم فيها إلا بعلم. 
وقد نهى النبي ﷺ عن الخروج على الدولة المسلمة الظالمة وبالأخص لما يترتب على ذلك من ظلم الذي يخالف الميزان الفطري والذي به يكون فساد المصالح العامة في الدنيا كسفك الدماء ونهب الأموال وهتك الأعراض، والتي هي أشد ظلما في الدنيا من الظلم الذي يكون بمخالفة الميزان الديني كمنع الزكاة أو الحكم بغير ما أنزل الله من باب الهوى. وقد تقدم معنا أن الله جل جلاله قدم في الدنيا إقامة الميزان بين الناس بالعدل على إقامة الحق في نفوس الناس لتقديم المصلحة العامة على الخاصة. ولذلك فقد نهى النبي ﷺ عن الخروج على ولاة الأمر المسلمين ولو كانوا ظالمين وعاصين لله ولرسوله فعن حذيفة بن اليمان رضي الله أن النبي ﷺ قال: يكونُ بَعْدِي أَئِمَّةٌ لا يَهْتَدُونَ بهُدَايَ، وَلَا يَسْتَنُّونَ بسُنَّتِي، وَسَيَقُومُ فيهم رِجَالٌ قُلُوبُهُمْ قُلُوبُ الشَّيَاطِينِ في جُثْمَانِ إنْسٍ، قُلتُ: كيفَ أَصْنَعُ يا رَسولَ اللهِ، إنْ أَدْرَكْتُ ذلكَ؟ قالَ: تَسْمَعُ وَتُطِيعُ لِلأَمِيرِ، وإنْ ضُرِبَ ظَهْرُكَ، وَأُخِذَ مَالُكَ، فَاسْمَعْ وَأَطِعْ {\footnotesize (صحيح مسلم)}. وقد أوصى بذلك النبي ﷺ في حجة الوداع فعن أم الحصين الأحمسية أنها قالت: سمعتُ رسولَ اللَّهِ صلَّى اللَّهُ عليْهِ وسلَّمَ يخطبُ في حجَّةِ الوداعِ  \comment{، وعليْهِ بردٌ قدِ التفعَ بِهِ من تحتِ إبطِهِ قالت فأنا أنظرُ إلى عضلةِ عضدِهِ تزتَجُّ، سمعتُهُ} يقولُ: يا أيُّها النَّاسُ اتَّقوا اللَّهَ وإن أمِّرَ عليْكم عبدٌ حبشيٌّ مجدَّعٌ فاسمعوا لَهُ وأطيعوا ما أقامَ لَكم كتابَ اللَّهِ  {\footnotesize (صحيح الترمذي، وصححه الألباني)}. وأيضا حديث العرباض بن سارية رضي الله عنه أنه قال: وعظَنا رسولُ اللَّهِ صلَّى اللَّهُ عليْهِ وسلَّمَ يومًا بعدَ صلاةِ الغداةِ موعِظةً بليغةً ذرِفَت منْها العيونُ ووجِلَت منْها القلوبُ، فقالَ رجلٌ إنَّ هذِهِ موعظةُ مودِّعٍ فماذا تعْهدُ إلينا يا رسولَ اللَّهِ، قالَ أوصيكم بتقوى اللَّهِ والسَّمعِ والطَّاعةِ وإن عبدٌ حبشيٌّ فإنَّهُ من يعِش منْكم يرَ اختلافًا كثيرًا وإيَّاكم ومحدَثاتِ الأمورِ فإنَّها ضَلالةٌ فمن أدرَكَ ذلِكَ منْكم فعليْهِ بِسُنَّتي وسنَّةِ الخلفاءِ الرَّاشدينَ المَهديِّينَ عضُّوا عليْها بالنَّواجذِ {\footnotesize (صحيح الترمذي، وصححه الألباني)}.

فلا يسعنا الخروج على ولاة الأمر الظالمين والعاصين لله ورسوله إلتزاما بأمر النبي ﷺ ولكن لا نصدقهم ولا نعينهم على ظلمهم كما صح ذلك عن النبي ﷺ أنه قال لكَعبِ بنِ عُجْرةَ: أعاذَكَ اللهُ من إمارةِ السُّفهاءِ، قال: وما إمارةُ السُّفهاءِ؟ قال: أُمراءُ يكونونَ بَعْدي، لا يَقتَدونَ بهَدْيي، ولا يَستَنُّونَ بسُنَّتي، فمَن صدَّقَهم بكذِبِهم، وأعانَهم على ظُلْمِهم، فأولئك ليسوا منِّي، ولستُ منهم، ولا يَرِدوا عليَّ حَوْضي، ومَن لم يُصدِّقْهم بكذِبِهم، ولم يُعِنْهم على ظُلْمِهم، فأولئك منِّي وأنا منهم، وسيَرِدوا عليَّ حَوْضي {\footnotesize (صحيح ابن حبان)}. ويكفي ولي الأمر المسلم الظالم ذلا وخسرانا أن أن النبي ﷺ قد تبرأ منه كما جاء في حديث سعد بن تميم أنه قيل: يا رسولَ اللهِ، ما للخليفةِ مِن بعدِك؟ قال: مِثلُ الذي لي، ما عدَلَ في الحُكمِ، وقسَطَ في القِسطِ، ورَحِمَ ذا الرَّحِمِ، فمَن فعَلَ غيرَ ذلك فليس منِّي ولستُ منه {\footnotesize (صحيح، تخريج سنن أبي داود، وصححه الألباني)}.


\comment{
فلا يسعنا الخروج على ولاة الأمر وإن ظلموا ولكن لا نصدقهم ولا نعينهم على ظلمهم كما أخبر بذلك جابر بن عبد الله أن النبي ﷺ قال لكَعبِ بنِ عُجْرةَ: أعاذَكَ اللهُ من إمارةِ السُّفهاءِ، قال: وما إمارةُ السُّفهاءِ؟ قال: أُمراءُ يكونونَ بَعْدي، لا يَقتَدونَ بهَدْيي، ولا يَستَنُّونَ بسُنَّتي، فمَن صدَّقَهم بكذِبِهم، وأعانَهم على ظُلْمِهم، فأولئك ليسوا منِّي، ولستُ منهم، ولا يَرِدوا عليَّ حَوْضي، ومَن لم يُصدِّقْهم بكذِبِهم، ولم يُعِنْهم على ظُلْمِهم، فأولئك منِّي وأنا منهم، وسيَرِدوا عليَّ حَوْضي، يا كَعبُ بنَ عُجْرةَ، الصومُ جُنَّةٌ، والصَّدقةُ تُطفِئُ الخَطيئةُ، والصَّلاةُ قُربانٌ -أو قال: بُرهانٌ- يا كَعبَ بنَ عُجْرةَ، إنَّه لا يدخُلُ الجَنَّةَ لَحمٌ نبَتَ من سُحتٍ، النَّارُ أَوْلى به، يا كَعبُ بنَ عُجْرةَ، النَّاسُ غَاديانِ: فمُبتاعٌ نفْسَه فمُعتِقُها، وبائعٌ نفْسَه فموبِقُها {\footnotesize (صحيح ابن حبان)}. \comment{{\footnotesize (صحيح ابن حبان، حسنه الوادعي في الصحيح المسند، والمنذري في الترغيب والترهيب، وقال شعيب الأرناؤوط إسناده قوي على شرط مسلم ف تخريج المسند لشعيب، وقال غيرهم رجاله رجال الصحيح)}. }
}
ويفرق بين النصح لولي الأمر الظالم وبين إنكار المنكر بالعموم، فإنكار المنكر بالعموم واجب على كل مسلم، وبالأخص رد الظالمين لمن استطاع أن يغير ويصلح بدون أن يترتب على ذلك مفسدة أعظم، فقد جاء عن أبوبكر الصديق أنه قال بعد أن حمِد اللهَ وأثنَى عليه: يا أيُّها النَّاسُ، إنَّكم تقرءون هذه الآيةَ، وتضعونها على غيرِ موضعِها (عَلَيْكُمْ أَنْفُسَكُمْ لَا يَضُرُّكُمْ مَنْ ضَلَّ إِذَا اهْتَدَيْتُمْ)، وإنَّا سمِعنا النَّبيَّ صلَّى اللهُ عليه وسلَّم يقولُ: إنَّ النَّاسَ إذا رأَوُا الظَّالمَ فلم يأخُذوا على يدَيْه أوشك أن يعُمَّهم اللهُ بعقابٍ وإنِّي سمِعتُ رسولَ اللهِ صلَّى اللهُ عليه وسلَّم يقولُ: ما من قومٍ يُعمَلُ فيهم بالمعاصي، ثمَّ يقدِرون على أن يُغيِّروا، ثمَّ لا يُغيِّروا إلَّا يوشِكُ أن يعُمَّهم اللهُ منه بعقابٍ {\footnotesize (صحيح أبي داود، وصححه الألباني)}. ولقد بايع النبي ﷺ أصحابه على السمع والطاعة والنصح لكل مسلم كما جاء ذلك عن جرير بن عبدالله أنه قال: بايَعْتُ النبيَّ صَلَّى اللَّهُ عليه وسلَّمَ علَى السَّمْعِ والطَّاعَةِ، فَلَقَّنَنِي، فِيما اسْتَطَعْتُ، والنُّصْحِ لِكُلِّ مُسْلِمٍ {\footnotesize (صحيح مسلم)}. فالنصح يكون لكل مسلم سواء كان ولي الأمر وغير ولي الأمر ويكون بحسب الحاجة وبالحكمة والموعظة الحسنة. ومن المصلحة في أغلب الأحوال أن تكون النصيحة لولي الأمر بالسر لما قد يترتب على الجهر بها من الفتن أو التحريض. ولهذا فقد قال النبي ﷺ: مَن أرادَ أن ينصحَ لذي سلطانٍ في أمرٍ فلا يُبدِهِ عَلانيةً ولَكِن ليأخذْ بيدِهِ فيَخلوَ بهِ فإن قبِلَ منهُ فذاكَ وإلَّا كانَ قد أدَّى الَّذي علَيهِ لَهُ {\footnotesize (صححه الألباني في تخريج كتاب السنة)}. فلو كانت هذه النصيحة لسلطان ظالم فهذا من أفضل الجهاد كما جاء ذلك عن أبو سعيد الخدري أن النبي ﷺ قال: أفضَلُ الجِهادِ كلمةُ عدلٍ، وفي رواية: كلمة حق، عندَ سُلطانٍ جائرٍ {\footnotesize (صحيح ابن ماجه، وصححه الألباني)}.

وأما الخروج على ولاة الأمور فشرطه أن يكون عندهم كفرا بواحا ظاهرا لا شك فيه. فقد جاء عن عبادة بن الصامت أنه قال: دَعَانَا رَسولُ اللهِ صَلَّى اللَّهُ عليه وسلَّمَ فَبَايَعْنَاهُ، فَكانَ فِيما أَخَذَ عَلَيْنَا: أَنْ بَايَعَنَا علَى السَّمْعِ وَالطَّاعَةِ في مَنْشَطِنَا وَمَكْرَهِنَا، وَعُسْرِنَا وَيُسْرِنَا، وَأَثَرَةٍ عَلَيْنَا، وَأَنْ لا نُنَازِعَ الأمْرَ أَهْلَهُ، قالَ: إلَّا أَنْ تَرَوْا كُفْرًا بَوَاحًا عِنْدَكُمْ مِنَ اللهِ فيه بُرْهَانٌ {\footnotesize (صحيح مسلم)}. ومن المعلوم أنه ليس كل حكم بغير ما أنزل الله كفر ومن ذلك بلا شك القوانين الوضعية التي لا تعارض كتاب الله وسنة نبيه ﷺ. ومثال ذلك الأمور التي فيها مصالح الناس كالأمور التنظيمة المباحة فهذا أمر مطلوب ولازم وبه يؤجر ولي الأمر لما في ذلك من نفع عام لجميع المسلمين، كتنظيم طرق سير السيارات، وقوانين حماية البيانات، وغيرها من القوانين التي بها تحفظ الدماء، والأموال، والأعراض. والكفر البواح لا يكون بالحكم بغير ما أنزل الله مع الإقرار بالذنب دون الإعتقاد بجواز ذلك والجهر به كمن يفعل ذلك من باب الهوى. وإنما الكفر البواح هو الإعتقاد مع الجهر أن الحكم المخالف لشرع الله وكتابه هو حكم جائز على وجه التفضيل أو المساواة أو الرد أو غير ذلك. ومن ذلك من يعتقد بأفضلية حكم غير الله على حكم الله أو مساواة حكم غير الله مع حكم الله أو جواز حكم غير الله أو رد حكم الله، المخالف لشرع الله وكتابه والجهر بذلك. ولقد بين ذلك الشيخ ابن باز رحمه الله في بيان القوانين الوضعية والآراء البشرية \textbf{التي تخالف شرع الله} فقال: الحكم بغير ما أنزل الله [بالقوانين التي تخالف شرع الله] أقسام، تختلف أحكامهم بحسب اعتقادهم وأعمالهم، فمن حكم بغير ما أنزل الله يرى أن ذلك أحسن من شرع الله فهو كافر عند جميع المسلمين، وهكذا من يحكّم القوانين الوضعية بدلا من شرع الله ويرى أن ذلك جائز، ولو قال: إن تحكيم الشريعة أفضل فهو كافر لكونه استحل ما حرم الله. أما من حكم بغير ما أنزل الله اتباعا للهوى أو لرشوة أو لعداوة بينه وبين المحكوم عليه أو لأسباب أخرى وهو يعلم أنه عاص لله بذلك وأن الواجب عليه تحكيم شرع الله [وإنما خالفها فعلا لا عقيدة لهوى] فهذا يعتبر من أهل المعاصي والكبائر ويعتبر قد أتى كفرا أصغر وظلما أصغر وفسقا أصغر كما جاء هذا المعنى عن ابن عباس رضي الله عنهما وعن طاووس وجماعة من السلف الصالح وهو المعروف عند أهل العلم. والله ولي التوفيق {\footnotesize (مجموع فتاوى ومقالات الشيخ ابن باز: 4/416)}. 

\comment{
ومن ذلك ما أخبر به حذيفة بن اليمان رضي الله أن النبي ﷺ قال:
كان الناسُ يسألون رسولَ اللهِ صلَّى اللهُ عليهِ وسلَّم عن الخيرِ، وكنتُ أسألُه عن الشرِّ، مخافةَ أن يُدركني، فقلتُ يارسولَ اللهِ، إنَّا كنا في جاهليةٍ وشرٍّ، فجاءنا اللهُ بهذا الخيرِ [فنحنُ فيه]، [وجاء بك]، فهل بعد هذا الخيرِ من شرٍّ [كما كان قبلَه؟]، [قال ياحذيفةُ تعلَّمْ كتابَ اللهِ، واتَّبِعْ ما فيه، (ثلاثَ مراتٍ). قال: قلتُ: يارسولَ اللهِ أبعد هذا الشرِّ من خيرٍ؟] قال نعم، [قلتُ: فما العِصمةُ منه؟ قال: السيفُ]، قلتُ وهل بعد ذلك الشرِّ من خيرٍ؟(وفي طريقٍ: قلتُ: وهل بعدَ السيفِ بقيَّةٌ؟) قال: نعم، وفيه (وفي طريقٍ: تكونُ إمارةٌ (وفي لفظٍ: جماعةٌ) على أقذاءٍ، وهُدنةٍ على) دَخَنٍ، قال: قلتُ: وما دَخَنُه؟ قال: قومٌ، وفي طريقٍ أخرى: يكونُ بعدي أئمةٌ [يستنُّونَ بغيرِ سُنَّتِي، و] يَهدُون بغيرِ هديِي، تعرفُ منهم وتُنكرُ، [وسيقومُ فيهم رجالٌ قلوبُهم قلوبُ الشياطينِ، في جثمانِ إنسٍ] (وفي أخرى: الهُدنةُ على دَخَنٍ ما هيَ؟قال: لا ترجعُ قلوبُ أقوامٍ على الذي كانت عليه) فقلتُ: هل بعد ذلك الخيرِ من شرٍّ؟قال: نعم، [فتنةٌ عمياءُ صمَّاءُ عليها] دعاةٌ على أبوابِ جهنمَ، من أجابَهُم إليها قذفوهُ فيها فقلتُ: يا رسولَ اللهِ، صِفْهُمْ لنا؟قال: هم من جِلدَتِنا، و يتكلمون بألسِنَتِنا، قلتُ: [يا رسولَ اللهِ]، فما تأمُرني إذا أدركني ذلك؟قال: تلزمُ جماعةَ المسلمين، وإمامَهم [تسمعُ وتُطيعُ الأميرَ، وإن ضرب ظهرَك، وأخذ مالَك، فاسمع وأطِعْ] فقلتُ: فإن لم يكن لهم جماعةٌ ولا إمامٌ؟ قال: فاعتزل تلك الفِرقَ كلَّها، ولو أن تعضَّ على أصلِ شجرةٍ، حتى يُدرِكَك الموتُ وأنت على ذلك. (وفي طريقٍ) فإن تَمُتْ يا حذيفةُ وأنت عاضٌّ على جذلٍ خيرٌ لك من أن تتبعَ أحدًا منهم. (وفي أخرى) فإن رأيتَ يومئذٍ للهِ عزَّ وجلَّ في الأرضِ خليفةً، فالزَمْهُ وإن ضرب ظهرَك وأخذ مالَك، فإن لم ترَ خليفةً فاهرب [في الأرضِ] حتى يُدرِكَك الموتُ وأنت عاضٌّ على جذلِ شجرةٍ. قال: قلتُ: ثم ماذا؟ قال: ثم يخرجُ الدجالُ. قال: قلتُ: فبم يجيءُ؟ قال: بنهرٍ أو قال: ماءٍ ونارٍ فمن دخل نهرَه حُطَّ أجرُه ووجب وِزْرُه، ومن دخل نارَه وجب أجرُه، وحُطَّ وِزْرُه . [قلتُ: يا رسولَ اللهِ: فما بعد الدجالِ؟ قال: عيسى بنُ مريمَ] قال: قلتُ: ثم ماذا؟ قال: لو أنتجتَ فرسًا لم تركب فَلُوَّها حتى تقومَ الساعةُ. {\footnotesize (جمعه الألباني من عدة طرق في صحيح البخاري وأورده في السلسلة الصحيحة)}.
}


\section{مسألة التفرق في الدين }
\label{sec:app_division}




\section{مسألة تجريح الأعيان}
\label{sec:app_division}

يقول شيخ الإسلام بن تيمية: وليعلم أن المؤمن تجب موالاته، وإن ظلمك واعتدى عليك، والكافر تجب معاداته، وإن أعطاك وأحسن إليك؛ فإن الله سبحانه بعث الرسل وأنزل الكتب؛ ليكون الدين كله لله، فيكون الحب لأوليائه، والبغض لأعدائه، والإكرام لأوليائه، والإهانة لأعدائه، والثواب لأوليائه، والعقاب لأعدائه. وإذا اجتمع في الرجل الواحد خير وشر وفجور، وطاعة ومعصية، وسنة وبدعة: استحق من الموالاة والثواب بقدر ما فيه من الخير، واستحق من المعاداة والعقاب بحسب ما فيه من الشر، فيجتمع في الشخص الواحد موجبات الإكرام والإهانة، فيجتمع له من هذا وهذا، كاللص الفقير تقطع يده لسرقته، ويعطى من بيت المال ما يكفيه لحاجته. هذا هو الأصل الذي اتفق عليه أهل السنة والجماعة، وخالفهم الخوارج، والمعتزلة، ومن وافقهم عليه، فلم يجعلوا الناس لا مستحقًّا للثواب فقط، ولا مستحقًّا للعقاب فقط. 

وقال أيضًا: معلوم أنه في كل طائفة بر، وفاجر، وصديق، وزنديق، والواجب موالاة أولياء الله المتقين من جميع الأصناف، وبعض الكفار والمنافقين من جميع الأصناف، والفاسق الِملِّيُّ يعطى من الموالاة بقدر إيمانه، ويعطى من المعاداة بقدر فسقه، فإن مذهب أهل السنة والجماعة أن الفاسق الملي له الثواب والعقاب إذا لم يعف الله عنه، وإنه لا بد أن يدخل النار من الفساق من شاء الله، وإن كان لا يخلد في النار أحد من أهل الإيمان، بل يخلد فيها المنافقون كما يخلد فيها المتظاهرون بالكفر. اهـ.

وراجع للفائدة الفتوى رقم: 113503.
