\chapter{حساب الله}

\section{مقدمة}

العد والإحصاء والحساب كلها داخلة صفات الله جل جلاله وأفعاله، وهذا من كمال عدله سبحانه حتى يعطي لكي ذي حق حقه يوم الحساب. ولقد أثبت سبحانه في مواضع كثيره في كتابه العظيم أنه سبحانه أحصى كل شئ عددا، وأنه سبحانه سريع الحساب كما يليق بجلاله بدون تشبيه أو تكييف أو تعطيل أو تمثيل. ومن ذلك أن الله جل جلاله قدر المقادير كلها وأحصاها عددا ويقوم سبحانه بمحاسبة كل نفس يوم الحساب وهو سبحانه سريع الحساب وهو أسرع الحاسبين.


ومن حكمة الله وكمال علمه أنه سبحانه أحصى كل شئ كما في قوله تعالى: \quranayah*[72][28][10]{\footnotesize \surahname*[72]}. بل إن علمه سبحانه بذلك قد سبق فعله كما في قوله تعالى: \quranayah*[78][29]{\footnotesize \surahname*[78]}. وجاء في تفسير السعدي: (وَكُلُّ شَيْءٍ) من قليل وكثير، وخير وشر (أَحْصَيْنَاهُ كِتَابًا) أي: كتبناه في اللوح المحفوظ \href{https://shamela.ws/book/42/2064#p9}{\faExternalLink} \cite{tafsir_Saadi}. ولهذا فإن الحسيب من أسماء الله عز وجل كما في قوله تعالى: \quranayah*[4][86]{\footnotesize \surahname*[4]}. وقوله تعالى: \quranayah*[33][39]{\footnotesize \surahname*[33]}. والله يحاسب خلقه في تدبيره لهذا الكون كل بحسب حاله كما في قوله تعالى في الكافرين: \quranayah*[65][8]{\footnotesize \surahname*[65]}. فهو سبحانه المحاسب لعباده والمجازي لهم في الدنيا والآخرة \href{https://shamela.ws/book/42/2002#p3}{\faExternalLink} \cite{tafsir_Saadi}.

ومن كمال عدله سبحانه أنه يحصى أعمال العباد ليوم الحساب كما في قوله تعالى: \quranayah*[36][12]{\footnotesize \surahname*[36]}. وقوله تعالى: \quranayah*[58][6]{\footnotesize \surahname*[58]}. ولهذا يوم القيامة يقول تعالى: \quranayah*[18][49]{\footnotesize \surahname*[18]}. ومن كمال عدله سبحانه أنه أحصى الأعمال كلها ويضع لها موازين القسط لحساب المكلفين من الجن والإنس حسابا ظاهرا لإظهار عدله سبحانه كما في قوله تعالى: \quranayah*[21][47]{\footnotesize \surahname*[21]}. فيتنازع الخصماء يوم القيامة عند الله كما في قوله تعالى: \quranayah*[39][31]{\footnotesize \surahname*[39]}. يحكم الله جل جلاله بعدله يوم القيامة بين الناس كما في قوله تعالى: \quranayah*[22][69]{\footnotesize \surahname*[22]}. فيجد كل مكلف كتابا له مكتوب فيه كل أعماله كما في قوله تعالى: \quranayah*[17][13]{\footnotesize \surahname*[17]}. ومن كمال عدله سبحانه أنه يعطي لكل مكلف كتابه يقرأه بنفسه كما في قوله تعالى: \quranayah*[17][14]{\footnotesize \surahname*[14]}.

وكل هذا الإحصاء حتى يوفي الله جل جلاله أجور المكلفين في حكمه الجزائي من جنة أو نار، وبحسب ما كلفهم به في حكمه الشرعي من طاعة أو عصيان، وبحسب ما شاء لهم في حكمه الكوني من حياة أو موت ومن هداية أو ضلال، كما في قوله تعالى: \quranayah*[3][185]{\footnotesize \surahname*[3]}. ولهذا فقد سمى الله جل جلاله يوم القيامة بيوم الحساب الذي فيه يكون الجزاء كما في قوله تعالى: \quranayah*[38][53]{\footnotesize \surahname*[38]}. فيحاسب الؤمنين حسابا يسيرا كما في قوله تعالى: \quranayah*[84][7-9]{\footnotesize \surahname*[84]}. ويحاسب الكافرين حسابا عسيرا كما في قوله تعالى: \quranayah*[84][10-15]{\footnotesize \surahname*[84]}. وقد صح أن عائشة أم المؤمنين أنها قالت: سمعتُ النبيَّ ﷺ يقولُ في بعضِ صلاتهِ : اللهمَّ حاسِبني حسابًا يسيرًا، فلمَّا انصرف قلتُ: يا نبيَّ اللهِ ما الحسابُ اليسيرُ؟ قال: أنْ ينظرَ في كتابهِ فيتجاوزَ عنه، إنَّه من نوقِشَ الحسابَ يومئذٍ يا عائشةُ هلكَ {\footnotesize (بإسناد جيد، الألباني أصل صفة الصلاة)}. وهذا فيه أن الحساب اليسير هو العرض الذي فيه يتجاوز عن السيئات وتجزى الحسنات وتضاعف برحمة الله وأن الحساب العسير هو النقاش الذي فيه حساب الحسنات والسيئات كبيرها وصغيرها ولا يتجاوز عن شيئ منها بعدل الله وكلاهما حساب. وهذا فيه أن الأنبياء والرسل سيحاسبون يوم القيامة. 

ولهذا فإن الله عز وجل أحصى كل شئ عددا حتى يقوم بنفسه جل جلاله بحاسب المكلفين بعدله يوم الحساب وهو سبحانه سريع الحساب كما في قوله تعالى: \quranayah*[13][40-41]{\footnotesize \surahname*[13]}. ومن عدل الله أنه سبحانه سريع الحساب وحسابه لا يكون إلا بما كسبت كل نفس كما في قوله تعالى: \quranayah*[14][51]{\footnotesize \surahname*[14]}. وأنه سبحانه سريع الحساب ولكن حسابه لا ظلم فيه كما في قوله تعالى: \quranayah*[40][17]{\footnotesize \surahname*[40]}. ولهذا فقد وصف الله جل جلاله نفسه بأنه أسرع الحاسبين كما في قوله تعالى: \quranayah*[6][62]{\footnotesize \surahname*[6]}. وقد كان النبي يدعو الله بصفة سريع الحساب ومن ذلك عندما دعا علَى الأحْزَابِ، فَقالَ: اللَّهُمَّ مُنْزِلَ الكِتَابِ، سَرِيعَ الحِسَابِ، اهْزِمِ الأحْزَابَ، اهْزِمْهُمْ وزَلْزِلْهُمْ {\footnotesize (صحيح البخاري)}



هذه هي المقدمة للفصل الأول.


من المعلوم أن الناس يتفاوتون في قدراتهم الحسابية كل بحسب إجتهاده وبما أودعه الله جل جلاله فيهم من إدراك وعقل. ولهذا فإن حساب الإنسان يكون بحسب إدراكه وعقله وتقديره. وهذا فيه أن الإنسان قد يخفى عليه العديد من الأمور وقد يتخلل حسابه النقص وبالأخص في الأمور التي يصعب حسابها. ولكن الله جعل الناس متوافقين في الحساب بفطرتهم. والله جل جلاله له كمال العدل في حسابه وهذا لأن تقديره تقدير كامل لا نقص فيه لأن الله هو العليم بكل شئ والقادر على كل شئ. وفي معرفة حساب الله العديد من الفوائد منها أن الله عدل وقائم بالقسط في الدنيا والآخرة، وفي الميزان الكوني والميزان الشرعي. وفي معرفة ذلك الإستعداد ليوم الحساب الذي فيه يحاسب الله المكلفين كما سيأتي بيانه في الفصل القادم. وفي هذه المعرفة أيضا الترغيب في إقامة الحق والميزان مع الأخذ بالأسباب والتي بها يمكن تحقيق الحكم الرشيد كما سيأتي بيانه في الفصل الذي يلي الفصل القادم.

\section{صفة العد والحساب}

 

\quranayah*[9][111]{\footnotesize \surahname*[9]}


وقد كان النبي يدعو الله بصفة سريع الحساب ومن ذلك عندما دعا علَى الأحْزَابِ، فَقالَ: اللَّهُمَّ مُنْزِلَ الكِتَابِ، سَرِيعَ الحِسَابِ، اهْزِمِ الأحْزَابَ، اهْزِمْهُمْ وزَلْزِلْهُمْ {\footnotesize (صحيح البخاري)}

{\footnotesize (صحيح البخاري، صحيح مسلم)}

يَدْخُلُ الجَنَّةَ مِن أُمَّتي سَبْعُونَ ألْفًا بغيرِ حِسابٍ، قالوا: ومَن هُمْ يا رَسولَ اللهِ؟ قالَ: هُمُ الَّذِينَ لا يَكْتَوُونَ ولا يَسْتَرْقُونَ، وعلَى رَبِّهِمْ يَتَوَكَّلُونَ، فَقامَ عُكّاشَةُ، فقالَ: ادْعُ اللَّهَ أنْ يَجْعَلَنِي منهمْ، قالَ: أنْتَ منهمْ، قالَ: فَقامَ رَجُلٌ، فقالَ: يا نَبِيَّ اللهِ، ادْعُ اللهَ أنْ يَجْعَلَنِي منهمْ، قالَ: سَبَقَكَ بها عُكّاشَةُ{\footnotesize (صحيح مسلم)}.


جاءتِ امرأَةٌ إلى النَّبيِّ صلَّى اللهُ عليه وسلَّمَ بها لَمَمٌ ، فقالت: يا رسولَ اللهِ، ادْعُ اللهَ أنْ يَشفيَني. قال: إنْ شِئتِ دعَوتُ اللهَ أنْ يَشفيَكِ، وإنْ شِئتِ فاصْبِري، ولا حِسابَ عليكِ. قالت: بَلْ أصبِرُ، ولا حِسابَ علَيَّ {\footnotesize (إسناده حسن، تخريج المسند لشعيب)}.


أنَّ رجلًا سأل أيُّ الشهداءِ أفضلُ قال : الذين إن يُلْقُوا في الصفِّ لا يَلفِتون وجوهَهم حتى يُقتَلوا ، أولئك يَنطلِقون في الغُرَفِ العُلا من الجنَّةِ ، ويضحكُ إليهم ربُّهم ، وإذا ضحِك ربُّك إلى عبدٍ في الدنيا فلا حسابَ عليه 
{\footnotesize (صححه الألباني في صحيح الترغيب)}.

أوَّلُ ما يُحاسَبُ به العبدُ يومَ القيامةِ صَلاتُه، فإنْ كان أكمَلَها كُتِبَتْ كامِلَةً، وإنْ لم يكنْ أكمَلَها قال اللهُ عزَّ وجلَّ لملائكتِه: انظُروا هل تَجِدونَ لعَبدي من تطوُّعٍ فأكْمِلوا به ما ضيَّعَ من فَريضَتِه، والزَّكاةُ مِثلُ ذلك، ثم تُؤخَذُ الأعمالُ على حِسابِ ذلك {\footnotesize (صحيح، تخريج مشكل الآثار، شعيب الأرناؤوط)}.

\section{التجارة مع الله في الدنيا}

\section{القبر أول منازل الآخرة}

وما تقول ؟ قلتُ : تقول أعاذَكُم اللهُ من فتنةِ الدجالِ، ومن فتنةِ عذابِ القبرِ قالت عائشةُ : فقام رسولُ اللهِ فرفع يديْهِ مدًّا، يستعيذُ باللهِ من فتنةِ الدجالِ، ومن عذابِ القبرِ ثم قال : أما فتنةُ الدجالِ، فإنه لم يكن نبيٌّ إلا ( قد ) حذَّرَ أُمَّتَه، وسأُحدِّثُكم ( وه ) بحديثٍ لم يُحذِّرْه نبيٌّ أُمَّتَه : إنَّهُ أعورُ، وإنَّ اللهَ ليس بأعورَ، مكتوبٌ بين عينيه كافرٌ، يقرؤُه كلُّ مؤمنٍ. فأما فتنةُ القبرِ فبي تُفتنونَ، وعنِّي تُسألونَ، فإذا كان الرجلُ الصالحُ أُجْلِسَ في قبرِه غيرَ فزعٍ ولا مشعوفٍ، ثم يقال له : فيم كنتَ ؟ فيقول في الإسلامِ فيقال ماهذا الرجلُ الذي كان فيكم ؟ فيقول : محمدٌ رسولُ اللهِ، جاءنا بالبيناتِ من عند اللهِ فصدَّقناهُ، فيفرجُ له فُرجةً قِبَلَ النارِ، فينظر إليها يُحطِّمُ بعضُها بعضًا، فيقال له : انظر إلى ماوقاك اللهُ، ثم يفرجُ له فُرجةً إلى الجنةِ، فينظرُ إلى زهرتِها وما فيها، فيقال له هذا مقعدُك منها ويقال على اليقينِ كنتَ وعليه مُتَّ، وعليه تُبعثُ إن شاء اللهُ وإذا كان الرجلُ السوءُ، أُجْلِسَ في قبرِه فزعًا مشعوفًا فيقال له : فيم كنتَ ؟ فيقول : سمعتُ الناسَ يقولون قولًا فقلتُ كما قالوا، فيُفرجُ له فُرجةً إلى الجنةِ، فينظرُ إلى زهرتِها ومافيها، فيقال له انظر إلى ماصرف اللهُ عنك ثم يُفرجُ له فُرجةً قِبَلَ النارِ فينظرُ إليها يُحطِّمُ بعضُها بعضًا ويقال ( له ) هذا مقعدُك منها، على الشكِّ كنتَ، وعليه مُتَّ وعليه تُبعثُ إن شاء اللهُ ثم يُعذَّبُ

الراوي : عائشة أم المؤمنين | المحدث : الألباني | المصدر : صحيح الترغيب
الصفحة أو الرقم : 3557 | خلاصة حكم المحدث : صحيح 


جاءت يهوديةٌ فاستطعمتْ على بابي، فقالَتْ : أطعموني أعاذَكُم اللهُ من فتنةِ الدجالِ ومن فتنةِ عذابِ القبرِ، فلم أزل أحبِسُها حتى أتى رسولُ اللهِ صلَّى اللهُ عليْهِ وسلَّمَ فقلتُ : يا رسولَ اللهِ ما تقولُ هذه اليهوديةُ ؟ قال : وما تقولُ ؟ قلتُ : تقولُ : أعاذَكُم اللهُ من فتنةِ الدجالِ ومن فتنةِ عذابِ القبرِ. قالَتْ عائشةُ : فقال رسولُ اللهِ صلَّى اللهُ عليْهِ وسلَّمَ فرفع يديه مدًّا يستعيذُ باللهِ من فتنةِ الدجالِ ومن فتنةِ عذابِ القبرِ، ثم قال : أما فتنةُ الدجالِ فإنَّهُ لم يكن نبيٌّ إلا وقد حذَّرَ أُمَّتَه وسأُحذِّرُكُموه بحديثٍ لم يُحذِّرُه نبيٌّ أُمَّتَه، إنَّهُ أعورُ واللهُ ليس بأعورَ، مكتوبٌ بين عينيه كافرٌ يقرؤُه كلُّ مؤمنٍ، فأما فتنةُ القبرِ فبي تُفتنونَ وعني تُسألونَ، فإن كان الرجلُ الصالحُ أُجلسَ في قبرِه غيرَ فزعٍ ولا مشغوفٍ، ثم يقال له : فيم كنتَ ؟ فيقول : في الإسلامِ، فيقال : ما هذا الرجلُ الذي كان فيكم ؟ فيقول : محمدٌ رسولُ اللهِ جاءنا بالبيناتِ من عِندَ اللهِ فصدَّقناهُ، فيُفرجُ له فرجةً قبلَ النارِ فينظر إليها يُحطِّمُ بعضُها بعضًا، فيقال له : انظر إلى ما وقاك اللهُ، ثم يُفْرَجُ له فُرجةً على الجنةِ فينظرُ إلى زهرتها وما فيها، فيقال له : هذا مقعدُك منها ويقال : على اليقينِ كنتَ وعليْهِ مُتَّ وعليْهِ تُبعثُ إن شاء اللهُ. وإذا كان الرجلُ السوءُ جلس في قبرِه فزعًا مشعوفًا فيقال له : فيم كنتَ ؟ فيقول : لا أدري. فيقال : ما هذا الرجلُ الذي كان فيكم ؟ فيقول : سمعتُ الناسَ يقولون قولًا فقلتُ كما قالوا : فيُفرجُ له فُرجةً قِبَلَ الجنةِ فينظرُ إلى زهرتِها وما فيها فيقال له : انظر إلى ما صرف اللهُ عنك، ثم يُفرجُ له فُرجةً قِبَلَ النارِ فينظرُ إليها يُحطِّمُ بعضُها بعضًا ويقال له : هذا مقعدُك منها، على الشكِّ كنتَ وعليْهِ مُتَّ وعليْهِ تُبعثُ إن شاء اللهُ. ثم يُعذَّبُ

الراوي : عائشة أم المؤمنين | المحدث : السيوطي | المصدر : شرح الصدور
الصفحة أو الرقم : 193 | خلاصة حكم المحدث : إسناده صحيح 

الراوي : عائشة أم المؤمنين | المحدث : شعيب الأرناؤوط | المصدر : تخريج المسند لشعيب
الصفحة أو الرقم : 25089 | خلاصة حكم المحدث : إسناده صحيح على شرط الشيخين 

\section{يوم القيامة}


جَاءَ حَبْرٌ مِنَ اليَهُودِ، فَقالَ: إنَّه إذَا كانَ يَوْمُ القِيَامَةِ جَعَلَ اللَّهُ السَّمَوَاتِ علَى إصْبَعٍ، والأرَضِينَ علَى إصْبَعٍ، والمَاءَ والثَّرَى علَى إصْبَعٍ، والخَلَائِقَ علَى إصْبَعٍ، ثُمَّ يَهُزُّهُنَّ، ثُمَّ يقولُ: أنَا المَلِكُ أنَا المَلِكُ، فَلقَدْ رَأَيْتُ النبيَّ صَلَّى اللهُ عليه وسلَّمَ يَضْحَكُ حتَّى بَدَتْ نَوَاجِذُهُ تَعَجُّبًا وتَصْدِيقًا لِقَوْلِهِ، ثُمَّ قالَ النبيُّ صَلَّى اللهُ عليه وسلَّمَ: {وَما قَدَرُوا اللَّهَ حَقَّ قَدْرِهِ} [الأنعام: 91] إلى قَوْلِهِ {يُشْرِكُونَ} [الزمر: 67].
الراوي : عبدالله بن مسعود | المحدث : البخاري | المصدر : صحيح البخاري
الصفحة أو الرقم : 7513 | خلاصة حكم المحدث : [صحيح]

جَاءَ حَبْرٌ إلى النبيِّ صَلَّى اللَّهُ عليه وَسَلَّمَ، فَقالَ: يا مُحَمَّدُ، أَوْ يا أَبَا القَاسِمِ إنَّ اللَّهَ تَعَالَى يُمْسِكُ السَّمَوَاتِ يَومَ القِيَامَةِ علَى إصْبَعٍ، وَالأرَضِينَ علَى إصْبَعٍ، وَالْجِبَالَ وَالشَّجَرَ علَى إصْبَعٍ، وَالْمَاءَ وَالثَّرَى علَى إصْبَعٍ، وَسَائِرَ الخَلْقِ علَى إصْبَعٍ، ثُمَّ يَهُزُّهُنَّ، فيَقولُ: أَنَا المَلِكُ، أَنَا المَلِكُ، فَضَحِكَ رَسُولُ اللهِ صَلَّى اللَّهُ عليه وَسَلَّمَ تَعَجُّبًا ممَّا قالَ الحَبْرُ ، تَصْدِيقًا له، ثُمَّ قَرَأَ: {وَما قَدَرُوا اللَّهَ حَقَّ قَدْرِهِ وَالأرْضُ جَمِيعًا قَبْضَتُهُ يَومَ القِيَامَةِ وَالسَّمَوَاتُ مَطْوِيَّاتٌ بيَمِينِهِ سُبْحَانَهُ وَتَعَالَى عَمَّا يُشْرِكُونَ} [الزمر: 67]. 7148- [20-...] حَدَّثَنَا عُثْمَانُ بنُ أَبِي شيبَةَ، وإسْحَاقُ بنُ إبْرَاهِيمَ كِلَاهُمَا، عن جَرِيرٍ، عن مَنْصُورٍ بهذا الإسْنَادِ، قالَ: جَاءَ حَبْرٌ مِنَ اليَهُودِ إلى رَسُولِ اللهِ صَلَّى اللَّهُ عليه وَسَلَّمَ،... بمِثْلِ حَديثِ فُضَيْلٍ وَلَمْ يَذْكُرْ: ثُمَّ يَهُزُّهُنَّ. وَقالَ: فَلقَدْ رَأَيْتُ رَسُولَ اللهِ صَلَّى اللَّهُ عليه وَسَلَّمَ ضَحِكَ حتَّى بَدَتْ نَوَاجِذُهُ تَعَجُّبًا لِما قالَ تَصْدِيقًا له، ثُمَّ قالَ رَسُولُ اللهِ صَلَّى اللَّهُ عليه وَسَلَّمَ: {وَما قَدَرُوا اللَّهَ حَقَّ قَدْرِهِ} وَتَلَا الآيَةَ.
عرض مختصر..
الراوي : عبدالله بن مسعود | المحدث : مسلم | المصدر : صحيح مسلم
الصفحة أو الرقم : 2786 | خلاصة حكم المحدث : [صحيح]

\subsection{البعث}

\subsection{المحشر}





\subsection{الحساب يبدأ عند الميزان وقبل الجزاء}


نقل القرطبي أن الحساب يكون قبل الميزان وفيه تعرض الأعمال فقط قبل أن توزن على الميزان وهذا لا يصح. لأن الحساب لا يكتمل إلا بعد وزن الأعمال كلها كبيرها وصغيرها وعدها وجمعها. ولهذا فإن الحساب يبدأ مع بداية وزن الأعمال وينتهي عند الإنتهاء من وزنها وعدها وجمعها. يبدأ الحساب مع وزن الأعمال وهنا تعرض الأعمال وتوزن على الميزان كلها صغيرها وكبيرها حتى يحسب حسابها الحساب الكامل والوافي الذي لا ظلم فيه، وبناءا على الحساب الكلي يعطى الجزاء إما جنة وإما نار. لكن هذا الحساب نوعان، إما الحساب اليسير وإما الحساب العسير. فمن كان حسابه يسيرا يعطى كتاب أعماله بيمينه وأما من كان حسابه عسيرا يعطى كتاب أعماله من وراء ظهره في يده الشمال 

قال تعالى:   
\quranayah*[84][7-15]{\footnotesize \surahname*[84]}. وقد جاء عن الطبري قوله في تفسير هذه الآيات: وقوله: (فَأَمَّا مَنْ أُوتِيَ كِتَابَهُ بِيَمِينِهِ) يقول تعالى ذكره: فأما من أعطي كتاب أعماله بيمينه. (فَسَوْفَ يُحَاسَبُ حِسَابًا يَسِيرًا) بأن ينظر في أعماله، فيغفر له سيئها، ويُجازى على حُسنها. وقوله: (وَيَنْقَلِبُ إِلَى أَهْلِهِ مَسْرُورًا) يقول: وينصرف هذا المحاسَبُ حسابًا يسيرًا إلى أهله في الجنة مسرورًا. (وَأَمَّا مَنْ أُوتِيَ كِتَابَهُ وَرَاءَ ظَهْرِهِ) وأما من أعطي كتابه منكم أيها الناس يومئذ وراء ظهره، وذلك أن جعل يده اليمنى إلى عنقه وجعل الشمال من يديه وراء ظهره، فيتناول كتابه بشماله من وراء ظهره، ولذلك وصفهم جلَّ ثناؤه أحيانًا أنهم يؤتون كتبهم بشمائلهم، وأحيانًا أنهم يؤتونها من وراء ظهورهم. (فَسَوْفَ يَدْعُو ثُبُورًا) يقول: فسوف ينادي بالهلاك، وهو أن يقول: واثبوراه، واويلاه، وهو من قولهم: دعا فلان لهفه: إذا قال: والهفاه. وقوله: (وَيَصْلَى سَعِيرًا) اختلفت القرّاء في قراءة ذلك، فقرأته عامة قرّاء مكة والمدينة والشام: (وَيُصَلَّى) بضم الياء وتشديد اللام، بمعنى: أن الله يصليهم تصلية بعد تصلية، وإنضاجة بعد إنضاجة، كما قال تعالى: كُلَّمَا نَضِجَتْ جُلُودُهُمْ بَدَّلْنَاهُمْ جُلُودًا غَيْرَهَا ، واستشهدوا لتصحيح قراءتهم ذلك كذلك، بقوله: ثُمَّ الْجَحِيمَ صَلُّوهُ وقرأ ذلك بعض المدنيين وعامة قرّاء الكوفة والبصرة: (وَيَصْلَى) بفتح الياء وتخفيف اللام، بمعنى: أنهم يَصْلونها ويَرِدونها، فيحترقون فيها، واستشهدوا لتصحيح قراءتهم ذلك كذلك بقول الله: يَصْلَوْنَهَا و إِلا مَنْ هُوَ صَالِ الْجَحِيمِ . وقوله: (إِنَّهُ كَانَ فِي أَهْلِهِ مَسْرُورًا) يقول تعالى ذكره: إنه كان في أهله في الدنيا مسرورا لما فيه من خلافه أمرَ الله، وركوبه معاصيه. وقوله: (إِنَّهُ ظَنَّ أَنْ لَنْ يَحُورَ* بَلَى) يقول تعالى ذكره: إنّ هذا الذي أُوتي كتابه وراء ظهره يوم القيامة، ظنّ في الدنيا أن لن يرجع إلينا، ولن يُبعث بعد مماته، فلم يكن يبالي ما ركب من المآثم؛ لأنه لم يكن يرجو ثوابًا، ولم يكن يخشى عقابًا، يقال منه: حار فلان عن هذا الأمر: إذا رجع عنه، ومنه الخبر الذي رُوي عن رسول الله صلى الله عليه وسلم أنه كان يقول في دعائه: " اللَّهُمَّ إنّي أعُوذُ بِكَ مِنَ الحَوْرِ بَعْدَ الكَوْرِ" يعني بذلك: من الرجُوع إلى الكفر، بعد الإيمان. وقوله: (بَلَى) يقول تعالى ذكره: بلى لَيَحُورَنَّ وَلَيَرْجِعَنّ إلى ربه حيا كما كان قبل مماته. وقوله: (إِنَّ رَبَّهُ كَانَ بِهِ بَصِيرًا) يقول جلّ ثناؤه: إن ربّ هذا الذي ظن أن لن يحور، كان به بصيرا، إذ هو في الدنيا بما كان يعمل فيها من المعاصي، وما إليه يصير أمره في الآخرة، عالم بذلك كلِّه [هـ].

وقد صح أن عائشة أم المؤمنين أنها قالت: سمعتُ النبيَّ ﷺ يقولُ في بعضِ صلاتهِ : اللهمَّ حاسِبني حسابًا يسيرًا، فلمَّا انصرف قلتُ: يا نبيَّ اللهِ ما الحسابُ اليسيرُ؟ قال: أنْ ينظرَ في كتابهِ فيتجاوزَ عنه، إنَّه من نوقِشَ الحسابَ يومئذٍ يا عائشةُ هلكَ {\footnotesize (بإسناد جيد، الألباني أصل صفة الصلاة)}. وهذا فيه أن الحساب اليسير هو العرض الذي فيه يتجاوز عن السيئات وتجزى الحسنات وتضاعف برحمة الله وأن الحساب العسير هو النقاش الذي فيه حساب الحسنات والسيئات كبيرها وصغيرها ولا يتجاوز عن شيئ منها بعدل الله وكلاهما حساب. وهذا فيه أن الأنبياء والرسل سيحاسبون يوم القيامة. 

وقد صح عن النبي ﷺ أنه قال: مَن حُوسِبَ عُذِّبَ قَالَتْ عَائِشَةُ: فَقُلتُ أوَليسَ يقولُ اللَّهُ تَعَالَى: {فَسَوْفَ يُحَاسَبُ حِسَابًا يَسِيرًا} [الانشقاق: 8] قَالَتْ: فَقَالَ: إنَّما ذَلِكِ العَرْضُ، ولَكِنْ: مَن نُوقِشَ الحِسَابَ يَهْلِكْ، وفي رواية: عذب {\footnotesize (صحيح البخاري، صحيح مسلم)}. والمقصود والمراد بقول النبي ﷺ: (من حوسب عذب)،  هو الحساب العسير الذي يكون فيه النقاش ولهذا جاء بيان ذلك في نهاية الحديث كما في أغلب الأحاديث الأخرى: (من نوقش الحساب عذب) كما سبق بيانه، وهذا هو المعنى الصحيح وهو الحساب العسير الذي فيه تناقش الأعمال كبيرها وصغيرها. فمن المعلوم أن كل المكلفين من الجن والإنس وحتى الأنبياء سيحاسبون يوم القيامة، فلا يكون المعنى أن كل من حوسب عموما عذب فهذا لا يتوافق ما عدل الله ومع كتابه وما صح عن نبيه ﷺ. 

وقد صح عن النبي ﷺ أنه قال: 

النبي ﷺ فقالت: 










ومن عدل الله جل جلاله أنه سبحانه يقضي في المظالم بين الناس يوم الحاسب ولهذا فقد قال النبي ﷺ:  من كَانَتْ له مَظْلِمَةٌ لأخِيهِ مِن عِرْضِهِ أَوْ شيءٍ، فَلْيَتَحَلَّلْهُ منه اليَومَ، قَبْلَ أَنْ لا يَكونَ دِينَارٌ وَلَا دِرْهَمٌ، إنْ كانَ له عَمَلٌ صَالِحٌ أُخِذَ منه بقَدْرِ مَظْلِمَتِهِ، وإنْ لَمْ تَكُنْ له حَسَنَاتٌ أُخِذَ مِن سَيِّئَاتِ صَاحِبِهِ فَحُمِلَ عليه {\footnotesize (صحيح البخاري)}. ومن ذلك أيضا أن الله يجزي عباده كل بحسب ما دل عليه من العمل فعن جرير بن عبدالله أن النبي ﷺ قال: مَن سَنَّ في الإسْلَامِ سُنَّةً حَسَنَةً، فَعُمِلَ بهَا بَعْدَهُ، كُتِبَ له مِثْلُ أَجْرِ مَن عَمِلَ بهَا، وَلَا يَنْقُصُ مِن أُجُورِهِمْ شيءٌ، وَمَن سَنَّ في الإسْلَامِ سُنَّةً سَيِّئَةً، فَعُمِلَ بهَا بَعْدَهُ، كُتِبَ عليه مِثْلُ وِزْرِ مَن عَمِلَ بهَا، وَلَا يَنْقُصُ مِن أَوْزَارِهِمْ شيءٌ {\footnotesize (صحيح مسلم)}.

العجائب: 

إنَّ اللَّهَ سيُخَلِّصُ رجلًا من أمَّتي على رؤوسِ الخلائقِ يومَ القيامةِ فينشُرُ علَيهِ تسعةً وتسعينَ سجلًّا، كلُّ سجلٍّ مثلُ مدِّ البصرِ ثمَّ يقولُ: أتنكرُ من هذا شيئًا ؟ أظلمَكَ كتبتي الحافِظونَ ؟يقولُ: لا يا ربِّ، فيقولُ: أفلَكَ عذرٌ ؟ فيقولُ: لا يا ربِّ، فيقولُ: بلَى، إنَّ لَكَ عِندَنا حسنةً، وإنَّهُ لا ظُلمَ عليكَ اليومَ، فيخرجُ بطاقةً فيها أشهدُ أن لا إلَهَ إلَّا اللَّهُ، وأشهدُ أنَّ محمَّدًا عبدُهُ ورسولُهُ، فيقولُ: احضُر وزنَكَ فيقولُ يا ربِّ، ما هذِهِ البطاقةُ مع هذِهِ السِّجلَّاتِ ؟ فقالَ: فإنَّكَ لا تُظلَمُ، قالَ: فتوضَعُ السِّجلَّاتُ في كفَّةٍ، والبطاقةُ في كفَّةٍ فطاشتِ السِّجلَّاتُ وثقُلتِ البطاقةُ، ولا يثقلُ معَ اسمِ اللَّهِ شيءٌ {\footnotesize (صحيح الترمذي، صححه الألباني)}


\subsubsection{الجزاء إما جنة أو نار}

الجنة 

مائة درجة 

وهي مائة درجة كما جاء ذلك عن النبي ﷺ حيث قال: مَن آمَنَ باللَّهِ وبِرَسولِهِ، وأَقامَ الصَّلاةَ، وصامَ رَمَضانَ؛ كانَ حَقًّا علَى اللَّهِ أنْ يُدْخِلَهُ الجَنَّةَ، جاهَدَ في سَبيلِ اللَّهِ أوْ جَلَسَ في أرْضِهِ الَّتي وُلِدَ فيها، فقالوا: يا رَسولَ اللَّهِ، أفَلا نُبَشِّرُ النَّاسَ؟ قالَ: إنَّ في الجَنَّةِ مِئَةَ دَرَجَةٍ، أعَدَّها اللَّهُ لِلْمُجاهِدِينَ في سَبيلِ اللَّهِ، ما بيْنَ الدَّرَجَتَيْنِ كما بيْنَ السَّماءِ والأرْضِ، فإذا سَأَلْتُمُ اللَّهَ، فاسْأَلُوهُ الفِرْدَوْسَ؛ فإنَّه أوْسَطُ الجَنَّةِ وأَعْلَى الجَنَّةِ -أُراهُ- فَوْقَهُ عَرْشُ الرَّحْمَنِ، ومِنْهُ تَفَجَّرُ أنْهارُ الجَنَّةِ. {\footnotesize (صحيح البخاري)}.

أعلى مراتب الجنة هي الفردوس الأعلى كما جاء عن أم المؤمنين عائشة رضي الله عنها أن النبيُّ صَلَّى اللهُ عليه وسلَّمَ كانَ يقولُ وهو صَحِيحٌ: إنَّه لَمْ يُقْبَضْ نَبِيٌّ حتَّى يَرَى مَقْعَدَهُ مِنَ الجَنَّةِ، ثُمَّ يُخَيَّرَ فَلَمَّا نَزَلَ به، ورَأْسُهُ علَى فَخِذِي غُشِيَ عليه، ثُمَّ أفَاقَ فأشْخَصَ بَصَرَهُ إلى سَقْفِ البَيْتِ، ثُمَّ قالَ: اللَّهُمَّ الرَّفِيقَ الأعْلَى \.فَقُلتُ: إذًا لا يَخْتَارُنَا، وعَرَفْتُ أنَّه الحَديثُ الذي كانَ يُحَدِّثُنَا وهو صَحِيحٌ، قالَتْ: فَكَانَتْ آخِرَ كَلِمَةٍ تَكَلَّمَ بهَا: اللَّهُمَّ الرَّفِيقَ الأعْلَى {\footnotesize (صحيح البخاري)}. فعرفت عائشة رضي الله عنه أن الرسول ﷺ أوري مقعده في الجنة فاختار الرفيق الأعلى  قبل أن تقض روحه ﷺ.


وأدناهم مرتبة هو آخر رجل يدخل الجنة وهو آخر رجل يخرج من النار كما صح ذلك عن النبي ﷺ حيث قال: إنِّي لأعلَمُ آخِرَ أهلِ النَّارِ خُرُوجًا مِنها، وآخِرَ أهلِ الجَنَّةِ دُخُولًا الجَنَّةَ: رَجُلٌ يَخرُجُ مِنَ النَّارِ حَبْوًا، فيَقُولُ اللهُ تباركَ وتعالى لَه: اذهَب فادخُلِ الجَنَّةَ، فيَأتيها فيُخَيَّلُ إليه أنَّها مَلأى، فيَرجِعُ فيَقُولُ: يا رَبِّ، وجَدْتُها مَلأى، فيَقُولُ اللهُ تباركَ وتعالى لَه: اذهَبْ فادخُلِ الجَنَّةَ، قال: فيَأتيها فيُخَيَّلُ إليه أنَّها مَلأى، فيَرجِعُ فيَقُولُ: يا رَبِّ، وجَدْتُها مَلأى، فيَقُولُ اللهُ لَه: اذهَبْ فادخُلِ الجَنَّةَ؛ فإنَّ لَكَ مِثلَ الدُّنيا وعَشَرةَ أمثالِها -أو إنَّ لَكَ عَشَرةَ أمثالِ الدُّنيا- قال: فيَقُولُ: أتَسخَرُ بي -أو أتضحَكُ بي- وأنتَ المَلِكُ؟ قال: لَقَد رَأيتُ رَسولَ اللهِ صلَّى اللهُ عليه وسلَّم ضَحِكَ حَتَّى بَدَت نَواجِذُه، قال: فكانَ يُقالُ: ذاكَ أدنى أهلِ الجَنَّةِ مَنزِلةً {\footnotesize (صحيح البخاري، صحيح مسلم واللفظ له)}.




