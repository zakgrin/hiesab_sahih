
\chapter{الفصل الثاني}

\section{جداول}
فيما يلي مثال على جدول:

\begin{table}[h]
\centering
\begin{tabular}{|c|c|c|}
\hline
العنوان 1 & العنوان 2 & العنوان 3 \\
\hline
الخلية 1 & الخلية 2 & الخلية 3 \\
\hline
الخلية 4 & الخلية 5 & الخلية 6 \\
\hline
\end{tabular}
\caption{مثال على جدول}
\end{table}

\newpage

\section{مراجع}
يمكنك إضافة مراجع كما يلي:

\begin{itemize}
    \item المرجع الأول
    \item المرجع الثاني
    \item المرجع الثالث
\end{itemize}

% Example of a reference using BibTeX
\section{مراجع باستخدام BibTeX}
لإضافة مراجع باستخدام BibTeX، يمكن استخدام الملف التالي `references.bib`:

\begin{verbatim}
@book{example,
  author    = "المؤلف",
  title     = "عنوان الكتاب",
  publisher = "دار النشر",
  year      = "السنة"
}
\end{verbatim}

ثم تضمين المراجع في المستند الرئيسي:

\begin{verbatim}
\bibliographystyle{plain}
\bibliography{references}
\end{verbatim}

\newpage

\section{نص الفصل الثاني - الصفحة الثانية}

هذه الصفحة الثانية للفصل الثاني تحتوي على نص إضافي لاختبار تقسيم الصفحات وظهور الرؤوس والأقدام بشكل صحيح في النصوص العربية.

\newpage

\section{نص الفصل الثاني - الصفحة الثالثة}

هذه الصفحة الثالثة للفصل الثاني تحتوي على المزيد من النصوص لاختبار تقسيم الصفحات وظهور الرؤوس والأقدام بشكل صحيح في النصوص العربية.
