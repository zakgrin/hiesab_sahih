
\chapter{ميزان الله}

\section{مقدمة}
هذه هي المقدمة للفصل الأول.


\section{فضل الميزان}


فالميزان يوضع يوم القيامة بعد الصراط:
حَدَّثَنَا عَبْدُ اللَّهِ بْنُ الصَّبَّاحِ الْهَاشِمِيُّ، حَدَّثَنَا بَدَلُ بْنُ الْمُحَبَّرِ، حَدَّثَنَا حَرْبُ بْنُ مَيْمُونٍ الأَنْصَارِيُّ أَبُو الْخَطَّابِ، حَدَّثَنَا النَّضْرُ بْنُ أَنَسِ بْنِ مَالِكٍ، عَنْ أَبِيهِ، قَالَ سَأَلْتُ النَّبِيَّ صلى الله عليه وسلم أَنْ يَشْفَعَ لِي يَوْمَ الْقِيَامَةِ فَقَالَ " أَنَا فَاعِلٌ " ز قَالَ قُلْتُ يَا رَسُولَ اللَّهِ فَأَيْنَ أَطْلُبُكَ قَالَ " اطْلُبْنِي أَوَّلَ مَا تَطْلُبُنِي عَلَى الصِّرَاطِ " ز قَالَ قُلْتُ فَإِنْ لَمْ أَلْقَكَ عَلَى الصِّرَاطِ قَالَ " فَاطْلُبْنِي عِنْدَ الْمِيزَانِ " ز قُلْتُ فَإِنْ لَمْ أَلْقَكَ عِنْدَ الْمِيزَانِ قَالَ " فَاطْلُبْنِي عِنْدَ الْحَوْضِ فَإِنِّي لاَ أُخْطِئُ هَذِهِ الثَّلاَثَ الْمَوَاطِنَ " ز قَالَ أَبُو عِيسَى هَذَا حَدِيثٌ حَسَنٌ غَرِيبٌ لاَ نَعْرِفُهُ إِلاَّ مِنْ هَذَا الْوَجْهِ

وفي هذا الميزان من من شئ أثقل فيه من حسن الخلق كما ذكر الرسول:





حَدَّثَنَا مُحَمَّدُ بْنُ الْمُثَنَّى، حَدَّثَنَا مُحَمَّدُ بْنُ عَبْدِ اللَّهِ الأَنْصَارِيُّ، حَدَّثَنَا الأَشْعَثُ، عَنِ الْحَسَنِ، عَنْ أَبِي بَكْرَةَ، أَنَّ النَّبِيَّ صلى الله عليه وسلم قَالَ ذَاتَ يَوْمٍ " مَنْ رَأَى مِنْكُمْ رُؤْيَا " ز فَقَالَ رَجُلٌ أَنَا رَأَيْتُ كَأَنَّ مِيزَانًا نَزَلَ مِنَ السَّمَاءِ فَوُزِنْتَ أَنْتَ وَأَبُو بَكْرٍ فَرُجِحْتَ أَنْتَ بِأَبِي بَكْرٍ وَوُزِنَ عُمَرُ وَأَبُو بَكْرٍ فَرُجِحَ أَبُو بَكْرٍ وَوُزِنَ عُمَرُ وَعُثْمَانُ فَرُجِحَ عُمَرُ ثُمَّ رُفِعَ الْمِيزَانُ فَرَأَيْنَا الْكَرَاهِيَةَ فِي وَجْهِ رَسُولِ اللَّهِ صلى الله عليه وسلم.


حَدَّثَنَا هِشَامُ بْنُ عَمَّارٍ، حَدَّثَنَا صَدَقَةُ بْنُ خَالِدٍ، حَدَّثَنَا ابْنُ جَابِرٍ، قَالَ سَمِعْتُ بُسْرَ بْنَ عُبَيْدِ اللَّهِ، يَقُولُ سَمِعْتُ أَبَا إِدْرِيسَ الْخَوْلاَنِيَّ، يَقُولُ حَدَّثَنِي النَّوَّاسُ بْنُ سَمْعَانَ الْكِلاَبِيُّ، قَالَ سَمِعْتُ رَسُولَ اللَّهِ ـ صلى الله عليه وسلم ـ يَقُولُ " مَا مِنْ قَلْبٍ إِلاَّ بَيْنَ إِصْبَعَيْنِ مِنْ أَصَابِعِ الرَّحْمَنِ إِنْ شَاءَ أَقَامَهُ وَإِنْ شَاءَ أَزَاغَهُ " ز وَكَانَ رَسُولُ اللَّهِ ـ صلى الله عليه وسلم ـ يَقُولُ " يَا مُثَبِّتَ الْقُلُوبِ ثَبِّتْ قُلُوبَنَا عَلَى دِينِكَ " ز قَالَ " وَالْمِيزَانُ بِيَدِ الرَّحْمَنِ يَرْفَعُ أَقْوَامًا وَيَخْفِضُ آخَرِينَ إِلَى يَوْمِ الْقِيَامَةِ " ز

