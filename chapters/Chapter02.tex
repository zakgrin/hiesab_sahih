\chapter{حساب الله}

\section{مقدمة}
هذه هي المقدمة للفصل الأول.



\section{صفة العد والحساب}

العد والإحصاء والحساب كلها من صفات الله جل جلاله، وهذا من كمال عدله سبحانه حتى يعطي لكي ذي حق حقه يوم الحساب. ولقد أثبت سبحانه في مواضع كثيره في كتابه العظيم أنه سبحانه أحصى كل شئ عددا، وأنه سبحانه سريع الحساب كما يليق بجلاله بدون تشبيه أو تكييف أو تعطيل أو تمثيل. ومن ذلك أن الله جل جلاله قدر المقادير كلها وعلمها عددا واحصائها وحسابها

\quranayah*[9][111]{\footnotesize \surahname*[9]}


\section{التجارة مع الله في الدنيا}

\section{يوم الحساب}

\subsection{البعث}


\subsection{الحساب يبدأ عند الميزان وقبل الجزاء}


نقل القرطبي أن الحساب يكون قبل الميزان وفيه تعرض الأعمال فقط قبل أن توزن على الميزان وهذا لا يصح. لأن الحساب لا يكتمل إلا بعد وزن الأعمال كلها كبيرها وصغيرها وعدها وجمعها. ولهذا فإن الحساب يبدأ مع بداية وزن الأعمال وينتهي عند الإنتهاء من وزنها وعدها وجمعها. يبدأ الحساب مع وزن الأعمال وهنا تعرض الأعمال وتوزن على الميزان كلها صغيرها وكبيرها حتى يحسب حسابها الحساب الكامل والوافي الذي لا ظلم فيه، وبناءا على الحساب الكلي يعطى الجزاء إما جنة وإما نار. لكن هذا الحساب نوعان، إما الحساب اليسير وإما الحساب العسير. فمن كان حسابه يسيرا يعطى كتاب أعماله بيمينه وأما من كان حسابه عسيرا يعطى كتاب أعماله من وراء ظهره في يده الشمال 

قال تعالى:   
\quranayah*[84][7-15]{\footnotesize \surahname*[84]}. وقد جاء عن الطبري قوله في تفسير هذه الآيات: وقوله: (فَأَمَّا مَنْ أُوتِيَ كِتَابَهُ بِيَمِينِهِ) يقول تعالى ذكره: فأما من أعطي كتاب أعماله بيمينه. (فَسَوْفَ يُحَاسَبُ حِسَابًا يَسِيرًا) بأن ينظر في أعماله، فيغفر له سيئها، ويُجازى على حُسنها. وقوله: (وَيَنْقَلِبُ إِلَى أَهْلِهِ مَسْرُورًا) يقول: وينصرف هذا المحاسَبُ حسابًا يسيرًا إلى أهله في الجنة مسرورًا. (وَأَمَّا مَنْ أُوتِيَ كِتَابَهُ وَرَاءَ ظَهْرِهِ) وأما من أعطي كتابه منكم أيها الناس يومئذ وراء ظهره، وذلك أن جعل يده اليمنى إلى عنقه وجعل الشمال من يديه وراء ظهره، فيتناول كتابه بشماله من وراء ظهره، ولذلك وصفهم جلَّ ثناؤه أحيانًا أنهم يؤتون كتبهم بشمائلهم، وأحيانًا أنهم يؤتونها من وراء ظهورهم. (فَسَوْفَ يَدْعُو ثُبُورًا) يقول: فسوف ينادي بالهلاك، وهو أن يقول: واثبوراه، واويلاه، وهو من قولهم: دعا فلان لهفه: إذا قال: والهفاه. وقوله: (وَيَصْلَى سَعِيرًا) اختلفت القرّاء في قراءة ذلك، فقرأته عامة قرّاء مكة والمدينة والشام: (وَيُصَلَّى) بضم الياء وتشديد اللام، بمعنى: أن الله يصليهم تصلية بعد تصلية، وإنضاجة بعد إنضاجة، كما قال تعالى: كُلَّمَا نَضِجَتْ جُلُودُهُمْ بَدَّلْنَاهُمْ جُلُودًا غَيْرَهَا ، واستشهدوا لتصحيح قراءتهم ذلك كذلك، بقوله: ثُمَّ الْجَحِيمَ صَلُّوهُ وقرأ ذلك بعض المدنيين وعامة قرّاء الكوفة والبصرة: (وَيَصْلَى) بفتح الياء وتخفيف اللام، بمعنى: أنهم يَصْلونها ويَرِدونها، فيحترقون فيها، واستشهدوا لتصحيح قراءتهم ذلك كذلك بقول الله: يَصْلَوْنَهَا و إِلا مَنْ هُوَ صَالِ الْجَحِيمِ . وقوله: (إِنَّهُ كَانَ فِي أَهْلِهِ مَسْرُورًا) يقول تعالى ذكره: إنه كان في أهله في الدنيا مسرورا لما فيه من خلافه أمرَ الله، وركوبه معاصيه. وقوله: (إِنَّهُ ظَنَّ أَنْ لَنْ يَحُورَ* بَلَى) يقول تعالى ذكره: إنّ هذا الذي أُوتي كتابه وراء ظهره يوم القيامة، ظنّ في الدنيا أن لن يرجع إلينا، ولن يُبعث بعد مماته، فلم يكن يبالي ما ركب من المآثم؛ لأنه لم يكن يرجو ثوابًا، ولم يكن يخشى عقابًا، يقال منه: حار فلان عن هذا الأمر: إذا رجع عنه، ومنه الخبر الذي رُوي عن رسول الله صلى الله عليه وسلم أنه كان يقول في دعائه: " اللَّهُمَّ إنّي أعُوذُ بِكَ مِنَ الحَوْرِ بَعْدَ الكَوْرِ" يعني بذلك: من الرجُوع إلى الكفر، بعد الإيمان. وقوله: (بَلَى) يقول تعالى ذكره: بلى لَيَحُورَنَّ وَلَيَرْجِعَنّ إلى ربه حيا كما كان قبل مماته. وقوله: (إِنَّ رَبَّهُ كَانَ بِهِ بَصِيرًا) يقول جلّ ثناؤه: إن ربّ هذا الذي ظن أن لن يحور، كان به بصيرا، إذ هو في الدنيا بما كان يعمل فيها من المعاصي، وما إليه يصير أمره في الآخرة، عالم بذلك كلِّه [هـ].

وقد صح أن عائشة أم المؤمنين أنها قالت: سمعتُ النبيَّ ﷺ يقولُ في بعضِ صلاتهِ : اللهمَّ حاسِبني حسابًا يسيرًا، فلمَّا انصرف قلتُ: يا نبيَّ اللهِ ما الحسابُ اليسيرُ؟ قال: أنْ ينظرَ في كتابهِ فيتجاوزَ عنه، إنَّه من نوقِشَ الحسابَ يومئذٍ يا عائشةُ هلكَ {\footnotesize (بإسناد جيد، الألباني أصل صفة الصلاة)}. وهذا فيه أن الحساب اليسير هو العرض الذي فيه يتجاوز عن السيئات وتجزى الحسنات وتضاعف برحمة الله وأن الحساب العسير هو النقاش الذي فيه حساب الحسنات والسيئات كبيرها وصغيرها ولا يتجاوز عن شيئ منها بعدل الله وكلاهما حساب. وهذا فيه أن الأنبياء والرسل سيحاسبون يوم القيامة. 

وقد صح عن النبي ﷺ أنه قال: مَن حُوسِبَ عُذِّبَ قَالَتْ عَائِشَةُ: فَقُلتُ أوَليسَ يقولُ اللَّهُ تَعَالَى: {فَسَوْفَ يُحَاسَبُ حِسَابًا يَسِيرًا} [الانشقاق: 8] قَالَتْ: فَقَالَ: إنَّما ذَلِكِ العَرْضُ، ولَكِنْ: مَن نُوقِشَ الحِسَابَ يَهْلِكْ، وفي رواية: عذب {\footnotesize (صحيح البخاري، صحيح مسلم)}. والمقصود والمراد بقول النبي ﷺ: (من حوسب عذب)،  هو الحساب العسير الذي يكون فيه النقاش ولهذا جاء بيان ذلك في نهاية الحديث كما في أغلب الأحاديث الأخرى: (من نوقش الحساب عذب) كما سبق بيانه، وهذا هو المعنى الصحيح وهو الحساب العسير الذي فيه تناقش الأعمال كبيرها وصغيرها. فمن المعلوم أن كل المكلفين من الجن والإنس وحتى الأنبياء سيحاسبون يوم القيامة، فلا يكون المعنى أن كل من حوسب عموما عذب فهذا لا يتوافق ما عدل الله ومع كتابه وما صح عن نبيه ﷺ. 

وقد صح عن النبي ﷺ أنه قال: 

النبي ﷺ فقالت: 










ومن عدل الله جل جلاله أنه سبحانه يقضي في المظالم بين الناس يوم الحاسب ولهذا فقد قال النبي ﷺ:  من كَانَتْ له مَظْلِمَةٌ لأخِيهِ مِن عِرْضِهِ أَوْ شيءٍ، فَلْيَتَحَلَّلْهُ منه اليَومَ، قَبْلَ أَنْ لا يَكونَ دِينَارٌ وَلَا دِرْهَمٌ، إنْ كانَ له عَمَلٌ صَالِحٌ أُخِذَ منه بقَدْرِ مَظْلِمَتِهِ، وإنْ لَمْ تَكُنْ له حَسَنَاتٌ أُخِذَ مِن سَيِّئَاتِ صَاحِبِهِ فَحُمِلَ عليه {\footnotesize (صحيح البخاري)}. ومن ذلك أيضا أن الله يجزي عباده كل بحسب ما دل عليه من العمل فعن جرير بن عبدالله أن النبي ﷺ قال: مَن سَنَّ في الإسْلَامِ سُنَّةً حَسَنَةً، فَعُمِلَ بهَا بَعْدَهُ، كُتِبَ له مِثْلُ أَجْرِ مَن عَمِلَ بهَا، وَلَا يَنْقُصُ مِن أُجُورِهِمْ شيءٌ، وَمَن سَنَّ في الإسْلَامِ سُنَّةً سَيِّئَةً، فَعُمِلَ بهَا بَعْدَهُ، كُتِبَ عليه مِثْلُ وِزْرِ مَن عَمِلَ بهَا، وَلَا يَنْقُصُ مِن أَوْزَارِهِمْ شيءٌ {\footnotesize (صحيح مسلم)}.

العجائب: 

إنَّ اللَّهَ سيُخَلِّصُ رجلًا من أمَّتي على رؤوسِ الخلائقِ يومَ القيامةِ فينشُرُ علَيهِ تسعةً وتسعينَ سجلًّا، كلُّ سجلٍّ مثلُ مدِّ البصرِ ثمَّ يقولُ: أتنكرُ من هذا شيئًا ؟ أظلمَكَ كتبتي الحافِظونَ ؟يقولُ: لا يا ربِّ، فيقولُ: أفلَكَ عذرٌ ؟ فيقولُ: لا يا ربِّ، فيقولُ: بلَى، إنَّ لَكَ عِندَنا حسنةً، وإنَّهُ لا ظُلمَ عليكَ اليومَ، فيخرجُ بطاقةً فيها أشهدُ أن لا إلَهَ إلَّا اللَّهُ، وأشهدُ أنَّ محمَّدًا عبدُهُ ورسولُهُ، فيقولُ: احضُر وزنَكَ فيقولُ يا ربِّ، ما هذِهِ البطاقةُ مع هذِهِ السِّجلَّاتِ ؟ فقالَ: فإنَّكَ لا تُظلَمُ، قالَ: فتوضَعُ السِّجلَّاتُ في كفَّةٍ، والبطاقةُ في كفَّةٍ فطاشتِ السِّجلَّاتُ وثقُلتِ البطاقةُ، ولا يثقلُ معَ اسمِ اللَّهِ شيءٌ {\footnotesize (صحيح الترمذي، صححه الألباني)}


\subsubsection{الجزاء إما جنة أو نار}

الجنة 

مائة درجة 

وهي مائة درجة كما جاء ذلك عن النبي ﷺ حيث قال: مَن آمَنَ باللَّهِ وبِرَسولِهِ، وأَقامَ الصَّلاةَ، وصامَ رَمَضانَ؛ كانَ حَقًّا علَى اللَّهِ أنْ يُدْخِلَهُ الجَنَّةَ، جاهَدَ في سَبيلِ اللَّهِ أوْ جَلَسَ في أرْضِهِ الَّتي وُلِدَ فيها، فقالوا: يا رَسولَ اللَّهِ، أفَلا نُبَشِّرُ النَّاسَ؟ قالَ: إنَّ في الجَنَّةِ مِئَةَ دَرَجَةٍ، أعَدَّها اللَّهُ لِلْمُجاهِدِينَ في سَبيلِ اللَّهِ، ما بيْنَ الدَّرَجَتَيْنِ كما بيْنَ السَّماءِ والأرْضِ، فإذا سَأَلْتُمُ اللَّهَ، فاسْأَلُوهُ الفِرْدَوْسَ؛ فإنَّه أوْسَطُ الجَنَّةِ وأَعْلَى الجَنَّةِ -أُراهُ- فَوْقَهُ عَرْشُ الرَّحْمَنِ، ومِنْهُ تَفَجَّرُ أنْهارُ الجَنَّةِ. {\footnotesize (صحيح البخاري)}.

أعلى مراتب الجنة هي الفردوس الأعلى كما جاء عن أم المؤمنين عائشة رضي الله عنها أن النبيُّ صَلَّى اللهُ عليه وسلَّمَ كانَ يقولُ وهو صَحِيحٌ: إنَّه لَمْ يُقْبَضْ نَبِيٌّ حتَّى يَرَى مَقْعَدَهُ مِنَ الجَنَّةِ، ثُمَّ يُخَيَّرَ فَلَمَّا نَزَلَ به، ورَأْسُهُ علَى فَخِذِي غُشِيَ عليه، ثُمَّ أفَاقَ فأشْخَصَ بَصَرَهُ إلى سَقْفِ البَيْتِ، ثُمَّ قالَ: اللَّهُمَّ الرَّفِيقَ الأعْلَى \.فَقُلتُ: إذًا لا يَخْتَارُنَا، وعَرَفْتُ أنَّه الحَديثُ الذي كانَ يُحَدِّثُنَا وهو صَحِيحٌ، قالَتْ: فَكَانَتْ آخِرَ كَلِمَةٍ تَكَلَّمَ بهَا: اللَّهُمَّ الرَّفِيقَ الأعْلَى {\footnotesize (صحيح البخاري)}. فعرفت عائشة رضي الله عنه أن الرسول ﷺ أوري مقعده في الجنة فاختار الرفيق الأعلى  قبل أن تقض روحه ﷺ.


وأدناهم مرتبة هو آخر رجل يدخل الجنة وهو آخر رجل يخرج من النار كما صح ذلك عن النبي ﷺ حيث قال: إنِّي لأعلَمُ آخِرَ أهلِ النَّارِ خُرُوجًا مِنها، وآخِرَ أهلِ الجَنَّةِ دُخُولًا الجَنَّةَ: رَجُلٌ يَخرُجُ مِنَ النَّارِ حَبْوًا، فيَقُولُ اللهُ تباركَ وتعالى لَه: اذهَب فادخُلِ الجَنَّةَ، فيَأتيها فيُخَيَّلُ إليه أنَّها مَلأى، فيَرجِعُ فيَقُولُ: يا رَبِّ، وجَدْتُها مَلأى، فيَقُولُ اللهُ تباركَ وتعالى لَه: اذهَبْ فادخُلِ الجَنَّةَ، قال: فيَأتيها فيُخَيَّلُ إليه أنَّها مَلأى، فيَرجِعُ فيَقُولُ: يا رَبِّ، وجَدْتُها مَلأى، فيَقُولُ اللهُ لَه: اذهَبْ فادخُلِ الجَنَّةَ؛ فإنَّ لَكَ مِثلَ الدُّنيا وعَشَرةَ أمثالِها -أو إنَّ لَكَ عَشَرةَ أمثالِ الدُّنيا- قال: فيَقُولُ: أتَسخَرُ بي -أو أتضحَكُ بي- وأنتَ المَلِكُ؟ قال: لَقَد رَأيتُ رَسولَ اللهِ صلَّى اللهُ عليه وسلَّم ضَحِكَ حَتَّى بَدَت نَواجِذُه، قال: فكانَ يُقالُ: ذاكَ أدنى أهلِ الجَنَّةِ مَنزِلةً {\footnotesize (صحيح البخاري، صحيح مسلم واللفظ له)}.




