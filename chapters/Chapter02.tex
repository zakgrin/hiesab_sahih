
\chapter{الحكم الرشيد}

\section{مقدمة}


قال تعالى:


يَا دَاوُودُ إِنَّا جَعَلْنَاكَ خَلِيفَةً فِي الْأَرْضِ فَاحْكُم بَيْنَ النَّاسِ بِالْحَقِّ وَلَا تَتَّبِعِ الْهَوَىٰ فَيُضِلَّكَ عَن سَبِيلِ اللَّهِ ۚ إِنَّ الَّذِينَ يَضِلُّونَ عَن سَبِيلِ اللَّهِ لَهُمْ عَذَابٌ شَدِيدٌ بِمَا نَسُوا يَوْمَ الْحِسَابِ [ص : 26]


تفسير ابن كثير:

هذه وصية من الله - عز وجل - لولاة الأمور أن يحكموا بين الناس بالحق المنزل من عنده تبارك وتعالى ولا يعدلوا عنه فيضلوا عن سبيله وقد توعد [ الله ] تعالى من ضل عن سبيله ، وتناسى يوم الحساب ، بالوعيد الأكيد والعذاب الشديد .

قال ابن أبي حاتم : حدثنا أبي حدثنا هشام بن خالد حدثنا الوليد ، حدثنا مروان بن جناح ، حدثني إبراهيم أبو زرعة - وكان قد قرأ الكتاب - أن الوليد بن عبد الملك قال له : أيحاسب الخليفة فإنك قد قرأت الكتاب الأول ، وقرأت القرآن وفقهت ؟ فقلت : يا أمير المؤمنين أقول ؟ قال : قل في أمان . قلت يا أمير المؤمنين أنت أكرم على الله أو داود ؟ إن الله - عز وجل - جمع له النبوة والخلافة ثم توعده في كتابه فقال : ( يا داود إنا جعلناك خليفة في الأرض فاحكم بين الناس بالحق ولا تتبع الهوى فيضلك عن سبيل الله إن الذين يضلون ) الآية .

وقال عكرمة : ( لهم عذاب شديد بما نسوا يوم الحساب ) هذا من المقدم والمؤخر لهم عذاب شديد يوم الحساب بما نسوا .

وقال السدي : لهم عذاب شديد بما تركوا أن يعملوا ليوم الحساب .

وهذا القول أمشى على ظاهر الآية فالله أعلم .

\section{أركان الحكم الرشيد}

يقام الحكم الرشيد على ثلاثة أركان وهي: (1) إقامة الحق بالعلم الصحيح، (2) إقامة الميزان الشرعي بالعدل والقسط، (3) الأخذ بأسباب القوى كالحديد وما يلزم ذلك من علوم كالحساب والطب وغيرها من العلوم التي تكمن المسلمين من دحر الأعداء ونشر الحق ونصرته. فهذا كله من نصر الله ورسله وقد وعد سبحانه بنصر من ينصره كما في قوله تعالى:
\quranayah*[47][7]{\footnotesize \surahname*[47]}. 

وبهذه الأمور الثلاثة التي يبنى عليها الحكم الرشيد يكون التمكين الذي وعد الله به كما ذكر سبحانه في قصة ذي القرنين في قوله تعالى: 
\quranayah*[18][84-85]{\footnotesize \surahname*[18]}. وقد بين معنى ذلك الشيخ العثيمين رحمه الله أن معنى "من كل شئ سببا" أن الله أتاه كل الأسباب التي بها يكون التمكين في الأرض من قوة السلطة وتمام الملك فانتفع بما أعطاه الله من الأسباب. فهذا التمكين جاء بتسخير الله وهذا لأن ذي القرنين أخذ بالأسباب التي أعطاها الله له مع إقامة الحق وإقامة العدل كما في قوله تعالى: 
\quranayah*[18][87-89]{\footnotesize \surahname*[18]}. وقد جاء في تفسير السعدي رحمه الله أن هذا يدل على كونه من الملوك الصالحين الأولياء، العادلين العالمين، حيث وافق مرضاة الله في معاملة كل أحد، بما يليق بحاله [هـ]. وقد أثبت  سبحانه له التمكين والرشد لما له من الخبرة في إتباع الأسباب كما في قوله تعالى: 
\quranayah*[18][91-92]{\footnotesize \surahname*[18]}. ومعنى ذلك أي: أحطنا بما عنده من الخير والأسباب العظيمة كما جاء في تفسير السعدي. ومن ذلك أنه كان لديه من الأسباب العلمية ما يمكنه من فهم العديد من العلوم التي تمكنه من الإنتقال إلى مشارق الأرض ومغاربها وفهم اللغات الأخرى، ومن ذلك ما فقه به ألسنة أولئك القوم (الذين لا يفقهون قولا) الذين اشتكوا إليه ضرر يأجوج ومأجوج [.] إفسادهم في الأرض، فلم يكن ذو القرنين ذا طمع، ولا رغبة في الدنيا، ولا تاركا لإصلاح أحوال الرعية، بل كان قصده الإصلاح، فلذلك أجاب طلبتهم لما فيها من المصلحة، ولم يأخذ منهم أجرة، وشكر ربه على تمكينه واقتداره [هـ]. فلم يطلب منهم إلا أن يعينوه على حمل زبر أي قطع الحديد ووضعه في مكانه بين الجبلين وإشعال النار له بالمنافيخ الشديدة والآلات العظيمة لإذابة النحاس حتى يكون سائلا فيصبه عليها ليستحكم السد استحكاما هائلا يعجز يأجوج ومأجوج على الصعود فوقه فضلا عن ثقبه. 

وقد علم ذا القرنين أن كل ذلك من فضل الله عليه حيث قال تعالى: 
\quranayah*[18][98]{\footnotesize \surahname*[18]}. وما أجمل ما أورده السعدي في تفسيره هذه الآية حيث قال: 
فلما فعل هذا الفعل الجميل والأثر الجليل، أضاف النعمة إلى موليها وقال: { هَذَا رَحْمَةٌ مِنْ رَبِّي ْ} أي: من فضله وإحسانه عليَّ، وهذه حال الخلفاء الصالحين، إذا من الله عليهم بالنعم الجليلة، ازداد شكرهم وإقرارهم، واعترافهم بنعمة الله كما قال سليمان عليه السلام، لما حضر عنده عرش ملكة سبأ مع البعد العظيم، قال: 
\quranayah*[27][40][19]{\footnotesize \surahname*[27]}.
بخلاف أهل التجبر والتكبر والعلو في الأرض فإن النعم الكبار، تزيدهم شرا وبطرا. كما قال قارون، لما آتاه الله من الكنوز، ما إن مفاتحه لتنوء بالعصبة أولي القوة، قال: 
\quranayah*[28][78][1-6]{\footnotesize \surahname*[28]}. 

وقد علم أيضا ذا القرنين بما لديه من الخبرة بأسباب الحديد وما قد يطرأ عليه من صدإ وتآكل بعد زمن أنه سياتي يوم وينهار هذا السد العظيم ويخرج يأجوج ومأجوج في آخر الزمان كما في قوله تعالى: 
\quranayah*[18][98][6]{\footnotesize \surahname*[18]}. وجاء في تفسير السعده رحمه الله أن قوله: { فَإِذَا جَاءَ وَعْدُ رَبِّي ْ} أي: لخروج يأجوج ومأجوج { جَعَلَهُ ْ} أي: ذلك السد المحكم المتقن { دَكَّاءَ ْ} أي: دكه فانهدم، واستوى هو والأرض { وَكَانَ وَعْدُ رَبِّي حَقًّا ْ} [هـ]. ولهذا فقد أخبر سبحانه بوقوع ذلك لا محالة في قوله تعالى: 
\quranayah*[21][96]{\footnotesize \surahname*[21]}. وقد قال السعدي رحمه الله: أنه في آخر الزمان، ينفتح السد عنهم، فيخرجون إلى الناس في هذه الحالة والوصف، الذي ذكره الله من كل من مكان مرتفع، وهو الحدب ينسلون أي: يسرعون. وفي هذا دلالة على كثرتهم الباهرة، وإسراعهم في الأرض، إما لبذواتهم، وإما لما خلق الله لهم من الأسباب التي تقرب لهم البعيد، وتسهل عليهم الصعب، وأنهم يقهرون الناس، ويعلون عليهم في الدنيا، وأنه لا يد لأحد بقتالهم. 

وكل ما تقدم فيه أن ذا القرنين لم يكن فقط يأخذ بالأسباب وإنما كان يقيم الحق والعدل مع الأخذ بالأسباب والعلم بها وذلك من فضل الله عليه وتوفيقه له رحمه الله تعالى ورضي عنه. وقد قال عنه الشيخ ابن باز رحمه الله: ذو القرنين ملك عظيم صاحب خير، وإحسان، وإصلاح، واختلف الناس في نبوته، والمشهور أنه ملك صالح. وفي موضع آخر رجح الشيخ ابن باز رحمه الله أن ذا القرنين نبيا من الأنبياء لأنه كان يتبع أمر الله في الأرض وظاهر الآيات أنه كان يتلقى هذه الأوامر والتوجيهات من ربه جل جلاله وهذا شأن النبي.


\subsection{الميزان الشرعي}

الأمانة:
قال ﷺ :ما مِنْ عبدٍ يسترْعيه اللهُ رعيَّةً، يموتُ يومَ يموتُ، وهوَ غاشٌّ لرعِيَّتِهِ، إلَّا حرّمَ اللهُ عليْهِ الجنَّةَ {\footnotesize (صحيح الجامع وصححه الألباني)}. وفي رواية: ما مِن عَبْدٍ اسْتَرْعاهُ اللَّهُ رَعِيَّةً، فَلَمْ يَحُطْها بنَصِيحَةٍ، إلَّا لَمْ يَجِدْ رائِحَةَ الجَنَّةِ {\footnotesize (صحيح البخاري)}.


\section{الحكم الرشيد في زمن الصحابة}


البيان الواضح لسنين الخلافة الراشدة


قال صلى الله عليه وسلم 

خلافةُ النُّبوَّةِ ثلاثون سنةً ، ثم يُؤتي اللهُ الملكَ مَن يشاءُ

الراوي : سفينة مولى رسول الله صلى الله عليه وسلم | المحدث : الألباني | المصدر : صحيح الجامع | الصفحة أو الرقم : 3257 | خلاصة حكم المحدث : صحيح
   
قال سعيدٌ: قال لي سَفينةُ: أمسِكْ عليكَ : أبو بكرٍ سنتين، وعمرُ عشرًا، وعثمانُ اثنتي عشرةَ، وعليُّ كذا، قال سعيدٌ : قلتُ لسفينةَ : إنَّ هؤلاء يزعمون أنَّ عليًّا لم يكن بخليفةٍ، قال : كذبَتْ أستاهُ بني الزرقاءِ – يعني : بني مرْوانَ -الراوي : سفينة مولى رسول الله صلى الله عليه وسلم | المحدث : الألباني | المصدر : صحيح أبي داود
الصفحة أو الرقم: 4646 | خلاصة حكم المحدث : حسن

وجاء في تفسير ابن كثير 
وجاء في تفسيير القرطبي ان الشعبي قال: كان بين عمر وأبي خصومة ، فتقاضيا إلى زيد بن ثابت ، فلما دخلا عليه أشار لعمر إلى وسادته ، فقال عمر : هذا أول جورك ، أجلسني وإياه مجلسا واحدا ، فجلسا بين يديه .

قال أبو بكرٍ ، بعد أن حمِد اللهَ وأثنَى عليه : يا أيُّها النَّاسُ ، إنَّكم تقرءون هذه الآيةَ ، وتضعونها على غيرِ موضعِها عَلَيْكُمْ أَنْفُسَكُمْ لَا يَضُرُّكُمْ مَنْ ضَلَّ إِذَا اهْتَدَيْتُمْ وإنَّا سمِعنا النَّبيَّ صلَّى اللهُ عليه وسلَّم يقولُ : إنَّ النَّاسَ إذا رأَوُا الظَّالمَ فلم يأخُذوا على يدَيْه أوشك أن يعُمَّهم اللهُ بعقابٍ وإنِّي سمِعتُ رسولَ اللهِ صلَّى اللهُ عليه وسلَّم يقولُ : ما من قومٍ يُعمَلُ فيهم بالمعاصي ، ثمَّ يقدِرون على أن يُغيِّروا ، ثمَّ لا يُغيِّروا إلَّا يوشِكُ أن يعُمَّهم اللهُ منه بعقابٍ
الراوي : أبو بكر الصديق | المحدث : الألباني | المصدر : صحيح أبي داود
الصفحة أو الرقم: 4338 | خلاصة حكم المحدث : صحيح



لما بويع أبو بكر بالخلافة بعد بيعة السقيفة تكلم أبو بكر، فحمد الله وأثنى عليه ثم قال:
"أما بعد أيها الناس فإني قد وليت عليكم ولست بخيركم، فإن أحسنت فأعينوني، وإن أسأت فقوموني، الصدق أمانة، والكذب خيانة، والضعيف فيكم قوي عندي حتى أريح عليه حقه إن شاء الله، والقوى فيكم ضعيف حتى آخذ الحق منه إن شاء الله، لا يدع قوم الجهاد في سبيل الله إلا ضربهم الله بالذل، ولا تشيع الفاحشة في قوم قط إلا عمهم الله بالبلاء، أطيعوني ما أطعت الله ورسوله، فإذا عصيت الله ورسوله فلا طاعة لي عليكم".

(يا أيُّها الناس، قد وُلِّيت عليكم ولست بخيركم، فإن رأيتموني على حقٍّ فأعينوني، وإن رأيتموني على باطل فسدِّدوني. أطيعوني ما أطعتُ الله فيكم، فإذا عصيتُه فلا طاعة لي عليكم. ألا إنَّ أقواكم عندي الضعيف حتى آخذ الحقَّ له، وأضعفكم عندي القويُّ حتى آخذ الحقَّ منه. أقول قولي هذا وأستغفر الله لي ولكم).

أنَّ رجلًا، قال لرسولِ اللهِ صلَّى اللهُ عليه وسلَّم : رأيتُ كأنَّ ميزانًا دُلِّيَ منَ السماءِ، فوُزِنتَ بأبي بكرٍ فرجَحتَ بأبي بكرٍ، ثم وُزِن أبو بكرٍ بعُمرَ، فرجَح أبو بكرٍ، ثم وُزِن عُمرُ بعثمانَ، فرَجَح عُمرُ، ثم رُفِع الميزانُ، فاستَهَلَّها رسولُ اللهِ صلَّى اللهُ عليه وسلَّم خلافةَ نبوةٍ، ثم يؤتي اللهُ المُلكَ مَن يشاءُ

الراوي : سفينة مولى رسول الله صلى الله عليه وسلم | المحدث : البوصيري | المصدر : إتحاف الخيرة المهرة
الصفحة أو الرقم : 5/ 11 | خلاصة حكم المحدث : إسناده صحيح | أحاديث مشابهة | شرح حديث مشابه

أنَّ رجلًا قال : يا رسولَ اللهِ رأيتُ كأنَّ مِيزانًا دُلِّي مِنَ السماءِ فَوُزِنْتَ فيه أنت وأبوبكرٍ فَرَجَحْتَ بأبي بكرٍ ثم وُزِنَ فيه أبوبكرٍ وعمرُ فَرَجَحَ أبو بكرٍ بعمرَ ثم وُزِنَ فيه عمرُ وعثمانُ فَرَجَحَ عمرُ بعثمانَ ثم رُفِعَ الميزانُ فاسْتآلهَا يعني تَأَوَّلَها ثم قال : خِلافَةُ نُبُوَّةٍ ثم يُؤتِي اللهُ الملكَ مَنْ يَشاءُ

الراوي : أبو بكرة نفيع بن الحارث | المحدث : الألباني | المصدر : تخريج كتاب السنة
الصفحة أو الرقم : 1135 | خلاصة حكم المحدث : صحيح 

\section{الحساب في زمن الحكم الرشيد}
البحث والعناية بعلم الحساب هو غاية جليلة ومهمة عظيمة أعتنى بها المسلمون اللاحقون في زمن الخليفة الراشد هارون الرشيد التي أسس دار الحكمة في بغداد العراق حتى أصبح المسلمين في ذلك الوقت روادا في علم الحساب والذى كان مفتاحا لهم لشتى العلوم الأخرى حتى عرف ذلك الزمان بالعصر الإسلامي الذهبي. ومن أبرز من بحث وألف في علم الحساب هو العالم الفذ محمد بن موسى الخوارزمي رحمه الله تعالى والذي وصل صيته أقطاب الأرض حتى دخل أسمه معاجم وقواميس كافة اللغات الأخرى. فاللوغرتميات جاءت من الترجمة اللاتينية لإسمه وهو ما عرف عند العرب المتأخرين بالخوارزميات. وهذا مفهوم يبنى عليه كافة الحسابات المركبة والمعقدة التي نراها اليوم من انظمة الحساب والمنطق بشتى أنواعها بما فيها أنظمة الصواريخ والطيران وحتى انظمة الذكاء الإصطناعي. وقد ألف الخوارزمي كتابه "المختصر في الجبر والمقابلة" وكان هذا الكتاب نافعا للمسلمين وغيرهم وهو أساس تقدم البشرية في شتى المجالا إلى يومنا هذا. ولهذا سمي علم الموازنة والمقابلة بعلم "الجبر" كما سماه الخوارزمي بذلك وتمت اضافة كلمة "الجبر" أيضا إلى كافة معاجم اللغات الأخرى. ويعتبر الخوارزمي إلى يومنا هو مؤسس علم الجبر والحساب والخوارزميات ومن أهم علماء الحساب في تاريخ البشرية.

وللأسف فقد غاب وغيب على أغلب المسلمين في زماننا هذا أهمية ميراث الخوارزمي في علم الجبر والحساب. وهو ميراث حري بنا جمع شتاته وإعادة بناء أركانه لتقوم الأمة بالميزان الذي أمرنا الله به. فقد جهل الكثير من المسليمن ميراث الخوارزمي حتى بخس قدره ونسي علمه فكان بين مفرط أو مدلس. ومن ذلك ضياع كتابه في الجبر والمقابلة من المسلمين حتى تمت طباعة أول نسحة عربية منه في عام 1939م (1357هـ) بناء على النسخة الأصلية الوحيدة التي سرقت من مصر ونقلت إلى بريطانيا والتي يرجع تاريخها إلى عام 1439م (843هـ) أي بعد وفاة الخوارزمي بحوالي 500 عام شمسية.
ليرجع لنا كتاب الخوارزمي بعد حوالي ألف عام من تأليفه.
وفي كل هذه الأعوام ترجم كتابه إلى شتى اللغات ومنها الأنجليزية والألمانية والفرنسية وأصبحت مرجعا لجميع الحضارات الأوروبية وغيرها.
ليتفاجأ المسلمين بوجود كلمات عربية في هذه الثقافات ومنها algorithms والتي تعني الخوارزميات وكلمة algebra وهي الجبر في معجم اللغة الانجليزية على سبيل المثال لا الحصر.

ومن التدليس الذي تعرض له الخوارزمي في تقديم كتابه هو نسبة عمله إلى الحضارة المصرية في طرح مخالف للطرح الذي وضعه الخوارزمي في كتابه. وهذا ليس إلا إحقاقا للحق ولا يجب أن يحمل هذا على محمل الإستنقاص لمن نقل هذا العمل لنا تقديما وتعليقا فجزاهم الله خير الجزاء. ومن التدليس أيضا طرح كتابه في الحساب مجردا من الغاية التي كتب لها ومنه عدم ذكر سبب تأليف كتابه في الجبر والذي كان في الأساس سعيا منه رحمه الله لتحقيق الحكم الرشيد بناء على الحساب الصحيح في الميراث والبيع والشراء والكراء وما بتعلق بذلك من حساب المسافات والأرض. وليتبين طرح الخوارزمي نضع مقدمة كتابه رحمه الله والتي جاء فيها:\footnote{مع تصرف يسير من حذف لكلمات التي تخالف السياق وفي الغالب قد يظن انها أخطاء خلال النسخ.}

\newpage
\fbox{\small
    \begin{minipage}{34em}
        \begin{center}
            بسم الله الرحمن الرحيم
        \end{center}
        هذا كتاب وضعه محمد بن موسى الخوارزمي افتتحه بأن قال:

        الحمد الله على نعمه بما هو أهله من محامده التي بأداء ما افترض منها على من يعبده من خلقه يقع اسم الشكر ويستوجب المزيد إقرارا بروبويته وتذللا لعزته وخشوعا لعظمته. بعث محمدا صلى الله عليه وعلى آله وسلم بالنبوة على حين فترة من الرسل نورا من الحق ودروس من الهدي فبصر به من العمى واستنقذ به من الهلكة وكثر به بعد قلة وألف به بعد الشتات.

        تبارك الله ربنا وتعالى جده وتقدست أسماؤه ولا إله غيره, وصلى الله على محمد النبي وآله وسلم. ولم تزل العلماء في الأزمنة الخالية والأمم الماضية يكتبون الكتب مما يصنفون من صنوف العلم ووجوه الحكمة نظرا لمن بعدهم واحتسابا للأجر بقدر الطاقة ورجاء أن يلحقهم من أجر ذلك وذخره وذكره ويبقى لهم من لسان صدق ما يصغر في جنبه كثير مما كانوا يتكفلونه من المؤونة ويحملونه على أنفسهم من المشقة في كشف أسرار العلم وغامضه. إما رجل سبق إلى مالم يكن مستخرجا قبله فورثه من بعده. وإما رجل شرح مما أبقى الأولون ما كان مستغلقا فأوضح طريقه وسهل مسلكه وقرب مأخذه. وإما رجل وجد في بعض الكتب خللا فلم شعثه وأقام أوده وأحسن الظن بصاحبه غير راد عليه ولا مفتخر بذلك من فعل نفسه.

        وقد شجعني ما فضل الله به الامام المأمون أمير المؤمنين مع الخلافة التي حاز له إرثها وأكرمه بلباسها وحلاه بزينتها, من الرغبة في الأدب وتقريب أهله وإدنائهم وبسط كنفه لهم ومعونته إياهم على إيضاح ما كان مستبهما وتسهيل ما كان مستوعرا. على أن ألفت من كتاب الجبر والمقابلة كتابا مختصرا حاصرا للطيف الحساب وجليله لما يلزم الناس من الحاجة إليه في مواريثهم ووصياهم وفي مقاسمتهم وأحكامهم وتجارتهم, وفي جميع ما بتعاملون به بينهم من مساحة الأرضين وكرى الأنهار والهندسة وغير ذلك من وجوهه وفنونه, مقدما لحسن النية فيه وراجيا لأن ينزله أهل الأدب بفضل ما استودعوا من نعم الله تعالى وجليل آلائه وجميل بلائه عندهم منزلته وبالله توفيقي في هذا لا في غيره عليه توكلت وهو رب العرش العظيم وصلى الله على جميع الأنبياء والمرسلين.
    \end{minipage}
}

وعليه يلعم أن الخوارزمي رحمه الله إنما ألف كتابه هذا لتوضيح علم الحساب الصحيح الذي يحتاج إليه الناس في أمور دينهم ودنياهم. فقد إفتتح الخوارزمي رحمه الله كتابه بالبسملة متبعا سنة الأنبياء في ذلك. وكان رحمه الله حريصا وراجيا بأن يعتنى أهل الأدب بهذا الكتاب ويعطونه حقه وينزلونه منزلته لما علم ما فيه من أسس وقواعد لا غنى عنها في علم الحساب الصحيح. وختم مقدمته سائلا الله التوفيق في ذلك ومتوكلا عليه. وبهذا يتبين حسن مقصد الخوارزمي من تأليف كتابه فنسأل الله العلي العظيم أن يرحمه رحمة واسعة وأن يرفغ قدره في الجنة وأن يجزيه عنا خير الجزاء.

فبدأو بالعناية بعلم الحساب وجمع مؤلفاته من كافة أقطاب الدنيا فعكفوا على ترجمتها حتى فهموها وعقلوها وعرفوا ما شابها من خطأ ونقصان. فأسسوا نظام الأرقام الذي نعرفه اليوم فقسموا الأرقام إلى ارقام فردية وأسسوا علم الجبر وحساب المثلثات وغيرها من علوم الحساب بشكل لم تعرفه البشرية من قبل. وكان ذلك سببا في تحقيق الحكم الرشيد في المعاملات والبيع والشراء والكراء. فكان علم الحساب مفتاحا في تطور المسلمين في شتى مجالات الدنيا ومنها مجال الهندسة والطب في العصر الإسلامي الذهبي.


سيرة هارون الرشيد
اللحيدان
%\href{https://www.youtube.com/watch?v=0QDv__BL1VQ}{video}

الفوزان
%\href{https://www.alfawzan.af.org.sa/ar/node/7212}{text}


كلام الفوزان في المأمون
%\href{https://www.youtube.com/watch?v=dwUnqfHH7eI}{video}
غرر به المعتزلة

ومن الحكم الرشيد مصالحة الكفار لدرء المفاسد كما في العهد المكي
وفيه ايضا ان النجاشي لم يكن مسلم ولا  يظلم عنده أحد فهو حقق العدل
%\href{https://www.youtube.com/watch?v=bZH3NTKYoOQ}{video}



\section{عودة الحكم الرشيد في آخر الزَّمانِ}

قالَ: كانَتْ بَنُو إسْرائِيلَ تَسُوسُهُمُ الأنْبِياءُ، كُلَّما هَلَكَ نَبِيٌّ خَلَفَهُ نَبِيٌّ، وإنَّه لا نَبِيَّ بَعْدِي، وسَيَكونُ خُلَفاءُ فَيَكْثُرُونَ. قالوا: فَما تَأْمُرُنا؟ قالَ: فُوا ببَيْعَةِ الأوَّلِ فالأوَّلِ، أعْطُوهُمْ حَقَّهُمْ؛ فإنَّ اللَّهَ سائِلُهُمْ عَمَّا اسْتَرْعاهُمْ.

‎يَكونُ في آخِرِ الزَّمانِ خَلِيفَةٌ يَقْسِمُ المالَ ولا يَعُدُّهُ.
‎الراوي : أبو سعيد الخدري وجابر بن عبدالله | المحدث :مسلم | المصدر : صحيح مسلم
‎الصفحة أو الرقم: 2913 | خلاصة حكم المحدث : [صحيح]


‎مِنْ خُلَفائِكُمْ خَلِيفَةٌ يَحْثُو المالَ حَثْيًا، لا يَعُدُّهُ عَدَدًا. وفي رِوايَةِ ابْنِ حُجْرٍ: يَحْثِي المالَ.
‎الراوي : أبو سعيد الخدري | المحدث : مسلم | المصدر : صحيح مسلم | الصفحة أو الرقم : 2914 | خلاصة حكم المحدث : [صحيح]
   





\section{معادلات}
فيما يلي مثال على معادلة رياضية:

\begin{equation}
    E = mc^2
\end{equation}

ومثال آخر على معادلة معقدة:

\begin{equation}
    \int_0^\infty e^{-x^2} \, dx = \frac{\sqrt{\pi}}{2}
\end{equation}

\newpage

\section{نص الفصل الأول - الصفحة الثانية}

هذه الصفحة الثانية للفصل الأول تحتوي على نص إضافي لتوضيح كيفية تنسيق النصوص في كتب اللاتكس باللغة العربية.

\newpage

\section{نص الفصل الأول - الصفحة الثالثة}

هذه الصفحة الثالثة للفصل الأول تحتوي على المزيد من النصوص لاختبار تقسيم الصفحات وظهور الرؤوس والأقدام بشكل صحيح في النصوص العربية.
