
\chapter{مقدمة}

\begin{center}
    بسم الله الرحمن الرحيم
\end{center}

الحمد الله العزيز العليم فالق الحب والنوى خلق كل شئ بقدر معلوم فقدره تقديرا. جعل الشمس والقمر حسبانا فقدر للقمر منازلا وجعل الشمس تجري لمستقر لها وكل في فلك يسبحون. لا إله إلا هو وحده لا شريك له  إيمانا بروبويته وإقرارا بألوهيته وتصديقا بأسمائه وصفاته على الوجه الذي يحب ويرضى من غير تحريف ولا تمثيل. بعث الرسل مبشرين ومنذرين بالحق وليبينوا للناس أمور دنيهم ومن أجلها إفراده سبحانه وحده بالعبادة بالطريقة التي ارتضاها. ومن رحمته أنه أرسل محمدا ﷺ خاتما للنبيئين وأنزل عليه القرآن هدى وبشرى للمتقين. أما بعد:

هذا كتاب ألفته لشرح علم الحساب الصحيح والحديث في دقه وجليله باللغة العربية سالكا بذلك مسلك أهل الأدب في فنون ووجوه علم الرياضيات والفيزياء وعلوم الآلة وما يمكن حسابه ومتبعا في ذلك ما جاء في كتاب الله عز وجل وسنة نبه ﷺ وراجعا لأهل العلم الشرعي فيما يتعلق بمسائل الحساب التي تفيد الناس في أمور دينهم ودنياهم ومن أجلها إقامة الميزان والذي هو من أركان الحكم الرشيد الذي أمر الله به.

فعلم الحساب من العلم الذي حث الله تعالى عليه في موضعين في كتابه.
قال تعالى: \quranayah*[17][12]{\footnotesize \surahname*[17]}.
يقول السعدي رحمه الله في تفسيره:
{ وَلِتَعْلَمُوا } بتوالي الليل والنهار واختلاف القمر { عَدَدَ السِّنِينَ وَالْحِسَابَ } فتبنون عليها ما تشاءون من مصالحكم. { وَكُلَّ شَيْءٍ فَصَّلْنَاهُ تَفْصِيلًا } أي: بينا الآيات وصرفناه لتتميز الأشياء ويستبين الحق من الباطل كما قال تعالى: { مَا فَرَّطْنَا فِي الْكِتَابِ مِنْ شَيْءٍ }
[هـ].

وقال تعالى: \quranayah*[10][5]{\footnotesize \surahname*[10]}.
يقول السعدي رحمه الله في تفسير هذه الأيات:
وفي هذه الآيات الحث والترغيب على التفكر في مخلوقات الله، والنظر فيها بعين الاعتبار، فإن بذلك تنفتح البصيرة، ويزداد الإيمان والعقل، وتقوى القريحة، وفي إهمال ذلك، تهاون بما أمر الله به، وإغلاق لزيادة الإيمان، وجمود للذهن والقريحة
[هـ].

فعلم الحساب يحتاج إليه لغايات عديدة وأجلها إقامة الميزان ومنها اتباع أيات الله الشرعية وفهم ايات الله الكونية. من الأمثلة لتطبيقات علم الحساب في اتباع ايات الله الشرعية كعلم المواريث والبيع والشراء وغيرها من المعاملات التي يحتاج اليها الناس في أمور دينهم ودنياهم.  ومن الأمثلة على تطبيقات علم الحساب في فهم ايات الله الكونية كدراسة الظواهر الطبيعية التي خلقها الله كحركة الشمس والقمر لما في ذلك من تفكر في عظمة الله وزيادة في الإيمان.

ورغم أن علم الحساب قد يدرس لغايات كونية محضة إلا أن العلم به من الضروريات لإقامة الحق والميزان. ولهذا جاء الحث عليه في كتاب الله عز وجل. فإن وافق هذا الحساب أيات الله الشرعية أو الفطرة السليمة فهو من العدل والإصلاح الذي أمر الله به. وإن خالف أمر الله أو الفطرة التي فطر الله الناس عليها فهو من الظلم والفساد الذي لا يرضى الله به ومن أسباب تعجيل سخط الله في الدنيا قبل الأخرة. وهذا بالعموم على المسلمين وغيرهم. فقد قال تعالى:
وَمَا كَانَ رَبُّكَ لِيُهْلِكَ ٱلْقُرَىٰ بِظُلْمٍۢ وَأَهْلُهَا مُصْلِحُونَ ١١٧
(هود)
وبهذا يكون الحساب سبيلا إلى الحكم بالعدل بما يوافق الفطرة السليمة ونجاة من سخط الله وعذابه العاجل في الدنيا. فإن وافق الحساب الحكم الشرعي مع الفطرة كان ذلك نجاة في الدنيا والأخرة وكان حكما راشدا. ولهذا كان الحساب لإقامة الحق والميزان من بنيان الحكم الرشيد. وعليه يكون علم الحساب من الدين بالضرورة وليس بخلاف ذلك إذ يتعذر إقامة الميزان حق إقامته من دون حساب صحيح.

والحساب الصحيح لا يقام الا بالميزان. فقد سماه الخوارزمي رحمه الله تعالى الجبر والمقابلة بحيث يبنى الحساب على التساوي بين المتغيرات لجبر ما اختل من الميزان, عليه يمكن حساب ما جهل منها. فقد قال ابن تيمية رحمه الله عن هذا:
وأما حساب الفرائض ومعرفة أصول المسائل وتصحيحها والمناسخات وقسمة التركات، وهذا الثاني كله علم معقول يُعلم بالعقل، كسائر حساب المعاملات وغير ذلك من الأنواع التي يحتاج إليها الناس [.] ثم قد ذكروا حساب المجهول الملقب بحساب الجبر والمقابلة في ذلك، وهو علم قديم [.] أول من عرف أنه أدخله فيها محمد بن موسى الخوارزمي. وبعض الناس يذكر عن علي بن أبي طالب أنه تكلم فيه، وأنه تعلم ذلك من يهودي، وهذا كذب على علي
[هـ].
{\footnotesize (مجموع الفتاوى 9/214)}.
ويقول اين تيمية فيه أيضا:
وكذلك كثير من متأخري أصحابنا يشتغلون وقت بطالتهم بعلم الفرائض والحساب والجبر والمقابلة والهندسة ونحو ذلك، لأن فيه تفريحاً للنفس، وهو علم صحيح لا يدخل فيه غلط
[هـ].
{\footnotesize (مجموع الفتاوى ج9/ص129)}.

ومن أعظم البلايا في زماننا هذا أن المسلمين في غالبهم قد أضاعوا هذا العلم العظيم ظنا منهم أنه ليس من الدين في شئ بعد أن كانوا روادا فيه ووضعوا أسسه وقواعده في زمن هارون الرشيد كما سيأتي. فاعتنت وتسابقت وتهالفت عليه الأمم الأخرى وكان سببا في نهوضها وإزدهارها بل وأيضا تسلطها على أمة الإسلام. فتضيع علم الحساب من الأمية التي جاء الإسلام بالحث على خلافها من طلب العلم ونشره وإقامة الحق والميزان. فالأمية لا تكون فقط بعدم القدرة على القراءة والكتابة كما هو شائع, وإنما ايضا بعدم القدرة على الحساب.
ومما يأكد هذا الطرح قوله ﷺ عندما سأل عن عدد الأيام في الشهر فقال ﷺ:
إنَّا أُمَّةٌ أُمِّيَّةٌ، لا نَكْتُبُ ولَا نَحْسُبُ، الشَّهْرُ هَكَذَا وهَكَذَا. يَعْنِي مَرَّةً تِسْعَةً وعِشْرِينَ، ومَرَّةً ثَلَاثِينَ"
{\footnotesize (صحيح البخاري)}.
فجعل ﷺ الجهل بعلم الحساب من الأمية.

ولهذا جاءت الشريعة بالحث أولا على القراءة ومن ثم الحساب. فكان أول ما أنزل الله "إقرا" وفيه الحث لأمة الإسلام على تعلم القراءة والكتابة وطلب العلم ونشره.
يقول تعالى:
\quranayah*[96][1-5]{\footnotesize \surahname*[96]}.
ومن ثم جاء الحث على الحساب لإقامة الحق والميزان في قوله تعالى:
\quranayah*[10][5]{\footnotesize \surahname*[10]}.

ومن المعلوم أن الرسول ﷺ كان أميا كما قال تعالى:
\quranayah*[7][158]{\footnotesize \surahname*[7]}.
وكذلك الصحابة رضى الله عنهم أجمعين وصفهم الله بالأمية في قوله تعالى:
\quranayah*[62][2]{\footnotesize \surahname*[62]}.

فالرسول ﷺ والصحابة في غالبهم كانوا أمة أمية لا يقرأون ولا يكتبون ولا بحسبون كما ورد هذا في كتاب الله وسنة نبيه ﷺ.
يقول الشيخ اين باز رحمه الله تعالى:
فكان النبي ﷺ مثلما قال الله: وَوَجَدَكَ ضَالًّا فَهَدَى [الضحى:7]، جاهلاً بالعلوم التي جاء بها الوحي، لم يكن عنده علم بما شرع الله له في كتابه العظيم، ولم يكن عنده علم بعلوم الأولين المرسلين، ولم يكن يكتب ويخط حتى جاءه هذا الخير العظيم والوحي العظيم عليه الصلاة والسلام، فكل إنسان لم يتعلم ولم يكتب يقال له: أمي، والأمة العربية هكذا كان الغالب عليها أنها أمية لا تكتب ولا تقرأ، هذا الغالب على أمة محمد ﷺ
[هـ].

ونزيد على ذلك أن كل إنسان لم يتعلم الحساب مع القراءة والكتابة يكون أميا كما بين ذلك النبي ﷺ. والله حث هذه الأمة الأمية في كتابه العظيم على العلم الذي يتأتي بالقراءة والكتابة حتى تقيم الحق والميزان الذي يتأتى بالحساب الصحيح. فالله جل جلاله أنزل كتابه لغاية عظيمة وهي إقامة الحق والميزان فقال تعالى:
\quranayah*[42][17]{\footnotesize \surahname*[42]}.
فذكر الله الميزان إلحاقا بالحق لان الحق لا يقام إلا بالميزان. بل إن الله جعل الميزان  عمادا للدنيا والاءخرة. يقول جل في علاه:
\quranayah*[55][7-9]{\footnotesize \surahname*[55]}. وهذا فيه الحث على إقامة الوزن بدون خسران ولهذا لا يقام الميزان إلا بالحساب الصحيح.

فالله عز وجل أرسل رسله بالكتاب والميزان لغاية جليلة ألا وهي إقامة الحق والعدل يقول جل جلاله:
\quranayah*[57][25]{\footnotesize \surahname*[57]}.

ومن الميزان الكيل. فالميزان أعم وأشمل والكيل هو وزن الأشياء. فالعدل في الكيل والوزن من إقامة الميزان وهو من الأمور التي اوصى الله تعالى بها, فهي من الوصايا العشر من سورة الأنعام في قوله تعالى:
\quranayah*[6][152]{\footnotesize \surahname*[6]}.
وأيضا من الوصايا التي ذكرها الله في سورة الإسراء في قوله تعالى:
\quranayah*[17][35]{\footnotesize \surahname*[17]}.

ولما كان الأنبياء أعلم الناس بأمر الله وأحرصهم, فقد أقاموا المكيال والميزان حق إقامته في حكمهم الرشيد بين الخلق. ومثال ذلك يوسف عليه السلام في قوله تعالى:
\quranayah*[12][55]{\footnotesize \surahname*[12]}.
حيث قال لإخوته عن الكيل:
\quranayah*[12][59-60]{\footnotesize \surahname*[12]}.

فلا شك أن التفريط في الكيل والميزان من اعظم البلايا التي حذر الله عز وجل منها في كتابه الكريم. فالظلم في الكيل والميزان من الإفساد العظيم ومن أسباب تعجيل العذاب في الدنيا قبل الأخرة. وفي قصة مدين مع نبيهم شعيبا العبرة الواضحة في ذلك. يقول تعالى على لسان نبيه شعيب محذرا قومه:
\quranayah*[7][85]{\footnotesize \surahname*[7]}.
وفي موضع أخر من سورة الشعراء:
\quranayah*[26][181-183]{\footnotesize \surahname*[26]}.
وفي سورة هود:
\quranayah*[11][84-85]{\footnotesize \surahname*[11]}.

ونبينا ﷺ كان من أحرص الناس في إقامة الكيل والميزان وحذر من الفساد في ذلك في العديد من المواضع منها ما ورد عن ابن عباس رضي الله عنه أنه قَالَ: قَالَ رَسُولُ اللَّهِ صَلَّى اللَّهُ عَلَيْهِ وَسَلَّمَ لِأَصْحَابِ الْكَيْلِ وَالْمِيزَانِ: «إِنَّكُمْ قَدْ وُلِّيتُمْ أَمْرَيْنِ هَلَكَتْ فِيهِمَا الْأُمَمُ السَّابِقَة قبلكُمْ».
{\footnotesize رَوَاهُ التِّرْمِذِيّ}.
وعَنْ عَبْدِ اللَّهِ بْنِ عُمَرَ، قَالَ أَقْبَلَ عَلَيْنَا رَسُولُ اللَّهِ ـ ﷺ ـ فَقَالَ «يَا مَعْشَرَ الْمُهَاجِرِينَ خَمْسٌ إِذَا ابْتُلِيتُمْ بِهِنَّ وَأَعُوذُ بِاللَّهِ أَنْ تُدْرِكُوهُنَّ: لَمْ تَظْهَرِ الْفَاحِشَةُ فِي قَوْمٍ قَطُّ حَتَّى يُعْلِنُوا بِهَا إِلاَّ فَشَا فِيهِمُ الطَّاعُونُ وَالأَوْجَاعُ الَّتِي لَمْ تَكُنْ مَضَتْ فِي أَسْلاَفِهِمُ الَّذِينَ مَضَوْا, وَلَمْ يَنْقُصُوا الْمِكْيَالَ وَالْمِيزَانَ إِلاَّ أُخِذُوا بِالسِّنِينَ وَشِدَّةِ الْمَؤُنَةِ وَجَوْرِ السُّلْطَانِ عَلَيْهِمْ, وَلَمْ يَمْنَعُوا زَكَاةَ أَمْوَالِهِمْ إِلاَّ مُنِعُوا الْقَطْرَ مِنَ السَّمَاءِ وَلَوْلاَ الْبَهَائِمُ لَمْ يُمْطَرُوا, وَلَمْ يَنْقُضُوا عَهْدَ اللَّهِ وَعَهْدَ رَسُولِهِ إِلاَّ سَلَّطَ اللَّهُ عَلَيْهِمْ عَدُوًّا مِنْ غَيْرِهِمْ فَأَخَذُوا بَعْضَ مَا فِي أَيْدِيهِمْ, وَمَا لَمْ تَحْكُمْ أَئِمَّتُهُمْ بِكِتَابِ اللَّهِ وَيَتَخَيَّرُوا مِمَّا أَنْزَلَ اللَّهُ إِلاَّ جَعَلَ اللَّهُ بَأْسَهُمْ بَيْنَهُمْ».
{\footnotesize أخرجه ابن ماجه وصححه الألباني}.
فذكر ﷺ أن النقص في الكيل والميزان من أسباب البلاء العظيم ومنها الفقر والجوع وجور السلطان. وفيه الدليل على نبوته ﷺ فقد وقع ذلك كما أخبر بعد أن تهاون الكثير من المسلمين إلا من رحم الله في أمر الميزان والكيل.

وَعَنْ أَبِي سَعِيدٍ وَأَبِي هُرَيْرَةَ: أَنَّ رَسُولَ اللَّهِ صَلَّى اللَّهُ عَلَيْهِ وَسلم اسْتَعْمَلَ رَجُلًا عَلَى خَيْبَرَ فَجَاءَهُ بِتَمْرٍ جَنِيبٍ فَقَالَ: «أَكُلُّ تَمْرِ خَيْبَرَ هَكَذَا؟» قَالَ: لَا وَاللَّهِ يَا رَسُولَ اللَّهِ, إِنَّا لَنَأْخُذُ الصَّاعَ مِنْ هَذَا بِالصَّاعَيْنِ وَالصَّاعَيْنِ بِالثَّلَاثِ فَقَالَ: «لَا تَفْعَلْ بِعِ الْجَمْعَ بِالدَّرَاهِمِ ثُمَّ ابْتَعْ بِالدَّرَاهِمِ جَنِيبًا». وَقَالَ: «فِي الْمِيزَانِ مِثْلَ ذَلِكَ»
{\footnotesize مُتَّفَقٌ عَلَيْهِ}.
وهذا فيه حرص النبي ﷺ حيث انه من المعلوم أن من اخذ صاع اضافيا لا يثبت قيمة البيع فيكون من أخذ صاعين بدل صاع فقد اشترى بنصف قيمة ما باع بينما من أخذ ثلاثة بدل اثنين فقد اشترى بثلثي قيمة ما باع وهذا من الظلم الذي لا يقع إلا خطأ أو جهلا أو غشا. فأخبر النبي ﷺ أن هذا بخلاف الميزان وهو الحساب الصحيح في البيع والشراء, بل ونهى عن ذلك وأمر بأخذ القيمة عند البيع ومن ثم الشراء حتى تثبت القيمة. وهذا فيه دليل على نبوته ﷺ فهو أمي لا يحسب ولكن لا ينطق إلا بالحق كما أخبر ذلك الله عز وجل في كتابه العظيم:
\quranayah*[53][3-4]{\footnotesize \surahname*[53]}.

فقد علم الصحابة رضي الله عنهم بأهمية إقامة الميزان والكيل وأن الفساد فيهما من أساب سخط الله. ومنه ما ورد عَنْ يَحْيَى بْنِ سَعِيدٍ، أَنَّهُ سَمِعَ سَعِيدَ بْنَ الْمُسَيَّبِ، يَقُولُ إِذَا جِئْتَ أَرْضًا يُوفُونَ الْمِكْيَالَ وَالْمِيزَانَ فَأَطِلِ الْمُقَامَ بِهَا وَإِذَا جِئْتَ أَرْضًا يُنَقِّصُونَ الْمِكْيَالَ وَالْمِيزَانَ فَأَقْلِلِ الْمُقَامَ بِهَا.

ولهذا كان البحث والعناية بعلم الحساب هو غاية جليلة ومهمة عظيمة أعتنى بها المسلمون اللاحقون في زمن الخليفة الراشد هارون الرشيد التي أسس دار الحكمة في بغداد العراق حتى أصبح المسلمين في ذلك الوقت روادا في علم الحساب والذى كان مفتاحا لهم لشتى العلوم الأخرى حتى عرف ذلك الزمان بالعصر الإسلامي الذهبي. ومن أبرز من بحث وألف في علم الحساب هو العالم الفذ محمد بن موسى الخوارزمي رحمه الله تعالى والذي وصل صيته أقطاب الأرض حتى دخل أسمه معاجم وقواميس كافة اللغات الأخرى. فاللوغرتميات جاءت من الترجمة اللاتينية لإسمه وهو ما عرف عند العرب المتأخرين بالخوارزميات. وهذا مفهوم يبنى عليه كافة الحسابات المركبة والمعقدة التي نراها اليوم من انظمة الحساب والمنطق بشتى أنواعها بما فيها أنظمة الصواريخ والطيران وحتى انظمة الذكاء الإصطناعي. وقد ألف الخوارزمي كتابه "المختصر في الجبر والمقابلة" وكان هذا الكتاب نافعا للمسلمين وغيرهم وهو أساس تقدم البشرية في شتى المجالا إلى يومنا هذا. ولهذا سمي علم الموازنة والمقابلة بعلم "الجبر" كما سماه الخوارزمي بذلك وتمت اضافة كلمة "الجبر" أيضا إلى كافة معاجم اللغات الأخرى. ويعتبر الخوارزمي إلى يومنا هو مؤسس علم الجبر والحساب والخوارزميات ومن أهم علماء الحساب في تاريخ البشرية.

وللأسف فقد غاب وغيب على أغلب المسلمين في زماننا هذا أهمية ميراث الخوارزمي في علم الجبر والحساب. وهو ميراث حري بنا جمع شتاته وإعادة بناء أركانه لتقوم الأمة بالميزان الذي أمرنا الله به. فقد جهل الكثير من المسليمن ميراث الخوارزمي حتى بخس قدره ونسي علمه فكان بين مفرط أو مدلس. ومن ذلك ضياع كتابه في الجبر والمقابلة من المسلمين حتى تمت طباعة أول نسحة عربية منه في عام 1939م (1357هـ) بناء على النسخة الأصلية الوحيدة التي سرقت من مصر ونقلت إلى بريطانيا والتي يرجع تاريخها إلى عام 1439م (843هـ) أي بعد وفاة الخوارزمي بحوالي 500 عام شمسية.
ليرجع لنا كتاب الخوارزمي بعد حوالي ألف عام من تأليفه.
وفي كل هذه الأعوام ترجم كتابه إلى شتى اللغات ومنها الأنجليزية والألمانية والفرنسية وأصبحت مرجعا لجميع الحضارات الأوروبية وغيرها.
ليتفاجأ المسلمين بوجود كلمات عربية في هذه الثقافات ومنها algorithms والتي تعني الخوارزميات وكلمة algebra وهي الجبر في معجم اللغة الانجليزية على سبيل المثال لا الحصر.

ومن التدليس الذي تعرض له الخوارزمي في تقديم كتابه هو نسبة عمله إلى الحضارة المصرية في طرح مخالف للطرح الذي وضعه الخوارزمي في كتابه. وهذا ليس إلا إحقاقا للحق ولا يجب أن يحمل هذا على محمل الإستنقاص لمن نقل هذا العمل لنا تقديما وتعليقا فجزاهم الله خير الجزاء. ومن التدليس أيضا طرح كتابه في الحساب مجردا من الغاية التي كتب لها ومنه عدم ذكر سبب تأليف كتابه في الجبر والذي كان في الأساس سعيا منه رحمه الله لتحقيق الحكم الرشيد بناء على الحساب الصحيح في الميراث والبيع والشراء والكراء وما بتعلق بذلك من حساب المسافات والأرض. وليتبين طرح الخوارزمي نضع مقدمة كتابه رحمه الله والتي جاء فيها:\footnote{مع تصرف يسير من حذف لكلمات التي تخالف السياق وفي الغالب قد يظن انها أخطاء خلال النسخ.}

%\begin{mdframed}
\newpage
\fbox{\small
    \begin{minipage}{34em}
        \begin{center}
            بسم الله الرحمن الرحيم
        \end{center}
        هذا كتاب وضعه محمد بن موسى الخوارزمي افتتحه بأن قال:

        الحمد الله على نعمه بما هو أهله من محامده التي بأداء ما افترض منها على من يعبده من خلقه يقع اسم الشكر ويستوجب المزيد إقرارا بروبويته وتذللا لعزته وخشوعا لعظمته. بعث محمدا صلى الله عليه وعلى آله وسلم بالنبوة على حين فترة من الرسل نورا من الحق ودروس من الهدي فبصر به من العمى واستنقذ به من الهلكة وكثر به بعد قلة وألف به بعد الشتات.

        تبارك الله ربنا وتعالى جده وتقدست أسماؤه ولا إله غيره, وصلى الله على محمد النبي وآله وسلم. ولم تزل العلماء في الأزمنة الخالية والأمم الماضية يكتبون الكتب مما يصنفون من صنوف العلم ووجوه الحكمة نظرا لمن بعدهم واحتسابا للأجر بقدر الطاقة ورجاء أن يلحقهم من أجر ذلك وذخره وذكره ويبقى لهم من لسان صدق ما يصغر في جنبه كثير مما كانوا يتكفلونه من المؤونة ويحملونه على أنفسهم من المشقة في كشف أسرار العلم وغامضه. إما رجل سبق إلى مالم يكن مستخرجا قبله فورثه من بعده. وإما رجل شرح مما أبقى الأولون ما كان مستغلقا فأوضح طريقه وسهل مسلكه وقرب مأخذه. وإما رجل وجد في بعض الكتب خللا فلم شعثه وأقام أوده وأحسن الظن بصاحبه غير راد عليه ولا مفتخر بذلك من فعل نفسه.

        وقد شجعني ما فضل الله به الامام المأمون أمير المؤمنين مع الخلافة التي حاز له إرثها وأكرمه بلباسها وحلاه بزينتها, من الرغبة في الأدب وتقريب أهله وإدنائهم وبسط كنفه لهم ومعونته إياهم على إيضاح ما كان مستبهما وتسهيل ما كان مستوعرا. على أن ألفت من كتاب الجبر والمقابلة كتابا مختصرا حاصرا للطيف الحساب وجليله لما يلزم الناس من الحاجة إليه في مواريثهم ووصياهم وفي مقاسمتهم وأحكامهم وتجارتهم, وفي جميع ما بتعاملون به بينهم من مساحة الأرضين وكرى الأنهار والهندسة وغير ذلك من وجوهه وفنونه, مقدما لحسن النية فيه وراجيا لأن ينزله أهل الأدب بفضل ما استودعوا من نعم الله تعالى وجليل آلائه وجميل بلائه عندهم منزلته وبالله توفيقي في هذا عليه توكلت وهو رب العرش العظيم وصلى الله على جميع الأنبياء والمرسلين.
        [هـ].
    \end{minipage}
}
%\end{mdframed}

وعليه يلعم أن الخوارزمي رحمه الله انما ألف كتابه هذا لتوضيح علم الحساب الصحيح الذي يحتاج إليه الناس في أمور دينهم ودنياهم. فقد إفتتح الخوارزمي رحمه الله كتابه بالبسملة متبعا سنة الأنبياء في ذلك. وكان رحمه الله حريصا وراجيا بأن يعتنى أهل الأدب بهذا الكتاب ويعطونه حقه وينزلونه منزلته لما علم ما فيه من أسس وقواعد لا غنى عنها في علم الحساب الصحيح. وختم مقدمته سائلا الله التوفيق في ذلك ومتوكلا عليه. وبهذا يتبين حسن مقصد الخوارزمي في تأليف كتابه فنسأل الله العلي العظيم أن يرحمه رحمة واسعة وأن يرفغ قدره في الجنة وأن يجزيه عنا خير الجزاء.



فبدأو بالعناية بعلم الحساب وجمع مؤلفاته من كافة أقطاب الدنيا فعكفوا على ترجمتها حتى فهموها وعقلوها وعرفوا ما شابها من خطأ ونقصان. فأسسوا نظام الأرقام الذي نعرفه اليوم فقسموا الأرقام إلى ارقام فردية وأسسوا علم الجبر وحساب المثلثات وغيرها من علوم الحساب بشكل لم تعرفه البشرية من قبل. وكان ذلك سببا في تحقيق الحكم الرشيد في المعاملات والبيع والشراء والكراء. فكان علم الحساب مفتاحا في تطور المسلمين في شتى مجالات الدنيا ومنها مجال الهندسة والطب في العصر الإسلامي الذهبي.

