\chapter*{مقدمة}

\begin{center}
    بسم الله الرحمن الرحيم
\end{center}

الحمد الله العزيز العليم فالق الحب والنوى خلق كل شئ بقدر معلوم فقدره تقديرا. فالق الإصباح وجعل الليل سكنا والشمس والقمر حسبانا فقدر للقمر منازلا وجعل الشمس تجري لمستقر لها وكل في فلك يسبحون. لا إله إلا هو وحده لا شريك له  إيمانا بروبويته وتسليما وإقرارا بألوهيته وتصديقا بأسمائه وصفاته على الوجه الذي يحب ويرضى من غير تحريف ولا تمثيل ولا تكييف ولا تعطيل. بعث الرسل بالحق والميزان مبشرين ومنذرين وليبينوا للناس أمور دنيهم ودنياهم ومن أجلها إقامة التوحيد بإفراده سبحانه وحده بالعبادة بالطريقة التي ارتضاها وإقامة الميزان بالقسط والعدل بين الناس. ومن رحمته أنه أرسل محمدا ﷺ خاتما للنبيئين وأنزل عليه القرآن هدى وبشرى للمتقين، أما بعد:

هذا كتاب ألفته لشرح علم الحساب في دقه وجليله سالكا بذلك مسلك أهل العلم السببي في فنون ووجوه علم الرياضيات والعلوم الطبيعية كالفيزياء وعلوم الحاسب كعلوم الآلة والذكاء الإصطناعي وما يمكن حسابه مقدما ببيان وتفصيل الأمور الدينية المهمة في العقيدة والتفسيير لبيان أهمية الحساب في الحكم الرشيد متبعا في ذلك ما جاء في كتاب الله عز وجل وسنة نبه محمد ﷺ وسبيل المؤمنين من سلف هذه الأمة وعلماءها.

سميت هذا الكتاب: "الحساب الصحيح من بنيان الحكم الرشيد". فالحساب هو وسيلة يحتاجها الناس لغايات عديدة منها ما هو فرض, ومنها ما هو نافع, ومنها ما هو بخلاف ذلك. فالحساب هو مفتاح العلوم الكونية وهو السبيل لفهمها وبه تكشف العديد من حقائق وأسرار هذا الكون. فإن كانت الغاية من هذا الحساب هي منفعة الناس بالعموم في أمور دينهم ودنياهم وإقامة الميزان الشرعي الذي أمر الله به بالقسط والعدل كان هذا الحساب صحيحا وكان من بنيان الحكم الرشيد وكان سببا في الزيادة في الإيمان وإقامة العدل بين الناس وسبيلا للتطور العلمي والحضاري وقوة في دحر الأعداء ونشر الحق ونصرته.

اسئل الله العلي العظيم أن يجعل هذا العمل خالصا لوجهه الكريم وأن يبارك فيه ويجعله سببا لعودة أمة الإسلام إلى الطريق المستقيم وأن يجعل دعوتنا دعوة الراسخين في العلم كما في قوله تعالى:
\quranayah*[3][7][32]\quranayah*[3][8]{\footnotesize \surahname*[3]}. اللهم اجعل دعائنا كدعاء نبينا ﷺ كما جاء عن عائشة أم المؤمنين أن النبي ﷺ كان إذَا قَامَ مِنَ اللَّيْلِ افْتَتَحَ صَلَاتَهُ: اللَّهُمَّ رَبَّ جِبْرَائِيلَ، وَمِيكَائِيلَ، وإسْرَافِيلَ، فَاطِرَ السَّمَوَاتِ وَالأرْضِ، عَالِمَ الغَيْبِ وَالشَّهَادَةِ، أَنْتَ تَحْكُمُ بيْنَ عِبَادِكَ فِيما كَانُوا فيه يَخْتَلِفُونَ، اهْدِنِي لِما اخْتُلِفَ فيه مِنَ الحَقِّ بإذْنِكَ؛ إنَّكَ تَهْدِي مَن تَشَاءُ إلى صِرَاطٍ مُسْتَقِيمٍ {\footnotesize (صحيح مسلم)}. وكما جاء أيضا عن عائشة أم المؤمنين أن النبي ﷺ علمها هذا الدعاء: 
للَّهمَّ إنِّي أسألُكَ مِنَ الخيرِ كلِّهِ عاجلِهِ وآجلِهِ ، ما عَلِمْتُ منهُ وما لم أعلَمْ ، وأعوذُ بِكَ منَ الشَّرِّ كلِّهِ عاجلِهِ وآجلِهِ ، ما عَلِمْتُ منهُ وما لم أعلَمْ ، اللَّهمَّ إنِّي أسألُكَ من خيرِ ما سألَكَ عبدُكَ ونبيُّكَ ، وأعوذُ بِكَ من شرِّ ما عاذَ بِهِ عبدُكَ ونبيُّكَ ، اللَّهمَّ إنِّي أسألُكَ الجنَّةَ وما قرَّبَ إليها من قَولٍ أو عملٍ ، وأعوذُ بِكَ منَ النَّارِ وما قرَّبَ إليها من قولٍ أو عملٍ ، وأسألُكَ أن تجعلَ كلَّ قَضاءٍ قضيتَهُ لي خيرًا {\footnotesize (صحيح ابن ماجه وصححه الألباني)}. وكما جاء عن أم المؤمنين أم سلمة أن أكثر دعاء نبينا ﷺ كان: اللهم يا مقلب القلوب ثبت قلبي على دينك {\footnotesize (صحيح الترمذي وصححه الألباني)}.  وكما جاء عن عبدالله بن عمر أن نبينا ﷺ قلَّما يقوم من مجلس حتى يدعو بهؤلاء الدعوات لأصحابه: اللهمَّ اقسِمْ لنا مِنْ خشيَتِكَ ما تحولُ بِهِ بينَنَا وبينَ معاصيكَ ، ومِنْ طاعَتِكَ ما تُبَلِّغُنَا بِهِ جنتَكَ ، ومِنَ اليقينِ ما تُهَوِّنُ بِهِ علَيْنَا مصائِبَ الدُّنيا ، اللهمَّ متِّعْنَا بأسماعِنا ، وأبصارِنا ، وقوَّتِنا ما أحْيَيْتَنا ، واجعلْهُ الوارِثَ مِنَّا ، واجعَلْ ثَأْرَنا عَلَى مَنْ ظلَمَنا ، وانصرْنا عَلَى مَنْ عادَانا ، ولا تَجْعَلِ مُصِيبَتَنا في دينِنِا ، ولَا تَجْعَلْ الدنيا أكبرَ هَمِّنَا ، ولَا مَبْلَغَ عِلْمِنا ، ولَا تُسَلِّطْ عَلَيْنا مَنْ لَا يرْحَمُنا {\footnotesize (صحيح الترمذي)}. وكما جاء عن زيد بن أرقم رضي الله عنه أنه قال: لا أُعلِّمُكم إلا ما كان رسولُ اللهِ ﷺ يُعلِّمُنا: اللَّهمَّ إني أعوذُ بك من العجزِ والكسلِ، والبخلِ والجبنِ، والهَرَمِ وعذابِ القبرِ، اللَّهمَّ آتِ نفسي تقْوَاها وزكِّها أنت خيرُ من زكَّاها أنت ولِيُّها ومولاها، اللَّهمَّ إني أعوذُ بك من قلبٍ لا يخشعُ ومن نفسٍ لا تشبعُ وعلمٍ لا ينفعُ ودعوةٌ لا يُستجابُ لها {\footnotesize (صحيح النسائي)}. وعن أنس ابن مالك أنه قال: كثيرًا ما كُنتُ أسمعُ النَّبيَّ صلَّى اللَّهُ علَيهِ وسلَّمَ يدعو بِهَؤلاءِ الكلِماتِ (وفي رواية في صحيح النسائي: لا يدعهُنَّ) اللَّهمَّ إنِّي أعوذُ بِكَ منَ الهمِّ والحزنِ والعَجزِ والكَسلِ والبُخلِ وضَلَعِ الدَّينِ وغلبةِ الرِّجالِ {\footnotesize (صحيح الترمذي، صححه الألباني)} وفي زيادة: 


اللهم اعنا على إقامة الحق واجعلنا من المهتدين، واعنا على إقامة الميزان واجعلنا من المقسطين، واهدنا إلى الرشاد واجعلنا من المصلحين، وزدنا علما واجعلنا من المتقين. اللهم اهدنا إلى الإسلام واجعلنا من الذاكرين، واهدنا إلى الإيمان واجعلنا من المخلصين، واهدنا إلى الإحسان واجعلنا من الموقنين. اللهم ربنا نسألك الصلاح والصبر واليقين والهدى والتقى والعفاف والغنى واجعلنا اللهم من الشاكرين والفائزين، اللهم اغفر لنا ولوالدينا ولجميع المسلمين والمسلمات والمؤمنين والمؤمنات الأحياء منهم والأموات فأنت سبحانك أرحم الرحمين. وصلى الله على نبينا محمد وعلى آله وصحبه وسلم.

\newpage
\vspace*{\fill}
\textbf{ملاحظة:}

هذا البحث ما هو إلا اجتهاد شخصي للؤلف وقد يحتوي على أخطاء ونقص، فما وافق الحق فمن الله جل جلاله وما خالفه فمن نفسي واستغفر الله وأتوب إليه. يمكن للقارئ الكريم المساهمة بالنقض البنَّاء في تصحيح وتحسين هذا الكتاب بإرسال ملاحظاته ومقترحاته وتعليقاته على البريد الإلكتروني.
