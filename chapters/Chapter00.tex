\chapter{مقدمة}

\begin{center}
    بسم الله الرحمن الرحيم
\end{center}

الحمد الله العزيز العليم فالق الحب والنوى خلق كل شئ بقدر معلوم فقدره تقديرا. جعل الشمس والقمر حسبانا فقدر للقمر منازلا وجعل الشمس تجري لمستقر لها وكل في فلك يسبحون. لا إله إلا هو وحده لا شريك له  إيمانا بروبويته وتسليما وإقرارا بألوهيته وتصديقا بأسمائه وصفاته على الوجه الذي يحب ويرضى من غير تحريف ولا تمثيل ولا تكييف ولا تعطيل. بعث الرسل بالحق والميزان مبشرين ومنذرين وليبينوا للناس أمور دنيهم ودنياهم ومن أجلها إقامة التوحيد بإفراده سبحانه وحده بالعبادة بالطريقة التي ارتضاها وإقامة الميزان\comment{الشرعي} بالقسط والعدل بين الناس كما أمر الله بذلك. ومن رحمته أنه أرسل محمدا ﷺ خاتما للنبيئين وأنزل عليه القرآن هدى وبشرى للمتقين. أما بعد:

هذا كتاب ألفته لشرح علم الحساب في دقه وجليله سالكا بذلك مسلك أهل الأدب في فنون ووجوه علم الرياضيات والعلوم الطبيعية كالفيزياء وعلوم الحاسب كعلوم الآلة والذكاء الإصطناعي وما يمكن حسابه مصحوبا بشرح ما تيسر من بيان الحكم الرشيد متبعا في ذلك ما جاء في كتاب الله عز وجل وسنة نبه ﷺ وراجعا لأهل العلم الشرعي فيما يتعلق بمسائل الحساب التي تفيد الناس في أمور دينهم ودنياهم ومن أجلها إقامة الميزان الشرعي الذي أمر الله به والذي هو من أركان الحكم الرشيد. ولهذا سميت هذا الكتاب: "الحساب الصحيح من بنيان الحكم الرشيد". فالحساب هو وسيلة يحتاجها الناس لغايات عديدة منها ما هو فرض, ومنها ما هو نافع, ومنها ما هو بخلاف ذلك. فإن كانت الغاية من هذا الحساب هي منفعة الناس بالعموم في أمور دينهم ودنياهم وإقامة الميزان الشرعي الذي أمر الله به بالقسط والعدل كان هذا الحساب صحيحا وكان من بنيان الحكم الرشيد وكان سببا في الزيادة في الإيمان وإقامة العدل بين الناس وسبيلا للتطور العلمي والحضاري وقوة في دحر الأعداء ونشر الحق ونصرته. اسئل الله العلي العظيم أن يجعل هذا العمل خالصا لوجهه الكريم وأن يبارك فيه ويجعله سببا لعودة أمة الإسلام إلى الطريق المستقيم.
